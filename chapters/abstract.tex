% ------------------------------------------------------------------------------
\chapter*{Abstract}
% ------------------------------------------------------------------------------

% ca 500-600 words?

This thesis presents a search for Higgs boson pair production in final states
with two $b$-quarks and two \tauleptons using \SI{139}{\per\femto\barn} of
proton--proton collision data at a centre-of-mass energy of \SI{13}{\TeV}
recorded by the ATLAS detector at the Large Hadron Collider in 2015--2018.
Searches for Higgs boson pair production are important to solidify our
understanding of electroweak symmetry breaking in the Standard Model (SM) and,
in addition, serve as probes for physics beyond the SM.

The search presented in this thesis targets channels with at least one hadronic
\taulepton decay (i.e.~$\tau \to \nu_{\tau} + \text{hadrons}$). A crucial
component of this search is the ability to distinguish the signatures of
hadronic \taulepton decays from those of quark- or gluon-initiated jets in the
ATLAS detector. For this purpose, a novel \tauid algorithm based on recurrent
neural networks is introduced. The algorithm exploits information of
reconstructed charged-particle tracks, topological clusters of calorimeter cell
signals, and purposefully constructed discriminating variables
to\todo{TODO}. The new \tauid method significantly outperforms the methods used
perviously and has since become the default algorithm for \tauid at the ATLAS
experiment.

First, a search for non-resonant Higgs boson pair production predicted by the SM
is presented. The production of Higgs boson pairs via gluon--gluon fusion (\ggF)
and vector boson fusion (VBF) are considered in the search. No statistically
significant deviation from the background-only hypothesis is observed;
therefore, upper limits are set on the production cross section \xsecggfvbf and
the signal strength~$\mu = \xsecggfvbf / \xsecggfvbf^{\text{SM}}$ of SM Higgs
boson pair production, where $\xsecggfvbf^{\text{SM}}$ is the cross section
predicted by the SM. The search yields upper limits of $\mu < \num{4.7}$ and
$\xsecggfvbf < \SI{140}{\femto\barn}$ at \SI{95}{\percent} confidence level.

% BSM enhancements of the Higgs boson pair production cross section due to
% resonant production of

, for example, as it arises in models with extended Higgs sectors.

narrow width, scalar resonances with masses ranging from
\SIrange{251}{1600}{\GeV}.

% Depending on the mass of the resonance, the upper limits on the cross section
% of $\pp \to X \to \HH$ range from \SIrange{20}{900}{\femto\barn}

% Search for resonant \HH production.
% - No statistically significant excess is observed.
% - Limits are set for resonance masses ranging from 251 GeV to 1600 GeV
% - The largest excess is at 1 TeV with a local (global) significance of 3.1 (2.0) $\sigma$

Lastly, the search for non-resonant Higgs boson pair production is reinterpreted
in terms of anomalous values of the Higgs boson self-coupling constant
$\lambdahhh$. Upper limits are set at \SI{95}{\percent} as a function of the
self-coupling modifier $\klambda$, where
$\klambda = \lambdahhh / \lambdahhh^{\text{SM}}$. Based on these upper limits,
an allowed \klambda interval of $\num{-2.4} \leq \klambda \leq 9.2$ is obtained.



% Self coupling:

% The search is reinterpreted in the context of anomalous values of the Higgs
% boson self-coupling constant. The allowed \klambda interval yields
% $-2.4 \leq \klambda \leq 9.2$.


% Non-resonant and resonant
%
% Upper limits on the SM HH production cross section

% klambda

% Examples: Ray
% https://bonndoc.ulb.uni-bonn.de/xmlui/handle/20.500.11811/9447
%
% Tobias
% https://bonndoc.ulb.uni-bonn.de/xmlui/handle/20.500.11811/9266


%%% Local Variables:
%%% mode: latex
%%% TeX-master: "../phd_thesis"
%%% End:
