% ------------------------------------------------------------------------------
\chapter*{Abstract}
% ------------------------------------------------------------------------------

A search for Higgs boson pair (\HH) production in final states with two
$b$-quarks and two \tauleptons using \SI{139}{\per\femto\barn} of proton--proton
collision data at a centre-of-mass energy of \SI{13}{\TeV} recorded by the ATLAS
detector at the Large Hadron Collider is presented. Searches for Higgs boson
pair production are vital for solidifying our understanding of electroweak
symmetry breaking in the Standard Model (SM) since they serve as direct probes
of the Higgs boson self-coupling. In addition, they can provide sensitivity to
new physical phenomena introduced by theories beyond the SM (BSM).

The search presented in this thesis targets channels with one or more hadronic
\taulepton decays. Consequently, the search relies on effective \tauid
algorithms, which are algorithms that distinguish signatures of hadronic
\taulepton decays in the ATLAS detector from those of other sources, such as
quark- or gluon-initiated jets. A novel \tauid algorithm based on recurrent
neural networks is introduced. It exploits information from reconstructed
charged-particle tracks, topological clusters of calorimeter cell signals, and
purposefully constructed discriminating variables for \tauid. The new technique
significantly outperforms the method previously employed in the ATLAS experiment
and has since become the default algorithm for \tauid.

Three different types of Higgs boson pair production are considered in this
thesis: non-resonant \HH production predicted by the SM (SM~\HH production);
resonant \HH production via scalar, narrow-width resonances; and non-resonant
\HH production with anomalous values of the Higgs boson self-coupling
constant~\lambdahhh.
% In all cases, machine learning methods are used to separate signal from
% background events.

The search for SM~\HH production targets the gluon--gluon fusion (\ggF) and
vector boson fusion (VBF) production modes. No statistically significant
deviation from the background-only hypothesis is observed; therefore, upper
limits are set on the SM~\HH production cross section \xsecggfvbf and signal
strength~$\mu = \xsecggfvbf / \xsecggfvbf^{\text{SM}}$, where
$\xsecggfvbf^{\text{SM}}$ is the SM cross section prediction. The search yields
upper limits of $\mu < \num{4.7}$ and $\xsecggfvbf < \SI{140}{\femto\barn}$ at
\SI{95}{\percent} confidence level (CL).

Various BSM theories predict the existence of new scalar particles~$X$ that are
able to decay into pairs of Higgs bosons. Such scenarios are probed by
performing a search for \HH production via intermediate resonances with masses
ranging from \SIrange{251}{1600}{\GeV}. Over the considered mass range, the
largest deviation from the background-only hypothesis is observed for a mass of
\SI{1000}{\GeV} with a local (global) significance of $3.1\sigma$
($2.0\sigma$). Upper limits are set on the $\pp \to X \to \HH$ production cross
section, which range from \SIrange{20}{900}{\femto\barn} depending on the mass
of the resonance.

Finally, the search for SM~\HH production is reinterpreted in the context of
anomalous values of the Higgs boson self-coupling constant. Upper limits at
\SI{95}{\percent}~CL are set on the non-resonant \HH production cross section as
a function of the self-coupling modifier
$\klambda = \lambdahhh / \lambdahhh^{\text{SM}}$, where $\lambdahhh^{\text{SM}}$
is the Higgs boson self-coupling predicted by the SM. The upper limits on the
non-resonant \HH production cross section are compared to theory predictions,
providing constraints on the self-coupling modifier of
$\num{-2.4} \leq \klambda \leq 9.2$.


%%% Local Variables:
%%% mode: latex
%%% TeX-master: "../phd_thesis"
%%% End:
