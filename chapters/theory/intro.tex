The Standard Model (SM) of particle physics is among the most precisely tested
theories of physics, explaining numerous observed phenomena with high
precision. It provides a description of the known fundamental particles and
their interactions (excluding gravitation) in the framework of a relativistic
quantum field theory. The current formulation of the SM stems from the early
1970s after the development of the theories of the electroweak and strong
interaction~\cite{Glashow:1961tr,Salam:1964ry,Weinberg:1967tq,Englert:1964et,Higgs:1964pj,tHooft:1971qjg,Fritzsch:1973pi,Gross:1973id,Politzer:1973fx}.
This chapter lays the theoretical foundation for the searches for Higgs boson
pair production presented in this thesis.

Both SI and natural units ($\hbar = c = \varepsilon_0 = 1$) are used in the
following, whichever is most suitable in a given context. In addition, this
chapter adopts the Einstein summation convention implying summation over
repeated indices in a mathematical term. Greek indices represent the four
dimensions of space-time, while the meaning of Latin indices is
context-dependent. Finally, the metric tensor of special relativity is assumed
to be $\eta = \diag(+1, -1, -1, -1)$.

This chapter is structured as follows: First, an overview of the SM and its
particle content is given in \Cref{sec:sm_overview}. The fundamental
interactions are described in \Cref{sec:theo_symmetries_interactions} on the
basis of symmetries of the theory. Subsequently, the SM phenomenology of the
Higgs boson and Higgs boson pair production is presented in
\Cref{sec:higgs_boson}. The SM has several limitations in explaining certain
experimental or theoretical phenomena, which suggests the existence of physics
beyond the SM (BSM). \Cref{sec:bsm} lists some of these limitations as well as
examples of BSM theories that can lead to an enhanced production of Higgs boson
pairs at the LHC. This chapter concludes in \Cref{seq:experimental_status} with
the experimental status of searches for Higgs boson pair production prior to the
work presented in this thesis. The initial parts of this chapter
(\Cref{sec:sm_overview,sec:theo_symmetries_interactions}) are based on
Refs.~\cite{Halzen:1984mc,Thomson:2013zua,Djouadi:2005gi}.

%%% Local Variables:
%%% mode: latex
%%% TeX-master: "../../phd_thesis"
%%% End:
