\section{Probing the Standard Model at Hadron Colliders}

Description of the hard scatter event in pp collisions:
\begin{itemize}
\item Collisions of constitutents (partons) of the protons with
  various fractions of momentum of the parent proton.

\item Factorisation approach? Factorise long-distance (soft and
  non-perturbative) and short-distance (hard and perturbative) physics
  at the factorisation scale $\mu_\text{F}$. Long-distance effects are
  factored out into the PDFs while the partonic cross section only
  contains short-distance interactions allowing for perturbative
  expansion in \alphas. The factorisation scale separates partons
  described by the PDF (soft) and the hard scatter process. The
  calculation becomes less dependent on the scale the higher the order
  of the calculation (thus frequently used to estimate uncertainties
  on missing higher orders).

\item The partonic cross section is calculated in a perturbative
  expansion in \alphas up to some order.
\end{itemize}

\begin{align*}
  \sigma_{\Pproton\Pproton \ra \PX}() = \sum_{i,j} \int \mathrm{d}x_1 \mathrm{d}x_2 \, f_i(x_1, \muF^2) f_j(x_1, \muF^2) \, \hat{\sigma}_{i j \ra \PX}(x_1, x_2, \alphas, Q^2 / \muF^2)
\end{align*}
(Ellis, Stirling, Webber)

The inclusive cross section of proton-proton going to \PX is (factorisation approach)
\begin{align*}
  \frac{\mathrm{d}\sigma_{\Pproton\Pproton \ra \PX}}{\mathrm{d}\Omega} = \sum_{i,j} \int \mathrm{d}x_1 \mathrm{d}x_2 \int f_i(x_1) f_j(x_2) \, \mathrm{d}\hat{\sigma}_{i j \ra \PX}(x_1, x_2, \hat{s})
\end{align*}
where $i, j$ indicates the species of parton (i.e.\ gluons, quarks and
anti-quarks), $f_i(x)$ are the corresponding parton density functions,
and $\mathrm{d}\hat{\sigma}(i\,j \ra \PX)$ is the partonic cross
section at a c.m. of $\hat{s} = x_1 x_2 s$.

%%% Local Variables:
%%% mode: latex
%%% TeX-master: "../../phd_thesis"
%%% End:
