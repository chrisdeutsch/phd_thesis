\section{Physics Beyond the Standard Model}

The SM is among the most precisely tested theories of physics. It had numerous
successes in predicting phenomena before they were experimentally observed. Most
recently, the Higgs boson was discovered in 2012 about five decades after the
inclusion of the BEH mechanism into the SM by Weinberg in
1967~\cite{Weinberg:1967tq}. Similarly, the gluons, $W^\pm$ and $Z$ bosons were
all predicted by the SM before direct experimental evidence for their existence
could be obtained. Despite its many successes, the SM is known to be an
incomplete theory leaving a number of phenomena unexplained:
\begin{description}

\item[Matter-antimatter asymmetry] In the Big Bang cosmological model, equal
  amounts of matter and antimatter are produced in the initial phase of the
  evolution of the universe. However, the universe observed today mostly
  consists of matter particles. This fact is referred to as the
  matter-antimatter asymmetry of the universe. According to the conditions
  proposed by Sakharov~\cite{Sakharov:1967dj}, violation of CP symmetry is
  required for the generation of such an asymmetry in the early universe. While
  CP violation has been observed in the quark sector~\cite{Christenson:1964fg}
  and first indications of CP violation in the lepton sector
  exist~\cite{T2K:2019bcf}, it is unclear if its size is sufficient to yield an
  explanation of the observed asymmetry.

\item[Gravitation] The theory of general relativity provides an accurate
  description of gravitation in the context of a classical field
  theory. However, no generally accepted approach exists that reconciles general
  relativity with the quantum theory underlying our current formulation of the
  SM. Moreover, understanding the weakness of gravitational interactions at the
  scale of elementary particles remains one of the open questions in particle
  physics.

\item[Dark matter] Astrophysical
  observations~\cite{Zwicky:1933gu,Zwicky:1937zza,Rubin:1970zza,Rubin:1980zd,Clowe:2006eq}
  indicate that the vast majority of the universe consists of a form of matter,
  referred to as \emph{dark matter}, that does not interact via the
  electromagnetic interaction but can be inferred from its gravitational
  interaction with ordinary matter. Provided dark matter is microscopic in
  nature, the SM does not provide a suitable dark matter candidate that is
  consistent with current cosmological models.

\item[Neutrino masses] The observation of neutrino
  oscillations~\cite{Super-Kamiokande:1998kpq,SNO:2002tuh} constitutes
  experimental evidence of neutrinos being massive particles. In the SM, it is
  assumed that neutrinos are massless particles, however. Extending the SM to
  incorporate non-vanishing neutrino masses poses the question whether neutrinos
  are Dirac or Majorana~\cite{Majorana:1937vz} particles. In addition, upper
  limits on the neutrino masses are
  $\mathcal{O}(\SI{1}{\electronvolt})$~\cite{pdg2020} which, adopting arguments
  of \emph{naturalness}, appear to be unnaturally small compared to the mass
  scales of other fermions (\SI{1}{\MeV} to \SI{100}{\GeV}).

  % If neutrinos are Dirac particles, then a
  % Dirac mass term can be introduced into the SM Lagrangian via Yukawa
  % interactions between neutrinos and the Higgs boson. This mass term couples
  % left- and right-handed neutrino chiral states, implying the existence of a
  % non-interacting (\emph{sterile}) degree of freedom.

  % If neutrinos are Dirac particles, then the
  % addition of a Dirac mass term into the SM Lagrangian introduces a coupling
  % between left- and right-handed neutrino chiral states. This implies the
  % existence of a non-interacting (\emph{sterile}) degree of freedom for
  % neutrinos, since right-handed neutrinos (left-handed anti-neutrinos) do not
  % participate in the weak interaction. While the neutrino masses could be
  % generated via Yukawa couplings to the Higgs field, the coupling strengths
  % differ by many orders of magnitude compared to other fermions. The alternative
  % hypothesis of neutrinos being Majorana particles~\cite{Majorana:1937vz}, i.e.\
  % neutrinos are their own anti-particles,

  % - lepton number violation (neutrinoless double-$\beta$ decay)

  % \emph{see-saw mechanism}

  % Majorana:
  % - Neutrinos and antineutrinos are the same particle
  % - No sterile neutrinos
  % - Seesaw mechanism could explain the small mass of neutrinos

% \item[Higgs mass / hierarchy problem] % Naturalness?
\item[Hierarchy problem]



% Is this really a 'problem'?
% \item[Vacuum Stability] The present minimum with a vacuum expectation value of
%   $v \approx \si{246}{\GeV}$ might be either a global minimum in which case the
%   universe is stable or only a local minimum which leads to a metastable
%   universe. In the metastable case, the state of the Higgs field could tunnel to
%   a new local or global minimum with a smaller potential. Current experimental
%   data cannot distinguish whether the universe is stable or
%   meta-stable\todo{citation}.

% \item[Violation of lepton flavour universality] Not yet statistically
%   significant

\end{description}
These shortcomings do not disprove the SM, however, which still predicts and
describes many natural phenomena with excellent precision. Therefore, it is
often believed that the SM represents the low-energy manifestation of an
extended theory occurring at high energy scales, for example a \emph{Grand
  Unified Theory} that unifies the electroweak and strong interaction.


\subsection{Non-Resonant Higgs Boson Pair Production}%
\label{sec:bsm_nonresonant_hh}

Contributions of BSM physics might appear at energy scales beyond what can be
experimentally probed using direct searches at the LHC. Nevertheless, it is
possible to test such models indirectly through their effect on SM processes via
virtual corrections. These corrections can, for example, alter the total or
differential cross-section of a given process. Therefore, searches for
non-resonant \HH production are already of interest, since an enhancement in its
cross-section would be indicative of BSM physics.

Another way of exploring BSM contributions to non-resonant \HH production is to
investigate the Higgs boson self-coupling constant for possible deviations from
the SM value of $\lambda_{HHH}^\text{SM} = 3 m_{H}^2 / v$. Such deviations can
similarly arise due to virtual corrections involving massive BSM particles as
indicated in \Cref{fig:bsm_hh_prod_feyn}. If the mass scale of the particles
participating in the corrections is sufficiently large, then the dynamics of the
BSM theory can be reduced to an effective interaction vertex between three Higgs
bosons with a coupling constant
\begin{align*}
  \lambda_{HHH} = \klambda \times \lambda_{HHH}^{\text{SM}} \,\text{,}
\end{align*}
where \klambda is an arbitrary coupling modifier. Due to the interference with
the $\pp \to HH$ box diagram previously depicted in
\Cref{fig:dihiggs_ggf_feyn_box}, a change in \klambda will alter both the total
cross-section of non-resonant \HH production and the kinematics of the final
state particles. A detailed discussion of these effects will follow in
\Cref{sec:higgs_self_coupling}.

\begin{figure}[htbp]
  \centering

  \includegraphics[width=0.35\textwidth]{feynman_graphs/di_higgs_effective}

  \caption{Non-resonant production of Higgs boson pairs for anomalous values of
    the trilinear Higgs boson self-coupling constant.  Contributions of new
    physics, for example through loops of heavy BSM particles, are indicated as
    a hatched circle. The effective coupling constant in units of the
    self-coupling constant predicted by the SM is given by \klambda.}%
  \label{fig:bsm_hh_prod_feyn}
\end{figure}

% Unitarity: Scattering probability < 1
%
% Perturbativity: Higher-order corrections become smaller as opposed to larger
Current bounds on possible values of \klambda from requirements of perturbative
unitarity in $HH \to HH$ scattering allow for variations within
$|\klambda| \lessapprox 6$~\cite{DiLuzio:2017tfn}.\footnote{Similar arguments
  were made in the past to obtain upper limits on the Higgs boson mass from
  unitarity bounds in the scattering of longitudinally polarised vector
  bosons~\cite{Lee:1977eg}.} These bounds still allow for ample variation of the
Higgs boson self-coupling strength, further motivating searches for non-resonant
\HH production. These searches constitute a major part of
\Cref{sec:dihiggs,sec:higgs_self_coupling}, in which upper limits are set on the
non-resonant \HH production cross-section of a signal with SM-like
($\klambda = 1$) kinematics, and signals with anomalous \klambda
($\klambda \neq 1$), respectively.

% Anomalous values of the Higgs boson self-coupling within
% $|\klambda| \lessapprox 6$ are allowed given arguments based on the unitarity
% and perturbativity of $HH \to HH$
% scattering~\cite{DiLuzio:2017tfn}.\footnote{Similar arguments were made in the
%   past to obtain upper limits on the Higgs boson mass from unitarity bounds in
%   the scattering of longitudinally polarised vector bosons~\cite{Lee:1977eg}.}


\subsection{Resonant Higgs Boson Pair Production}%
\label{sec:bsm_resonant_hh}

If BSM physics appears at experimentally accessible energy scales, new particles
could be produced directly (on-shell) in collider experiments. Further assuming
that these particles are short-lived and decay into detectible SM particles, one
can reconstruct the mass of such particles using the four-momenta of their decay
products. The presence of BSM physics then appears as an enhancement of the
differential cross-section $\mathrm{d}\sigma / \mathrm{d}m$, $m$ refering to the
invariant mass of the final state particles, in a region close to the mass of
the new particle. This phenomenon is referred to as a \emph{resonance} and the
production of particles via an intermediate resonance as \emph{resonant
  production}.

The Higgs sector is often used as an entrypoint for physics beyond the SM. Aside
from aesthetic reasons, there are currently no arguments that require nature to
realise a \emph{minimal Higgs model} with a single
Higgs-doublet~\cite{Gunion:1989we}. In fact, the Higgs sector can be readily
extended with additional scalar fields with singlet and doublet representations
under the SM gauge group~\cite{Gunion:1989we}. Such extended Higgs sectors are
part of many BSM theories, resulting in a phenomenology with new scalar
particles. Under certain circumstances, these models allow for a sizable
production of SM-like Higgs boson pairs via intermediate scalar resonances. A
possible Feynman diagram of resonant Higgs boson pair production is depicted in
\Cref{fig:resonant_production_feyn}.

\begin{figure}[htbp]
  \centering

  \includegraphics[width=0.35\textwidth]{feynman_graphs/di_higgs_resonant}

  \caption{Resonant production of SM Higgs boson pairs via an intermediate
    scalar resonance $X$ produced in \ggF.}%
  \label{fig:resonant_production_feyn}
\end{figure}

In the following, two examples of extended Higgs sectors are given:
\begin{description}

\item[Additional Higgs-singlet models] The simplest extension of the SM Higgs
  sector is the addition of a real scalar field $\phi_{S}$ that transforms as a
  singlet under the SM gauge group. This scalar field, being a gauge singlet,
  does not interact with any of the SM fermions or vector bosons. It could
  therefore be part of a ``hidden sector'' which might provide a suitable
  candidate for dark matter. In so-called \emph{Higgs portal
    models}~\cite{Patt:2006fw}, the hidden sector can only be accessed through
  coupling/mixing of $\phi_{S}$ with the CP-even component of the SM Higgs
  field. Such models can allow resonant production of Higgs boson pairs through
  new scalar
  resonances~\cite{Schabinger:2005ei,Bowen:2007ia,Barger:2007im,Dolan:2012ac,No:2013wsa,Chen:2014ask,Robens:2016xkb,DiMicco:2019ngk}.

  A general choice for the potential of a Higgs sector extended by an additional
  scalar field
  reads~\cite{OConnell:2006rsp,No:2013wsa,Chen:2014ask,DiMicco:2019ngk}
  \begin{align*}
    V(\phi, \phi_{S}) = V(\phi)
    + \frac{a_1}{2} (\phi^\dagger \phi) \phi_{S}
    + \frac{a_2}{2} (\phi^\dagger \phi) \phi_{S}^2
    + b_1 \phi_{S} + \frac{b_2}{2} \phi_{S}^2 + \frac{b_3}{3} \phi_{S}^3 + \frac{b_4}{4} \phi_{S}^4 \,\text{,}
  \end{align*}
  where $\phi$ refers to the complex Higgs-doublet and $V(\phi)$ is the BEH
  potential. In unitary gauge, the fields can be expanded about the vacuum state
  as $\phi = (0 \quad v + H)^{\text{T}} / \sqrt{2}$ and $\phi_{S} = v_{S} + S$,
  where $v_{S}$ is the VEV of $\phi_{S}$. After the expansion, terms bilinear in
  $H$ and $S$ appear in the potential,\footnote{The bilinear terms only appear
    if either $a_1 \neq 0$, or $a_2 \neq 0$ and $\phi_{S}$ has non-vanishing
    VEV. See for example Ref.~\cite{Chen:2014ask}.} which indicate that the
  physical fields are mixtures of $H$ and $S$. The physical fields $H_1$ and
  $H_2$ with masses $m_1$ and $m_2$, respectively, can be expressed as
  \begin{align*}
    \begin{pmatrix}
      H_1 \\
      H_2
    \end{pmatrix}
    =
    \begin{pmatrix}
      \cos\theta & \sin\theta \\
      -\sin\theta & \cos\theta
    \end{pmatrix}
    \begin{pmatrix}
      H \\
      S
    \end{pmatrix} \,\text{,}
  \end{align*}
  with a mixing angle $\theta$. In the following, it is assumed that $H_1$ can
  be identified with the observed Higgs boson and $H_2$ is a new scalar with
  $m_2 > 2 m_1$.  The scalar $H_2$ can be produced via \ggF through the
  admixture of $H$ in $H_2$, however, suppressed by a factor of
  $\sin^2\theta$. Further, the interaction terms of the scalar potential allow
  for decays of $H_2$ into a pair of $H_1$. As a result, resonant production
  processes according to $\pp \to H_2 \to H_1 H_1$ are allowed given the prior
  assumptions.

\item[Two-Higgs-doublet models (2HDM)] Generic 2HDM extend the Higgs sector of
  the SM by introducing an additional $SU(2)_{\text{L}}$ doublet of complex
  scalar fields~\cite{Gunion:1989we,Branco:2011iw}. Such extensions are
  motivated by theories such as supersymmetry~\cite{Haber:1984rc}, which require
  at least two Higgs-doublets to generate masses of up- and down-type fermions,
  or models of electroweak baryogenesis~\cite{Trodden:1998ym}, in which 2HDM can
  provide CP-violating processes necessary to generate a matter-antimatter
  asymmetry in the universe~\cite{Sakharov:1967dj}.
  % The class of 2HDM encompass theories with vastly different phenomenology.
  % For example, 2HDM with flavour-changing neutral currents (FCNC) at
  % tree-level exist, however, such models are in tension with the experimental
  % non-observation of tree-level FCNC~\cite{Gunion:1989we,Branco:2011iw}.

  Further discussion is restricted to flavour- and CP-conserving 2HDM, which,
  for example, include models describing the Higgs sector of minimal
  supersymmetric extensions of the SM (MSSM).\footnote{For CP-violating 2HDM as
    possible explanations of electroweak baryogenesis, see for example the
    \emph{complex 2HDM} (C2HDM) in Ref.~\cite{Fontes:2017zfn}.}
  % The Higgs doublets in 2HDM have non-vanishing VEV denoted $v_1$ and $v_2$,
  % respectively.  , often expressed as the ratio
  % $\tan\beta \coloneqq v_2 / v_1$.
  The particle spectrum of these models consist of five scalar particles after
  EWSB: two CP-even Higgs bosons $H_1$ and $H_2$, a CP-odd Higgs boson $A$, and
  two charged Higgs bosons $H^\pm$. Similar to the model with an additional
  Higgs-singlet, the physical fields $H_1$ and $H_2$ are mixed states of the
  CP-even components of both Higgs-doublets, and interaction vertices of the
  form $H_1 H_1 H_2$ exist~\cite{Gunion:1989we,Branco:2011iw}. This can allow
  for resonant production processes according to $\pp \to H_2 \to H_1 H_1$,
  which are promising search channel for heavy, CP-even Higgs boson in certain
  BSM scenarios~\cite{Dolan:2012ac,Djouadi:2013vqa,Djouadi:2013uqa}.

\end{description}
The selected examples are not intended to be comprehensive but rather serve to
illustrate how resonant \HH production can arise in models with extended Higgs
sectors. In this thesis, the benchmark signal process is the decay of a CP-even
scalar resonance, $X$, produced via \ggF into a pair of SM Higgs bosons as
depicted in \Cref{fig:resonant_production_feyn}. The width of the resonance is
assumed to be narrow such that interference with SM \HH production can be
neglected.

%%% Local Variables:
%%% mode: latex
%%% TeX-master: "../../phd_thesis"
%%% End:
