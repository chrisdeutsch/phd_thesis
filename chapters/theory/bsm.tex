\section{Physics Beyond the Standard Model}

The SM is one of the most well-tested theories in physics. It had numerous
successes in predicting phenomena before they were experimentally observed. Most
recently, the Higgs boson was discovered in 2012 about five decades after the
inclusion of the BEH mechanism into the SM by Weinberg in
1967~\cite{Weinberg:1967tq}. Similarly, the gluons, $W^\pm$ and $Z$ bosons were
all predicted by the SM before direct experimental evidence for their existence
could be obtained. Despite its many successes, the SM is known to be an
incomplete theory leaving some phenomena unexplained:
\begin{description}

\item[Matter-antimatter asymmetry] In the Big Bang cosmological model, equal
  amounts of matter and antimatter are produced in the initial phase of the
  evolution of the universe. However, the universe observed today mostly
  consists of matter particles, thus resulting in a large asymmetry between
  matter and antimatter. The SM does not contain a mechanism that can explain
  the size of the matter-antimatter asymmetry observed in the universe.

\item[Gravitation] At the moment, no quantum field theory of gravitation exists
  that could be incorporated into the SM. Therefore, the SM does not attempt to
  describe gravitational interactions between massive particles. In addition, it
  is currently not understood why the gravitational interaction between
  elementary particles is much weaker than the strong or electroweak
  interaction.

\item[Dark matter] Based on astrophysical observations it is postulated that the
  vast majority of the universe consists of a form of matter, referred to as
  dark matter, that interacts gravitationally but not electromagnetically.

  The SM does not provide a dark matter candidate particle.

\item[]

\item[Neutrinos] Mass, Majorana, Dirac, etc.

\item[Vacuum Stability] The present minimum with a vacuum expectation
  value of $v \approx \si{246}{\GeV}$ might be either a global minimum
  in which case the universe is stable or only a local minimum which
  leads to a metastable universe. In the metastable case, the state of
  the Higgs field could tunnel to a new local or global minimum with a
  smaller vacuum expectation value. Current experimental data cannot
  distinguish whether the universe is stable or
  meta-stable\todo{citation}.

\item[Elektroweak phase transition] In baryogenesis a first order
  electroweak phase transition is needed.

\end{description}
These shortcomings do not disprove the SM, however, which still predicts and
describes many natural phenomena with excellent precision. Therefore, it is
often believed that the SM represents the low-energy manifestation of an
extended theory occurring at high energy scales, for example a \emph{Grand
  Unified Theory} that unifies the electroweak and strong interaction.


\subsection{Non-Resonant Higgs Boson Pair Production}%
\label{sec:bsm_nonresonant_hh}

Contributions of new physics might only appear at energy scales beyond what can
be experimentally probed using direct searches at the LHC. However, some
sensitivity to such models is provided by indirect searches using SM processes,
which can be sensitive to heavy BSM particles via virtual corrections. These
corrections can, for example, alter the total/differential cross-section of
non-resonant \HH production. Consequently, searches for non-resonant \HH
production are already of interest, since an enhancement in its cross-section
compared to the SM expectation would be a indicator of BSM contributions.

An avenue to explore BSM contributions to non-resonant \HH production is to
probe the Higgs boson self-coupling constant $\lambda_{HHH}$ for possible
deviations from the SM. Such deviations can arise through virtual corrections to
the Higgs boson self-interaction involving BSM particles as illustrated in
\Cref{fig:bsm_hh_prod_feyn}. If the mass scale of the BSM particles
participating in the virtual corrections is sufficiently large, then the
dynamics of the BSM theory can be reduced to an effective interaction vertex
between three Higgs bosons with a coupling constant
\begin{align*}
  \lambda_{HHH} = \klambda \times \lambda_{HHH}^{\text{SM}} \,\text{,}
\end{align*}
where $\lambda_{HHH}^{\text{SM}}$ is the self-coupling constant predicted by the
SM and \klambda is a self-coupling modifier constant. Due to the interference
with the $pp \to HH$ box diagram depicted in \Cref{fig:dihiggs_ggf_feyn_box}, a
change in \klambda will alter both the total cross-section of non-resonant \HH
production and the kinematic distributions of the final state particles.

\begin{figure}[htbp]
  \centering

  \includegraphics[width=0.35\textwidth]{feynman_graphs/di_higgs_effective}

  \caption{Non-resonant production of Higgs boson pairs for anomalous values of
    the trilinear self-coupling constant.  Contributions of new physics, for
    example through loops of heavy BSM particles, are indicated as a hatched
    circle. The effective coupling constant in units of the Higgs boson
    self-coupling constant predicted by the SM is given by \klambda.}%
  \label{fig:bsm_hh_prod_feyn}
\end{figure}

% Unitarity: Scattering probability < 1
%
% Perturbativity: Higher-order corrections become smaller as opposed to larger
On theoretical grounds, anomalous values of the Higgs boson self-coupling are
constrained to be within $|\klambda| \lessapprox 6$ based on arguments of
unitarity and perturbativity in $HH \to HH$
scattering~\cite{DiLuzio:2017tfn}.\footnote{Similar arguments were made in the
  past to obtain upper limits on the Higgs boson mass from unitarity bounds in
  the scattering of longitudinally polarised vector bosons~\cite{Lee:1977eg}.}
Therefore, relatively large variations of the Higgs boson self-coupling would
still be consistent given



A search for non-resonant \HH production with SM-like kinematics is presented in
. Constraints are set on anomalous values of \klambda in
\Cref{sec:higgs_self_coupling}.





\subsection{Resonant Higgs Boson Pair Production}%
\label{sec:bsm_resonant_hh}


\begin{figure}[htbp]
  \centering

  \includegraphics[width=0.35\textwidth]{feynman_graphs/di_higgs_resonant}

  \caption{Resonant production of Higgs boson pairs}%
  \label{fig:resonant_production_feyn}
\end{figure}



\todo[inline]{What models predict Spin-0 resonances decaying into pair of SM
  Higgs? 2HDM}

\todo[inline]{Mention Spin-2 resonances? KK-graviton -- theoretically not favoured}

% \item[BSM] Radions, 2HDM, Warped extra dimensions, composite Higgs, hMSSM, KK
%   Gravitons: Most could decay to pairs of SM Higgs bosons.

\todo[inline]{1 page}


%%% Local Variables:
%%% mode: latex
%%% TeX-master: "../../phd_thesis"
%%% End:
