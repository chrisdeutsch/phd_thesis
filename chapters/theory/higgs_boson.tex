\section{The Higgs Boson}

The Higgs boson is the only elementary scalar particle predicted by the SM. It
couples to all massive particles of the SM with coupling strengths related to
the particle's mass. In particular, the strength of the Higgs boson coupling to
gauge bosons is proportional to $m_V^2$, where $V$ refers to either the $W^\pm$
or the $Z$ boson, and the Yukawa coupling to fermions proportional to $m_{f}$.

About half a century after the proposal of the Glashow--Salam--Weinberg model of
the electroweak interaction, the Higgs boson was discovered in 2012 by the
ATLAS~\cite{HIGG-2012-27} and CMS~\cite{CMS-HIG-12-028} collaborations at the
Large Hadron Collider. Since its discovery, extensive measurements of its
properties have been performed showing agreement with the SM predictions. The
Higgs boson has a mass of about $\SI{125}{\GeV}$ with the most precise
measurement by the ATLAS collaboration being
\begin{align*}
  m_{H} = \num{124.94}%
  \valuesep\numerrt{0.17}{stat.}%
  \valuesep\numerrt{0.03}{syst.}%
  \valuesep\si{\GeV}
\end{align*}
in the $H \to Z Z^{*} \to 4\ell$ channel~\cite{HIGG-2020-07} using \pp-collision
events collected in the period from 2015 to 2018 at the Large Hadron Collider.
The mass-dependent couplings of the Higgs boson to massive gauge bosons and
fermions ($t$, $b$, $\tau$, $\mu$) are also experimentally established~CITE.





Production modes:

ggF, VBF, WH, ZH, ttH, (tH?)

HH

Decay modes:

Yukawa couplings:

To ``t''

To ``b''

To $\tau$

To $\mu$



Coupling strengths established:





Most precise measurement by the ATLAS collaboration


Discovery of the Higgs~CITE last remaining parameter of the electroweak theory
determined.


Higgs mass

Higgs boson production

Higgs couplings


Couplings of gauge bosons in excellent agreement with the SM prediction

Yukawa: top, bottom, tau, muon (?) also in agreement


\subsection{Production and Decay Modes}


\begin{figure}[htbp]
  \centering

  \begin{subfigure}{0.45\textwidth}
    \centering
    \includegraphics[scale=1]{feynman_graphs/higgs_prod_ggf}
    \subcaption{Gluon-gluon fusion (\ggF)}
  \end{subfigure}%
  \begin{subfigure}{0.45\textwidth}
    \centering
    \includegraphics[scale=1]{feynman_graphs/higgs_prod_vbf}
    \subcaption{Vector boson fusion (VBF)}
  \end{subfigure}

  \vspace*{0.5em}

  \begin{subfigure}{0.45\textwidth}
    \centering
    \includegraphics[scale=1]{feynman_graphs/higgs_prod_vh}
    \subcaption{Associated production with a vector boson ($VH$)}
  \end{subfigure}%
  \begin{subfigure}{0.45\textwidth}
    \centering
    \includegraphics[scale=1]{feynman_graphs/higgs_prod_tth}
    \subcaption{Associated production with \ttbar ($\ttbar H$)}
  \end{subfigure}%

  \caption{The dominant Higgs boson production modes in \pp-collisions at
    centre-of-mass energies of \SI{13}{\TeV}.}
\end{figure}


Pair production of Higgs bosons is covered in detail in
\Cref{fig:theory_higgs_pair_prod}.

\begin{figure}[htbp]
  \centering

  \begin{subfigure}[b]{0.47\textwidth}
    \centering

    \includegraphics[width=0.95\textwidth]{theory/h_prod_crossec_13tev}

    \subcaption{Cross sections of Higgs boson production modes as a function of
      $m_{H}$.}
  \end{subfigure}\hfill%
  \begin{subfigure}[b]{0.47\textwidth}
    \centering

    \renewcommand{\arraystretch}{1.1}%
    \begin{tabular}{lS[table-format=2.3]c}
  \toprule
  Decay mode  & {BR / \%} & Observed \\
  \midrule
  $\bbbar$        & 58    & \checkmark \\
  $W^{+} W^{-}$    & 21    & \checkmark \\
  $gg$            & 8.2   & \\
  $\tau^+ \tau^-$ & 6.3   & \checkmark \\
  $c\bar{c}$      & 2.9   & \\
  $ZZ^{*}$            & 2.6   & \checkmark \\
  $\gamma\gamma$  & 0.23  & \checkmark \\
  $Z\gamma$       & 0.15  & \\
  $\mu^{+}\mu^{-}$ & 0.022 & $\sim$~\cite{CMS-HIG-19-006} \\
  \bottomrule
\end{tabular}

%%% Local Variables:
%%% mode: latex
%%% TeX-master: "../phd_thesis"
%%% End:


    \subcaption{Branching ratios of Higgs boson decay modes. The $gg$,
      $\gamma\gamma$, and $Z\gamma$ decay modes occur via higher-order
      processes.}
  \end{subfigure}

  \caption{Higgs boson production cross section in \pp-collisions at a
    centre-of-mass energy of \SI{13}{\TeV} (a) and Higgs boson decay branching
    ratios for $m_{H} = \SI{125.0}{\GeV}$ (b). The figure and branching ratios
    are taken from Ref.~\cite{deFlorian:2016spz}.}
\end{figure}


\subsection{Higgs Boson Pair Production}
\label{fig:theory_higgs_pair_prod}

Production modes

\begin{figure}[htbp]
  \centering

  \begin{subfigure}{0.49\textwidth}
    \centering
    \includegraphics[width=0.7\textwidth]{feynman_graphs/di_higgs_box}
    \subcaption{}
  \end{subfigure}\hfill%
  \begin{subfigure}{0.49\textwidth}
    \centering
    \includegraphics[width=0.7\textwidth]{feynman_graphs/di_higgs_triangle}
    \subcaption{}
  \end{subfigure}

  \vspace*{1em}

  \begin{subfigure}{0.33\textwidth}
    \centering
    \includegraphics[width=0.95\textwidth]{feynman_graphs/di_higgs_vbf_kvklam}
    \subcaption{}
  \end{subfigure}\hfill%
  \begin{subfigure}{0.33\textwidth}
    \centering
    \includegraphics[width=0.95\textwidth]{feynman_graphs/di_higgs_vbf_kvkv}
    \subcaption{}
  \end{subfigure}\hfill%
  \begin{subfigure}{0.33\textwidth}
    \centering
    \includegraphics[width=0.95\textwidth]{feynman_graphs/di_higgs_vbf_ktwov}
    \subcaption{}
  \end{subfigure}

  \caption{Feynman graphs}
  \label{fig:hh_feynmans}
\end{figure}



\begin{figure}[htbp]
  \centering
  \includegraphics[width=0.45\textwidth]{theory/di_higgs_branching_ratio}
  \caption{Possible final states of decays of pairs of Standard Model Higgs
    bosons. Higgs boson branching ratios are taken from~\cite{deFlorian:2016spz}
    and assume~$m_{H} = \SI{125.0}{\GeV}$. The decay mode~$H \ra g g$ is
    excluded due to limited experimental feasibility.}
  \label{fig:hh_branching_ratios}
\end{figure}



\clearpage
\todo[inline]{Here be dragons}


\subsection{Higgs Boson Self-Interaction and Higgs Boson Pair Production}

Couplings of the Higgs boson to heavy vector bosons and heavy fermions
are quickly established after the Higgs boson discovery. However, the
Higgs boson self-coupling is still an open question and an important
probe to check EWSB.

% Super excellent talk by Katharine:
% https://indico.cern.ch/event/1065153/attachments/2351166/4011032/Seminar.pdf

All wrong...

Taylor expansion of the Higgs potential about the minimum after EWSB:
\begin{align*}
  V(h) &= \frac{1}{2} m_{\PH}^2 h^2 + \lambda v h^3 + \frac{1}{4} \lambda h^4 + \dots \\
  m_{\PH} &= \sqrt{2 \lambda v} \approx \SI{125}{\GeV} \\
  \lambda &\approx 0.13 \quad \text{(SM)}
\end{align*}
$h$ is a complex field with the $h^2$-term being the mass term,
$h^3$-term yielding the trilinear coupling, and the $h^4$-term
yielding the quartic coupling. All these terms are directly predicted
by the SM but an experimental probe is still justified to detect
possible deviations from the SM. The self-coupling strength~$\lambda$
can be directly probed in either double Higgs boson production or
triple Higgs boson production (although inaccessible
currently). Indirect probes also exist for example in single Higgs
boson production where the self-coupling contributes at loop-level.

Cross section of HH production is 1000 times smaller than single
Higgs. \todo{Cross section diagram would be nice?}

The Higgs potential in the SM is determined by two parameters:
$v = 1 / \sqrt{\sqrt{2} G_{\text{F}}^0}$ and $\lambda$



4b: Very large branching ratio but huge multi-jet background.

bbtautau: Can suppress multi-jet background reasonably but large
backgrounds from EW and top.

bbyy: Very clean, but low BR.

Sensitivity vs mass:

bbyy: Low mass sensitivity (due to low threshold photon triggers)

bbtautau: Medium mass sensitivity

bbbb: High mass sensitivity (BR and reduced problems in finding Higgs
candidates due to boost)

\todo[inline]{Experimental evidence for the top Yukawa coupling exists
  therefore non-resonant \HH production has to exist too (at least via
  the box diagram)}

\begin{description}


\item[Vacuum Stability] The present minimum with a vacuum expectation
  value of $v \approx \si{246}{\GeV}$ might be either a global minimum
  in which case the universe is stable or only a local minimum which
  leads to a metastable universe. In the metastable case, the state of
  the Higgs field could tunnel to a new local or global minimum with a
  smaller vacuum expectation value. Current experimental data cannot
  distinguish whether the universe is stable or
  meta-stable\todo{citation}.

\item[Elektroweak phase transition] In baryogenesis a first order
  electroweak phase transition is needed.

\item[BSM] Radions, 2HDM, Warped extra dimensions, composite Higgs,
  hMSSM, KK Gravitons: Most could decay to pairs of SM Higgs bosons.

\end{description}

Single Higgs boson production cross section
$\approx \SI{50}{\pico\barn}$. Compared to Higgs pair production cross
section $\approx \SI{30}{\femto\barn}$.

\todo[inline]{Shortcomings of the SM: Hierarchy problem, dark matter, quantum
  description of gravity}

\todo[inline]{Maybe: SUSY - superpartners - cancellation of loop corrections to $h$
  mass; In R-parity conserving SUSY - LSP - DM candidate}

\todo[inline]{What models predict Spin-0 resonances decaying into pair of SM
  Higgs? 2HDM}

\todo[inline]{Mention Spin-2 resonances? KK-graviton -- theoretically not favoured}

\todo[inline]{Scalar sector relatively unexplored?}

%%% Local Variables:
%%% mode: latex
%%% TeX-master: "../../phd_thesis"
%%% End:
