\section{Previous Searches for Higgs Boson Pair Production}%
\label{seq:experimental_status}

The following summarises the experimental status of searches for Higgs boson
pair production prior to the work performed as part of this thesis. The focus
lies on results of the ATLAS and CMS collaborations obtained using \pp~collision
datasets at $\sqrt{s} = \SI{13}{\TeV}$ collected during the 2015 and 2016
data-taking periods at the LHC.


\subsection*{Searches for SM Higgs Boson Pair Production}%
\label{sec:past_results_smhh}

% https://cms-results.web.cern.ch/cms-results/public-results/publications/HIG-17-030/index.html

The ATLAS and CMS collaborations set upper limits on the cross section of SM \HH
production via \ggF, $\sigma_{\ggF}$, at \SI{95}{\percent} CL. The results of
both collaborations are summarised in \Cref{fig:prior_status_smhh}. Searches for
SM \HH production were performed in various channels, the \bbbb, \bbtautau, and
\bbyy channels being the most sensitive ones. A statistical combination of SM
\HH searches was performed by both collaborations. The observed (expected) upper
limits on $\sigma_{\ggF} / \sigma_{\ggF}^{\text{SM}}$ of the combination are
$6.9$ ($10$) and $22.2$ ($12.8$) ATLAS and CMS collaboration,
respectively~\cite{HDBS-2018-58,CMS-HIG-17-030}. At the time of publication,
these were the most stringent limits on SM \HH production.

\begin{figure}[htbp]
  \centering

  \begin{subfigure}[b]{0.9\textwidth}
    \centering

    \includegraphics[width=0.62\textwidth, trim=0 0.4em 0 0,
    clip]{status/atlas_36ifb}

    \subcaption{Results of SM \HH searches by the ATLAS collaboration.  Upper
      limits excluding systematic uncertainties are given in the ``Exp.\ stat.''
      column.  The figure is taken from Ref.~\cite{HDBS-2018-58}.}
  \end{subfigure}

  \vspace{0.5em}

  \begin{subfigure}[b]{0.9\textwidth}
    \centering

    \includegraphics[width=0.62\textwidth]{status/cms_36ifb}

    \subcaption{Results of SM \HH searches by the CMS collaboration. The
      $\bbbar V V$ ($V = Z$ or $W^\pm$) channel targets final states with two
      charged leptons. The figure is taken from Ref.~\cite{CMS-HIG-17-030}.}
  \end{subfigure}

  \caption{Upper limits at \SI{95}{\percent} CL on the cross section of SM \HH
    production via \ggF by the ATLAS (a) and CMS (b) collaborations. The upper
    limits are normalised to a SM cross section prediction of
    $\sigma_{\ggF} = \SI{33.5}{\femto\barn}$ and given separately for the
    individual channels, and the statistical combination of all listed
    channels. In both cases, the expected limits are derived under the
    background-only hypothesis (i.e.\ no SM \HH production). The results are
    based on \pp~collision data taken at the beginning of Run~2 of the LHC.}%
  \label{fig:prior_status_smhh}
\end{figure}


\subsection*{Constraints on the Strength of the Higgs Boson Self-Coupling}%
\label{sec:past_results_klambda}

The ATLAS and CMS collaborations reinterpreted the searches for SM \HH
production in the context of anomalous values of the trilinear Higgs boson
self-coupling constant. Upper limits at \SI{95}{\percent} CL were set on the
cross section of non-resonant \HH production as a function of the Higgs boson
self-coupling modifier, \klambda. All other couplings were fixed to their SM
values. The results of both collaborations are summarised in
\Cref{fig:prior_status_klambda}. The \klambda interval in which the upper limit
on the cross section does not exclude cross section predicted by theory is
considered as the \emph{allowed \klambda interval}. The results depicted in
\Cref{fig:prior_status_klambda} yield allowed \klambda intervals of
\begin{align*}
  -5.0 < \klambda < 12.0 \,\text{(observed)} \qquad -5.8 < \klambda < 12.0 \,\text{(expected)}
\end{align*}
for the result of the ATLAS collaboration~\cite{HDBS-2018-58} and
\begin{align*}
  -11.8 < \klambda < 18.8 \,\text{(observed)} \qquad -7.1 < \klambda < 13.6 \,\text{(expected)}
\end{align*}
for the result of the CMS collaboration~\cite{CMS-HIG-17-030}.

\begin{figure}[htbp]
  \centering

  \begin{subfigure}[b]{0.9\textwidth}
    \centering

    \includegraphics[width=0.67\textwidth, trim=0 1.2em 0 0,
    clip]{status/atlas_36ifb_klambda}

    % trim={<left> <lower> <right> <upper>}

    \subcaption{Results of the ATLAS collaboration for the \bbbb, \bbtautau, and
      \bbyy channels and their combination. The figure is taken from
      Ref.~\cite{HDBS-2018-58}.}
  \end{subfigure}

  \vspace{0.5em}

  \begin{subfigure}[b]{0.9\textwidth}
    \centering

    \includegraphics[width=0.58\textwidth]{status/cms_36ifb_klambda}

    \subcaption{Results of the CMS collaboration for the combination of the
      \bbbb, \bbtautau, \bbyy, and $\bbbar VV$ channels. The figure is taken
      from Ref.~\cite{CMS-HIG-17-030}.}
  \end{subfigure}

  \caption{Upper limits at \SI{95}{\percent} CL on the cross section of
    non-resonant \HH production as a function of \klambda by the ATLAS (a) and
    CMS (b) collaborations. The expected upper limits are obtained under the
    background-only assumption (i.e.\ no non-resonant \HH production). Values of
    \klambda where the theoretical prediction exceeds the upper limit are
    excluded by the measurements. The results are based on \pp~collision data
    taken at the beginning of Run~2 of the LHC.}%
  \label{fig:prior_status_klambda}
\end{figure}


\subsection*{Searches for Resonant Production of Higgs Boson Pairs}%
\label{sec:past_results_resonant}

The ATLAS and CMS collaborations performed searches for CP-even, scalar
resonances with narrow width decaying into a pair of SM Higgs bosons. Resonance
masses ranging from the \HH production threshold up to \SI{3000}{\GeV} are
considered by both collaborations. The upper limits at \SI{95}{\percent} CL on
the production cross section of the scalar resonance as a function of its mass
are shown in \Cref{fig:prior_status_reso}. Neither the ATLAS nor the CMS result
shows a statistically significant excess in the search for resonant \HH
production.
% The excluded cross sections range from about \SI{1000}{\femto\barn} at low
% mass to about \SI{5}{\femto\barn} at high resonance mass.

\begin{figure}[htbp]
  \centering

  \newcommand*{\mybox}[1][red]{\textcolor{#1}{\rule{1.2ex}{1.2ex}}}
  \definecolor{cbbww}{RGB}{0, 153, 0}
  \definecolor{cbbtautau}{RGB}{153, 0, 153}
  \definecolor{cbbyy}{RGB}{255, 51, 102}
  \definecolor{cbbbb}{RGB}{51, 51, 255}
  \definecolor{cwwyy}{RGB}{0, 204, 204}
  \definecolor{cwwww}{RGB}{204, 102, 51}

  \begin{subfigure}[t]{0.90\textwidth}
    \centering

    \includegraphics[width=0.52\textwidth, trim=0 0.2em 0 0,
    clip]{status/atlas_36ifb_resonant}

    \subcaption{Results of the ATLAS collaboration for the channels:
      \bbbb~(\mybox[cbbbb]), \bbtautau~(\mybox[cbbtautau]),
      \bbyy~(\mybox[cbbyy]), $\bbbar W^+ W^-$~(\mybox[cbbww]),
      $W^+ W^- \gamma\gamma$~(\mybox[cwwyy]), and
      $W^+ W^- W^+ W^-$~(\mybox[cwwww]). The statistical combination of all
      channels is shown in black. The observed (expected) upper limits are
      depicted as solid (dashed) lines. The figure is taken from
      Ref.~\cite{HDBS-2018-58}.}
  \end{subfigure}

  \vspace{0.5em}

  \begin{subfigure}[t]{0.9\textwidth}
    \centering

    \includegraphics[width=0.6\textwidth]{status/cms_36ifb_resonant}

    \subcaption{Results of the CMS collaboration for the statistical combination
      of the \bbbb, \bbtautau, \bbyy, and $\bbbar VV$ ($V = Z$ or $W^\pm$)
      channels. The figure is taken from Ref.~\cite{CMS-HIG-17-030}.}
  \end{subfigure}

  \caption{Upper limits at \SI{95}{\percent} CL on the production cross section
    of CP-even, scalar resonances ($S$ / $X$) decaying into a pair of SM Higgs
    bosons by the ATLAS (a) and CMS (b) collaboration. The expected upper limits
    are derived assuming the background-only hypothesis. The results are based
    on \pp~collision data taken at the beginning of Run~2 of the LHC.}%
  \label{fig:prior_status_reso}
\end{figure}


%%% Local Variables:
%%% mode: latex
%%% TeX-master: "../../phd_thesis"
%%% End:
