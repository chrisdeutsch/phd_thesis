%------------------------------------------------------------------------------
\chapter{Useful information}
\label{sec:app}
%------------------------------------------------------------------------------

\todo[inline]{Add information of $\pT(Z)$ dependence of \ZHF
  normalisation.}

\todo[inline]{Add plots for basic kinematic variables in the \hadhad
  signal region.}

%------------------------------------------------------------------------------
\chapter{Additional figures}
\label{sec:app_additional_figures}
% ------------------------------------------------------------------------------


\begin{figure}[htbp]
  \centering

  \begin{subfigure}[t]{.49\textwidth}
    \includegraphics[width=\textwidth]{mva/correlations/NonResHH_pearson}
    \subcaption{SM \HH (gluon fusion)}
  \end{subfigure}\hfill %
  \begin{subfigure}[t]{.49\textwidth}
    \includegraphics[width=\textwidth]{mva/correlations/ttbar_pearson}
    \subcaption{\ttbar}
  \end{subfigure}

  \begin{subfigure}[t]{.49\textwidth}
    \includegraphics[width=\textwidth]{mva/correlations/Ztautau_pearson}
    \subcaption{$Z \rightarrow \tautau + \text{jets}$}
  \end{subfigure}\hfill %
  \begin{subfigure}[t]{.49\textwidth}
    \includegraphics[width=\textwidth]{mva/correlations/Fake_pearson}
    \subcaption{Multi-jet}
  \end{subfigure}

  \caption{Correlation matrices of the MVA input variables for the SM
    \HH signal and the three largest background contributions (b) -
    (d) in the \hadhad SR.}
  \label{fig:mva_input_correlations}
  \todo[inline]{Will move to appendix (or remove completely).}
\end{figure}


\begin{figure}[htbp]
  \centering

  \begin{subfigure}{0.495\textwidth}
    \centering

    \includegraphics[width=\textwidth]{results_res/postfit/Region_BMin0_incJet1_dist300_J2_D2HDMPNN_T2_SpcTauLH_Y2015_LTT0_L1_GlobalFit_conditionnal_mu0log}
    \subcaption{$\mX = \SI{300}{\GeV}$}
  \end{subfigure}\hfill%
  \begin{subfigure}{0.495\textwidth}
    \centering

    \includegraphics[width=\textwidth]{results_res/postfit/Region_BMin0_incJet1_dist500_J2_D2HDMPNN_T2_SpcTauLH_Y2015_LTT0_L1_GlobalFit_conditionnal_mu0log}
    \subcaption{$\mX = \SI{500}{\GeV}$}
  \end{subfigure}

  \begin{subfigure}{0.495\textwidth}
    \centering

    \includegraphics[width=\textwidth]{results_res/postfit/Region_BMin0_incJet1_dist1000_J2_D2HDMPNN_T2_SpcTauLH_Y2015_LTT0_L1_GlobalFit_conditionnal_mu0log}
    \subcaption{$\mX = \SI{1000}{\GeV}$}
  \end{subfigure}\hfill%
  \begin{subfigure}{0.495\textwidth}
    \centering

    \includegraphics[width=\textwidth]{results_res/postfit/Region_BMin0_incJet1_dist1600_J2_D2HDMPNN_T2_SpcTauLH_Y2015_LTT0_L1_GlobalFit_conditionnal_mu0log}
    \subcaption{$\mX = \SI{1600}{\GeV}$}
  \end{subfigure}

  \caption{Distribution of the PNN discriminant of the \lephad SLT
    channel after the maximum likelihood fit of the background-only
    model in all signal and control regions. The signal overlay is
    scaled to the expected upper limit on $\sigma(pp \to X \to HH)$.}
\end{figure}


\begin{figure}[htbp]
  \centering

  \begin{subfigure}{0.495\textwidth}
    \centering

    \includegraphics[width=\textwidth]{results_res/postfit/Region_BMin0_incJet1_dist300_J2_D2HDMPNN_T2_SpcTauLH_Y2015_LTT1_L1_GlobalFit_conditionnal_mu0log}
    \subcaption{$\mX = \SI{300}{\GeV}$}
  \end{subfigure}\hfill%
  \begin{subfigure}{0.495\textwidth}
    \centering

    \includegraphics[width=\textwidth]{results_res/postfit/Region_BMin0_incJet1_dist500_J2_D2HDMPNN_T2_SpcTauLH_Y2015_LTT1_L1_GlobalFit_conditionnal_mu0log}
    \subcaption{$\mX = \SI{500}{\GeV}$}
  \end{subfigure}

  \begin{subfigure}{0.495\textwidth}
    \centering

    \includegraphics[width=\textwidth]{results_res/postfit/Region_BMin0_incJet1_dist1000_J2_D2HDMPNN_T2_SpcTauLH_Y2015_LTT1_L1_GlobalFit_conditionnal_mu0log}
    \subcaption{$\mX = \SI{1000}{\GeV}$}
  \end{subfigure}\hfill%
  \begin{subfigure}{0.495\textwidth}
    \centering

    \includegraphics[width=\textwidth]{results_res/postfit/Region_BMin0_incJet1_dist1600_J2_D2HDMPNN_T2_SpcTauLH_Y2015_LTT1_L1_GlobalFit_conditionnal_mu0log}
    \subcaption{$\mX = \SI{1600}{\GeV}$}
  \end{subfigure}

  \caption{Distribution of the PNN discriminant of the \lephad LTT
    channel after the maximum likelihood fit of the background-only
    model in all signal and control regions. The signal overlay is
    scaled to the expected upper limit on $\sigma(pp \to X \to HH)$.}
\end{figure}


\label{app:limit_tables}

\begin{table}[htbp]
  \centering

  % Workspaces: comb_2022_01_29
  \begin{tabular}{S[round-mode=places, round-precision=1]S[round-mode=figures, round-precision=3]S[round-mode=figures, round-precision=3]S[round-mode=figures, round-precision=3]S[round-mode=figures, round-precision=3]S[round-mode=figures, round-precision=3]S[round-mode=figures, round-precision=3]}
\toprule
{$m_\text{X}$ / \si{\GeV}} & {Observed} & {-2$\sigma$} & {-1$\sigma$} & {Expected} & {+1$\sigma$} & {+2$\sigma$} \\
\midrule
                       251 & 638.872175 &   181.664494 &   243.884926 & 338.467987 &   471.049512 &   631.475223 \\
                       260 & 894.993159 &   388.485481 &   521.542490 & 723.806261 &  1007.328904 &  1350.395718 \\
                       280 & 494.395118 &   451.739851 &   606.461601 & 841.658565 &  1171.345214 &  1570.271194 \\
                       300 & 539.764125 &   354.034729 &   475.292291 & 659.619384 &   917.999342 &  1230.643115 \\
                       325 & 338.515601 &   253.530669 &   340.365402 & 472.365365 &   657.395923 &   881.285781 \\
                       350 & 227.751021 &   188.783586 &   253.442320 & 351.731914 &   489.509061 &   656.221555 \\
                       375 & 130.064846 &   116.904015 &   156.943861 & 217.809577 &   303.127914 &   406.364431 \\
                       400 &  81.305295 &    77.219107 &   103.666797 & 143.870688 &   200.226373 &   268.417629 \\
                       450 &  49.055225 &    36.059070 &    48.409370 &  67.183414 &    93.499875 &   125.343202 \\
                       500 &  47.398396 &    22.887089 &    30.725960 &  42.642053 &    59.345400 &    79.556711 \\
                       550 &  25.498057 &    17.666592 &    23.717433 &  32.915490 &    45.808839 &    61.409993 \\
                       600 &  21.141948 &    14.047597 &    18.858926 &  26.172764 &    36.424915 &    48.830178 \\
                       700 &  25.444431 &    10.155733 &    13.634091 &  18.921642 &    26.333451 &    35.301856 \\
                       800 &  30.803797 &     8.179250 &    10.980659 &  15.239160 &    21.208502 &    28.431499 \\
                       900 &  30.613120 &     7.192976 &     9.656583 &  13.401584 &    18.651128 &    25.003157 \\
                      1000 &  30.482489 &     6.546307 &     8.788430 &  12.196744 &    16.974339 &    22.755304 \\
                      1100 &  28.003556 &     7.206365 &     9.674559 &  13.426531 &    18.685847 &    25.049700 \\
                      1200 &  24.595625 &     7.388710 &     9.919357 &  13.766265 &    19.158659 &    25.683539 \\
                      1400 &  27.385901 &    10.730404 &    14.405588 &  19.992339 &    27.823552 &    37.299442 \\
                      1600 &  33.889635 &    16.667519 &    22.376175 &  31.054067 &    43.218277 &    57.937161 \\
\bottomrule
\end{tabular}

  \caption{$\text{CL}_\text{s}$ upper limits on the cross section of
    $\PX \ra \PHiggs \PHiggs \ra \bbtautau$ from the combined fit of
    all channels. Limits are in fb.}%
  \label{tab:comb_limits_resonant}
\end{table}


\begin{table}[htbp]
  \centering

  % Workspaces: comb_2022_01_29
  \setlength{\tabcolsep}{1.2em}

\begin{tabular}{
  S[round-mode=places, round-precision=1, table-format=4.0]
  S[round-mode=figures, round-precision=2, table-format=4.0]
  S[round-mode=figures, round-precision=2, table-format=4.1]
  S[round-mode=figures, round-precision=2, table-format=4.1]
  S[round-mode=figures, round-precision=2, table-format=4.1]
  S[round-mode=figures, round-precision=2, table-format=4.1]
  S[round-mode=figures, round-precision=2, table-format=4.1]
  }
  \toprule
  & \multicolumn{6}{c}{Upper limit on $\sigma(\pp \to X \to \HH)$ at \SI{95}{\percent}~CL / \si{\femto\barn}} \\ \cmidrule{2-7}
  {$\mX / \si{\GeV}$} &  {Observed} & {-2$\sigma$} & {-1$\sigma$} &  {Expected} & {+1$\sigma$} & {+2$\sigma$} \\
  \midrule
  251 &  973.495799 &   265.045066 &   355.823501 &  493.818400 &   687.252341 &   921.310423 \\
  260 & 1627.925794 &   557.380325 &   748.284137 & 1038.482489 &  1445.267171 &  1937.482972 \\
  280 &  885.081013 &   554.703191 &   744.690080 & 1033.494591 &  1438.325460 &  1928.177118 \\
  300 &  900.688379 &   426.880001 &   573.087206 &  795.340965 &  1106.884516 &  1483.857065 \\
  325 &  427.179604 &   285.774280 &   383.652509 &  532.440009 &   741.002448 &   993.366246 \\
  350 &  262.632595 &   216.567956 &   290.742890 &  403.498330 &   561.552936 &   752.801470 \\
  375 &  153.878061 &   126.844843 &   170.289441 &  236.330819 &   328.904125 &   440.919267 \\
  400 &  102.277962 &    84.818611 &   113.869145 &  158.029694 &   219.931613 &   294.833899 \\
  450 &   62.479143 &    41.225854 &    55.345787 &   76.809901 &   106.897159 &   143.303211 \\
  500 &   55.685134 &    28.092274 &    37.713931 &   52.340087 &    72.842259 &    97.650205 \\
  550 &   25.524188 &    21.202297 &    28.464123 &   39.503034 &    54.976794 &    73.700285 \\
  600 &   26.608906 &    17.384295 &    23.338449 &   32.389529 &    45.076853 &    60.428713 \\
  700 &   32.164818 &    12.811085 &    17.198906 &   23.868959 &    33.218685 &    44.531999 \\
  800 &   33.258635 &    10.820855 &    14.527018 &   20.160863 &    28.058088 &    37.613854 \\
  900 &   42.358711 &     9.759847 &    13.102613 &   18.184048 &    25.306933 &    33.925736 \\
  1000 &   43.661054 &     9.557592 &    12.831085 &   17.807216 &    24.782492 &    33.222686 \\
  1100 &   35.580690 &    10.783233 &    14.476511 &   20.090768 &    27.960536 &    37.483079 \\
  1200 &   35.027829 &    11.710499 &    15.721367 &   21.818402 &    30.364903 &    40.706303 \\
  1400 &   34.671442 &    16.796855 &    22.549810 &   31.295041 &    43.553643 &    58.386742 \\
  1600 &   38.223685 &    24.965885 &    33.516747 &   46.515158 &    64.735643 &    86.782712 \\
  \bottomrule
\end{tabular}

%%% Local Variables:
%%% mode: latex
%%% TeX-master: "../phd_thesis"
%%% End:

  \caption{$\text{CL}_\text{s}$ upper limits on the cross section of
    $\PX \ra \PHiggs \PHiggs \ra \bbtautau$ from the \hadhad-only
    fit. Limits are in fb.}
  \label{tab:hadhad_limits_resonant}
\end{table}


\begin{table}[htbp]
  \centering

  % Workspaces: comb_2022_01_29
  \begin{tabular}{S[round-mode=places, round-precision=1]S[round-mode=figures, round-precision=3]S[round-mode=figures, round-precision=3]S[round-mode=figures, round-precision=3]S[round-mode=figures, round-precision=3]S[round-mode=figures, round-precision=3]S[round-mode=figures, round-precision=3]}
\toprule
{$m_\text{X}$ / \si{\GeV}} & {Observed} & {-2$\sigma$} & {-1$\sigma$} &  {Expected} & {+1$\sigma$} & {+2$\sigma$} \\
\midrule
                       251 & 552.032644 &   270.335780 &   362.926294 &  503.675786 &   700.970971 &   939.701218 \\
                       260 & 829.243978 &   570.414317 &   765.782296 & 1062.766755 &  1479.063844 &  1982.789806 \\
                       280 & 663.073122 &   726.556706 &   975.403748 & 1353.683262 &  1883.935454 &  2525.548865 \\
                       300 & 705.021696 &   671.542428 &   901.546975 & 1251.183475 &  1741.285405 &  2334.316374 \\
                       325 & 682.929132 &   531.546469 &   713.602136 &  990.350170 &  1378.280909 &  1847.683146 \\
                       350 & 640.029842 &   413.434924 &   555.037165 &  770.290787 &  1072.021915 &  1437.121279 \\
                       375 & 404.946223 &   340.643223 &   457.314169 &  634.669015 &   883.275646 &  1184.093543 \\
                       400 & 225.954653 &   206.923331 &   277.794962 &  385.528957 &   536.544768 &   719.276250 \\
                       450 & 115.340563 &    83.070740 &   111.522626 &  154.773151 &   215.399447 &   288.758210 \\
                       500 &  83.539028 &    45.334477 &    60.861621 &   84.464877 &   117.550672 &   157.584999 \\
                       550 &  76.951902 &    37.325428 &    50.109457 &   69.542826 &    96.783494 &   129.745126 \\
                       600 &  46.233908 &    27.117318 &    36.405050 &   50.523598 &    70.314231 &    94.261205 \\
                       700 &  34.918442 &    18.616178 &    24.992254 &   34.684710 &    48.271081 &    64.710802 \\
                       800 &  42.007130 &    14.243050 &    19.121322 &   26.536922 &    36.931718 &    49.509583 \\
                       900 &  28.934176 &    11.908244 &    15.986840 &   22.186830 &    30.877649 &    41.393674 \\
                      1000 &  27.789253 &    10.066135 &    13.513805 &   18.754707 &    26.101126 &    34.990407 \\
                      1100 &  31.739464 &    10.250724 &    13.761616 &   19.098623 &    26.579758 &    35.632048 \\
                      1200 &  26.569981 &    10.110804 &    13.573773 &   18.837932 &    26.216951 &    35.145679 \\
                      1400 &  37.714240 &    15.133699 &    20.317019 &   28.196333 &    39.241138 &    52.605524 \\
                      1600 &  61.591676 &    23.504469 &    31.554794 &   43.792323 &    60.946244 &    81.702755 \\
\bottomrule
\end{tabular}

  \caption{$\text{CL}_\text{s}$ upper limits on the cross section of
    $\PX \ra \PHiggs \PHiggs \ra \bbtautau$ from the \lephad-only
    fit. Limits are in fb.}
  \label{tab:lephad_limits_resonant}
\end{table}


\label{app:breakdown_table}

\begin{table}[htbp]
  \centering

  \caption{Decomposition of the variance on $\hat{\sigma}$, the
    maximum likelihood estimate of the cross section
    $\sigma(pp \to X\to HH)$, by uncertainty category for the fit to
    Asimov data with $\mu = 0$ in all regions. The decomposition is
    determined analogously to~\Cref{tab:breakdown_nonres}, separately
    for four exemplary signal mass hypotheses. The fractions of
    subcategories do not necessarily sum to the fraction of the parent
    category due to correlations between nuisance parameters.}%
  \label{tab:breakdown_res_exp_mu0}

  \begin{tabular}{
  l
  S[table-format=2.0, table-space-text-pre=\textless, table-column-width=1.6cm]
  S[table-format=2.0, table-space-text-pre=\textless, table-column-width=1.6cm]
  S[table-format=2.0, table-space-text-pre=\textless, table-column-width=1.6cm]
  S[table-format=2.0, table-space-text-pre=\textless, table-column-width=1.6cm]
  }
  \toprule
         & \multicolumn{4}{c}{Explained fraction of variance on $\hat{\sigma}$}\\
         %& \multicolumn{4}{c}{of variance on $\hat{\mu}$}\\
  \cmidrule{2-5}
  Source & {$\SI{300}{\GeV}$} & {$\SI{500}{\GeV}$} & {$\SI{1000}{\GeV}$} & {$\SI{1600}{\GeV}$} \\
  \midrule
  \textbf{Data statistical uncertainty}
         & 59\,\si{\percent} & 81\,\si{\percent} & 82\,\si{\percent} & 82\,\si{\percent} \\
  \textbf{Systematic uncertainties}
         & 41\,\si{\percent} & 19\,\si{\percent} & 18\,\si{\percent} & 17\,\si{\percent} \\
  \hspace{0.8em} Instrumental uncertainties
         & 10\,\si{\percent} & 1\,\si{\percent} & 1\,\si{\percent} & {\textless} 1\,\si{\percent}\\
  \hspace{0.8em} Signal modelling uncertainties
         & 1\,\si{\percent}  & 1\,\si{\percent} & {\textless} 1\,\si{\percent} & 3\,\si{\percent} \\
  \hspace{0.8em} Background statistical uncertainties
         & 18\,\si{\percent} & 11\,\si{\percent} & 7\,\si{\percent} & 9\,\si{\percent} \\
  \hspace{0.8em} Background modelling uncertainties
         & 12\,\si{\percent} & 7\,\si{\percent} & 10\,\si{\percent} & 5\,\si{\percent} \\
  \midrule
  \hspace{1.6em} -- \hspace{0.2em} Top quark (incl.\ free normalisation)
         & 3\,\si{\percent} & 2\,\si{\percent} & 1\,\si{\percent} & {\textless} 1\,\si{\percent} \\
  \hspace{1.6em} -- \hspace{0.2em} \ZHF (incl.\ free normalisation)
         & 3\,\si{\percent} & 1\,\si{\percent} & 3\,\si{\percent} & 2\,\si{\percent} \\
  \hspace{1.6em} -- \hspace{0.2em} SM Higgs boson
         & {\textless} 1\,\si{\percent} & 2\,\si{\percent} & 3\,\si{\percent} & 2\,\si{\percent} \\
  \hspace{1.6em} -- \hspace{0.2em} Fake-\tauhadvis
         & 4\,\si{\percent} & {\textless} 1\,\si{\percent} & 1\,\si{\percent} & 1\,\si{\percent} \\
  \hspace{1.6em} -- \hspace{0.2em} Other
         & {\textless} 1\,\si{\percent} & {\textless} 1\,\si{\percent} & {\textless} 1\,\si{\percent} & {\textless} 1\,\si{\percent} \\
  \bottomrule
\end{tabular}

%%% Local Variables:
%%% mode: latex
%%% TeX-master: "../phd_thesis"
%%% End:

\end{table}

\begin{table}[htbp]
  \centering

  \caption{Summary of allowed \klambda intervals obtained from
    searches for non-resonant \HH production by the CMS collaboration
    using \SI{138}{\per\femto\barn} of \pp-collision data taken during
    Run~2 of the LHC.}%
  \label{tab:cms_klambda}

  \begin{tabular}{lccc}
    \toprule
    & \multicolumn{3}{c}{Allowed \klambda interval} \\
    \cmidrule{2-4}
    Search channel & Observed & Expected & Reference  \\
    \midrule
    \bbtautau & $[-1.7, 8.7]$ & $[-2.9, \phantom{0}9.8]$  & \cite{CMS-HIG-20-010} \\
    \bbbb     & $[-2.3, 9.4]$ & $[-5.0, 12.0]$            & \cite{CMS-HIG-20-005} \\
    \bbyy     & $[-3.3, 8.5]$ & $[-2.5, \phantom{0}8.2]$  & \cite{CMS-HIG-19-018} \\
    \bottomrule
  \end{tabular}
\end{table}


%------------------------------------------------------------------------------
\chapter{Control and validation regions plots}
\label{app:control_and_validation_regions}
% ------------------------------------------------------------------------------

\todo[inline]{Should add more validation region plots with all
  uncertainties.}

\section{Multi-Jet Validation Region}

\section{Z+Jets Validation Region}

\section{2-tag SS 'Control Region'}


%------------------------------------------------------------------------------
\chapter{Graphical illustration of the copula-based toy generation}
\label{app:graphical_illustration_copula_model}
% ------------------------------------------------------------------------------



%%% Local Variables:
%%% mode: latex
%%% TeX-master: "../phd_thesis"
%%% End:
