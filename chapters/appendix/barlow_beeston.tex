The expected number of background events in bin $b$ of a channel $c$,
$\nu_{cb}$, is often estimated using a finite sample of events (e.g.\ from MC
simulation); therefore, $\nu_{cb}$ does not correspond to the true expected
number of background events. The background predictions are subject to
statistical uncertainties that have to be considered when performing inference,
particularly when bins are only sparsely populated by events.

This uncertainty is included in the likelihood function, employing the method
proposed by Barlow and Beeston~\cite{barlow1993}, by replacing the expected
number of background events estimated using the finite sample with its true
value, which is unknown and has to be inferred from data. In practice, this is
done by performing the substitution $\nu_{cb} \rightarrow \gamma_{cb} \nu_{cb}$,
introducing new NPs $\gamma_{cb}$.
% which specify the
% relative difference between the estimated and true expected number of background
% events.
%
% Initially it was proposed to introduce one such nuisance parameter for every
% background source~\cite{barlow1993}, however, a simplified version proposed in
% Ref.~\cite{conway2011} is used such that only the combined uncertainty on the
% total background prediction is considered instead.
These NPs are constrained by auxiliary measurements that contribute terms of the
form
\begin{align*}
  \pois(m_{cb} ; \gamma_{cb} \tau_{cb})
  \qquad \text{with} \qquad
  \tau_{cb} = \frac{( \sum_i w_i )^2}{\sum_i w_i^2} = \text{const.}
\end{align*}
to the likelihood function~\cite{cranmer2012}, where the sums go over all events
in bin $b$ of channel $c$ with event weights $w_i$. This corresponds to a
measurement of the effective number of unweighted events based on the observed
value $m_{cb}$, which is nominally equal to $\tau_{cb}$,\footnote{Generally,
  $m_{cb}$ is not integer-valued and thus not covered by the support of the
  Poisson distribution; therefore, the factorial term in the Poisson PMF is
  replaced by the gamma function to generalise the distribution to
  $\mathbb{R}^+$.} for the finite sample of events.

This approach is based on the approximation of the \emph{compound Poisson
  distribution} (CPD), which describes the distribution of the sum of a Poisson
number of random weights, by a \emph{scaled Poisson distribution}
(SPD)~\cite{Bohm:2013gla}. Formally, the CPD can be defined as
% The CPD describes the distribution of rate predictions based on a finite
% sample of weighted events and can be defined as
\begin{align*}
  X = \sum_{i = 1}^{N} W_i \,\text{,}  % \quad \text{with} \quad N \sim \pois(\lambda) \text{ and i.i.d.\ } W_i \text{ (independent of }N\text{)} \,\text{.}
\end{align*}
with $N \sim \pois(\lambda)$ and i.i.d.\ $W_i$ that are independent of $N$. The
CPD can be approximated, see for example Ref.~\cite{Bohm:2013gla}, using the SPD
defined by
\begin{align*}
  \tilde{X} = s \cdot \tilde{N} \quad \text{with} \quad \tilde{N} \sim \pois(\tilde{\lambda})
\end{align*}
with
\begin{align*}
  s = \frac{\expect(W^2)}{\expect(W)} \qquad \tilde{\lambda} = \frac{\lambda \expect(W)^2}{\expect(W^2)}
\end{align*}
where $\expect(W)$ and $\expect(W^2)$ are the first and second moments of the
weight distribution, respectively. The Barlow--Beeston method makes the
assumption that the expectation values can be approximated by sample averages
such that
\begin{align*}
  s = \frac{\sum_i w_i^2}{\sum_i w_i} \qquad \tilde{\lambda} = \frac{\lambda}{n} \frac{(\sum_i w_i)^2}{\sum_i w_i^2}
\end{align*}
with sample size $n$. The form of the likelihood terms introduced by the
Barlow--Beeston method is motivated by defining $m_{cb}$ to be the observed
value of $\tilde{N}$ and $\gamma_{cb} = \lambda / n$. Using these definitions,
the terms in the likelihood function read
\begin{align*}
  \pois(m_{cb}; \tilde{\lambda}) = \pois(m_{cb}; \gamma_{cb} \tau_{cb}) \,\text{.}
\end{align*}
Finally, the expected rate can be expressed as
\begin{align*}
  \expect(\tilde{X}) = s \cdot \expect(\tilde{N}) = \gamma_{cb} \sum_i w_i =
  \gamma_{cb} \nu_{cb} \,\text{,}
\end{align*}
where $\nu_{cb} = \sum_i w_i$ is the background rate predicted using the finite
sample of events, thus motivating the substitution
$\nu_{cb} \to \gamma_{cb} \nu_{cb}$ performed in the Barlow--Beston method.

%%% Local Variables:
%%% mode: latex
%%% TeX-master: "../../phd_thesis"
%%% End:
