\begin{figure}[htbp]
  \centering

  \includegraphics[width=\textwidth]{event_displays/hadhad}

  \caption[Visualisation of the SM~\HH candidate event with the largest BDT
  score observed in data in the \hadhad channel.]{Visualisation of the SM~\HH
    candidate event with the largest BDT score observed in data in the \hadhad
    channel. The two $b$-tagged jets (blue cones) have transverse momenta of
    \SI{160}{\GeV} and \SI{100}{\GeV}. The two \tauhadvis candidates (red cones)
    have transverse momenta of \SI{100}{\GeV} and \SI{40}{\GeV}. Energy
    deposited in cells of the electromagnetic (hadronic) calorimeters is
    visualised as green (yellow) towers. Reconstructed charged-particle tracks
    are depicted as yellow lines. The event has $\mMMC = \SI{130}{\GeV}$,
    $\mBB = \SI{130}{\GeV}$, and $\mHH = \SI{510}{\GeV}$. The image is taken
    from Ref.~\cite{HDBS-2018-40}.}%
  \label{fig:event_display_hadhad}
\end{figure}


\begin{figure}[htbp]
  \centering

  \includegraphics[width=\textwidth]{event_displays/lephad}

  \caption[Visualisation of the SM~\HH candidate event with the largest NN score
  observed in data in the \lephad SLT channel.]{Visualisation of the SM~\HH
    candidate event with the largest NN score observed in data in the \lephad
    SLT channel. The two $b$-tagged jets (blue cones) have transverse momenta of
    \SI{190}{\GeV} and \SI{90}{\GeV}. The \tauhadvis candidate (red cone) has a
    transverse momentum of \SI{80}{\GeV}. The muon (red line) has a transverse
    momentum of \SI{30}{\GeV}. Energy deposited in cells of the electromagnetic
    (hadronic) calorimeters is visualised as green (yellow)
    towers. Reconstructed charged-particle tracks are depicted as yellow
    lines. The event has $\mMMC = \SI{120}{\GeV}$, $\mBB = \SI{120}{\GeV}$, and
    $\mHH = \SI{680}{\GeV}$. The image is taken from Ref.~\cite{HDBS-2018-40}.}%
  \label{fig:event_display_lephad}
\end{figure}

%%% Local Variables:
%%% mode: latex
%%% TeX-master: "../../phd_thesis"
%%% End:
