The recommendations of the ATLAS collaboration for the derivation of
uncertainties on the modelling of selected physics processes using simulation
are summarised. \Cref{app:top_uncertainties} describes the prescriptions adopted
to derive uncertainties for \ttbar backgrounds. \Cref{app:zjets_uncertainties}
summarises the prescriptions for \Zjets backgrounds.


\subsection{Uncertainties on the Modelling of \ttbar Production using
  Simulation}%
\label{app:top_uncertainties}

% https://twiki.cern.ch/twiki/bin/viewauth/AtlasProtected/PmgTopProcesses
The nominal event simulation for the \ttbar background uses
\POWHEGBOX[v2]~\cite{Frixione:2007nw} as a matrix element (ME) generator
interfaced to \PYTHIA[8.230]~\cite{Sjostrand:2014zea} for the parton shower (PS)
and hadronisation. Uncertainties on the modelling of these processes are derived
by varying the simulation setup. The following uncertainties are considered:
\begin{description}

\item[Hard scatter simulation and NLO+PS matching] An uncertainty due to the
  choice of ME generator is estimated by comparing the nominal setup with an
  alternative in which \POWHEGBOX[v2] is replaced by \MGNLO. This comparison
  also probes the effect of different schemes for matching NLO ME generators to
  the PS simulation.\footnote{\POWHEGBOX[v2] uses the \POWHEG
    method~\cite{Nason:2004rx,Frixione:2007vw,Alioli:2010xd} and \MGNLO the
    MC@NLO method~\cite{Frixione:2002ik} for NLO+PS matching.}

\item[Parton shower and hadronisation model] An uncertainty on the modelling of
  the PS and non-perturbative effects is estimated by replacing \PYTHIA[8] by
  \HERWIG[7] for the PS simulation.

\item[Missing higher order contributions] The renormalisation scale~(\muR) and
  factorisation scale~(\muF) is doubled (halved) to probe the effect of
  truncating the perturbative expansion in \alphas when simulating the hard
  scatter process. Perturbative QCD calculations to sufficiently high order
  should be approximately independent of the choice of scale.

\item[Initial and final state radiation (ISR / FSR)] An uncertainty on the
  emission of ISR is estimated by varying $\alphas^\text{ISR}$ in the A14 set of
  tuned parameters for \PYTHIA[8]~\cite{ATL-PHYS-PUB-2014-021}.
  % Tune of the MPI, ISR, FSR parameters in Pythia8
  An estimate of the uncertainty from the modelling of FSR emissions is
  estimated by doubling (halving) the renormalisation scale used for FSR
  branchings in
  \PYTHIA[8]~\cite{Sjostrand:2014zea,Mrenna:2016sih,pythia-variations-online}.
  % https://pythia.org/latest-manual/Variations.html

\item[Damping factor for additional emissions] The damping parameter~\hdamp in
  \POWHEGBOX[v2] is increased to $3 m_\text{top}$ (from $1.5 m_\text{top}$) and
  compared with the nominal simulation setup. The \hdamp parameter controls the
  transverse momentum of additional radiation when matching \POWHEGBOX[v2] to
  \PYTHIA[8] using the
  \POWHEG-method~\cite{ATL-PHYS-PUB-2016-020,ATL-PHYS-PUB-2020-023}.

\end{description}
% PDF uncertainties were neglected.
These prescriptions are a revised version of the methodology outlined in
Ref.~\cite{ATL-PHYS-PUB-2020-023}. Uncertainties on the \NNPDF[3.0nlo] set of
PDFs and the value of \alphas were found to be negligible in the search for
Higgs boson pair production presented in this thesis.

Variations of the renormalisation and factorisation scales are provided by
internal re-weighting in \POWHEGBOX[v2]. Similarly, \PYTHIA[8] provides
variations of initial and final state emissions by varying the renormalisation
scales in the PS by re-weighting~\cite{Mrenna:2016sih,pythia-variations-online}.
This approach allows to estimate uncertainties without changing the
particle-level predictions of the simulation program, thus avoiding the need to
re-run the detector simulation.


\subsection{Uncertainties on the Modelling of \Zjets production in Simulation}%
\label{app:zjets_uncertainties}

The nominal \Zjets event simulation uses \SHERPA[2.2.1] for the simulation of
the hard scatter event and parton showering. The following uncertainties are
considered:
\begin{description}

\item[Factorisation and renormalisation scales] Six variations of the
  factorisation and renormalisation scales are performed using internal
  re-weighting implemented in \SHERPA[2.2.1]~\cite{Bothmann:2019yzt}, altering
  the scales by factors of $\frac{1}{2}$ and $2$. The following variations are
  performed:
  \begin{align*}
    \left( \frac{\muF}{\muF^{\text{nom.}}}, \frac{\muR}{\muR^{\text{nom.}}} \right) \in
    \left\{ (\tfrac{1}{2}, \tfrac{1}{2}), (\tfrac{1}{2}, 1), (1, \tfrac{1}{2}), (1, 2), (2, 1), (2, 2) \right\} \,\text{,}
  \end{align*}
  where $\muF^{\text{nom.}}$ and $\muR^{\text{nom.}}$ are the nominal scale
  values.

\item[Resummation scale] The scale of the resummation of soft gluon emissions in
  the \SHERPA parton shower is varied by factors of $\frac{1}{2}$ and
  $2$. Variations of the resummation scale are provided in parameterised form
  with respect to the default \SHERPA configuration in Ref.~\cite{anders:2017}.

\item[Multi-jet merging scale] The simulation of \Zjets events with
  \SHERPA[2.2.1] uses matrix elements of NLO accuracy for up to two and LO for
  up to four partons. These multi-parton matrix elements are merged with the
  parton shower using an extension of the CKKW
  algorithm~\cite{Catani:2001cc,Hoeche:2009rj,Hoeche:2012yf}. The characteristic
  scale~$Q_{\text{cut}}$ of the multi-jet merging algorithm is varied from its
  nominal value of $Q_{\text{cut}} = \SI{20}{\GeV}$ to \SI{15}{\GeV} and
  \SI{30}{\GeV}~\cite{anders:2017}. These variations are provided, following the
  approach for the resummation scale, in parameterised form in
  Ref.~\cite{anders:2017}.

\item[PDF+\alphas and PDF choice] Uncertainties on the \NNPDF[3.0nnlo] set of
  PDFs~\cite{Ball:2014uwa} are evaluated using 100 replica sets provided through
  the \textsc{LHAPDF6} library~\cite{Buckley:2014ana} and implemented using
  internal re-weighting in \SHERPA. The uncertainty on \alphas propagated by
  comparing \NNPDF[3.0nnlo] PDF sets with $\alphas(\mZ^2) = 0.117$ and $0.119$
  with the nominal set using a value of $0.118$. Finally, an uncertainty on the
  choice of PDF set is estimated by comparing with two alternative PDF sets
  \MMHT[nnlo68cl]~\cite{Harland-Lang:2014zoa} and
  \CT[14nnlo]~\cite{Dulat:2015mca}.

\item[Alternative generator and parton shower] The prediction of \Zjets with the
  default configuration of \SHERPA[2.2.1] is compared to an alternative setup
  using~\MGNLO[2.2.2]~\cite{Alwall:2014hca} for the calculation of the hard
  interaction at LO interfaced to~\PYTHIA[8.186]~\cite{Sjostrand:2007gs} for
  parton showering.

\end{description}


%%% Local Variables:
%%% mode: latex
%%% TeX-master: "../../phd_thesis"
%%% End:
