\section{Signals with Varied Higgs Boson Self-Coupling}

Cross section is at $\NNLO_\text{FTapprox}$ with approximate finite top mass.

The samples are $\NLO_\text{FTapprox}$ (old sample) and $\NLO_\text{FT}$ (new
sample -- softer $m_{hh}$ spectrum).


\subsubsection{Understanding $\kappa_{\lambda}$ Dependency of Signal}
% From here: https://arxiv.org/pdf/1910.00012.pdf

$\kappa_\lambda = 2.2 - 2.4$ point of maximum destructive interference.

$\kappa_\lambda < -1$ and $\kappa_\lambda > 5$ triangle-type diagram
starts dominating (enhancing low $m_\text{HH}$-region).

Factorization of cross section amplitudes:
\begin{align*}
  A(\ktop, \klambda)
  = A_\square(\ktop) + A_\triangle(\ktop, \klambda)
  = \ktop^2 A_\square^\prime + \ktop \klambda A_\triangle^\prime
\end{align*}
At leading order write amplitude as sum of box diagram amplitude and
triangle diagram amplitude and rewrite with coupling independent parts
of amplitudes $A^\prime$.
\begin{align*}
  \sigma &\propto \int \left| A(\ktop, \klambda) \right|^2 \\
         &= \int \left|A_\square(\ktop)\right|^2 + \left|A_\triangle(\ktop, \klambda)\right|^2
           + 2 \operatorname{Re}\left( \overline{A_\square(\ktop)} \, A_\triangle(\ktop, \klambda) \right) \\
         &= \int \ktop^4 \left[ \left| A_\square^\prime \right|^2 + \left( \frac{\klambda}{\ktop}  \right)^2 \left| A_\triangle^\prime \right|^2 + 2 \frac{\klambda}{\ktop} \operatorname{Re}\left(\overline{A_\square^\prime} A_\triangle^\prime\right) \right]
\end{align*}
Findings:
\begin{itemize}
\item At leading order the kinematics of Higgs pair production is
  soley described by $\frac{\klambda}{\ktop}$ (the total cross section
  does depend on both individually)

\item For fixed \ktop the total LO cross section of Higgs pair
  production can be described by a second order polynomial in \klambda
\end{itemize}


The cross section can be divided into three distinct parts (the box
term, the triangle term and the interference term). If one produces
three samples with different \ktop and \klambda values, one can solve
for these terms and describe the general cross section (as a function
of \ktop and \klambda) as follows (in ATLAS the following basis is
used $S(\ktop, \klambda)$ with the samples
$S(1, 0), S(1, 1), S(1, 20)$). The basis is suitable since $S(1, 0)$
directly relates to the box diagram amplitude, $S(1, 20)$ is dominated
by the triangle diagram, and for the SM point $S(1, 1)$ there is large
interference between box and triangle diagram therefore giving access
to the interference term:
\begin{align*}
  \frac{\mathrm{d}\sigma}{\mathrm{d}\mathbf{x}} = \ktop^2 \left(
  \left[ \ktop^2 + \frac{\klambda^2 - 21 \klambda \ktop}{20} \right] S(1, 0, \mathbf{x})
  + \left[ \frac{20 \klambda \ktop - \klambda^2}{19} \right] S(1, 1, \mathbf{x})
  + \left[ \frac{\klambda^2 - \klambda \ktop}{380}  \right] S(1, 20, \mathbf{x})
  \right)
\end{align*}


\begin{align*}
  \frac{\mathrm{d}\sigma}{\mathrm{d}\mathbf{x}} (\ktop, \klambda) = \ktop^2 \left(
  \left[ \ktop^2 + \frac{\klambda^2 - 21 \klambda \ktop}{20} \right] \left. \frac{\mathrm{d}\sigma}{\mathrm{d}\mathbf{x}} \right|_{\ktop = 1, \klambda = 0}
  + \left[ \frac{20 \klambda \ktop - \klambda^2}{19} \right] \left. \frac{\mathrm{d}\sigma}{\mathrm{d}\mathbf{x}} \right|_{\ktop = 1, \klambda = 1}
  + \left[ \frac{\klambda^2 - \klambda \ktop}{380}  \right] \left. \frac{\mathrm{d}\sigma}{\mathrm{d}\mathbf{x}} \right|_{\ktop = 1, \klambda = 20}
  \right)
\end{align*}




%%% Local Variables:
%%% mode: latex
%%% TeX-master: "../../phd_thesis"
%%% End:
