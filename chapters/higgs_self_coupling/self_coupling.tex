With the \pp-collision datasets collected during Run~2 of the LHC,
direct searches for non-resonant Higgs boson pair production
constitute the most sensitive probes of the Higgs boson self-coupling
constant, \lambdahhh. This is due to the large sensitivity of the
% total and differential
non-resonant \HH production cross section to anomalous values of
\lambdahhh. Hereafter, the self-coupling constant is given in terms of
the modifier $\klambda = \lambdahhh / \lambdahhh^{\text{SM}}$ relating
an assumed value of the self-coupling constant to the value predicted
by the SM. Accessing the Higgs boson self-coupling is compelling to
test the predictions of the SM and to search for deviations that can,
for example, originate from BSM phenomena appearing at large energy
scales and thus manifest as changes in the (effective) Higgs boson
self-coupling constant.

This chapter presents a reinterpretation of the search for SM \HH
production (i.e.\ $\klambda = 1$) from~\Cref{sec:dihiggs} in terms of
non-resonant \HH production with anomalous values of the Higgs boson
self-coupling constant. The reinterpretation allows to set upper
limits on the non-resonant production cross section of Higgs boson
pairs as a function of \klambda. These upper limits allow, by
comparison with the theoretical cross section predictions, to exclude
regions of \klambda that are incompatible with the observations made
in the SM \HH search.

Previous constraints on \klambda were set by the ATLAS collaboration
using up to \SI{36.1}{\per\femto\barn} of \pp-collision data taken
during Run~2 of the LHC. An allowed range of $-5.0 < \klambda < 12.0$
at \SI{95}{\percent} CL was obtained by combining the results of
searches for non-resonant \HH production in the \bbtautau, \bbbb, and
\bbyy channels~\cite{HDBS-2018-58}. The methods used for the
reinterpretation performed in this chapter are largely adopted from
the earlier result published in Ref.~\cite{HDBS-2018-58}. The
following focuses on the reinterpretation of the search in the
$\bbbar\tautau$ channel with \SI{139}{\per\femto\barn} of
\pp-collision data. Some of the results of this chapter were published
by the ATLAS collaboration in Ref.~\cite{ATLAS-CONF-2021-052}.

This chapter is structured as follows: \Cref{sec:self_coupling_pheno}
describes the phenomenology of a non-resonantly produced \HH signal
with anomalous values of \klambda. The reinterpretation of the SM \HH
search, including the statistical model, assumptions, and limitations,
is discussed in \Cref{sec:reinterpretation}. The upper limits on the
cross section and the excluded intervals of \klambda are presented in
\Cref{sec:reinterpretation_results}. A conclusion and outlook is given
in \Cref{sec:reinterpretation_conclusion}.


\section{Phenomenology of Higgs Boson Pair Production with Anomalous
  Higgs Boson Self-Coupling Strength}%
\label{sec:self_coupling_pheno}

Before describing the reinterpretation of the SM \HH search, the
experimental signature of non-resonant \HH production with anomalous
values of \klambda is discussed. The aim is to understand the
sensitivity of direct searches for non-resonant \HH production, and of
the individual search channels.

The non-resonant \HH production cross section via \ggF and VBF is
shown in \Cref{fig:hh_xsec_incl} as a function of \klambda. The \ggF
production mode is the dominant contribution to Higgs boson pair
production throughout the considered \klambda range. For this
production mode, the destructive interference between the box and
triangle diagram becomes maximal at about $\klambda = 2.3$ at which
point the cross section reaches a minimum of approximately
\SI{13}{\femto\barn}. Similar behaviour is observed for the VBF
production mode, although involving different diagrams (cf.\
\Cref{fig:dihiggs_ggf_feyn,fig:dihiggs_vbf_feyn}) and resulting in a
different location of the minimum.

\begin{figure}[htbp]
  \centering

  \includegraphics[width=0.55\textwidth]{self_coupling/hh_xsec}

  \caption{Higgs boson pair production cross section via the \ggF and
    VBF production modes as a function of \klambda. The production
    cross section for \ggF is given at $\text{NNLO}_{\text{NLO-i}}$
    rescaled to
    $\text{NNLO}_{\text{FTapprox}}$~\cite{Grazzini:2018bsd} in the
    $\klambda = 1$
    limit~\cite{Amoroso:2020lgh,Baglio:2020wgt,LHCHWGHH}. The
    production cross section for VBF is obtained from simulation with
    \MGNLO at LO after applying an $\text{N}^3\text{LO}$ $k$-factor
    derived for the SM case~\cite{Dreyer:2018qbw,LHCHWGHH}. The cross
    sections are parameterised as quadratic functions of
    \klambda. Theoretical uncertainties are shown as coloured bands.}%
  \label{fig:hh_xsec_incl}
\end{figure}

In addition to the change in total cross section, anomalous values of
\klambda alter the differential \HH production cross section in terms
of the invariant mass of the pair of Higgs bosons. This is shown in
\Cref{fig:hh_xsec_mhh} for the dominant \ggF production mode and for
five exemplary values of \klambda. The \mHH spectra for different
values of \klambda show large differences in their \emph{hardness} as
measured by the median of the \mHH distribution
in~\Cref{fig:hh_median_mhh}. For \klambda values just below or at the
point of maximum destructive interference between the box and triangle
diagram, the \mHH spectra are moderately hard and have a pronounced
double peak structure. For other values of \klambda, particularly for
$\klambda \approx 3$, the cross section at low \mHH is enhanced
resulting in softer \mHH spectra.

\begin{figure}[htbp]
  \begin{subfigure}[t]{0.485\textwidth}
    \includegraphics[width=\textwidth]{self_coupling/hh_mhh_vs_klam}
    \subcaption{%Differential \HH production cross section with respect
      %to \mHH for the \ggF production mode and selected values of
      %\klambda.
      % Differential cross sections are normalised by dividing by the
      % total cross section. Only statistical uncertainties from the
      % finite number of generated events are shown.
    }%
    \label{fig:hh_xsec_mhh}
  \end{subfigure}\hfill%
  \begin{subfigure}[t]{0.485\textwidth}
    \includegraphics[width=\textwidth]{self_coupling/hh_median_mhh_vs_klam}
    \subcaption{%Median value of \mHH for Higgs boson pairs produced
      % via \ggF as a function of \klambda
    }%
    \label{fig:hh_median_mhh}
  \end{subfigure}

  \caption{Differential cross section of Higgs boson pair production
    with respect to \mHH for the \ggF production mode and selected
    values of \klambda (a) and the median value of \mHH as a function
    of \klambda (b). Both are given for \pp-collisions at
    $\sqrt{s} = \SI{13}{\TeV}$ and assuming $\mH =
    \SI{125.0}{\GeV}$. The cross sections are obtained from simulation
    with \POWHEGBOX[v2] at NLO including the top-quark mass
    dependence~\cite{Heinrich:2019bkc,Heinrich:2020ckp} for
    $\klambda = 0, 1, 10$. Differential cross sections for other
    values of \klambda are estimated using morphing techniques that
    are discussed in \Cref{sec:self_coupling_signals}.}
\end{figure}

The \klambda-dependency of $\mathrm{d}\sigma / \mathrm{d}\mHH$ is an
important factor affecting the sensitivity of searches to signals with
anomalous \klambda. This is due to experimental limitations in the
signal acceptance at low \mHH of most search channels due to
\pT-thresholds applied to objects at trigger-level. Particularly
searches targeting (visible) hadronic final states such as
$\bbbar\hadhad$ and \bbbb are affected. While these channels yield
stringent cross section limits for the SM case, scenarios where the
cross section at low \mHH is enhanced are less favourable due to the
high thresholds of $b$-jet and \tauhadvis triggers. In these cases,
the \bbyy channel, due to its use of di-photon triggers, yields larger
signal acceptance and the most stringent exclusion limits of a single
channel.


\section{Reinterpretation of the Search for SM \HH Production}%
\label{sec:reinterpretation}

The SM \HH search from \Cref{sec:dihiggs} is reinterpreted to set
upper limits on the cross section of non-resonant \HH production as a
function of an assumed valued of \klambda. %These upper limits allow,
%by comparison with the theoretical cross section predictions
%(cf.~\Cref{fig:hh_xsec_incl}), to exclude regions of \klambda that are
%incompatible with the observations made in the SM \HH search.
The reinterpretation adopts the statistical framework presented in
\Cref{sec:statistical_analysis} with few modifications. These
modifications are restricted to the signal model used in the
statistical interpretation. The background model and final
discriminants, including their binning, remain identical to those in
the SM \HH search.

The signal model used for the statistical interpretation is obtained
by replacing the SM \HH signal with signals from non-resonant \HH
production with arbitrary but fixed \klambda. Similar to the SM \HH
search, the combination of non-resonant \HH production via \ggF and
VBF is considered as the signal process. The signal is normalised
using the total non-resonant \HH production cross section via \ggF and
VBF, $\sigma_{\text{ggF+VBF}}$, which is allowed to vary freely and is
considered to be the POI.
% \footnote{The total cross section $\sigma_{\text{ggF+VBF}}$ is
% inclusive in the decay modes of the Higgs bosons. It is extrapolated
% from the measurement in the \bbtautau channel assuming Higgs boson
% branching ratios as predicted by the SM.}
The methods for obtaining signal templates in the signal regions are
described in \Cref{sec:self_coupling_signals}.

The adopted method of reinterpretation makes assumptions that are
given in the following: First, except for \klambda, all coupling
strengths are assumed to be at their SM values. Second, the single
Higgs boson production cross sections and branching ratios are fixed
to their SM values and are thus assumed to be independent of
\klambda. This is generally not the case since variations of \klambda
affect both the production cross sections and branching ratios due to
electroweak corrections. These effects and their sensitivity to
\klambda are discussed in
Refs.~\cite{ATL-PHYS-PUB-2019-009,Degrassi:2016wml,Maltoni:2017ims}.
Since backgrounds from single Higgs boson production are
non-negligible in the SM \HH search, it is instructive to gauge the
crudeness of this approximation over the allowed interval of
$-5.0 < \klambda < 12.0$ from previous results of the ATLAS
collaboration~\cite{HDBS-2018-58}. In the considered \klambda
interval, the single Higgs boson production cross sections deviate
from the SM prediction by up to \SI{12}{\percent} (\SI{25}{\percent})
for the \ggF, VBF, and $VH$ ($\ttbar H$) production
modes~\cite{ATL-PHYS-PUB-2019-009}. The Higgs boson branching ratios
to fermions show small relative deviations of up to \SI{3}{\percent}
from the SM over the relevant \klambda
range~\cite{ATL-PHYS-PUB-2019-009}. For the purpose of this
reinterpretation, signal processes were generated assuming SM Higgs
boson branching ratios for consistency with the treatment of single
Higgs boson backgrounds. The assumptions about single Higgs boson
production cross sections and branching ratios were dropped in a
follow-up analysis performed as part of
Ref.~\cite{ATL-HDBS-2022-03-002} by the ATLAS collaboration, which is
briefly discussed in the conclusion of this chapter.


\subsection{Signal Templates and Uncertainties}%
\label{sec:self_coupling_signals}

Templates of the contribution of a signal with anomalous values of
\klambda in all three signal regions are required for the
reinterpretation. These are obtained using morphing and re-weighting
techniques developed by the ATLAS collaboration that will be explained
hereafter. An ingredient for both methods are simulated event samples
for different assumed values of \klambda. The event simulation
proceeds using the same generator setup described in
\Cref{sec:data_and_simulation}. Non-resonant \HH events are generated
for $\klambda = 1, 10$ for the \ggF production mode and
$\klambda = 0, 1, 10, 20$ for the VBF production mode. Additionally,
large samples of events from non-resonant \HH production via \ggF are
generated for $\klambda = 0, 1, 10, 20$ without simulation of the
ATLAS detector. These event samples are normalised using the cross
sections previously shown in \Cref{fig:hh_xsec_incl}.

Established morphing techniques~\cite{ATL-PHYS-PUB-2015-047} are used
to obtain signal templates for arbitrary \klambda by linearly
combining a set of basis templates with fixed \klambda. This approach
is motivated by considering the squared matrix element of non-resonant
\HH production for either the \ggF or VBF production mode at leading
order in \klambda. The squared matrix element can be written as
\begin{align*}
  |\mathcal{M}|^2 = \klambda^2 |\mathcal{A}|^2 + 2 \klambda \operatorname{Re}(\mathcal{A} \mathcal{B}^*) + |\mathcal{B}|^2 \,\text{,}
\end{align*}
where $\mathcal{A}$ is the (complex) amplitude of diagrams involving
the Higgs boson-self coupling after factoring out the value of the
anomalous self-coupling constant, and $\mathcal{B}$ the amplitude of
diagrams not involving any Higgs boson self-interactions. Therefore,
knowing $|\mathcal{M}|^2$ at three suitably chosen values of \klambda
one can obtain the matrix element for any value of \klambda by solving
for the squared amplitudes of $\mathcal{A}$ and $\mathcal{B}$, and the
interference term. The same considerations apply to predictions of
differential cross sections such that the differential cross section
at any \klambda can be obtained by linearly combining the predictions
at $\klambda = a, b$ and $c$ according to
\begin{align*}
  \frac{\mathrm{d}\sigma}{\mathrm{d}\myvec{x}}(\klambda)
  = w_1(\klambda) \left. \frac{\mathrm{d}\sigma}{\mathrm{d}\myvec{x}} \right|_{\klambda = a}
  + w_2(\klambda) \left. \frac{\mathrm{d}\sigma}{\mathrm{d}\myvec{x}} \right|_{\klambda = b}
  + w_3(\klambda) \left. \frac{\mathrm{d}\sigma}{\mathrm{d}\myvec{x}} \right|_{\klambda = c} \,\text{,}
\end{align*}
where the coefficients $w_i(\klambda)$ are second degree polynomials
in \klambda.

For the \ggF production mode, this morphing technique is only used
indirectly. Differential cross section predictions at generator-level
are obtained by linearly combining the $\klambda = 0, 1, 20$ event
samples without detector simulation. These predictions are used to
derive a re-weighting in the generator-level \mHH for the
$\klambda = 1$ event sample to obtain any \klambda. This re-weighting
is then applied to the $\klambda = 1$ sample with full detector
simulation. For the VBF production mode, the morphing technique is
used directly by combining the $\klambda = 1, 2, 10$ samples. Event
samples not used for morphing or re-weighting are used for validation
purposes.

Uncertainties on the modelling of the signal processes with anomalous
\klambda are derived according to \Cref{sec:modelling_uncertainties}.
Variations of the $H \to \tautau$ and $H \to \bbbar$ branching ratios,
the parton shower program, the renormalisation and factorisation
scales, and the PDFs and $\alphas$ are considered. Additionally, the
statistical uncertainties of the \klambda re-weighting factors are
accounted for.


\subsection{Signal Acceptance in the \bbtautau Channel}%
\label{sec:self_coupling_bbtt_limitations}

The sensitivity of the $\bbbar\hadhad$ and $\bbbar\lephad$ SLT/LTT
channels to non-resonant \HH production can be qualitatively
understood by examining the acceptance times efficiency for signal
events, hereafter abbreviated as \emph{signal acceptance}, as a
function of \klambda. The signal acceptance is shown in
\Cref{fig:acceptance_vs_klambda} for the signal region selections in
all three channels and their combination. It reaches a maximum at
$\klambda \approx 2$ which corresponds to the value of the
self-coupling constant with the largest median \mHH for signal events
produced via \ggF. The limited signal acceptance at low \mHH due to
the trigger selection is illustrated in \Cref{fig:acceptance_vs_mhh}
for the dominant \ggF production mode. The \hadhad and \lephad SLT
channel show similar signal acceptance over a wide range of \mHH,
however, in the range of $\SI{250}{\GeV}< \mHH < \SI{320}{\GeV}$ the
\lephad SLT channel dominates the signal acceptance in this
search. The inclusion of the \lephad LTT channel, due to selecting
events with leptons below the lepton \pT-threshold of the \lephad~SLT
channel, is primarily intended to improve the signal acceptance at low
\mHH where its relative contribution to the total signal acceptance is
largest.

\begin{figure}[htbp]
  \centering

  \begin{subfigure}[t]{0.485\textwidth}
    \includegraphics[width=\textwidth]{self_coupling/acc_vs_klam}
    \subcaption{Signal (\ggF + VBF) acceptance as a function of
      \klambda. The combination of all channels is shown in black.}%
    \label{fig:acceptance_vs_klambda}
  \end{subfigure}\hfill%
  \begin{subfigure}[t]{0.485\textwidth}
    \includegraphics[width=\textwidth]{self_coupling/acc_vs_mhh}
    \subcaption{Signal (\ggF only) acceptance in bins of \mHH.}%
    \label{fig:acceptance_vs_mhh}
  \end{subfigure}

  \caption{Non-resonant \HH signal acceptance in the \hadhad,
    \lephad~SLT, and \lephad~LTT channel. The signal acceptance is
    calculated as the fraction of events with $\bbbar\tauhad\tauhad$
    or $\bbbar\taulep\tauhad$ final states passing the signal region
    selections of a given channel.}%
  \label{fig:acceptance_vs_klambda_vs_mhh}
\end{figure}

The signal acceptance is not the only factor determining the
sensitivity of the reinterpretation. Signals with enhanced cross
sections at low \mHH have larger overlap with background
processes. This is a consequence of using multivariate discriminants
trained to distinguish SM \HH events from backgrounds and the
naturally greater abundance of background events with low \mHH.


\section{Results}%
\label{sec:reinterpretation_results}

Upper limits are set on \xsecggfvbf at \SI{95}{\percent} CL by
combining the \hadhad, \lephad SLT, and \lephad LTT channels. The
exclusion limits obtained by the ATLAS collaboration are shown
in~\Cref{fig:klambda_scan} as a function of \klambda. They are
compared to the prediction of the combined non-resonant \HH production
cross section via \ggF and VBF from theory, previously shown
in~\Cref{fig:hh_xsec_incl}. The most stringent limits are set for
$\klambda \approx 2$ which follows from signal acceptance
considerations discussed previously. The theory prediction of the
non-resonant \HH production cross section exceeds the observed upper
limit outside of the interval of \mbox{$\klambda \in [-2.4, 9.2]$},
thus excluding non-resonant \HH production with
\mbox{$\klambda \notin [-2.4, 9.2]$} based on the upper limits on the
cross sections.\footnote{The allowed \klambda interval is not a
  confidence interval with well-defined coverage properties. Such an
  interval would require a more rigorous treatment that includes
  \klambda as a parameter in the statistical model directly.}

\begin{figure}[htbp]
  \centering

  \includegraphics[width=0.58\textwidth]{self_coupling/klam_scan_result}

  \caption{Upper limits on the Higgs boson pair production cross
    section via \ggF and VBF, \xsecggfvbf, for the combination of the
    \hadhad, \lephad SLT, and \lephad LTT channels at
    \SI{95}{\percent} CL as a function of \klambda. The theory
    prediction previously described in \Cref{fig:hh_xsec_incl} is
    overlaid. The expected upper limits assume the absence of
    non-resonant \HH production~($\xsecggfvbf = 0$). The \klambda
    intervals defined by the intersection of the theory prediction
    with the observed/expected limits are given in the lower left of
    the figure. The figure is taken from
    Ref.~\cite{ATLAS-CONF-2021-052}.}%
  \label{fig:klambda_scan}
\end{figure}

The result represents an improvement over earlier searches by the
ATLAS collaboration using \SI{36}{\per\femto\barn} of \pp-collision
data taken at the beginning of Run~2 of the LHC. These yielded allowed
intervals of \mbox{$\klambda \in [-7.4, 15.7]$} and
\mbox{$\klambda \in [-5.0, 12.0]$} for the \bbtautau channel and the
combination of the \bbtautau, \bbbb, and \bbyy channels,
respectively~\cite{HDBS-2018-58}.

A comparison between the allowed \klambda intervals of searches in the
\bbtautau, \bbbb, and \bbyy channels performed at the end of Run~2 by
the ATLAS collaboration is given in \Cref{tab:allowed_klambda}. The
results of searches in the \bbtautau and \bbyy channels are
complementary. While the \bbyy channel sets more stringent upper
bounds on the allowed \klambda interval due to its superior acceptance
of signal events with low \mHH, the \bbtautau channel provides a lower
bound that is competitive with the result of the \bbyy channel.
% At the current stage of searches for non-resonant \HH production
% with anomalous \klambda, the \bbbb channel
\Cref{tab:cms_klambda} in the appendix summarises the results obtained
by the CMS collaboration showing similar findings to those presented
in \Cref{tab:allowed_klambda}.

\begin{table}[htbp]
  \centering

  \caption{Summary of allowed \klambda intervals obtained from
    searches for non-resonant \HH production in \bbtautau, \bbbb, and
    \bbyy channels by the ATLAS collaboration using up to
    \SI{139}{\per\femto\barn} of \pp-collision data taken during Run~2
    of the LHC. The quoted results are obtained under the same
    assumptions discussed in this chapter. The expectation is derived
    from the upper limits using Asimov datasets with
    $\xsecggfvbf = 0$.}%
  \label{tab:allowed_klambda}

  \begin{tabular}{lccc}
    \toprule
    & \multicolumn{3}{c}{Allowed \klambda interval} \\
    \cmidrule{2-4}
    Search channel & Observed & Expected & Reference  \\
    \midrule
    \bbtautau & $[-2.4, \phantom{0}9.2]$ & $[-2.0, \phantom{0}9.0]$ & \cite{ATLAS-CONF-2021-052} \\
    \bbbb     & $[-3.9, 11.1]$           & $[-4.6, 10.8]$           & \cite{ATLAS-CONF-2022-035} \\
    \bbyy     & $[-1.5, \phantom{0}6.7]$ & $[-2.4, \phantom{0}7.7]$ & \cite{HDBS-2018-34} \\
    \bottomrule
  \end{tabular}
\end{table}


\section{Conclusion and Outlook}%
\label{sec:reinterpretation_conclusion}

An interpretation of the search for non-resonant \HH production in the
\bbtautau channel in terms of anomalous Higgs boson self-coupling
strengths was presented. Upper limits at \SI{95}{\percent} CL were set
on the total cross section of Higgs boson pair production via \ggF and
VBF as a function of the self-coupling constant. An allowed range of
$\klambda \in [-2.4, 9.2]$ was obtained by comparing the cross section
exclusion limits with cross section predictions from theory. The
results are based on assumptions that other coupling constants are as
predicted by the SM and neglecting changes in single Higgs boson
production cross sections and Higgs boson branching ratios with
\klambda due to electroweak corrections.

A follow-up analysis was performed by the ATLAS collaboration in
Ref.~\cite{ATL-HDBS-2022-03-002} combining the three channels most
sensitive to non-resonant \HH production, namely the \bbtautau, \bbbb,
and \bbyy channels. In contrast to the method presented in this
chapter, the statistical model is constructed using \klambda as a
parameter of interest, allowing to obtain confidence intervals using
likelihood-ratio tests directly. Moreover, the \klambda-dependency of
single Higgs boson production and the Higgs boson decay was accounted
for. The confidence intervals (\SI{95}{\percent} CL) resulting from
the combination are $\klambda \in [-0.6, 6.6]$ for the observed data
with an expectation of $\klambda \in [-2.1, 7.8]$ under the
$\klambda = 1$ hypothesis~\cite{ATL-HDBS-2022-03-002}. The \klambda
confidence interval based on the \bbtautau search only is
$\klambda \in [-2.7, 9.5]$ at \SI{95}{\percent}
CL~\cite{ATL-HDBS-2022-03-002} showing reasonable agreement with the
result obtained in this chapter that used simplifying assumptions.\todo{Should note that these numbers were read off a figure...}

Probing the nature of the Higgs boson self-coupling is an important
physics goal of the HL-LHC. Prospects of the sensitivity to the
self-coupling constant with \SI{3000}{\per\femto\barn} \pp-collision
data at $\sqrt{s} = \SI{14}{\GeV}$ were derived in
Ref.~\cite{ATL-PHYS-PUB-2021-044} by projecting the results of the SM
\HH search in the \bbtautau channel. This extrapolation yields an
expected confidence interval of $\klambda \in [-0.3, 7.4]$ at
\SI{95}{\percent} CL assuming the SM
hypothesis~\cite{ATL-PHYS-PUB-2021-044}. By combining with projections
in the \bbyy channel~\cite{ATL-PHYS-PUB-2022-001}, first exclusion of
the $\klambda = 0$ hypothesis at \SI{95}{\percent} CL could be
achieved using the dataset collected at the
HL-LHC~\cite{ATL-PHYS-PUB-2022-005}. This would represent a first
milestone in experimentally supporting the Higgs boson
self-interaction predicted by the SM, showing that a model without
Higgs boson self-coupling would be disfavoured.

% Prospects for HL-LHC \SI{95}{\percent} CL\\
% Baseline (bbtautau): [-0.3, 7.4]\\
% Baseline (bbyy): [-0.2, 4.6]\\
% Baseline (bbyy + bbtautau): [0.0, 2.7]

% The total Higgs boson pair production cross section with anomalous
% couplings. Following recommendations by the LHC Higgs Working
% Group~\cite{LHCHWGHH}:
% \begin{description}

% \item[\ggF production mode] The cross sections of \HH production with
%   anomalous self-couplings were calculated in
%   Ref.~\cite{Amoroso:2020lgh} at $\text{NNLO}_{\text{NLO-i}}$
%   (NLO-improved). This prediction is obtained from combining the
%   result with the full top-quark mass dependence at
%   NLO~\cite{Buchalla:2018yce}


%   with NNLO corrections in the $m_{t} \to \infty$
%   limit~\cite{deFlorian:2017qfk}.

%   Additionally, the prediction is rescaled such that it coincides with
%   the SM \HH cross section at $\text{NNLO}_{\text{FTapprox}}$ at
%   $\klambda = 1$.

%   The total \HH production cross section via \ggF was found to depend
%   quadratically on \klambda and is thus parameterised
%   accordingly~\cite{LHCHWGHH}.

% \item[VBF production mode]

% \end{description}

% \todo[inline]{Read YR about Higgs cross section predictions:
%   \url{https://cds.cern.ch/record/2227475/files/CERN-2017-002-M.pdf}}


%%% Local Variables:
%%% mode: latex
%%% TeX-master: "../../phd_thesis"
%%% End:
