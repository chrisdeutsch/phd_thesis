% Motivation paragraph
Theoretical motivation in~\Cref{sec:hh_motivation}.

In the current experimental environment of HEP, direct searches for
non-resonant Higgs boson pair production are the most sensitive probes
of the Higgs boson self-coupling constant, \lambdahhh, due to the
large sensitivity of the total and differential non-resonant \HH
production cross section to anomalous values of \lambdahhh. Hereafter,
the self-coupling constant is given in terms of the modifier
$\klambda = \lambdahhh / \lambdahhh^{\text{SM}}$ relating an assumed
value of the self-coupling constant to the value predicted by the SM.

The inclusive \HH cross section via the \ggF and VBF production modes
is shown in \Cref{fig:hh_xsec_incl} as a function of
\klambda. Non-resonant production of \HH via \ggF constitutes the
dominant contribution throughout the considered \klambda range,
exceeding the production cross section via VBF by at least a factor of
about five (18 for SM \HH production). For the \ggF production mode,
the destructive interference between the box and triangle diagram
becomes maximal for $\klambda \approx 2.3$ at which point the cross
section is at a minimum of about \SI{13}{\femto\barn}. Similar
behaviour is expected for the VBF production mode although involving
different diagrams (cf.\ \Cref{fig:hh_feynmans}) and yielding a cross
section minimum below $\klambda = 2$.

\begin{figure}[htbp]
  \centering

  \includegraphics[width=0.55\textwidth]{self_coupling/hh_xsec}

  \caption{Total \HH production cross section via \ggF and VBF as a
    function of \klambda for $\mH = \SI{125.0}{\GeV}$ in
    \pp-collisions at $\sqrt{s} = \SI{13}{\GeV}$. The production cross
    section via \ggF is given at $\text{NNLO}_{\text{NLO-i}}$ rescaled
    to $\text{NNLO}_{\text{FTapprox}}$ in the $\klambda = 1$
    limit~\cite{Amoroso:2020lgh,Baglio:2020wgt,LHCHWGHH,Grazzini:2018bsd}. The
    production cross section via VBF is obtained from simulation with
    \MGNLO at LO after applying an $\text{N}^3\text{LO}$ $k$-factor
    derived for the SM case~\cite{Dreyer:2018qbw,LHCHWGHH}. The cross
    sections are parameterised as quadratic functions of
    \klambda. Theoretical uncertainties are shown as coloured bands.}%
  \label{fig:hh_xsec_incl}
\end{figure}

In addition to the change in total cross section, anomalous values of
\klambda alter the differential \HH production cross section
(primarily) in terms of the invariant mass of the Higgs boson
pair. This is shown in \Cref{fig:hh_xsec_mhh} for the dominant \ggF
production mode and for five exemplary values of \klambda. The \mHH
spectra for different values of \klambda show large differences in
their \emph{hardness} as measured by the median of the \mHH
distribution in~\Cref{fig:hh_median_mhh}. For \klambda values just
below or at the point of maximum destructive interference between the
box and triangle diagram the \mHH spectra are moderately hard and have
a pronounced double peak structure (shown in
\Cref{fig:hh_xsec_mhh}). For other values of \klambda, particularly
for $\klambda \approx 3$, the cross section at low \mHH is enhanced
leading to softer \mHH spectra.

\begin{figure}[htbp]
  \begin{subfigure}[t]{0.485\textwidth}
    \includegraphics[width=\textwidth]{self_coupling/hh_mhh_vs_klam}
    \subcaption{Differential \HH production cross section with respect
      to \mHH for the \ggF production mode and selected values of
      \klambda. The differential cross sections are normalised by
      dividing by the total cross section. Only statistical
      uncertainties from the finite number of generated events are
      shown.}%
    \label{fig:hh_xsec_mhh}
  \end{subfigure}\hfill%
  \begin{subfigure}[t]{0.485\textwidth}
    \includegraphics[width=\textwidth]{self_coupling/hh_median_mhh_vs_klam}
    \subcaption{Median value of \mHH for Higgs boson pairs produced
      via \ggF as a function of \klambda.}%
    \label{fig:hh_median_mhh}
  \end{subfigure}

  \caption{Differential cross section of Higgs boson pair production
    via \ggF for selected values of \klambda (a) and the corresponding
    median value of \mHH as a function of \klambda (b). Both are shown
    for \pp-collisions at $\sqrt{s} = \SI{13}{\TeV}$ and assuming
    $\mH= \SI{125.0}{\GeV}$. The cross sections are obtained from
    simulation with \POWHEGBOX[v2] at NLO including the top-quark mass
    dependence~\cite{Heinrich:2019bkc,Heinrich:2020ckp} for
    $\klambda = 0, 1, 10$. Differential cross sections for other
    values of \klambda are estimated using morphing techniques derived
    at LO~\cite{ATL-PHYS-PUB-2019-007}.}
\end{figure}

This chapter presents a reinterpretation of the search for SM \HH
production presented in~\Cref{sec:dihiggs} in terms of non-resonant
\HH production with anomalous values of the Higgs boson self-coupling
constant. A reinterpretation of this search yields constraints on the
allowed values of \klambda as the expected number of signal events in
the signal regions is sensitive to the value of the self-coupling
constant. This sensitivity is due to the \klambda dependency of both
the non-resonant \HH production cross section and of the signal
acceptance of the signal region selections which is closely linked to
the characteristic \mHH spectrum for a given \klambda hypothesis. The
results presented in this chapter are based on
Ref.~\cite{ATLAS-CONF-2021-052} published by the ATLAS collaboration.


\section{Reinterpretation of the Search for SM \HH Production}%
\label{sec:reinterpretation}

The reinterpretation of the search for SM \HH production
($\klambda = 1$) in terms of anomalous values of \klambda is performed
based on the statistical model developed in
\Cref{sec:statistical_analysis}. With respect to the SM \HH search
presented previously, only the signal model used for the statistical
interpretation is altered. The discriminants entering the fit and
their binning remain at the configuration optimised for the SM \HH
signal.

The SM \HH signal model used for the statistical interpretation is
replaced by a model of non-resonant \HH production with arbitrary but
fixed \klambda. Both the \ggF and the VBF production mode are
considered. This signal is normalised according to a cross section of
$\sigma_{HH}^{\klambda}$ which is considered as a POI in the
statistical analysis. Upper limits are set on $\sigma_{HH}^{\klambda}$
as a function of \klambda and compare to the theoretical cross section
for a given value of \klambda. Intervals of \klambda are excluded if
the cross section predicted by theory is larger than the
experimentally determined upper limit.

This reinterpretation makes the assumption that the background model
remains unchanged for different assumed values of \klambda. However,
this is not the case since single Higgs boson production is a small
but non-negligible background and both the production cross sections
and the branching ratios of the Higgs boson decay depend on \klambda.

References on H sensitivity to
\klambda~\cite{ATL-PHYS-PUB-2019-009,Degrassi:2016wml,Maltoni:2017ims}.

In the \klambda range (-3 to 12 based on old results) relevant for
this search change in H->bb or H->tautau branching ratio is below
5\%. This is also not considered for the Higgs decay in signal
processes.

Production cross sections for all except ttH below 10\%. ttH can
change up to 15\%.


\subsection{Signal Model}

\Cref{sec:data_and_simulation}

ggF: Re-weighting

VBF: Linear combination


\todo[inline]{Hadhad channel: Acceptance times efficiency vs mHH.}

\todo[inline]{Hadhad channel: Acceptance times efficiency vs kLambda.}

Should say that bbtautau is only good because the SM is already pretty
hard. -> acceptance vs mHH in \hadhad.


\section{Results}%
\label{sec:reinterpretation_results}


\begin{figure}[htbp]
  \centering

  \includegraphics[width=0.6\textwidth]{self_coupling/klam_scan_result}

  \caption{Klambda scan. The figure is taken from
    Ref.~\cite{ATLAS-CONF-2021-052}.}%
  \label{fig:klambda_scan}
\end{figure}


\section{Conclusion and Outlook}%
\label{sec:reinterpretation_conclusion}


The total Higgs boson pair production cross section with anomalous
couplings. Following recommendations by the LHC Higgs Working
Group~\cite{LHCHWGHH}:
\begin{description}

\item[\ggF production mode] The cross sections of \HH production with
  anomalous self-couplings were calculated in
  Ref.~\cite{Amoroso:2020lgh} at $\text{NNLO}_{\text{NLO-i}}$
  (NLO-improved). This prediction is obtained from combining the
  result with the full top-quark mass dependence at
  NLO~\cite{Buchalla:2018yce}


  with NNLO corrections in the $m_{t} \to \infty$
  limit~\cite{deFlorian:2017qfk}.

  Additionally, the prediction is rescaled such that it coincides with
  the SM \HH cross section at $\text{NNLO}_{\text{FTapprox}}$ at
  $\klambda = 1$.

  The total \HH production cross section via \ggF was found to depend
  quadratically on \klambda and is thus parameterised
  accordingly~\cite{LHCHWGHH}.

\item[VBF production mode]

\end{description}


\todo[inline]{Read YR about Higgs cross section predictions:
  \url{https://cds.cern.ch/record/2227475/files/CERN-2017-002-M.pdf}}

Sample combination method:~\cite{ATL-PHYS-PUB-2019-007}


\todo{Updated result from H+HH
  combination~\cite{ATLAS-CONF-2022-050}.}


%%% Local Variables:
%%% mode: latex
%%% TeX-master: "../../phd_thesis"
%%% End:
