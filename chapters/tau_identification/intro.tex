The main goal of the \tauhadvis reconstruction employed at the ATLAS
experiment is to reconstruct \tauhadvis candidates that originate from
hadronic decays of \tauleptons (\truetauhadvis) with high
efficiency. Therefore, objects from sources other than \tauhad are
frequently reconstructed as \tauhadvis candidates
(\faketauhadvis). Tau identification aims to distinguish between
\tauhadvis candidates from \tauhad and other sources.

The primary source of \faketauhadvis are quark- or gluon-initiated
jets due to the purely hadronic and jet-like signature of \tauhadvis
at the ATLAS experiment. Electrons can be another, less abundant,
source of \faketauhadvis which have to be distinguished from
\truetauhadvis using a dedicated algorithm. In the following, the
former classification task is referred to as \tauid while the latter
is referred to as \emph{electron veto}. Consequently, this chapter is
concerned with classifying the origin of \tauhadvis candidates as
either originating from \tauhad or from quark- or gluon-initiated
jets.

A number of features can be exploited to distinguish between
\tauhadvis candidates originating from \tauhad and quark- or
gluon-initiated jets:
\begin{description}

\item[\taulepton mass] The \taulepton has a mass of
  \SI{1.777}{\GeV}~\cite{pdg2020} and is therefore sufficiently
  massive to decay hadronically while still having a small mass
  compared to the energy scales of processes studied at the ATLAS
  experiment.

  The \taulepton mass can be used directly as a feature by considering
  the invariant mass of the reconstructed secondary particles
  associated to a \tauhadvis candidate. Ignoring reconstruction
  effects, the invariant mass of the visible decay products of \tauhad
  is bounded by the \taulepton mass. This is not the case for
  \tauhadvis candidates originating from quark- or gluon-initiated
  jets which do not have a strict bound.

  The other features, described hereafter, are consequences of, or
  closely related to the mass of the \taulepton.

\item[Particle multiplicity] Hadronic decays of \tauleptons produce
  few (visible) daughter particles. Most decays produce one or three
  charged hadrons (most frequently $\pi^{\pm}$) and zero to two
  neutral pions.
  % Ellis, Stirling, Webber: 6.4 Quark and gluon jet differences
  In contrast, the average multiplicity of charged and neutral
  particles inside of jets originating from the fragmentation of
  partons produced in hard scattering interactions is large and
  increases with the momentum of the
  jet~\cite{Ellis:1996mzs,STDM-2015-12}. Therefore, requirements on
  (charged) particle multiplicties are effective at rejecting
  \tauhadvis candidates originating from quark- or gluon-initiated
  jets.\footnote{Gluon-initiated jets have, on average, a larger
    particle multiplicity and a broader angular distribution of
    particles compared to quark-initiated jets due to the larger
    effective colour charge of
    gluons~\cite{Ellis:1996mzs}. Consequently, quark-initiated jets
    are more likely to be reconstructed and mis-identified as
    \tauhadvis candidates.}

\item[Collimated daughter particles] Analyses typically consider
  \tauhadvis candidates with transverse momenta exceeding
  \SI{20}{\GeV}. At these momentum scales the decay products of
  \tauleptons are collimated due to the Lorentz boost of the
  lepton. This leads to the characteristic signature of a narrow jet
  with few visible particles. Requirements on the isolation of
  \tauhadvis candidates allow to reject candidates originating from
  quark- or gluon-initiated jets which have wider angular distribution
  of hadrons.

  % Mean flight path of a p = 20 GeV tau is
  % L = beta * gamma * c * tau = p/m0 * 87mu ~ 1mm
\item[\taulepton lifetime] The \taulepton has a proper lifetime of
  \SI{2.9e-13}{\second}
  ($c \tau = \SI{87}{\micro\metre}$)~\cite{pdg2020} and can thus
  travel for a few millimetres before decaying. The macroscopic
  distance traversed by the \taulepton before decaying produces a
  decay vertex that is displaced from the primary vertex (PV) of the
  hard interaction. For \taulepton decay modes with three charged
  hadrons, this secondary vertex can be reconstructed to determine its
  displacement from the PV. The secondary vertex cannot be
  reconstructed for decay modes with only one charged hadron. However,
  the longitudinal and transverse impact parameters of the track of
  the charged hadron can be used to gauge to incompatibility of the
  track with the PV, thus being sensitive to displaced decays of
  \tauleptons.

  Features related to the lifetime of the \taulepton can be used to
  distingush them from \tauhadvis candidates from quarks- or
  gluon-initiated jets as the hadrons constituting a jet are produced
  promptly at the PV of the interaction.
\end{description}
Prior to the introduction of the method described in this chapter, the
ATLAS collaboration used BDTs as binary classifiers using high-level
discriminating variables, i.e.\ variables purposefully constructed for
the classification task, as inputs.

This chapter introduces a method of performing \tauid using neural
networks that combines the information of high-level discriminating
variables with information from reconstructed charged-particle tracks
and clusters of energy in the calorimeters associated to \tauhadvis
candidates. The number of associated tracks and clusters varies
between candidates requiring the use of neural network architectures
able to process sequences of varying length. This is achieved by
employing a recurrent neural network (RNN) architecture. The method
will be referred to as RNN \tauid hereafter.

The RNN \tauid algorithm was initially proposed in
Ref.~\cite{cdeutsch-master} motivated by a similar approach developed
for track impact parameter based
$b$-tagging~\cite{ATL-PHYS-PUB-2017-003}. The algorithm was
implemented in the reconstruction software of the ATLAS
collaboration~\cite{ATL-SOFT-PUB-2021-001} and the results presented
in this chapter were partially published in
Ref.~\cite{ATL-PHYS-PUB-2019-033}. The RNN \tauid was adopted by the
ATLAS collaboration as the recommended \tauid algorithm for analyses
using the \SI{139}{\per\femto\barn} $pp$-collision dataset recorded
with the ATLAS detector during Run~2 of the LHC.

This chapter is structured as follows. The simulated events used for
the development and performance evaluation of \tauid are introduced in
\Cref{sec:tauid_mc}. The identification method based on RNN is
described in \Cref{sec:tauid_rnn}. Its performance is estimated based
on simulation and compared to the BDT-based approach in
\Cref{sec:tauid_perf}.  \Cref{sec:tauid_conclusion} concludes and
gives an outlook.

%%% Local Variables:
%%% mode: latex
%%% TeX-master: "../../phd_thesis"
%%% End:
