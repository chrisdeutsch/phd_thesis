This chapter describes a novel algorithm used at the ATLAS experiment to
identify the visible decay products of hadronic \taulepton decays. The algorithm
is applied to \tauhadvis candidates passing \tauhadvis reconstruction and aims
to differentiate between candidates originating from \tauhad (\truetauhadvis)
and those originating from non-\tauhad sources (\faketauhadvis). This step is
necessary since \tauhadvis reconstruction is not optimised to reject
\faketauhadvis but rather to correctly reconstruct \truetauhadvis and maintain
high \tauhadvis reconstruction efficiencies.
% More generally, the process of distinguishing between true- and \faketauhadvis
% is called \tauid.

The dominant source of \faketauhadvis at the ATLAS experiment are quark- or
gluon-initiated jets due to their similarity to the hadronic and jet-like
signature of \tauhadvis. Electrons can be another, less abundant source of
\faketauhadvis that have to be distinguished from \truetauhadvis using a
dedicated algorithm (\emph{electron veto}). This chapter is concerned with the
former source of \faketauhadvis and, in particular, with classifying the source
of \tauhadvis candidates as either originating from \tauhad or from quark- or
gluon-initiated jets. This process is referred to as \tauid hereafter.
% More motivation? Say something about reducing mis-identification
% rates to reduce backgrounds?

A number of features can be exploited to differentiate between
\tauhadvis candidates originating from \tauhad and quark- or
gluon-initiated jets:
\begin{description}

\item[\taulepton mass] The \taulepton has a mass of
  \SI{1.777}{\GeV}~\cite{pdg2020} and is therefore sufficiently massive to decay
  hadronically while still having a small mass compared to the energy scales
  typically studied at the ATLAS experiment.

  The \taulepton mass can be used as a feature directly by considering the
  invariant mass of the visible daughter particles of hadronic \taulepton
  decays. Ignoring reconstruction effects, this invariant mass is bounded by the
  mass of the \taulepton. This is not the case for \tauhadvis candidates
  originating from quark- or gluon-initiated jets, which do not have a strict
  upper bound.

  The features described hereafter are consequences of, or closely
  related to the mass of the \taulepton.

\item[Particle multiplicity] Hadronic decays of \tauleptons produce
  few (visible) daughter particles. Most decays produce one or three
  charged hadrons (most frequently $\pi^{\pm}$) and zero to two
  neutral pions.
  % Ellis, Stirling, Webber: 6.4 Quark and gluon jet differences
  In contrast, the average multiplicity of charged and neutral particles in jets
  originating from the fragmentation of partons produced in hard scattering
  interactions is large and increases with the momentum of the
  jet~\cite{Ellis:1996mzs,STDM-2015-12}. Therefore, particle multiplicity
  requirements are effective at rejecting \tauhadvis candidates originating from
  quark- or gluon-initiated jets.\footnote{Gluon-initiated jets have, on
    average, a larger particle multiplicity and a broader angular distribution
    of particles compared to quark-initiated jets due to the larger effective
    colour charge of gluons~\cite{Ellis:1996mzs}. Consequently, quark-initiated
    jets are more likely to be reconstructed and misidentified as \tauhadvis
    candidates.}

\item[Collimated daughter particles] The \tauhadvis candidates typically
  considered by analyses at the ATLAS experiment have transverse momenta
  exceeding \SI{20}{\GeV}. At these momentum scales the decay products of
  \tauleptons are collimated due to the Lorentz boost of the \taulepton. This
  leads to the characteristic detector signature of a narrow jet with few
  visible particles. Requirements on the isolation\footnote{Isolation of a
    reconstructed object refers to a lack of activity in the vicinity of the
    object. The activity is often quantified using reconstructed
    charged-particle tracks or topo-clusters in a cone or annulus surrounding
    the object.} of \tauhadvis candidates allow to reject candidates originating
  from quark- or gluon-initiated jets, which have a wider angular distribution
  of hadrons.

  % Mean flight path of a p = 20 GeV tau is
  % L = beta * gamma * c * tau = p/m0 * 87mu ~ 1mm
\item[\taulepton lifetime] The \taulepton has a proper lifetime of
  \SI{2.9e-13}{\second} ($c \tau = \SI{87}{\micro\metre}$)~\cite{pdg2020} and
  thus typically travels for a few millimetres inside the beampipe before
  decaying. The distance traversed by the \taulepton before its decay results in
  a decay vertex that is displaced from the PV. For \taulepton decay modes with
  three charged hadrons, this secondary vertex can be reconstructed and its
  displacement from the PV determined. For decay modes with only one charged
  hadron, the secondary vertex cannot be reconstructed directly. However, the
  longitudinal and transverse impact parameters of the reconstructed
  charged-hadron track can be used to gauge the incompatibility of the track
  with the PV, thus being sensitive to displaced decays of \tauleptons.

  Features sensitive to the \taulepton lifetime can be used to distinguish
  \tauhad from light-quark- or gluon-initiated jets, in which hadrons are
  produced promptly at the PV. An exception are jets originating from $b$- and
  $c$-quarks. These also contain displaced decays of $b$- or $c$-flavoured
  hadrons. However, the other features remain effective in discerning \tauhad
  from $b$- and $c$-jets.

\end{description}
Prior to the introduction of the method described in this chapter, the
ATLAS collaboration used BDTs as binary classifiers using high-level
discriminating variables, i.e.\ variables purposefully constructed for
the classification task, as inputs.

A method of performing \tauid using neural networks that combines the
information of high-level discriminating variables with information from
reconstructed charged-particle tracks and topo-clusters in the calorimeters is
presented. Tracks and topo-clusters in the vicinity of \tauhadvis candidates and
their associated features are included as inputs to neural networks. Since the
number of tracks and topo-clusters associated to \tauhadvis candidates varies,
an RNN architecture is used that allows to operate on sequences of varying
length. The method is referred to as the RNN \tauid hereafter.

The RNN \tauid algorithm was initially proposed in Ref.~\cite{cdeutsch-master}
motivated by a similar approach developed for track impact parameter based
$b$-tagging~\cite{ATL-PHYS-PUB-2017-003}. The algorithm was implemented in the
reconstruction software of the ATLAS collaboration~\cite{ATL-SOFT-PUB-2021-001}
and some of the results presented in this chapter were published in
Ref.~\cite{ATL-PHYS-PUB-2019-033}.
% The RNN \tauid was adopted by the ATLAS collaboration as the recommended
% \tauid algorithm for analyses using the \SI{139}{\per\femto\barn}
% \pp~collision dataset recorded with the ATLAS detector during Run~2 of the
% LHC.

This chapter is structured as follows: The simulated events used for the
development and performance evaluation of \tauid are introduced in
\Cref{sec:tauid_mc}. The identification method based on RNN is described in
\Cref{sec:tauid_rnn}. Its performance is estimated based on simulation and
compared to the BDT-based approach in \Cref{sec:tauid_perf}.
\Cref{sec:tauid_conclusion} concludes and gives an outlook on possible future
developments.

%%% Local Variables:
%%% mode: latex
%%% TeX-master: "../../phd_thesis"
%%% End:
