\section{Simulated Event Samples}%
\label{sec:tauid_mc}

The tau reconstruction and identification algorithms employed at the
ATLAS experiment for Run~2 of the LHC were developed using simulated
events that provide samples of \tauhadvis candidates. For \tauid,
simulated $\gammastar \to \tautau$ and di-jet events are used to yield
samples of true- and \faketauhadvis, respectively.

% y->tautau:
% https://gitlab.cern.ch/atlas-physics/pmg/infrastructure/mc15joboptions/-/blob/master/share/DSID425xxx/MC15.425200.Pythia8EvtGen_A14NNPDF23LO_Gammatautau_MassWeight.py
An artificial $\gammastar \to \tautau$ event sample was generated
using \PYTHIA[8.212]~\cite{Sjostrand:2014zea} for the matrix element
calculation at leading order (LO), parton showering, hadronisation,
and \taulepton decays. The contribution of the $Z$ boson propagator to
the hard scattering process was removed to provide an unpolarised
sample of \tauleptons. In addition, the cross-section of the process
was modified at generator-level to enhance the number of events with
high invariant $\tautau$ masses to increase the number of \tauhadvis
candidates with high transverse momenta. Finally, both \tauleptons are
enforced to decay hadronically to minimise the statistical
uncertainties from the size of the \truetauhadvis sample.

% Di-jet samples: JZ1W up to JZ6W
% https://gitlab.cern.ch/atlas-physics/pmg/infrastructure/mc15joboptions/-/blob/master/share/DSID361xxx/MC15.361021.Pythia8EvtGen_A14NNPDF23LO_jetjet_JZ1W.py
Di-jet events are generated using
\PYTHIA[8.186]~\cite{Sjostrand:2014zea} for the matrix element
calculation at LO, parton showering, and hadronisation. The generation
is performed in slices of \pT of the leading jet (anti-\kt with
$R = 0.6$) constructed from generator-level particles. Slices with
large jet transverse momenta are oversampled to increase the number of
events with jets (\faketauhadvis) of large transverse momentum.

The $\gammastar \to \tautau$ and di-jet samples use the A14 set of
tuned parameters for \PYTHIA[8]~\cite{ATL-PHYS-PUB-2014-021} and the
\NNPDF[2.3lo]~\cite{Ball:2012cx} set of parton distribution functions.
Decays of hadrons containing $b$- or $c$-quarks are simulated using
\EVTGEN[v1.2.0]~\cite{Lange:2001uf}. The contamination of the hard
scattering interaction with soft, inelastic proton--proton collision
events (pile-up) is accounted for by overlaying the event with
additional minimum-bias events. The response of the ATLAS detector is
simulated for all events~\cite{SOFT-2010-01} which are then
reconstructed using the \textsc{Athena} software
suite~\cite{ATL-SOFT-PUB-2021-001}.


\subsection{\tauhadvis Candidate Selection}
\label{sec:tauid_candidate_selection}

The simulated events are used to construct large samples of \tauhadvis
candidates for the development \tauid algorithms and performance
evaluation. Only \tauhadvis candidates passing the reconstruction are
considered and the following selections are applied:
\begin{itemize}

\item The number of charged particle tracks associated to the
  \tauhadvis candidate, \Ntracks, is either 1 or 3. Candidates with
  one or three associated tracks are referred to as 1- and 3-prong
  \tauhadvis candidates, respectively.

\item The (visible) transverse momentum of the candidate needs to
  fulfil $\pT > \SI{20}{\GeV}$.

\item The \tauhadvis candidate needs to be within $|\eta| < 2.5$ but
  outside of $1.37 < |\eta| < 1.52$, which is the transition region
  between between the barrel and endcap electromagnetic calorimeters.

\end{itemize}
In addition, reconstructed \tauhadvis candidates from
$\gammastar \to \tautau$ events are required to be geometrically
matched to a \tauhad at generator-level within a cone of
$\Delta R < 0.2$.
% And the reco cuts are required to be fulfilled at truth-level.
All efficiencies and rejection rates given in the remainder of this
chapter do not include the effect of these selections.

The $\gammastar \to \tautau$ and di-jet events provide samples of
true- and \faketauhadvis with a size of 20 million and 46 million
candidates, respectively. The distributions of \tauhadvis candidate
\pT is shown for both samples and separately for 1- and 3-prong
candidates in~\Cref{fig:tauid_candidate_pt}. The difference in \pT
spectra between 1- and 3-prong \truetauhadvis in
\Cref{fig:tauid_candidate_pt_gammastar} result from a reduction in
track association efficiency for 3-prong \tauhadvis candidates with
increasing candidate \pT due to the decrease in angular separation of
charged hadrons from the \taulepton decay. In contrast, the \pT
spectrum of 3-prong \tauhadvis candidates from di-jet events, depicted
in \Cref{fig:tauid_candidate_pt_dijet}, shows a comparatively heavier
tail towards large \pT due to the increase in the average charged
particle multiplicity in jets with increasing jet \pT. For the
development and performance evaluation of \tauid algorithms, the
sample of \faketauhadvis candidates is re-weighted, separately for 1-
and 3-prong candidates, to match the \pT spectrum of \truetauhadvis
from $\gammastar \to \tautau$.

\begin{figure}[htbp]
  \begin{subfigure}{0.498\textwidth}
    \includegraphics[width=\textwidth]{tauid/pubnote/taupt_gammastar}
    \subcaption{True-\tauhadvis from $\gammastar \to \tautau$ events}%
    \label{fig:tauid_candidate_pt_gammastar}
  \end{subfigure}\hfill%
  \begin{subfigure}{0.498\textwidth}
    \includegraphics[width=\textwidth]{tauid/pubnote/taupt_dijet}
    \subcaption{Fake-\tauhadvis from di-jet events}%
    \label{fig:tauid_candidate_pt_dijet}
  \end{subfigure}

  \caption{Transverse momentum of 1- and 3-prong \tauhadvis candidates
    in $\gammastar \to \tautau$ (a) and di-jet events (b). Statistical
    uncertainties are shown as coloured bands surrounding the central
    value. Figures taken from Ref.~\cite{ATL-PHYS-PUB-2019-033}.}%
  \label{fig:tauid_candidate_pt}
\end{figure}


%%% Local Variables:
%%% mode: latex
%%% TeX-master: "../../phd_thesis"
%%% End:
