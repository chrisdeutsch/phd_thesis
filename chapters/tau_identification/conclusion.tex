\section{Conclusion and Outlook}%
\label{sec:tauid_conclusion}

This chapter introduced a \tauid algorithm that exploits the discrimination
power of high-level input variables in conjunction with low-level detector
signatures of charged and neutral particles in the vicinity of \tauhadvis
candidates. The main feature of the algorithm is the use of an RNN architecture,
which allows it to operate on sequences of charged-particle tracks and clusters
of energy in the calorimeters.

The new algorithm improves the rejection of \faketauhadvis reconstructed as 1-
and 3-prong \tauhadvis candidates by about \SI{80}{\percent} and
\SIrange[range-units=single]{40}{80}{\percent}, respectively, compared to the
BDT-based method previously employed at the ATLAS experiment. For \tauhadvis
candidates with large transverse momenta ($\pT > \SI{70}{\GeV}$) the improvement
in \faketauhadvisC rejection exceeds \SI{100}{\percent}. This improvement is
mostly driven by the inclusion of charged-particle tracks in the network to
exploit the track-based isolation of \tauhadvis.

The method is considered by the ATLAS collaboration as the recommended \tauid
algorithm for analyses of the \SI{139}{\per\femto\barn} \pp~collision dataset
recorded during Run~2 of the LHC. In addition, the approach was adopted for
\tauid at the high-level trigger in 2018~\cite{ATL-DAQ-PUB-2019-001}. The
superior background rejection of the RNN \tauid allowed for an increase in the
efficiency of \tauhadvis-triggers while remaining within the allocated trigger
bandwidth~\cite{ATL-DAQ-PUB-2019-001}.

The improvements in \tauid, both at the high-level trigger and during offline
event reconstruction, are utilised in~\Cref{sec:dihiggs} for the search for
Higgs boson pair production. Due to the improved background rejection of the
RNN, it is possible to loosen the identification requirements for \tauhadvis in
the search without incurring large increases in backgrounds from the
misidentification of quark- or gluon-initiated jets as \tauhadvis.

The \tauid algorithms employed by the CMS collaboration underwent a similar
evolution as the one presented in this chapter. A BDT-based discriminant based
on high-level features~\cite{CMS-TAU-16-003} was replaced by a deep neural
network that combines high-level information with information from reconstructed
particles in the vicinity of the \tauhadvis~\cite{CMS-TAU-20-001}. Similarly,
the CMS collaboration observed a large reduction in the \faketauhadvisC
misidentification efficiency by factors exceeding 1.8 compared to the previous
method~\cite{CMS-TAU-20-001}, which only relied on high-level features. While
the concept is similar to the approach presented here, the technical
implementation adopted by the CMS collaboration differs and is documented in
Ref.~\cite{CMS-TAU-20-001}.
% exploiting particle-level information in the vicinity of \tauhadvis
% candidates using convolutional layers instead.

% https://atlas-glance.cern.ch/atlas/analysis/analyses/details.php?id=7262
% https://atlas-glance.cern.ch/atlas/analysis/pubnotes/details.php?id=6503
The RNN \tauid continues to be used by the ATLAS experiment at the beginning of
Run~3 of the LHC for identification at the high-level trigger and for offline
event reconstruction~\cite{ATL-PHYS-PUB-2022-044}. In the long term, changes to
the \tauid strategy are expected. A new RNN-based \tauhadvis track
classification algorithm~\cite{Maerker:2021hro} was deployed for \tauhadvis
reconstruction in Run~3~\cite{ATL-PHYS-PUB-2022-044}. This algorithm could be
integrated into or combined with the \tauid procedure to provide isolation
information directly. Such an approach is motivated by the large overlap in
track observables used by both algorithms and the inherent connection between
\tauhadvis track selection and \tauid. Second, other types of networks operating
on variable size collections of inputs are explored by the collaboration. For
\tauid, \deepsets~\cite{NIPS2017_f22e4747} are an alternative to the RNN-based
approach. \deepsets operate on unordered sets of inputs and therefore, unlike
RNN, do not require a specific ordering of the inputs. Preliminary results show
that \tauid based on \deepsets yields results that are competitive with the
implementation using RNN while requiring significantly less time for training
and prediction.\footnote{Similar findings were obtained for \deepsets-based
  $b$-tagging algorithms in Ref.~\cite{ATL-PHYS-PUB-2020-014}.}

% Write something about the reduction in time between iterations
% allowing for more thorough optimisation of input variables, and
% other hyperparameters of the algorithm.



%%% Local Variables:
%%% mode: latex
%%% TeX-master: "../../phd_thesis"
%%% End:
