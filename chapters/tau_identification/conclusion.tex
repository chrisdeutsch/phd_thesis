\section{Conclusion and Outlook}%
\label{sec:tauid_conclusion}

This chapter introduced the a novel \tauid algorithm exploiting the
discrimination power of high-level input variables in conjuction with
low-level detector signatures of charged and neutral particles in the
vicinity of \tauhadvis candidates. A main feature of the algorithm is
the use of a recurrent neural network architecture that allows to
operate on sequences of charged-particle tracks and clusters of energy
in the calorimeters.

The RNN \tauid improves the rejection of 1- and 3-prong \faketauhadvis
by about \SI{80}{\percent} and \SIrange{40}{80}{\percent},
respectively, compared to the method previously employed at the ATLAS
experiment. For \tauhadvis candidates with large transverse momenta
($\pT > \SI{70}{\GeV}$) the improvement in \faketauhadvis rejection
exceeds \SI{100}{\percent}. This improvement is largely driven by the
inclusion of charged-particle tracks.

The \tauid algorithm introduced in this chapter is considered by the
ATLAS collaboration as the recommended algorithm for analyses using
the \SI{139}{\per\femto\barn} $pp$-collision dataset recorded during
Run~2 of the LHC. In addition, the algorithm was adopted by the
collaboration for \tauid at the high-level trigger in
2018~\cite{ATL-DAQ-PUB-2019-001}. The superior background rejection of
the RNN \tauid allowed to improve the efficiency of \tauhadvis
triggers while staying within the allocated trigger
bandwidth~\cite{ATL-DAQ-PUB-2019-001}.

The improvements in \tauid, both at the high-level trigger and during
offline event reconstruction, are utilised in~\Cref{sec:dihiggs} for
the search for Higgs boson pair production. The improved background
rejection of the RNN allows to relax the identification requirements
applied to \tauhadvis and thus increase the acceptance of signal
events without incurring large increases in backgrounds from
misidentified \tauhadvis.

The \tauid algorithms employed by the CMS collaboration underwent a
similar evolution as the one presented in this chapter. A BDT-based
discriminant based on high-level features~\cite{CMS-TAU-16-003} was
replaced by a deep neural network that combines high-level information
with information from reconstructed particles in the vicinity of the
\tauhadvis~\cite{CMS-TAU-20-001}. Similarly, the CMS collaboration
observed a large reduction in the \faketauhadvis misidentification
efficiency by factors exceeding 1.8~\cite{CMS-TAU-20-001} compared to
the previous method only relying on high-level features. While the
concept is similar to the approach presented here, the technical
implementation adopted by the CMS collaboration differs and is
documented in Ref.~\cite{CMS-TAU-20-001}.
% exploiting particle-level information in the vicinity of \tauhadvis
% candidates using convolutional layers instead.


- Tau track classification~\cite{Maerker:2021hro}.

- In the future: To be replaced by ``deep sets'' (permutation
invariance). Based on the same idea and same expected performance but
significantly improved training and prediction time.

\cite{NIPS2017_f22e4747}


%%% Local Variables:
%%% mode: latex
%%% TeX-master: "../../phd_thesis"
%%% End:
