\section{Conclusion and Outlook}%
\label{sec:tauid_conclusion}

- Novel \tauid algorithm used in ATLAS

- Special feature is the recurrent architecture that allow sequences
of objects.

- Majority of the improvement comes from the detailed tracking
information (calorimeter clusters improve the classification
performance comparatively little).

- Based on learning on sequences

- Large improvements allow to choose looser working points with fairly
minimal (if at all) increases in background rejection.

- Trigger improvements \cite{ATL-DAQ-PUB-2019-001}

- Tau track classification~\cite{Maerker:2021hro}.

- Both the improved triggers and the offline ID are used in chapter
X. Allowing to significantly improve the selection efficiency of
signal events.


A similar transition in \tauid algorithms was performed by the CMS
collaboration. A BDT-based discriminant based on high-level
features~\cite{CMS-TAU-16-003} was replaced by a deep neural network
that combines high-level information with information from all
detected particles in the vicinity of the \tauhadvis
axis~\cite{CMS-TAU-20-001}. Similarly, the CMS collaboration observed
a large reduction in the \faketauhadvis misidentification efficiency
by factors exceeding 1.8~\cite{CMS-TAU-20-001} compared to the
previous approach.  While the concept is similar to the approach
presented here, the technical implementation adopted by the CMS
collaboration differs by exploiting particle-level information in the
vicinity of \tauhadvis candidates using convolutional layers instead.

- In the future: To be replaced by ``deep sets'' (permutation
invariance). Based on the same idea and same expected performance but
significantly improved training and prediction time.

\cite{NIPS2017_f22e4747}


%%% Local Variables:
%%% mode: latex
%%% TeX-master: "../../phd_thesis"
%%% End:
