The turn of the 20\textsuperscript{th} century marked the origin of modern
physics, ushering in a century of remarkable advancements in fundamental
physics. Key to this success was the development of the \emph{theory of special
  relativity} and the rise of \emph{quantum mechanics}. On the one hand, the
theory of special relativity
% , proposed by Einstein in
% 1905~\cite{einstein:1905zedbk,einstein:1905idtekvsea},
%
allowed describing the mechanics of physical systems with energies much larger
than the energy of the system at rest. On the other hand, the introduction of
quantum mechanics was instrumental in describing phenomena occurring at length
scales of atomic and sub-atomic physics. The field of experimental high-energy
physics (HEP) is situated at the intersection of special relativity and quantum
mechanics, probing the nature of elementary particles and their interactions at
the largest, experimentally accessible energy scales.

% Special relativity and quantum mechanics are both fundamental theories that
% play important roles in the study of elementary particles. Special relativity
% is necessary to understand the behavior of particles that travel at high
% speeds, such as those found in particle accelerators. Quantum mechanics is
% needed to describe the behavior of particles on a very small scale, such as
% subatomic particles. Together, these theories provide a theoretical framework
% for understanding the properties and interactions of elementary particles.


The \emph{Standard Model} (SM) is the theory at the centre of particle physics
and combines special relativity and quantum mechanics in a paradigm known as a
\emph{relativistic quantum field theory}. It describes the currently known
elementary particles and three (out of four) fundamental forces of nature. The
formulation of the SM traces back to the 1970s, when the theory of the strong
interaction~\cite{Fritzsch:1973pi,Gross:1973id,Politzer:1973fx} and the
Glashow--Salam--Weinberg
model~\cite{Glashow:1961tr,Salam:1964ry,Weinberg:1967tq} of the electroweak
interaction were consolidated in a single theory. Since then, the SM had
tremendous success in describing the properties of elementary particles and
their interactions, and notably in predicting the existence of several
elementary particles before their experimental discovery. Among the predicted
particles is the \emph{Higgs boson} ($H$), which was discovered in 2012 by the
ATLAS and CMS experiments~\cite{HIGG-2012-27,CMS-HIG-12-028} at the Large Hadron
Collider (LHC). The Higgs boson arises as part of the Brout--Englert--Higgs
(BEH) mechanism~\cite{Englert:1964et,Higgs:1964pj} that is employed in the SM to
explain how elementary particles acquire mass in a process referred to
\emph{electroweak symmetry breaking} (EWSB).


In the SM, the Higgs boson has the ability to interact with itself---a property
referred to as the \emph{Higgs boson self-coupling}. The strength of the
self-coupling is determined by the shape of the Higgs potential, a central
element of the BEH mechanism responsible for EWSB.
% This self-coupling is a result of the BEH mechanism, and its strength is
% directly related to the shape of the Higgs potential. The choice of Higgs
% potential is central to the BEH mechanism in the SM as it induces the process
% of EWSB that is instrumental in generating the masses of elementary particles.
Therefore, direct measurements of the Higgs boson self-coupling are instrumental
to test our understanding of EWSB in the SM. These measurements can be performed
using high-energy particle collision events, such as the ones occurring at the
LHC, that result in the production of pairs of Higgs bosons. Higgs boson pair
production is an exceedingly rare process, for which experimental evidence has
yet to be established.


In the presence of new physics, the phenomenology of Higgs boson pair production
might differ considerably from the SM. While the SM is a very successful theory,
it is known to be incomplete as it leaves a number of phenomena
(e.g.~gravitation, dark matter, etc.) unexplained. Eventually, these
shortcomings have to be addressed, requiring a theory that goes beyond the SM
(BSM). Some BSM theories predict Higgs boson pair production via new massive
resonances, or deviations of the (effective) Higgs boson self-coupling strength
from the SM expectation. The presence of such signatures would be indicative of
new physics, making searches for Higgs boson pair production an important probe
of the BSM sector.


% The SM is an effective theory in many regards, however, it is known to be
% incomplete as it leaves a number of phenomena (e.g.\ gravitation, dark matter,
% etc.) unexplained. Eventually, these shortcomings have to be addressed,
% requiring a theory that goes beyond the SM (BSM). The phenomenology of Higgs
% boson pair production in BSM theories might differ significantly from the SM.
% For example, Higgs boson pair production could occur via new massive
% resonances, or BSM physics could alter the (effective) strength of the Higgs
% boson self-coupling. The presence of such signatures would be indiciative of
% new physics, thus making searches for Higgs boson pair production an important
% probe of the BSM sector.


This thesis presents a search for Higgs boson pair production in final states
with two $b$-quarks and two \tauleptons. More specifically, final states with
two hadronic \taulepton decays ($\bbbar\hadhad$) or one hadronic and one
leptonic \taulepton decay ($\bbbar\lephad$) are considered.\footnote{Decays of
  the form $\tau^- \to \text{hadrons} + \nu_\tau$ are referred to as a hadronic
  \taulepton decays (\tauhad), and $\tau^- \to \ell^- \bar{\nu}_\ell \nu_\tau$
  for $\ell = e$ or $\mu$ as leptonic \taulepton decays (\taulep). The charge
  conjugate modes are named analogously.} The search is performed using data of
proton--proton (\pp) collision events at a centre-of-mass energy of
\SI{13}{\TeV} recorded by the ATLAS detector in 2015--2018. The size of the
analysed dataset amounts to an integrated luminosity of
\SI{139}{\per\femto\barn}. Three different modes of Higgs boson pair production
are considered: non-resonant Higgs boson pair production as predicted by the SM
(SM \HH production); production of Higgs boson pairs via new, scalar resonances
with masses of up to \SI{1.6}{\TeV} (resonant \HH production); and non-resonant
Higgs boson pair production with anomalous values of the Higgs boson
self-coupling constant.

The structure of this thesis is as follows: \Cref{sec:standard_model} lays the
foundation for searches for Higgs boson pair production. The SM is summarised,
and the role of EWSB and the Higgs boson is illustrated. Moreover, the
phenomenology of Higgs boson and Higgs boson pair production is discussed in the
context of the SM and BSM theories. The chapter concludes with a summary of
experimental results on searches for Higgs boson pair production prior to the
work presented in this thesis.  \Cref{sec:atlas_and_lhc} describes the LHC, the
ATLAS experiment, and the techniques used to reconstruct particle collision
events with the ATLAS detector. \Cref{sec:experimental_methods} describes the
statistical framework used to interpret the collision events recorded by the
detector.

The bulk of this thesis is concerned with the search for Higgs boson pair
production in the $\bbbar\tautau$ final state. Particular focus lies on the
channel with two hadronic \taulepton decays, referred to as the \hadhad channel.
A crucial element of searches in this final state is the ability to reconstruct
and identify hadronic \taulepton decays based on their signature in the ATLAS
detector. \emph{\Tauid} refers to the process of identifying hadronic \taulepton
decays, aiming to distinguish detector signatures originating from \tauhad from
those of non-\tauhad sources such as quark- or gluon-initiated jets. A novel
method of performing \tauid using \emph{recurrent neural networks} is introduced
in \Cref{sec:tauid} that exploits low-level tracking and calorimeter
information, as well as discriminating variables purposefully constructed for
\tauid. This method has since become the default \tauid algorithm at the ATLAS
experiment for analyses of data recorded during Run~2 (2015--2018) and the
beginning of Run~3 (2022--) of the
LHC~\cite{ATL-PHYS-PUB-2019-033,ATL-PHYS-PUB-2022-044}. It is used in
\Cref{sec:dihiggs} for the search for SM~\HH production and resonant \HH
production in BSM models with an additional CP-even scalar particle. Moreover,
the results of the search for SM~\HH production is reinterpreted in
\Cref{sec:higgs_self_coupling} to set constraints on models predicting anomalous
values of the Higgs boson self-coupling constant. The thesis is
concluded with a summary in \Cref{sec:conclusion}.


%%% Local Variables:
%%% mode: latex
%%% TeX-master: "../phd_thesis"
%%% End:
