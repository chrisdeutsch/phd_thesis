% 1954: Yang & Mills -> Extended gauge theory to non-abelian groups (QCD)
%
% 1961: Glashow: Electroweak unification -> predict Z boson based on 'partial'-symmetries
%       \cite{Glashow:1961tr}
%
% 1964: "Broken Symmetry and the Mass of Gauge Vector Mesons"
%       Englert/Brout \cite{Englert:1964et}
%
% 1964: "Broken Symmetries and the Masses of Gauge Bosons"
%       Higgs \cite{Higgs:1964pj}
% -> Spontaneous symmetry breaking to have massive gauge bosons aquire their mass
%
% 1967: Weinberg & Salam: Higgs mechanism included into the electroweak theory
%       Weinberg \cite{Weinberg:1967tq}
%
% Birth of the 'Standard Model' as it is today.

% Why study particle physics
% - Study of natural phenomena / what laws govern the universe?
% - Study matter and radiation, and its dynamics across space and time
% - Where do we come from where do we go?
% - Matter and radiation
% - Particle collider experiments reproduce the conditions of the universe


% Modern particle physics born in the 20th century: The Standard Model
% - First collider experiments
% - In its current form developed in the 1960s
% - Modern particle physics -> QFT

% The Higgs boson

% Contents of the thesis

The turn of the 20\textsuperscript{th} century marked the origin of modern
physics, ushering in a century of remarkable advancements in our understanding
of fundamental physics. Key to this success was the development of the
\emph{theory of special relativity} and the rise of \emph{quantum mechanics}. On
the one hand, the theory of special relativity, proposed by Einstein in
1905~\cite{einstein:1905zedbk,einstein:1905idtekvsea}, allowed to describe
physical systems with energies much larger than the energy of the system at
rest. On the other hand, the introduction of quantum mechanics was instrumental
in describing phenomena occurring at length scales of atomic and sub-atomic
physics. The field of (high-energy) particle physics is situated at the
intersection of special relativity and quantum mechanics, probing the nature of
elementary particles and their interactions at large energy scales.
% Nowadays, a cornerstone of the field are particle collider experiments (energy
% frontier).

The \emph{Standard Model} (SM) is at the centre of particle physics, combining
special relativity and quantum mechanics in a paradigm known as a
\emph{relativistic quantum field theory}. It describes the currently known
elementary particles and three (out of four) fundamental forces of nature.  The
formulation of the SM traces back to the 1970s, consolidating the theory of the
strong interaction\todo{Citations?} and the Glashow--Salam--Weinberg (GSW) model
of the electroweak
interaction~\cite{Glashow:1961tr,Salam:1964ry,Weinberg:1967tq} in a single
theory.

SM is great and no fundamental changes have been made since its inception.

A crucial ingredient to the electroweak theory was the Brout--Englert--Higgs
mechanism


A crucial ingredient


no fundamental changes since

was formulated in the 1970s

conceptually the same since then.



The SM describes the known
elementary particles and three of the four fundamental forces of nature.





It describes the elementary particles and three of the four fundamental
interactions







Nowadays,

1970s

SM

- describes the elementary particles and three of the four fundamental forces of nature

- Combines the theory of the strong interaction with the GSW model of the

- Key ingredient was the BEH mechanism explaining how elementary particle acquire their mass



the elementary particles and three out of



their interactions are described by the
\emph{Standard Model} (SM) of particle physics.






success story: SM



Brout--Englert--Higgs mechanism~\cite{Englert:1964et,Higgs:1964pj} incorporated
into the


Unifies the Glashow--Salam--Weinberg model of the electroweak
interaction~ with the theory
of the strong interaction.


% For millenia humankind has attempted to understand the universe and the laws
% that govern it. The nature of the universe was already debated by the ancient
% philosophers who eventually postulated that all matter consists of fundamental
% building blocks referred to as \emph{atoms}, derived from the Greek átomos
% meaning indivisible. This concept was adopted in the 19\textsuperscript{th}
% century to develop the atomic theory leading to the concept known today as
% \emph{the atom}. It turned out, however, that this term is a misnomer when the
% atom was found to be composed of electrons and a massive nucleus. With the
% ability to probe increasingly smaller length scales in scattering experiments of
% the 20\textsuperscript{th} century, it was found that the nucleus itself is
% composed of nucleons, which in turn consist of bound systems of quarks.

The Standard Model of particle physics

Unification of the electromagnetic and weak interaction by Glashow, Salam, and
Weinberg~\cite{Glashow:1961tr,Salam:1964ry,Weinberg:1967tq}. The Standard Model
is built on the assumption of a fundamental symmetry, referred to as local gauge
invariance, which

A point of tension was the non-vanishing mass of the gauge bosons of the weak
interaction which would violate the fundamental symmetries

acquire mass through spontaneous symmetry breaking in a mechanism now known as
the Higgs mechanism.




In 2012, first experimental evidence of a new boson with a mass of about
\SI{125}{\GeV} was established by the ATLAS~\cite{HIGG-2012-27} and
CMS~\cite{CMS-HIG-12-028} collaborations at the Large Hadron Collider of CERN.

theory of electroweak unification by Glashow, Salam, and Weinberg



Our current understanding of physics at the subatomic scale originates from the
20\textsuperscript{th} century. Quarks and leptons are now viewed as the
fundamental building blocks of matter, their interactions being mediated by the
exchange of particles referred to as \emph{gauge bosons}.

The Standard Model of particle physics, formulated in the 1970s, provides the
framework to describe the elementary particles.

provides the framework to describe the
elementary particles and their interactions.

It is one of the most scutinised theories

The mathematical framework for the


With the advent of quantum theory and the theory of relativity in the
20\textsuperscript{th}.



The studies continue to this day

Current model: Big bang, general relativity, Standard Model (QFT)
-> indications that it is not complete exist

In the current model of cosmology believed to reproduce the conditions fractions
of a second after the Big Bang.

% Big bang

The Standard Model of particle physics described in its current form in the
1960s by Glashow, Salam, and Weinberg.

With the completion of the Standard Model of particle physics as it is known
today in the 1960s

The Higgs field and the corresponding excitation, the Higgs boson, postulated in
the 1960s

\cite{Englert:1964et,Higgs:1964pj}


Postulated in the 1960s as part of the Higgs mechanism allowing for massive
gauge bosons while maintaining the fundamental symmetries the Standard Model of
physics is built upon.


-> Mass terms for W/Z first break local gauge invariance and second lead to
infinities that cannot be renormalised.

-> Mass generation via spontaneous symmetry breaking


. After searches spanning multiple decades at the Large Electron
Positron Collider and the Tevatron,

After a search spanning multiple decades, the Higgs boson was discovered in 2012
by the ATLAS and CMS experiments at the Large Hadron Collider

later awarded the Nobel Prize






Two parameters describe the Higgs potential: $m_{H}$ and $v$ (or $\mu$ and
$\lambda$)

After decades of searches at tevatron, lep, etc....

Discovery of a boson \cite{HIGG-2012-27,CMS-HIG-12-028} in 2012

Higgs boson discovery



The standard model one of the most precisely tested theories...

Stood up to scrutiny, thus far fulfilling all the predictions made by the SM.

Quantum numbers, Mass measurement

What is not yet tested? (While predicted by the SM)
- Higgs boson coupling to electrons/muons
- Higgs boson coupling to charm- and lighter quarks
- Trilinear self-coupling


Couplings: Z, W, top, bottom, tau, muon (evidence)

\clearpage

This thesis is structured as follows: \Cref{sec:standard_model} lays the
foundation for searches for Higgs boson pair production. The SM is summarised,
and the role of electroweak symmetry breaking and the Higgs boson in the SM is
illustrated. Moreover, the phenomenology of the Higgs boson and Higgs boson pair
production is discussed in the context of the SM and theories beyond the SM. The
chapter concludes with a summary of experimental results on searches for Higgs
boson pair production prior to the work presented in this thesis.
\Cref{sec:atlas_and_lhc} describes the Large Hadron Collider, the ATLAS
experiment, and the techniques used to reconstruct particle collision events
with the ATLAS detector. The collision events recorded by the ATLAS detector are
analysed using a suite of statistical methods that are introduced in
\Cref{sec:experimental_methods}.

The main part of this thesis is concerned with searches for Higgs boson pair
production in final states with two $b$-quarks and two hadronically decaying
\tauleptons ($\tau^{-} \to \text{hadrons} + \nu_\tau$). A crucial element of
these searches is the ability to reconstruct and identify hadronic \taulepton
decays.\todo{Why?} A novel method of performing \tauid using neural networks is
introduced in \Cref{sec:tauid}. This method is subsequently used in
\Cref{sec:dihiggs} to perform searches for resonant and non-resonant Higgs boson
pair production using \SI{139}{\per\femto\barn} of proton--proton collision data
collected at the ATLAS experiment during Run~2 of the Large Hadron Collider. In
\Cref{sec:higgs_self_coupling} the results of \Cref{sec:dihiggs} are
re-interpreted to set constraints on the possible values of the strength of the
Higgs boson self-coupling. The thesis is concluded in \Cref{sec:conclusion}.

\clearpage

\todo[inline]{This thesis presents results published as part of (list
  of publications)...}


Notes from chat with Jochen:
\begin{itemize}

\item Keep it short \ra about 120 pages
  \begin{itemize}
  \item SM about 15 pages and focus on \HH production

  \item LHC and ATLAS experiment about 10 pages: For every detector
    component a picture and a paragraph

  \item Experimental methods: 10 pages

  \item Try to keep it way below 160 pages???
  \end{itemize}

\item Cut out or move less interesting things to the backup

\item Make it more interesting! Cut out repetition...

\end{itemize}


%%% Local Variables:
%%% mode: latex
%%% TeX-master: "../phd_thesis"
%%% End:
