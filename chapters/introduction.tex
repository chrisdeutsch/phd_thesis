% 1954: Yang & Mills -> Extended gauge theory to non-abelian groups (QCD)
%
% 1961: Glashow: Electroweak unification -> predict Z boson based on 'partial'-symmetries
%       \cite{Glashow:1961tr}
%
% 1964: "Broken Symmetry and the Mass of Gauge Vector Mesons"
%       Englert/Brout \cite{Englert:1964et}
%
% 1964: "Broken Symmetries and the Masses of Gauge Bosons"
%       Higgs \cite{Higgs:1964pj}
% -> Spontaneous symmetry breaking to have massive gauge bosons aquire their mass
%
% 1967: Weinberg & Salam: Higgs mechanism included into the electroweak theory
%       Weinberg \cite{Weinberg:1967tq}
%
% Birth of the 'Standard Model' as it is today.

% Why study particle physics
% - Study of natural phenomena / what laws govern the universe?
% - Study matter and radiation, and its dynamics across space and time
% - Where do we come from where do we go?
% - Matter and radiation
% - Particle collider experiments reproduce the conditions of the universe


% Modern particle physics born in the 20th century: The Standard Model
% - First collider experiments
% - In its current form developed in the 1960s
% - Modern particle physics -> QFT

The turn of the 20\textsuperscript{th} century marked the origin of modern
physics, ushering in a century of remarkable advancements in our understanding
of fundamental physics. Key to this success was the development of the
\emph{theory of special relativity} and the rise of \emph{quantum mechanics}. On
the one hand, the theory of special relativity, proposed by Einstein in
1905~\cite{einstein:1905zedbk,einstein:1905idtekvsea}, allowed to describe
physical systems with energies much larger than the energy of the system at
rest. On the other hand, the introduction of quantum mechanics was instrumental
in describing phenomena occurring at length scales of atomic and sub-atomic
physics. The field of (high-energy) particle physics is situated at the
intersection of special relativity and quantum mechanics, probing the nature of
elementary particles and their interactions at large energy scales.
% Nowadays, a cornerstone of the field are particle collider experiments (energy
% frontier).

The \emph{Standard Model} (SM) is the theory at the centre of particle physics,
combining special relativity and quantum mechanics in a paradigm known as a
\emph{relativistic quantum field theory}. It describes the currently known
elementary particles and three (out of four) fundamental forces of nature. The
formulation of the SM traces back to the 1970s, when the theory of the strong
interaction~\cite{Fritzsch:1973pi,Gross:1973id,Politzer:1973fx}\todo{Check me!}
and the Glashow--Salam--Weinberg (GSW)
model~\cite{Glashow:1961tr,Salam:1964ry,Weinberg:1967tq} of the electroweak
interaction were consolidated in a single theory. Since then, the SM had
tremendous success in describing the properties of elementary particles and
their interactions, and notably in predicting the existence of several
elementary particles before their experimental discovery. Among the predicted
particles is the \emph{Higgs boson}, which was discovered in 2012 by the ATLAS
and CMS experiments~\cite{HIGG-2012-27,CMS-HIG-12-028} at the Large Hadron
Collider (LHC). The Higgs boson arises as part of the Brout--Englert--Higgs
(BEH) mechanism~\cite{Englert:1964et,Higgs:1964pj}\todo{Is BEH technically only
  for gauge boson masses?} that is employed in the SM to explain how elementary
particles acquire mass in a process referred to \emph{electroweak symmetry
  breaking} (EWSB).
% \footnote{The details of the mass generation process differ between gauge
% bosons and fermions. These differences are discussed in
% \Cref{sec:standard_model}.}
While the discovery of the Higgs boson and the measurement of its properties
lends confidence in the SM, further tests of the nature of the Higgs boson are
required.

% is a crucial ingredient in the GSW model of the electroweak interaction that
% describes how elementary particles can acquire mass.

% SM is great and no fundamental changes have been made since its inception.
%
% conceptually the same since then.


The Higgs boson is predicted to interact with itself--a characteristic referred
to as the Higgs boson self-coupling. This self-coupling is a result of the BEH
mechanism, and its strength is directly related to the shape of the Higgs
potential. The choice of Higgs potential is central to the BEH mechanism in the
SM as it induces the process of EWSB that is instrumental in generating the
masses of elementary particles. Direct measurements of the Higgs boson
self-coupling are thus of interest to test our understanding of EWSB in the SM.
These measurements can be performed using high-energy particle collision events
resulting in the production of a pair of Higgs bosons. Higgs boson pair
production is a very rare process, which is why experimental evidence for such
processes has yet to be established.

This thesis is concerned with a search for Higgs boson pair production in final
states with a pair of $b$-quarks and a pair of \tauleptons, i.e.\
$HH \to \bbtautau$. The search is performed using proton--proton collision data
at a centre-of-mass energy of \SI{13}{\TeV} recorded by the ATLAS detector
during Run~2 of the LHC. The data were collected by the ATLAS collaboration over
a period spanning from 2015 to 2018, yielding a dataset with an integrated
luminosity of \SI{139}{\per\femto\barn}.

Two channels are considered for the search: the \hadhad channel targeting events
with two hadronic \taulepton decays,\footnote{In the parlance of particle
  physics, decays of the form $\tau^- \to \text{hadrons} + \nu_\tau$ are
  referred to as a hadronic \taulepton decays (\tauhad), and
  $\tau^- \to \ell^- \bar{\nu}_\ell \nu_\tau$ for $\ell = e$ or $\mu$ as
  leptonic \taulepton decays (\taulep).} and the \lephad channel targeting
events with one leptonic and one hadronic \taulepton decay.

% The work presented in this
% thesis focuses.



The focus of this thesis lies on the channel where both \tauleptons decay
$\tau^- \to \text{hadrons} + \nu_\tau$.

- Searching for SM \HH production

- Searching for resonant production of Higgs boson pairs in scenarios involving
physics beyond the SM (BSM)

- Setting constraints on the Higgs boson self-coupling


- Tau identification: Novel machine learning



\vspace{5em}

This thesis is structured as follows: \Cref{sec:standard_model} lays the
foundation for searches for Higgs boson pair production. The SM is summarised,
and the role of electroweak symmetry breaking and the Higgs boson in the SM is
illustrated. Moreover, the phenomenology of the Higgs boson and Higgs boson pair
production is discussed in the context of the SM and theories beyond the SM. The
chapter concludes with a summary of experimental results on searches for Higgs
boson pair production prior to the work presented in this thesis.
\Cref{sec:atlas_and_lhc} describes the Large Hadron Collider, the ATLAS
experiment, and the techniques used to reconstruct particle collision events
with the ATLAS detector. The collision events recorded by the ATLAS detector are
analysed using a suite of statistical methods that are introduced in
\Cref{sec:experimental_methods}.

The main part of this thesis is concerned with searches for Higgs boson pair
production in final states with two $b$-quarks and two hadronically decaying
\tauleptons ($\tau^{-} \to \text{hadrons} + \nu_\tau$). A crucial element of
these searches is the ability to reconstruct and identify hadronic \taulepton
decays.\todo{Why?} A novel method of performing \tauid using neural networks is
introduced in \Cref{sec:tauid}. This method is subsequently used in
\Cref{sec:dihiggs} to perform searches for resonant and non-resonant Higgs boson
pair production using \SI{139}{\per\femto\barn} of proton--proton collision data
collected at the ATLAS experiment during Run~2 of the Large Hadron Collider. In
\Cref{sec:higgs_self_coupling} the results of \Cref{sec:dihiggs} are
re-interpreted to set constraints on the possible values of the strength of the
Higgs boson self-coupling. The thesis is concluded in \Cref{sec:conclusion}.

\clearpage

\todo[inline]{This thesis presents results published as part of (list
  of publications)...}


Notes from chat with Jochen:
\begin{itemize}

\item Keep it short \ra about 120 pages
  \begin{itemize}
  \item SM about 15 pages and focus on \HH production

  \item LHC and ATLAS experiment about 10 pages: For every detector
    component a picture and a paragraph

  \item Experimental methods: 10 pages

  \item Try to keep it way below 160 pages???
  \end{itemize}

\item Cut out or move less interesting things to the backup

\item Make it more interesting! Cut out repetition...

\end{itemize}

\todo[inline]{Should clarify: BEH mechanism (i.e. W / Z eating the goldstone
  bosons) only gives masses to gauge bosons. EWSB gives masses to the fermions.}


%%% Local Variables:
%%% mode: latex
%%% TeX-master: "../phd_thesis"
%%% End:
