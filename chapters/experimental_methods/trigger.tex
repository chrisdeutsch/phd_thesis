% ==============================================================================
\section{Trigger}
\label{sec:exp_trigger}
% ==============================================================================
\todo[inline]{This should be elsewhere.}

\todo[inline]{For more information on 2017 and 2018 triggers see the
  TauCP INT note.}

The L1 tau trigger (calorimeter only) remained unchanged during Run~2
(aside from topological requirements).

Taxonomy of trigger chains (in athena: HLT threshold, selection, preselection):

\vspace{1em}

\verb|HLT_tau25_medium1_tracktwo| (2015--2016)~\cite{ATLAS-CONF-2017-061}:\\
Three step process:
\begin{enumerate}
\item Calorimeter preselection: The calorimeter is read out in full
  granularity for the region of interest marked by the L1
  trigger. Topoclusters are built from calorimeter cells and
  calibrated with the local hadronic calibration (LC). The vectorial
  sum of clusters defines the jet seed for tau reconstruction. The
  core clusters of the jet seed are used to estimate the tau energy
  which is then calibrated using parametrised methods (subtracting
  pileup and correcting the response). The HLT tau \pT cuts are
  applied at this stage. \verb|tau25|

\item Track preselection (two-staged): Fast tracking (Fast Track
  Finder) is performed in a narrow cone around the center tau jet seed
  to find the highest momentum track to determine the $z$-location of
  the primary vertex from which the tau likely originated. Once the
  vertex is determined, fast tracking is performed over the full ROI
  but restricted in $|\Delta z|$ to be close to the leading
  track. Tracks with at least 1 GeV are kept as core tracks
  $\Delta R < 0.2$ and isolation tracks~$0.2 < \Delta R <
  0.4$. Multiplicity requirements are enforced, so 1 to 3 core tracks
  and maximum 1 isolation track. \verb|tracktwo|

\item Offline-like selection: Precision tracking (similar to offline
  reconstruction) is performed which is seeded using FTF tracks from
  the previous step (i.e. it is refining the tracks found by FTF).


  A BDT (two one for
  1-prong one for multi-prong) is used to identify taus similar to
  what would be used for offline identification of taus. A medium
  working point is chosen with efficiencies: 96\% for 1-prong and 82\%
  for 3-prong in Ztautau. \verb|medium1|
\end{enumerate}

For 2017 data-taking the triggers were similar to the 15--16
algorithms with some tunings of the track preselection to prevent
pile-up from impacting the multiplicity requirements (mostly due to
the upper bound of 3 core tracks for 3-prong taus). Moreover, the
track multiplicity and identification cuts were relaxed for very high
momenta (200 GeV and higher) but this is barely relevant for the work
presented in this thesis.

\vspace{1em}

\verb|HLT_tau25_medium1_tracktwoEF| (2018--):\\
The track multiplicity (core and iso.) which was applied in the track
preselection stage using Fast-Track-Finder tracks introduced
introduced inefficiencies for 3-prong taus in high pileup
conditions. Therefore, the track multiplicity selection was removed
and applied at a later stage on precision tracks that have better
pileup and fake track rejection. \verb|tracktwoEF|

\vspace{1em}

\verb|HLT_tau25_mediumRNN_tracktwoMVA| (2018 K--):\\
The (parametric) calibration in the first step is replaced by an
MVA-based energy calibration using Boosted Regression Trees similar to
the algorithm for offline tau calibration but without track und tau
substructure information. Additionally, the tau identification step is
replaced using the RNN algorithm (similar to offline) with minor
adjustments of input variables (dead sensors in hit counts). Separate
networks are trained for 0-prong and 1-prong.  The multi-prong network
is using the offline tune.  A dedicated medium working point for the
HLT is used with the following efficiency targets: 0-prong: 65\%;
96.5\% 1-prong; 86.5\% multi-prong. The recovery 0-track taus recovers
efficiency for cases (predominantely low momentum / high pileup),
which can happen due to inefficiencies in FTF that are used to seed
precision tracking. These can be recovered in offline reconstruction
with full tracking.


%%% Local Variables:
%%% mode: latex
%%% TeX-master: "../../phd_thesis"
%%% End:
