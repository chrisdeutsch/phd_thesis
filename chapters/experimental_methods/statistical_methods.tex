% ==============================================================================
\section{Statistical Methods}
\label{sec:statistical_methods}
% ==============================================================================

The following gives an introduction into the techniques used for the
statistical analysis. Statistical inference is used to extract useful
information from measurement results and to test competing hypotheses.


\subsection{The Method of Maximum Likelihood}

Let~$f(\myvec{x} \mid \boldtheta)$ be the joint probability
density\footnote{The principles outlined in the following also hold
  for discrete probabiltiy distributions where the equivalent of the
  probability density is the probability mass.} of a vector of random
variables~$\myvec{X} = (X_1, \dots, X_n)$ that depends on
parameters~$\boldtheta$. For a given
realisation~$\myvec{x} = (x_1, \dots, x_n)$ of~$\myvec{X}$ one can
consider the joint density at the point~$\myvec{x}$ as a function of
the parameter vector~$\boldtheta$
\begin{align*}
  L(\boldtheta \mid \boldx) = f(\boldx \mid \boldtheta) \overset{\text{i.i.d.}}{=} \prod_{i=1}^n f(x_i \mid \boldtheta)
\end{align*}
which is called the likelihood function. The last equality holds for
independent and identically
distributed~$X_i \sim f(x \mid \boldtheta)$. The likelihood function
connects the probability density of the (fixed)
observation~$\myvec{x}$ with the parameter vector~\boldtheta. The
parameters of the joint distribution can be inferred from the
observation using of maximum likelihood estimation by
\begin{align*}
  \hat{\boldtheta}(\boldx) = \argmax_{\boldtheta} L(\boldtheta \mid \boldx)
\end{align*}
where~$\hat{\boldtheta}$ is an estimator of the parameters based on
the observed value of~\myvec{X}, i.e.\ the estimator is chosen such
that the likelihood of observing~\boldx is maximised.

The likelihood function is a fundamental part of frequentist
statistical inference used to analyse results of high-energy physics
experiments. Maximum likelihood estimators have convenient properties
in the large sample limit under some general regularity
conditions\footnote{For a detailed description of these conditions see
  for example~\cite{casella2001}.} on the likelihood
function. Let~$\hat{\theta}_n$ denote the maximum likelihood estimator
for~$\theta$ on a sample of size~$n$, then the estimator has the
following asymptotic properties~\cite{casella2001}:
\begin{description}
\item[Consistency] The estimator converges to the true value~$\theta$
  with an arbitrarily small deviation as the sample size tends to
  infinity, i.e.\
  $\lim_{n \ra \infty} P(|\hat{\theta}_n - \theta| < \epsilon) = 1$
  for any $\epsilon > 0$. In this limit the estimator is unbiased with
  vanishing variance.

\item[Asymptotic normality and efficiency] In the large sample
  limit~$\sqrt{n}(\hat{\theta}_n - \theta)$ converges in distribution
  to a normal distribution with zero mean and variance~$v(\theta)$,
  where~$\theta$ is the true value of the parameter and~$v(\theta)$
  the \textit{asymptotic variance} of the estimator. The estimator is
  \textit{asymptotically efficient} in the sense that~$v(\theta)$
  reaches the Cram\'er--Rao lower bound for the asymptotic variance of
  estimators.
\end{description}

Frequently one minimises the negative logarithm of the likelihood
(NLL) instead, which being a monotonic transformation does not change
the parameter values extremum, to convert the products / exponentials
frequently appearing in the liklihood function into sums / factors.

The probability density function~$f$ is frequently intractable
analytically but can be estimated using Monte Carlo
simulation. Therefore, one can discretise the the observable of
interest into~$k$~bins such that the probability for an event to fall
into bin~$i$ can be described by
\begin{align*}
  p_i(\boldtheta) = \int_\text{bin} f(x; \boldtheta) \, \mathrm{d}x
\end{align*}
or using frequentist methods when only MC simulation is available. In
the case of the binned distribution, the likelihood function for a
fixed sample of size~$n$ is given by the probability mass function of
the multinomial distribution
\begin{align*}
  L(\boldtheta \mid \boldn) = \frac{n!}{\prod_{i=1}^k n_i!} \prod_{i = 1}^k \left[ p_i(\boldtheta) \right]^{n_i}
\end{align*}
where~$n_i$ corresponds to the number of observations in bin~$i$ with
the condition that
\begin{align*}
  \sum_{i = 1}^k n_i = n \quad \text{and} \quad \sum_{i = 1}^k p_i(\boldtheta) = 1 \,\text{.}
\end{align*}

In high energy physics, the size of the collected dataset is not fixed
a priori but is varying according to a Poisson distribution with an
expected rate~$\nu$. This is accounted for in the so-called ``extended
likelihood function''~\cite{cowan1998}:
\begin{align*}
  L(\nu, \boldtheta \mid \boldn) =
  \frac{\nu^n e^{-\nu}}{n!} \times \frac{n!}
  {\prod_{i = 1}^k n_{i}!}
  \prod_{i = 1}^k \left[ p_i(\boldtheta) \right]^{n_i}
\end{align*}
which follows from the joint distribution given by the product of the
Poisson and multinomial probability mass functions (conditional on the
number of events observed?).  With the expectation value of the number
of events in bin~$i$ given
by~$\nu_i(\boldtheta) = \nu p_i(\boldtheta)$\todo{Can $\nu$ depend on
  \boldtheta? In principle it can and should!?} and the previously
mentioned conditions it follows that
\begin{align*}
  L(\nu_1, \dots, \nu_k, \boldtheta \mid \boldn) = \prod_{i = 1}^k \frac{\nu_i^{n_i} e^{-\nu_i}}{n_i!}
\end{align*}
which is the product of Poisson probabilities of observing~$n_i$
events in a bin with an expected number of events~$\nu_i$ (the
dependency on \boldtheta is omitted).

% The Poisson part gives the probability of observing~$n$ events while
% the multinomial describes the probability of distributing these events
% among~$k$ bins. In general~$\nu_i$ can depend on~\boldtheta.

In general we are interested in mixture models, where events can
originate from different processes and the observable, which is called
the discriminant in this context as it aims to distinguish betwen the
different parts of the mixture, follows a process-specific
distribution. For illustration consider two species of events, one
from a signal process with PDF $f_\text{S}(x; \boldtheta)$ and from a
background process with PDF $f_\text{B}(x; \boldtheta)$. Consider a
model where the signal process contributes with a fraction~$\alpha$
such that the PDF of the mixture is:
\begin{align*}
  f(x; \mu, \boldtheta) = \alpha f_\text{S}(x; \boldtheta) + (1 - \alpha) f_\text{B}(x; \boldtheta) \,\text{.}
\end{align*}
In practice~$f_\text{B}$ is frequently itself a mixture model
consisting of multiple background processes. ($f_\text{S}$ should only
depend on $\boldtheta_\text{s}$ same for bkg.?) In the binned representation this
mixture model yields:
\begin{align*}
  \nu_i(\mu, \boldtheta) = \mu \nu_i^\text{S}(\boldtheta) + \nu_i^\text{B}(\boldtheta)
\end{align*}
with the signal strength~$\mu$ being a parameter of the model, which
is frequently expressed as the cross section of the signal process
with respect to a baseline value like the SM expectation.

The parameter~$\mu$ of the mixture model is usually the parameter of
interest (POI) while the other parameters~\boldtheta, which need to be
determined to estimate the signal strength, are called nuisance
parameters.

\todo[inline]{Maybe stress that at this point it's a particular choice
  model?}

The binned / extended likelihood function is further extended by
incorporating multiple analysis channels, additional free parameters
(like the signal strength), systematic uncertainties, and auxiliary
measurements. With a set of analysis channels~$\mathcal{C}$, a set of
bins in a given channel~$\mathcal{B}_\mathcal{C}$, and a set of
auxiliary measurements~$\mathbb{S}$, the likelihood function can be
written as~\cite{cranmer2012}:
\begin{align*}
  L(\boldalpha, \boldphi, \boldgamma) =
  \prod_{c \in \mathcal{C}} \prod_{b \in \mathcal{B}_{\mathcal{C}}}
  \mathrm{Pois}\left( n_{cb} \mid \nu_{cb}(\boldalpha, \boldphi, \boldgamma) \right)
  \mathrm{Pois}\left(m_{cb} \mid  \gamma_{cb} \tau_{cb} \right)
  \prod_{p \in \mathbb{S}} f_p\left(a_p \mid \alpha_p \right)
\end{align*}
Latin letter correspond to observables while greek letters correspond
to nuisance parameters.

\begin{align*}
  \nu_{cb}(\boldalpha, \boldphi, \boldgamma) = \sum_{s \in \mathcal{S}} \gamma_{cb} \, \phi_{cs} \, \eta_{cs}(\boldalpha) \, \sigma_{csb}(\boldalpha)
\end{align*}
$\boldalpha = (\alpha_i)_i$, $\boldphi = (\phi_{cs})_{cs}$,
$\boldgamma = (\gamma_{cb})$.

\todo[inline]{Describe: histogram-based uncertainties, normalisation
  uncertainties, free normalisation factors, MC stat uncertainty}

\todo[inline]{Describe what is meant by 'observables' and 'global
  observables'.}


\subsection{Barlow-Beeston}%
\label{sec:barlow_beeston}

\cite{barlow1993,conway2011}


\subsection{Hypothesis Testing}
\label{sec:hypothesis_testing}

What is a hypothesis? A population that is described by a specific set
of parameters. Hypothesis tests distinguish between two complementary
hypotheses. Hypothesis test provides a rule to either accept (reject)
the null hypothesis, rejecting (accepting) the alternative based on an
observed sample from a population.


Hypothesis testing aims to distinguish between a null hypothesis~$H_0$
(e.g.\ the Standard Model) and an alternative hypothesis~$H_1$. The
value of a test statistic~$t(\mathbf{x})$, which is commonly a real
number, serves to indicate agreement with $H_0$ or $H_1$,
respectively. Various methods exist to construct suitable test
statistics.

A rejection region of the test can be defined by setting a threshold
on the test statistic, i.e.\ $t > t_\text{cut}$, which specifies the
significance~$\alpha$ of the test, which is the probability to reject
$H_0$ in favour of $H_1$ if $H_0$ was true (type I error
rate). Moreover, the power of the test, $1 - \beta$, is specified,
which is related to the probability~$\beta$ of accepting $H_1$ when
$H_0$ was true (type II error rate). The field of particle physics
adopts the convention that the null hypothesis needs to be rejected at
a significance level of~$\alpha = \num{2.87e-7}$ ($5\sigma$) to claim
discovery of new phenomena (which is intentionally low to prevent
false discoveries). Exclusion of signal models is usually done at 95\%
confidence level ($\alpha = 0.05$).


Composite hypothesis: hypothesis depending on some parameter, e.g.\
signal strength, nuisance parameters.


$\boldtheta \in \boldsymbol{\Theta}_0$ (usually a simple hypothesis,
i.e.\ the set has one element)

$\boldtheta \in \boldsymbol{\Theta}_1$ (composite hypothesis)

Nested model: The b-only model is a limiting case of the
alternative~$\mu \ra 0$.



\subsubsection{Likelihood Ratio Test and Neyman-Pearson Lemma}

The (profile) Likelihood Ratio Test is the optimal test based on the
Neyman-Pearson lemma~\cite{neyman1933} meaning for a given
significance level yielding the largest power.
\begin{align*}
  \lambda = \frac{L(\boldtheta \mid H_0)}{L(\boldtheta \mid H_1)}
\end{align*}

\todo[inline]{The Likelihood Ratio Test is only the optimal test if
  there are no free parameters.}

Likelihood ratio test for nested / composite models (i.e.\ the
alternativ model contains the null hypothesis as a
subset~$\Theta_0 \subset \Theta$:
\begin{align}
  \lambda = \frac{\sup_{\theta \in \Theta_0} L(\theta \mid \boldx)}{\sup_{\theta \in \Theta\phantom{_0}} L(\theta \mid \boldx)}
\end{align}
where the numerator is the restricted MLE over $\Theta_0$ and the
numerator the unrestricted MLE. Frequently, the test statistics are
expressed as~$-2 \ln \lambda$ due to the asymptotic properties of the
sampling distributions.

The likelihood ratio still depends on both the parameters of interest
as well as the nuisance parameters. The profile likelihood given by
\begin{align*}
  L_\text{p}(\mu) = \sup_\theta L(\mu, \theta) % \\
%  \hat{\hat{\theta}}(\mu) = \argmax_\theta \mathcal{L}(\mu, \theta \mid \mathbf{x})
\end{align*}
and represents a profile of the likelihood function along the line
given by~$\hat{\hat{\theta}}(\mu) = \argmax_\theta L(\mu, \theta)$
(sometimes called the conditional maximum likelihood estimate of the
nuisance parameters given~$\mu$ -- double hat to distinguish it from
the maximum likelihood estimate~$\hat{\theta}$). The maximum of the
profile likelihood coincides with the maximum likelihood estimate.


In HEP we usually have a model where the signal contribution is
controlled by the signal strenght~$\mu \geq 0$ where~$\mu = 0$
corresponds to the background-only hypothesis. Thus the likelihood
ratio test for a null hypothesis with signal strength~$\mu$ is written as
\begin{align}
  \lambda(\mu) = \frac{\sup_\theta L(\mu, \theta \mid \boldx)}{\sup_{\mu,\theta} L(\mu, \theta \mid \boldx)}
\end{align}


\begin{align*}
  \Lambda(\mu) = \frac{\mathcal{L}\left( \mu, \hat{\hat{\theta}} \mid \mathbf{x} \right)}
                      {\mathcal{L}\left( \hat{\mu}, \hat{\theta} \mid \mathbf{x} \right)}
\end{align*}


hat: estimator to distinguish it from the true value of the parameter.

Asymptotic distributions: Wilk's theorem (valid under some regularity
conditions: asymptotic, nested, either model is correct, identifiable,
i.e.\ any value of parameters is uniquely assigned to a distinct
hypothesis)



\subsubsection{Discovery Significance}


The test statistic for the discovery of a positive signal
is~\cite{Cowan:2010js}:
\begin{align}
  q_0 =
  \begin{cases}
    -2 \log \lambda(0) & \muhat \geq 0 \\
    0 & \muhat < 0
  \end{cases}
\end{align}




Asymptotic distribution: non-central chi-square with degrees of
freedom according to the number of POIs. In the case of the null
hypothesis, this reduces to the result from Wilk (non-centrality
factor is zero) but really it is a half chi-square distribution.


\subsubsection{Confidence Intervals \& Upper Limits}

A confidence interval (comparison to point estimate, e.g.\ MLE) is an
interval of a parameter derived from the observed sample that covers
the true value with a given coverage probabilty. (The interval is the
random quantity and not the parameter).

The calculation of confidence intervals and upper limits can be
depicted as an inversion of hypothesis tests. The question is which
$\mu$ null hypothesis would be rejected at a fixed significance
level~$\alpha$. The confidence interval is then the region which
cannot be rejected at the specified coverage $1 - \alpha$.

Asymptotic properties of likelihood ratio -> should be roughly normal
in the minimum, can be taylor expanded and then confidence intervals
can be extracted from the Hessian at the MLE.


\begin{align}
  \tilde{q}_\mu =
  \begin{cases}
    -2 \ln \tilde{\lambda}(\mu) & \muhat \leq \mu \\
    0 & \muhat > \mu
  \end{cases}
\end{align}

Asimov (for CLs method)?

CLs method: \cite{Read:2002hq}

\begin{align}
  \text{CL}_\text{s+b} = \int^\infty_{q_\text{obs}} f(\tilde{q}_\mu \mid \mu) \, \mathrm{d}\tilde{q}_\mu \\
  \text{CL}_\text{b} = \int^\infty_{q_\text{obs}} f(\tilde{q}_\mu \mid 0) \, \mathrm{d}\tilde{q}_\mu
\end{align}


\begin{align}
  \text{CL}_\text{s} = \frac{\text{CL}_\text{s+b}}{\text{CL}_\text{b}}
\end{align}



% Statistical Analysis:

% Hypothesis testing: Neyman Pearson lemma, Likelihood ratio (\LambdaΛ) test
% Wilk's theorem (asymptotic approximation)
% Discovery significance (q0)
% Limit setting (CLs / CLs+b)

%%% Local Variables:
%%% mode: latex
%%% TeX-master: "../../phd_thesis"
%%% End:
