This thesis presents a search for Higgs boson pair production in final states
with two $b$-quarks and two \tauleptons using \SI{139}{\per\femto\barn} of
\pp~collision data recorded by the ATLAS experiment during Run~2 of the
LHC. Searches for Higgs boson pair production are instrumental in testing our
understanding of EWSB in the SM, allowing the shape of the Higgs potential---a
fundamental component of the theory proposed by Brout, Englert, and Higgs---to
be probed. These searches are challenging, however, since Higgs boson pair
production is three orders of magnitude less abundant than single Higgs boson
production at the LHC. Nevertheless, searches for Higgs boson pair production
are already of interest to probe the BSM sector. This is because the presence of
new physics could alter the non-resonant \HH production cross section or
introduce resonant enhancements of the cross section due to BSM particles
decaying into pairs of Higgs bosons.


The search for Higgs boson pair production in the \bbtautau channel considers
final states with one or two hadronic \taulepton decays. A particular focus is
put on the sub-channel with two hadronic \taulepton decays in this thesis. This
channel relies on the ability to efficiently identify \tauhadvis while
suppressing backgrounds from quark- or gluon-initiated jets. For this purpose, a
novel \tauid method is introduced that uses sequences of reconstructed
charged-particle tracks and topo-clusters in the calorimeters, as well as
high-level information about \tauhadvis candidates, as inputs. This is made
possible by the use of an RNN architecture, which allows to process
variable-length sequences. The new method significantly improves the \tauid
performance over the BDT-based algorithm previously used at the ATLAS
experiment. At fixed \tauhadvis efficiency working points, the RNN-based \tauid
improves the rejection of \faketauhadvis from quark- or gluon-initiated jets by
\SI{80}{\percent} (\SIrange{40}{80}{\percent}) for 1-prong (3-prong) \tauhadvis
candidates. Moreover, the improvement in \faketauhadvis rejection exceeds
\SI{100}{\percent} for candidates with $\pT > \SI{70}{\GeV}$. The RNN-based
\tauid was adopted by the ATLAS collaboration as the recommended \tauid
algorithm for analyses of the Run~2 \pp~collision dataset. Furthermore, it was
adapted for the use in \tauhadvis-triggers for data-taking at the end of Run~2
and the beginning of Run~3 of the LHC.


In this thesis, three different modes of Higgs boson pair production are
considered: non-resonant \HH production predicted by the SM; resonant \HH
production via massive scalar particles with small decay widths; and
non-resonant \HH production with anomalous values of the Higgs boson
self-coupling constant.


% SM HH
The search for non-resonant \HH production predicted by the SM targets the \ggF
and VBF production modes. Improved object reconstruction and identification
techniques are employed, among them the RNN-based \tauid, leading to an increase
in signal acceptance ranging from \SIrange{50}{100}{\percent} with respect to
earlier searches by the ATLAS collaboration in the \bbtautau channel. No
statistically significant signal is observed in the search; therefore, upper
limits are set on the SM~\HH signal strength and production cross section. The
observed (expected) upper limit at \SI{95}{\percent}~CL on the SM~\HH signal
strength is 4.7 (3.9).  The observed (expected) upper limit at
\SI{95}{\percent}~CL on the SM~\HH production cross section is
\SI{140}{\femto\barn} (\SI{110}{\femto\barn}). To date, this search provides the
highest expected sensitivity to SM~\HH production of any individual channel.
% of the ATLAS and CMS collaborations.


% Resonant
The search for resonant \HH production targets resonances with masses between
\num{251} and \SI{1600}{\GeV} decaying into pairs of Higgs bosons. The \bbtautau
channel provides upper limits that are competitive with other searches in a mass
range of \SIrange{375}{800}{\GeV}. Over this range, the upper limits on the
$pp \to X \to \HH$ cross section range from \SIrange{130}{30}{\femto\barn} at
\SI{95}{\percent}~CL. The largest excess is observed for a resonance mass of
\SI{1000}{\GeV} with a local (global) significance of $3.1\sigma$
($2.0\sigma$). This excess is not statistically significant given the
conventional significance thresholds for the discovery of new physics. In
addition, it is not supported by other search channels of the ATLAS and CMS
collaborations, which set more stringent upper limits for
$\mX = \SI{1000}{\GeV}$.


% KLAMBDA
Lastly, the search for SM~\HH production is reinterpreted in terms of
non-resonant \HH production with anomalous values of the Higgs boson
self-coupling constant. Upper limits at \SI{95}{\percent}~CL are set on the
non-resonant \HH production cross section for different hypothesised values of
\klambda. Based on the observed (expected) upper limits on the cross section,
the Higgs boson self-coupling modifier is constrained to be within
$-2.4 \leq \klambda \leq 9.2$ ($-2.0 \leq \klambda \leq 9.0$).


% COMBINATION
At present, no golden channel for searches for non-resonant \HH production
exists. Instead, the searches in the \bbbb, \bbtautau, and \bbyy channels are
complementary. As a result, combinations of these and other channels are
important to obtain the most stringent constraints on the SM~\HH signal strength
and the Higgs boson self-coupling constant. Such combinations were performed by
the ATLAS and CMS collaborations after Run~2 of the LHC, yielding upper limits
at \SI{95}{\percent}~CL on the SM~\HH signal strength of 2.4~\cite{HDBS-2022-03}
and 3.4~\cite{CMS-HIG-22-001}, respectively, outperforming any individual search
channel. Similarly, the best experimental constraints on the Higgs boson
self-coupling are obtained by combining multiple channels. The Higgs boson
self-coupling modifier is constrained (at \SI{95}{\percent}~CL) to be within
$-0.6 \leq \klambda \leq 6.6$ by the ATLAS collaboration~\cite{HDBS-2022-03} and
$-1.2 \leq \klambda \leq 6.5$ by the CMS collaboration~\cite{CMS-HIG-22-001} for
the combination of the most sensitive channels.


% Future outlook
Searches for Higgs boson pair production remain an important part of physics
programmes at the LHC and future colliders. Under the SM assumption, first
evidence for SM~\HH production is likely to be obtained at the HL-LHC. When
combining the most sensitive searches of the ATLAS and CMS collaborations,
discovery of SM~\HH production might be achieved using the full HL-LHC
\pp~collision dataset. Furthermore, the Higgs boson self-coupling constant will
be measured with a precision of the order of \SI{50}{\percent} at the
HL-LHC~\cite{DiMicco:2019ngk}. Among the main goals of post-LHC colliders---for
example, the FCC~\cite{FCC:2018byv}, ILC~\cite{Bambade:2019fyw}, or
CLIC~\cite{CLIC:2018fvx}---is the measurement of the Higgs boson self-coupling
with uncertainties of \SI{10}{\percent} and below, thus greatly expanding the
sensitivity to BSM physics.


\newpage\noindent\textbf{Notes:}\\

\todo[inline]{Reduce the number of footnotes.}

\todo[inline]{Use consistent abbreviations.}

\todo[inline]{Check consistent use of topo-cluster (particularly in the RNN ID
  part)}

\todo[inline]{Check uses of 'independent sentence; therefore, independent
  sentence... (or similar with however).}

\todo[inline]{Fix usage of - (hyphen), -- (en dash), and --- (em dash).}

\todo[inline]{Make sure $V$ is defined as vector boson.}

\todo[inline]{Put validation region plots in appendix.}

\todo[inline]{Make sure it's clear what a $b$-jet candidate is.}

\todo[inline]{There should be a comma before ``while'' if it doesn't have
  temporal meaning. \url{https://www.grammarly.com/blog/comma-before-while/}.}

\todo[inline]{Check that all figure captions are self explanatory.}

\todo[inline]{Update from CONF to bbtautau paper.}

\todo[inline]{Run~2 of the LHC -> LHC~Run~2 ?}

\todo[inline]{Update bbbb result to paper version!}

\todo[inline]{``allow to do sth.'' should be ``allow doing sth.'' in most cases?}

%%% Local Variables:
%%% mode: latex
%%% TeX-master: "../phd_thesis"
%%% End:
