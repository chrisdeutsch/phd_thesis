This thesis presents a search for Higgs boson pair production in final states
with two $b$-quarks and two \tauleptons using \SI{139}{\per\femto\barn} of
\pp~collision data recorded by the ATLAS experiment during Run~2 of the
LHC. Searches for Higgs boson pair production are instrumental in testing our
understanding of EWSB in the SM, allowing to probe the shape of the Higgs
potential---a fundamental component of the theory proposed by Brout, Englert,
and Higgs. These searches are challenging, however, since Higgs boson pair
production is three orders of magnitude less abundant than single Higgs boson
production at the LHC. Nevertheless, searches for Higgs boson pair production
are already of interest to probe the BSM sector. This is because the presence of
new physics could alter the non-resonant \HH production cross section or
introduce resonant enhancements of the cross section due to BSM particles
decaying into pairs of Higgs bosons.


The search for Higgs boson pair production in the \bbtautau channel considers
final states with one or two hadronic \taulepton decays. A particular focus is
put on the sub-channel with two hadronic \taulepton decays in this thesis. This
channel relies on the ability to efficiently identify \tauhadvis while
suppressing backgrounds from quark- or gluon-initiated jets. For this purpose, a
novel \tauid method is introduced that uses sequences of reconstructed
charged-particle tracks and topo-clusters in the calorimeters, as well as
high-level information about \tauhadvis candidates, as inputs. This is made
possible by the use of an RNN architecture that allows to process
variable-length sequences. This method significantly improves the \tauid
performance over the BDT-based algorithm previously used at the ATLAS
experiment. At fixed \tauhadvis efficiency working points, the RNN-based \tauid
improves the rejection of \faketauhadvis from quark- or gluon-initiated jets by
\SI{80}{\percent} (\SIrange{40}{80}{\percent}) for 1-prong (3-prong) \tauhadvis
candidates. Moreover, the improvement in \faketauhadvis rejection exceeds
\SI{100}{\percent} for candidates with $\pT > \SI{70}{\GeV}$. The RNN-based
\tauid was adopted by the ATLAS collaboration as the recommended \tauid
algorithm for analyses of the Run~2 \pp~collision dataset. Furthermore, it was
adapted for the use in \tauhadvis-triggers for data-taking at the end of Run~2
and the beginning of Run~3 of the LHC.


In this thesis, three different modes of Higgs boson pair production are
considered: non-resonant \HH production predicted by the SM; resonant \HH
production via massive scalar particles with small decay widths; and
non-resonant \HH production with anomalous values of the Higgs boson
self-coupling constant.


% SM HH
The search for non-resonant \HH production predicted by the SM targets the \ggF
and VBF production modes. Improved object reconstruction and identification
techniques are employed, among them the RNN-based \tauid, leading to an increase
in signal acceptance ranging from \SIrange{50}{100}{\percent} with respect to
earlier searches by the ATLAS collaboration in the \bbtautau channel. No
statistically significant signal is observed in the search; therefore, upper
limits are set on the SM~\HH signal strength and production cross section. The
observed (expected) upper limit at \SI{95}{\percent}~CL on the SM~\HH signal
strength is 4.7 (3.9).  The observed (expected) upper limit at
\SI{95}{\percent}~CL on the SM~\HH production cross section is
\SI{140}{\femto\barn} (\SI{110}{\femto\barn}). To date, this search provides the
highest expected sensitivity to SM~\HH production of any single channel of both
the ATLAS and CMS collaborations.


% Resonant
The search for resonant \HH production targets resonances with masses between
\num{251} and \SI{1600}{\GeV} decaying into pairs of Higgs bosons. The \bbtautau
channel provides upper limits that are competitive with other searches in a mass
range of \SIrange{375}{800}{\GeV}. Over this range, the upper limits on the
$pp \to X \to \HH$ cross section range from \SIrange{130}{30}{\femto\barn} at
\SI{95}{\percent}~CL. The largest excess is observed for a resonance mass of
\SI{1000}{\GeV} with a local (global) significance of $3.1\sigma$
($2.0\sigma$). This excess is not statistically significant given the
conventional significance thresholds for the discovery of new physics. In
addition, it is not supported by other search channels of the ATLAS and CMS
collaborations, which set more stringent upper limits for
$\mX = \SI{1000}{\GeV}$.


% KLAMBDA
Lastly, the search for SM~\HH production is reinterpreted in terms of
non-resonant \HH production with anomalous values of the Higgs boson
self-coupling constant. Upper limits at \SI{95}{\percent}~CL are set on the
non-resonant \HH production cross section for different hypothesised values of
\klambda. Based on the observed (expected) upper limits on the cross section,
\klambda is constrained to be within $-2.4 \leq \klambda \leq 9.2$
($-2.0 \leq \klambda \leq 9.0$).\todo{\bbyy is generally better of \klambda?}


% COMBINATION
At present, no single golden channel for searches for non-resonant \HH
production exists. Instead, searches in the \bbbb, \bbtautau, and \bbyy channels
are complementary. As a result, combinations of these and other channels are
instrumental in obtaining the most stringent constraints on the SM~\HH signal
strength and the Higgs boson self-coupling constant. The ATLAS and CMS
collaborations performed these combinations after Run~2 of the LHC, yielding
upper limits on the SM~\HH signal strength of 2.4 and 3.4,
respectively~\cite{HDBS-2022-03,CMS-HIG-22-001}, vastly outperforming any
individual search channel. Furthermore, the best experimental constraints on the
Higgs boson self-coupling are obtained by combinations of multiple channels.
After Run~2 of the LHC, the Higgs boson self-coupling is constrained to be
within $-0.6 \leq \klambda \leq 6.6$ by the ATLAS
collaboration~\cite{HDBS-2022-03} and $-1.2 \leq \klambda \leq 6.5$ by the CMS
collaboration~\cite{CMS-HIG-22-001} for combinations of the most sensitive
channels.


% Future outlook
Searches for Higgs boson pair production remain an important part of physics
programmes at the LHC and future colliders. Under the SM expectation, first
evidence for SM~\HH production is likely to be obtained at the HL-LHC. Moreover,
when combining the most sensitive searches of the ATLAS and CMS collaborations
discovery of SM~\HH production might even be in reach at the end of the HL-LHC
data-taking.

Uncertainties on the self-coupling of \SI{50}{\percent}~\cite{DiMicco:2019ngk}.

Precision measurements of the Higgs boson self-coupling, that is
measurements of \klambda with uncertainties at the percent-level, will require
dedicated collider experiments

(e.g.\ Higgs boson factories).




\vspace{10em}\noindent\textbf{Notes:}\\

\todo[inline]{Reduce the number of footnotes.}

\todo[inline]{Use consistent abbreviations for signal region (SR) and
  control region (CR) etc.}
\todo[inline]{Check consistent use of topo-cluster (particularly in the RNN ID part)}
\todo[inline]{Check uses of 'independent sentence; therefore, independent sentence... (or similar with however).}
\todo[inline]{Fix usage of - (hyphen), -- (en dash), and --- (em dash).}

\todo[inline]{Make sure $V$ is defined as vector boson.}

\todo[inline]{Put validation region plots in appendix.}

\todo[inline]{Make sure it's clear what a $b$-jet candidate is.}

\todo[inline]{There should be a comma before ``while'' if it doesn't have
  temporal meaning. \url{https://www.grammarly.com/blog/comma-before-while/}.}

\todo[inline]{Check consistent hyphen usage. $b$ quark, $b$-quark. top quark
  etc.}

\todo[inline]{top~quark should be top-quark.}

%%% Local Variables:
%%% mode: latex
%%% TeX-master: "../phd_thesis"
%%% End:
