This thesis presents a search for Higgs boson pair production in final states
with two $b$-quarks and two \tauleptons using \SI{139}{\per\femto\barn} of
\pp~collision data recorded by the ATLAS experiment during Run~2 of the
LHC. Searches for Higgs boson pair production are instrumental in testing our
understanding of EWSB in the SM, allowing to directly probe the shape of the
Higgs potential---a fundamental component of the theory proposed by Brout,
Englert, and Higgs. These searches are challenging, however, since Higgs boson
pair production is three orders of magnitude less abundant than single Higgs
boson production at the LHC. Nevertheless, searches for Higgs boson pair
production are already of interest to probe the BSM sector. For example, new
physics could enhance the non-resonant \HH production cross section or introduce
resonant production modes of Higgs boson pairs via new resonances.

The focus of this thesis lies on the channel with two hadronically decaying
\tauleptons. This channel relies on the ability to efficiently identify hadronic
\taulepton decays while suppressing backgrounds from quark- or gluon-initiated
jets. For this purpose, a novel \tauid method is introduced that uses sequences
of charged-particle tracks, sequences of topo-clusters in the calorimeters, and
additional high-level information about \tauhadvis candidates as inputs. This is
possible through the use of a recurrent neural network architecture that allows
processing of variable-length sequences. This method represents a significant
improvement over the previously used BDT-based algorithm only used high-level
information. At fixed \tauhadvis efficieny working points, the RNN-based \tauid
improves the rejection of \faketauhadvis from quark- or gluon-jets by
\SI{80}{\percent} (\SIrange{40}{80}{\percent}) for 1-prong (3-prong) \tauhadvis
candidates. The improvement in rejection of \faketauhadvis with high \pT exceeds
a factor of two. The method presented in this thesis was adopted by the ATLAS
collaboration as the recommended algorithm for \tauid for analyses of the Run~2
\pp~collision dataset. Furthermore, it was adapted for the use in
\tauhadvis-triggers at the end of Run~2 and the beginning of Run~3 of the LHC.

A search for Higgs boson pair production via scalar, narrow-width resonances
with masses in the range from \SIrange{251}{1600}{\GeV} is presented in this
thesis. The \bbtautau channel provides competitive upper limits on the cross
section of resonant \HH production in a mass range from
\SIrange{375}{800}{\GeV}. Over this mass range, upper limits on the cross
sections of $pp \to X \to \HH$ production are set from
\SIrange{130}{30}{\femto\barn} at \SI{95}{\percent}~CL. The largest excess in
the search is observed at $\mX = \SI{1000}{\GeV}$ with a local (global)
significance of $3.1\sigma$ ($2.0\sigma$). The excess is not statistically
significant given the conventional significance thresholds for discovery adopted
in HEP. In addition, the excess is not supported by other search channels of the
ATLAS and CMS collaborations, which yield more stringent upper limits at
$\mX = \SI{1000}{\GeV}$.

The non-resonant production of Higgs boson pairs predicted by the SM cannot be
established given the size of the \pp~collision datasets collected during Run~2
of the LHC. In this thesis a search for SM~\HH production via \ggF and VBF is
presented. No statistically significant signal contribution is
observed. Therefore, upper limits are set on the SM~\HH signal strength and
production cross section, yielding upper limits of
\begin{align*}
  &\mu  < 4.7 \, \text{(observed)} && \mu < 3.9\,\text{(expected)}
\end{align*}
and
\begin{align*}
  &\xsecggfvbf < \SI{140}{\femto\barn} \,\text{(observed)} && \xsecggfvbf < \SI{110}{\femto\barn} \,\text{(expected)}
\end{align*}
at \SI{95}{\percent}~CL, where the expected upper limits are derived under the
background-only hypothesis. The \bbtautau channel provides the highest expected
sensitivity to SM~\HH production of any single channel to date.

Finally, the search for SM~\HH production is reinterpreted to set upper limits
on the non-resonant \HH production cross sections for anomalous values of the
Higgs boson self-coupling constant.
in the context of
anomalous values of the Higgs boson self-coupling. Upper limits are set on the
cross section of non-resonant \HH production for different hypothesised values
of \klambda. Based on the cross section upper limits, \klambda is constrained to
be within
\begin{align*}
  &-2.4 \leq \klambda \leq 9.2 \,\text{(observed)} && -2.0 \leq \klambda \leq 9.0 \,\text{(expected)} \,\text{.}
\end{align*}

% Paragraph about future prospects.
% Run 3: First evidence?

Under the SM assumption, the first observation of SM~\HH production possible at
the HL-LHC. Moreover, the Higgs boson self-coupling.

Therefore, searches for Higgs boson pair production remain an important part of
future physics programmes...


\vspace{10em}\noindent\textbf{Notes:}\\

In~\cite{ATL-PHYS-PUB-2015-046} the HL-LHC prospects claim to set an
upper limit of $4.3 \times \sigma_\text{SM}$. 6 years later we already
exceed this prospect.

\todo[inline]{Reduce the number of footnotes.}

\todo[inline]{Use consistent abbreviations for signal region (SR) and
  control region (CR) etc.}
\todo[inline]{Check consistent use of topo-cluster (particularly in the RNN ID part)}
\todo[inline]{Check uses of 'independent sentence; therefore, independent sentence... (or similar with however).}
\todo[inline]{Fix usage of - (hyphen), -- (en dash), and --- (em dash).}

\todo[inline]{Make sure $V$ is defined as vector boson.}

\todo[inline]{Put validation region plots in appendix.}

\todo[inline]{Make sure it's clear what a $b$-jet candidate is.}

\todo[inline]{There should be a comma before ``while'' if it doesn't have
  temporal meaning. \url{https://www.grammarly.com/blog/comma-before-while/}.}

\todo[inline]{Check consistent hyphen usage. $b$ quark, $b$-quark. top quark
  etc.}

\todo[inline]{top~quark should be top-quark.}

%%% Local Variables:
%%% mode: latex
%%% TeX-master: "../phd_thesis"
%%% End:
