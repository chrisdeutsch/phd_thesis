\section{Event Reconstruction}

\todo[inline]{The same reconstruction and identification algorithms are applied
  to simulation and data. Dedicated corrections are applied to account for
  differences in efficiency / calibrations between simulation and data.}

\subsection{Primary Vertex Selection}

\subsection{Jets}
\todo[inline]{Jet building from clusters; calibration; bad jet veto; JVT;
  b-tagging}

\subsection{Hadronically Decaying Tau Leptons}

\todo[inline]{taus: seeding; energy scale (LC \& BRT); track selection;
  acceptance; Tau-ID (jet \& e)}

\subsection{Electrons}

\todo[inline]{electrons: reco \& ID / isolation}

\subsection{Muons}

\todo[inline]{muons: reco \& ID / isolation}

\subsection{Missing Transverse Momentum}

\todo[inline]{\MET: negative vector sum of reconstructed objects; soft term}

\subsection{Overlap Removal}

% https://gitlab.cern.ch/atlas/athena/tree/21.2/PhysicsAnalysis/AnalysisCommon/AssociationUtils/
% Harmonization note: https://cds.cern.ch/record/1743654/files/ATL-COM-PHYS-2014-451.pdf

After the event is reconstructed a overlap-removal procedure is applied to
resolve ambiguities when a physical object is reconstructed as multiple
particles in the ATLAS detector. The angular distance $\dRy =
\sqrt{\Delta y^2 + \Delta \phi^2}$ is used to measure the overlap of two
reconstructed objects. \todo{Why rapidity instead of pseudorapidity?}

The following step-by-step procedure is used to resolve ambiguities in the
reconstructed objects:
\begin{itemize}
  % https://indico.cern.ch/event/807799/contributions/3362328/attachments/1818680/2973610/Presentation.pdf
  % https://indico.cern.ch/event/814156/contributions/3396841/attachments/1830226/2997072/Presentation.pdf
%\item jet - $\mu$: Reject jet if not \btagged, $N_\text{tracks} < 4$ and $\dRy
%  < 0.4$

% https://gitlab.cern.ch/atlas/athena/-/blob/21.2/PhysicsAnalysis/AnalysisCommon/AssociationUtils/Root/EleEleOverlapTool.cxx
\item $e_1$ - $e_2$: Reject $e_1$ if both electrons share the track and ${\pT}_1
  < {\pT}_2$

% https://gitlab.cern.ch/atlas/athena/-/blob/21.2/PhysicsAnalysis/AnalysisCommon/AssociationUtils/Root/TauLooseEleOverlapTool.cxx
\item \tauhadvis - $e$: Reject \tauhadvis if $\dRy < 0.2$ and $e$ passes \verb|DFCommonElectronsLHLoose|

% https://gitlab.cern.ch/atlas/athena/-/blob/21.2/PhysicsAnalysis/AnalysisCommon/AssociationUtils/Root/TauLooseMuOverlapTool.cxx
\item \tauhadvis - $\mu$: Reject \tauhadvis if $\dRy < 0.2$:\\
  Case 1 ($\tauhadvis \pT > \SI{50}{\GeV}$): $p_{\text{T}, \mu} > \SI{2}{\GeV}$
  and combined muon\\
  Case 2 ($\tauhadvis \pT \leq \SI{50}{\GeV}$): $p_{\text{T}, \mu} > \SI{2}{\GeV}$

% https://gitlab.cern.ch/atlas/athena/-/blob/21.2/PhysicsAnalysis/AnalysisCommon/AssociationUtils/Root/EleMuSharedTrkOverlapTool.cxx
\item $\mu$ - $e$: Reject $\mu$ if calo-muon and shared ID track
\item $e$ - $\mu$: Reject $e$ if shared ID track

% https://gitlab.cern.ch/atlas/athena/-/blob/21.2/PhysicsAnalysis/AnalysisCommon/AssociationUtils/Root/EleJetOverlapTool.cxx
\item jet - $e$: Reject jet if $\dRy < 0.2$, not \btagged or \btagged and
  ${\pT}_{e} > \SI{100}{\GeV}$
\item $e$ - jet: Reject $e$ if $\dRy < 0.4$

% https://gitlab.cern.ch/atlas/athena/-/blob/21.2/PhysicsAnalysis/AnalysisCommon/AssociationUtils/Root/MuJetOverlapTool.cxx
\item jet - $\mu$: Reject jet if not \btagged, $N_\text{track} < 3$ ($p_\text{T,
  track} > \SI{500}{\MeV}$), and $\dRy < 0.2$
\item $\mu$ - jet: Reject $\mu$ if $\dRy < 0.4$
\end{itemize}
Due to a bug in the PFlow jet reconstruction, an additional OLR step removing
prompt muons being reconstructed as fake jets is implemented in the official ASG
OLR tool.

Additionally, an analysis-specific overlap-removal procedure for \tauhadvis,
anti-\tauhadvis and jets is implemented:
\begin{itemize}
\item jet - \tauhadvis: Reject jet if $\dRy < 0.2$
\item anti-\tauhadvis - jet: Reject anti-\tauhad if jet is \btagged and $\dRy <
  0.2$
\item jet - anti-\tauhadvis: Reject jet  if $\dRy < 0.2$
\end{itemize}
This establishes the following priority: \tauhadvis > \btagged jet >
anti-\tauhadvis > un-tagged jet.

The priority, \btagged jet > \tauhadvis > anti-\tauhadvis > un-tagged jet, was
investigated as an alternative but found to reduce signal acceptance in the
2-tag region significantly due to limited \tauhad rejection of the DL1r \btag
algorithm at the \SI{77}{\percent} working point. With the alternative priority
the signal acceptance is reduced by about \SI{8}{\percent} (\SI{13}{\percent})
in $\tauhad\taulep$ ($\tauhad\tauhad$).

\todo[inline]{Is the above still up to date for PflowJets?}

\todo[inline]{Special treatment of lepton OLR for b-jet corrections}


%%% Local Variables:
%%% mode: latex
%%% TeX-master: "../../phd_thesis"
%%% End:
