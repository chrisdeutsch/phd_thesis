\section{Multivariate analysis}

The event selection described in~\Cref{sec:event_selection} only
serves as a preselection with the intention that selected events
follow the expected topology of the signal and that basic kinematic
requirements, ensuring trigger efficiencies are in saturation, are
fulfilled.

The non-resonant and resonant production of SM Higgs boson pairs have
distinct kinematic properties that can be used to reject backgrounds
in the signal region. An example that is independent of the production
mode of SM Higgs boson pairs is the invariant mass of the \bbbar pair
and the \hadhad which can be reconstructed. A number of reconstructed
quantities can be defined that offer discrimination power to
distinguish between the various signals and backgrounds.

Multivariate methods are employed to exploit the discrimination power
of multiple reconstructed quantities and their correlations to
classify events regarding their signal- and
background-likeness. Depending on the search different methods are
used.

The search of non-resonant \HH production in the SM uses Boosted
Decision Trees and Neural Networks to distinguish between signal and
background in the \hadhad and \lephad channel, respectively. When
searching for resonant \HH production in BSM scenarios, multiple mass
hypotheses for the scalar resonance decaying into SM Higgs boson pairs
are considered. The signal event kinematics are therefore dependent on
the mass of the resonance, \mX, and as a result the classification
task continuously varies as a function of \mX. This is in contrast to
the former case where the kinematic properties of signal events are
fixed. Classification tasks that vary as a function of a parameter,
for example the resonance mass, can be performed by \emph{Parametric
  Neural Networks} (PNN), first introduced to HEP
in~\cite{Baldi:2016fzo}. PNN provide a single classifier that is able
to handle multiple classification tasks, while being able to smoothly
interpolate the parameter to values not seen during training.

The scores provided by these multivariate classification methods,
hereafter called MVA scores, are later used as a discriminant in the
maximum likelihood fit to extract the signal of interest and set upper
limits on signal strengths and cross-sections. No further selections
are applied to events entering the signal extraction procedure such
that the preselection regions are also the signal regions in the
respective channels.

In~\Cref{sec:mva_discriminating variables} the choice of
discriminating variables to classify signal and background processes
will be motivated. Afterwards, the training and optimisation of the
classifiers used to extract non-resonant signal is described
in~\Cref{sec:mva_smbdt}. Finally, \Cref{sec:mva_pnn} will explain the
interpolation properties of PNN as well as the training and
optimisation procedures used. \todo{Variable importance?}


\subsection{Discriminating variables}
\label{sec:mva_discriminating variables}

Discriminating variables were rechecked (in \hadhad) from previous
publication and found to be all significant except for MET centrality.

The discriminating variables used: \mMMC, \mBB, \mHH, \dRtautau, \dRbb.


\subsection{Extraction of the non-resonant signal using boosted decision trees}
\label{sec:mva_smbdt}

\todo[inline]{Optimisation}
\todo[inline]{Performance?}


\subsection{Extraction of resonant signals with parametric neural networks}
\label{sec:mva_pnn}


\todo[inline]{Optimisation}
\todo[inline]{Performance: Detuning studies}


\subsection{Others?}

\todo[inline]{Variable ranking}


%%% Local Variables:
%%% mode: latex
%%% TeX-master: "../../phd_thesis"
%%% End:
