\section{Multivariate analysis}

The event selection described in~\Cref{sec:event_selection} only
serves as a preselection with the intention that selected events
follow the expected topology of the signal and that basic kinematic
requirements, ensuring trigger efficiencies are in saturation, are
fulfilled.

The non-resonant and resonant production of SM Higgs boson pairs have
distinct kinematic properties that can be used to reject large
fractions of background contributions. An example independent of the
production mode of SM Higgs boson pairs is the invariant mass of the
\bbbar pair and the \hadhad which can be reconstructed. A number of
reconstructed quantities can be defined that offer discrimination
power to distinguish between the various signals and backgrounds.

Multivariate methods are employed to exploit the discrimination power
of multiple reconstructed quantities and their correlations to
classify events regarding their signal- and
background-likeness. Depending on the search different methods are
used.

\todo[inline]{Should talk about strategy: Fitting MVA score directly.}

The search of non-resonant \HH production in the SM uses Boosted
Decision Trees (BDT) to distinguish between signal and background.

The search for resonant \HH production in BSM scenarios considers
multiple different mass hypotheses for the resonance decaying into SM
Higgs boson pairs. Therefore, this case represents a classification
task that varies as a function of the resonance mass. Parametric
neural networks (PNN), first introduced to HEP
in~\Cref{Baldi:2016fzo}, are used as a convenient approach to handle
such tasks.


\subsection{Discriminating variables}

Discriminating variables were rechecked (in \hadhad) from previous
publication and found to be all significant except for MET centrality.

The discriminating variables used: \mMMC, \mBB, \mHH, \dRtautau, \dRbb.


\subsection{Extraction of the non-resonant signal using boosted decision trees}

\todo[inline]{Optimisation}
\todo[inline]{Performance?}


\subsection{Extraction of resonant signals with parametric neural networks}


\todo[inline]{Optimisation}
\todo[inline]{Performance: Detuning studies}


\subsection{Others?}

\todo[inline]{Variable ranking}


%%% Local Variables:
%%% mode: latex
%%% TeX-master: "../../phd_thesis"
%%% End:
