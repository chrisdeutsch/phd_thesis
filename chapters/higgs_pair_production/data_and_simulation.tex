\section{Data and Simulated Event Samples}
\label{sec:data_and_simulation}

This search is performed using \pp-collision data at a centre of mass
energy of $\sqrt{s} = \SI{13}{\TeV}$ collected by the ATLAS experiment
during \RunTwo of the LHC (2015--2018). All recorded events have to
pass the data quality criteria of the ATLAS
collaboration~\cite{DAPR-2018-01} requiring a fully operational
detector and stable beams at the LHC. The integrated luminosity of the
dataset of events passing the quality criteria corresponds to
\SI{139.0}{\ifb} with an uncertainty of
\SI{1.7}{\percent}~\cite{ATLAS-CONF-2019-021}.
% as measured by LUCID~\cite{LUCID2}.
Events recorded by the ATLAS detector are reconstructed using the
\textsc{Athena} software suite~\cite{ATL-SOFT-PUB-2021-001}.

% https://twiki.cern.ch/twiki/bin/view/Atlas/LuminosityForPhysics
% https://twiki.cern.ch/twiki/bin/view/AtlasProtected/GoodRunListsForAnalysisRun2
% 2015 (OflLumi-13TeV-008): 3219.56 ipb
% 2016 (OflLumi-13TeV-009): 32988.1 ipb
% 2017 (OflLumi-13TeV-010): 44307.4 ipb
% 2018 (OflLumi-13TeV-010): 58450.1 ipb

Monte Carlo event generators are used to estimate the contributions of
signal and (most) background processes in regions entering this
search. The response of the ATLAS detector to generated events is
obtained either from a \emph{full simulation} of the
detector~\cite{SOFT-2010-01} based on \GEANT~\cite{Agostinelli:2002hh}
or from a hybrid approach relying on a parametric description of
calorimeter response~\cite{SOFT-2010-01} (\emph{fast simulation}). All
generated events are reconstructed from the detector response using
the same reconstruction algorithms used for \pp-collision data. The
effect of pile-up is accounted for by overlaying events with
additional inelastic \pp-collisions obtained from
simulation. Additionally, simulated events are re-weighted such that
the pile-up conditions match those observed in data.

In the following, a description of the simulated event samples used in
this analysis is given with a focus on the simulation of signal
processes. Simulated background processes are summarised
in~\Cref{tab:monte_carlo}.

\begin{sidewaystable}[p]
  \centering

  \caption{Table of Monte Carlo samples.\\
    $\dagger$: V+0,1,2j@NLO+3,4j@LO \\
    $*$: 0,1j@NLO + 2,3j@LO\\
    $\ddag$: gluon loop induced ZH
  }%
  \label{tab:monte_carlo}

  \resizebox{\textwidth}{!}{% From VH->bb evidence paper:
% Events containing W or Z bosons with jets (V +jets) were simulated using the Sherpa
% 2.2.1 generator. Matrix elements were calculated for up to two partons at NLO and four
% partons at LO using the OpenLoops [67] and Comix [68] matrix-element generators. The
% number of expected V + jets events is rescaled using the NNLO cross-sections [71].


% Separate PS PDF?

\begin{tabular}{lllllll}
  \toprule
  Process                             & ME Generator    & ME PDF         & ME Order & PS and Hadronisation & UE Model Tune & Cross-Section Order \\
  \midrule
  \multicolumn{7}{l}{\textbf{Signals}} \\
  \midrule
  Non-resonant $\Pgluon\Pgluon \ra \PHiggs\PHiggs$ (ggF) &&&&&& \\
  Non-resonant $\Pquark\Pquark \ra \Pquark\Pquark\PHiggs\PHiggs$ (VBF) &&&&&& \\
  Resonant $\Pgluon\Pgluon \ra \PX \ra \PHiggs\PHiggs$ &&&&&& \\
  \midrule
  \multicolumn{7}{l}{\textbf{Top quark}} \\
  \midrule
  \ttbar &&&&&& \\
  Single \Ptop ($s$-channel) &&&&&& \\
  Single \Ptop ($t$-channel) &&&&&& \\
  Single \Ptop ($\PW\Ptop$-channel) &&&&&& \\
  $\ttbar\PZ$ &&&&&& \\
  $\ttbar\PW$ &&&&&& \\
  \midrule
  \multicolumn{7}{l}{\textbf{Vector boson + jets}} \\
  \midrule
  $\PW \ra \Plepton \Pnu$              & \SHERPA{2.2.1} & \NNPDF{3.0NNLO} & NLO? & \SHERPA{2.2.1}      & Default       & NNLO \\
  $\PZ / \Pphoton^{*} \ra \ell\ell$    & \SHERPA{2.2.1} & \NNPDF{3.0NNLO} & NLO? & \SHERPA{2.2.1}      & Default       & NNLO \\
  \midrule
  \multicolumn{7}{l}{\textbf{Diboson}} \\
  \midrule
  $\PW\PW$, $\PW\PZ$, $\PZ\PZ$ & \SHERPA{2.2.1} & ?               & & \SHERPA{2.2.1}      & ?             & ? \\
  \midrule
  \multicolumn{7}{l}{\textbf{Single Higgs boson}} \\
  \midrule
  $\PHiggs \ra $ &&&&&& \\
  $V \PHiggs \ra $ &&&&&& \\
  \bottomrule
\end{tabular}

%%% Local Variables:
%%% mode: latex
%%% TeX-master: "../phd_thesis.tex"
%%% End:
}
\end{sidewaystable}

The non-resonant production of Higgs boson pairs via \ggF is simulated
using \POWHEGBOX[v2]~\cite{Alioli:2010xd} at NLO (QCD) and accounts
for the finite top-quark mass in heavy-quark
loops~\cite{Borowka:2016ehy,Baglio:2018lrj}. The generator uses the
\PDFforLHC[15nlo]~\cite{Butterworth:2015oua} set of PDFs and is
interfaced to \PYTHIA[8]~\cite{Sjostrand:2014zea} with the A14 set of
tuned parameters~\cite{ATL-PHYS-PUB-2014-021} for the parton shower
and hadronisation. Decays of hadrons containing $b$ or $c$ quarks are
simulated using \EVTGEN[1.7.0]~\cite{Lange:2001uf}. The sample of
simulated events is normalised using a theory cross section of
\SI{31.05}{\femto\barn} at
$\text{NNLO}_{\text{FTapprox}}$~\cite{Grazzini:2018bsd}\todo{Should
  explain what this is.} with a cross section uncertainties of
$^{+6\,\%}_{-23\,\%}$ from renormalisation and factorisation scales
and the top-quark mass scheme~\cite{Baglio:2020wgt} and
$\pm\SI{3}{\percent}$ from PDFs and $\alphas$~\cite{LHCHWGHH}.

% https://gitlab.cern.ch/atlas-physics/pmg/infrastructure/mc15joboptions/-/blob/master/share/DSID450xxx/MC15.450149.MadGraphHerwig7EvtGen_PDF23LO_X300tohh_bbtautau_hadhad.py
The resonant di-Higgs production in a 2HDM model $\pp \ra X \ra hh$ in ggF was
generated using \MADGRAPH for the matrix element at LO interfaced to
\HERWIG{7.1.0.3} LO for the parton shower. An alternative sample replacing
\HERWIG with \PYTHIA{8} for the parton shower. Using narrow width approximation
where the heavy scalar is assumed to have a decay width of $\Gamma_X =
\SI{10}{\MeV}$. \todo{Which masses are available?} The cross-section for the
resonant production is set to $\sigma(\pp \ra X \ra hh) = \SI{1}{\pb}$.


\todo[inline]{Say that the resonances are narrow width. Why narrow
  width? -> Broad resonances are more model dependent.}


The production of single Higgs bosons is considered as a background.

Other backgrounds are briefly summarised in~\Cref{tab:monte_carlo}.



%%% Local Variables:
%%% mode: latex
%%% TeX-master: "../../phd_thesis"
%%% End:
