\section{Data and Simulated Event Samples}%
\label{sec:data_and_simulation}

This search uses \pp~collision data at a centre-of-mass energy of
$\sqrt{s} = \SI{13}{\TeV}$ collected by the ATLAS experiment during \RunTwo of
the LHC. All recorded events have to pass data quality
criteria~\cite{DAPR-2018-01}, requiring a fully operational ATLAS detector and
stable beams at the LHC. The integrated luminosity of events passing the quality
criteria corresponds to \SI{139}{\per\femto\barn} with an uncertainty of
\SI{1.7}{\percent}~\cite{ATLAS-CONF-2019-021}. Events recorded by the ATLAS
detector are reconstructed using the \textsc{Athena} software
suite~\cite{ATL-SOFT-PUB-2021-001}.

% https://twiki.cern.ch/twiki/bin/view/Atlas/LuminosityForPhysics
% https://twiki.cern.ch/twiki/bin/view/AtlasProtected/GoodRunListsForAnalysisRun2
% 2015 (OflLumi-13TeV-008): 3219.56 ipb
% 2016 (OflLumi-13TeV-009): 32988.1 ipb
% 2017 (OflLumi-13TeV-010): 44307.4 ipb
% 2018 (OflLumi-13TeV-010): 58450.1 ipb

MC event generators are used to estimate the contributions of signal and most
background processes in this search. The response of the ATLAS detector to
generated events is obtained either from a \emph{full simulation} of the
detector based on \GEANT~\cite{SOFT-2010-01,Agostinelli:2002hh} or from a hybrid
approach referred to as \emph{fast simulation} that replaces the simulation of
the calorimeter response with a parametric description
thereof~\cite{SOFT-2010-01}. Events are reconstructed from simulated detector
responses using the same algorithms used to reconstruct collision events
recorded by the ATLAS detector. The effect of pile-up is accounted for by
overlaying all generated events with additional inelastic \pp~collisions
obtained from simulation. Simulated events are then re-weighted to ensure that
the pile-up conditions match those of the recorded data.

In the following, a description of the MC event generators used for the
simulation of signal processes is given. The generator configurations used for
the simulation of background processes are summarised in~\Cref{tab:monte_carlo}.

\begin{sidewaystable}[p]
  \centering

  \caption[Summary of generators used to simulate signal and background
  processes for the search for Higgs boson pair production.]{Summary of
    generators used to simulate signal and background processes for the search
    for Higgs boson pair production. The order of the perturbative expansion in
    $\alphas$ is given unless qualified by ``EW'', which indicates higher order
    electroweak corrections. $*$:~$V+\text{jets}$ (diboson) event generation
    with \SHERPA[2.2.1] merges matrix elements with NLO accuracy for up to two
    (one) and LO accuracy for up to four (three) final state
    partons. $\dagger$:~$q\bar{q} / qg$ induced production of $ZH$ is normalised
    using the total $pp \to ZH$ cross section (NNLO+NLO EW) and subtracting the
    $gg \to ZH$ cross section (NLO+NLL) using predictions from
    Ref.~\cite{deFlorian:2016spz_book}. The table is adapted from
    Ref.~\cite{HDBS-2018-40}.}%
  \label{tab:monte_carlo}

  \resizebox{\textwidth}{!}{% From VH->bb evidence paper:
% Events containing W or Z bosons with jets (V +jets) were simulated using the Sherpa
% 2.2.1 generator. Matrix elements were calculated for up to two partons at NLO and four
% partons at LO using the OpenLoops [67] and Comix [68] matrix-element generators. The
% number of expected V + jets events is rescaled using the NNLO cross-sections [71].


% Separate PS PDF?

\begin{tabular}{lllllll}
  \toprule
  Process                             & ME Generator    & ME PDF         & ME Order & PS and Hadronisation & UE Model Tune & Cross-Section Order \\
  \midrule
  \multicolumn{7}{l}{\textbf{Signals}} \\
  \midrule
  Non-resonant $\Pgluon\Pgluon \ra \PHiggs\PHiggs$ (ggF) &&&&&& \\
  Non-resonant $\Pquark\Pquark \ra \Pquark\Pquark\PHiggs\PHiggs$ (VBF) &&&&&& \\
  Resonant $\Pgluon\Pgluon \ra \PX \ra \PHiggs\PHiggs$ &&&&&& \\
  \midrule
  \multicolumn{7}{l}{\textbf{Top quark}} \\
  \midrule
  \ttbar &&&&&& \\
  Single \Ptop ($s$-channel) &&&&&& \\
  Single \Ptop ($t$-channel) &&&&&& \\
  Single \Ptop ($\PW\Ptop$-channel) &&&&&& \\
  $\ttbar\PZ$ &&&&&& \\
  $\ttbar\PW$ &&&&&& \\
  \midrule
  \multicolumn{7}{l}{\textbf{Vector boson + jets}} \\
  \midrule
  $\PW \ra \Plepton \Pnu$              & \SHERPA{2.2.1} & \NNPDF{3.0NNLO} & NLO? & \SHERPA{2.2.1}      & Default       & NNLO \\
  $\PZ / \Pphoton^{*} \ra \ell\ell$    & \SHERPA{2.2.1} & \NNPDF{3.0NNLO} & NLO? & \SHERPA{2.2.1}      & Default       & NNLO \\
  \midrule
  \multicolumn{7}{l}{\textbf{Diboson}} \\
  \midrule
  $\PW\PW$, $\PW\PZ$, $\PZ\PZ$ & \SHERPA{2.2.1} & ?               & & \SHERPA{2.2.1}      & ?             & ? \\
  \midrule
  \multicolumn{7}{l}{\textbf{Single Higgs boson}} \\
  \midrule
  $\PHiggs \ra $ &&&&&& \\
  $V \PHiggs \ra $ &&&&&& \\
  \bottomrule
\end{tabular}

%%% Local Variables:
%%% mode: latex
%%% TeX-master: "../phd_thesis.tex"
%%% End:
}
\end{sidewaystable}

SM \HH production via \ggF is simulated using
\POWHEGBOX[v2]~\cite{Nason:2004rx,Frixione:2007vw,Alioli:2010xd} at NLO
accounting for the finite top-quark mass in real and virtual
corrections~\cite{Borowka:2016ehy,Baglio:2018lrj,Heinrich:2017kxx,Heinrich:2019bkc,Heinrich:2020ckp}.\footnote{As
  opposed to earlier calculations in the $m_{t} \to \infty$ limit using
  effective field theory approximations to simplify top-quark loops to effective
  couplings. See for example Ref.~\cite{Dawson:1998py} (NLO) and
  Ref.~\cite{deFlorian:2013jea} (NNLO).} The generator uses the
\PDFforLHC[15nlo] set of PDFs~\cite{Butterworth:2015oua} and is interfaced to
\PYTHIA[8]~\cite{Sjostrand:2014zea} with the A14 set of tuned
parameters~\cite{ATL-PHYS-PUB-2014-021} for parton showering, hadronisation, and
simulation of the underlying event. The sample of simulated events is normalised
using an inclusive $gg \to \HH$ cross section of \SI{31.05}{\femto\barn} at
$\text{NNLO}_{\text{FTapprox}}$~\cite{Grazzini:2018bsd}, which is a combination
of the full-theory prediction at NLO with additional NNLO corrections derived in
the heavy top-quark mass limit. The theoretical uncertainties on the cross
section prediction are
$^{\hspace{0.25pt}+\hspace{0.25pt}\phantom{0}6\,\%}_{-23\,\%}$ from scale
variations and the treatment of the finite top-quark mass~\cite{Baglio:2020wgt}
and $\pm\SI{3}{\percent}$ from uncertainties on PDFs and
$\alphas$~\cite{LHCHWGHH}.

SM \HH production via VBF is simulated using
\MGNLO~\cite{Alwall:2014hca} % [2.7.3]
at LO %~\cite{Bishara:2016kjn}
using the \NNPDF[3.0nlo] set of PDFs~\cite{Ball:2014uwa}. The matrix element
generator is interfaced to \PYTHIA[8] with the A14 tune for parton showering,
hadronisation, and simulation of the underlying event. An inclusive cross
section of \SI{1.726}{\femto\barn}
at~$\text{N}^3\text{LO}$~\cite{Dreyer:2018qbw,LHCHWGHH} is used to normalise the
sample of generated events. The theoretical uncertainties on the cross section
prediction are $^{\hspace{0.25pt}+\hspace{0.25pt}0.03\,\%}_{-0.04\,\%}$ from
scale variations and $\pm\SI{2.1}{\percent}$ from uncertainties on PDFs and
$\alphas$~\cite{LHCHWGHH}.

% https://gitlab.cern.ch/atlas-physics/pmg/infrastructure/mc15joboptions/-/blob/master/share/DSID450xxx/MC15.450149.MadGraphHerwig7EvtGen_PDF23LO_X300tohh_bbtautau_hadhad.py
Higgs boson pair production via scalar resonances produced in \ggF is simulated
using \MGNLO at LO using the \NNPDF[2.3lo] set of PDFs~\cite{Ball:2012cx}. The
matrix element generator is interfaced to
\HERWIG[7.1]~\cite{Gieseke:2012ft,Bellm:2017jjp} with the default tune for
parton shower, hadronisation, and simulation of the underlying event. Twenty
benchmark signals are generated for resonances with masses~$\mX$ ranging from
\SIrange{251}{1600}{\GeV} and a decay width of \SI{10}{\MeV}.
% ($\Gamma_{X} = \SI{10}{\MeV}$ in the event generation).
The interference between resonant production of Higgs boson pairs and SM~\HH
production is neglected.
% Why narrow width? -> Broad resonances are very model dependent

Lastly, decays of hadrons containing $b$- and $c$-quarks are simulated using
\EVTGEN~\cite{Lange:2001uf} for all signal, top-quark, and single Higgs boson
processes (cf.~\Cref{tab:monte_carlo}). Moreover, full simulation of the ATLAS
detector is used for all processes except for resonant \HH production with
$\mX \leq \SI{1000}{\GeV}$ and alternative samples used for the derivation of
uncertainties. In these cases, fast simulation is used instead.\footnote{The
  fast detector simulation was only validated for scalar resonances with
  $\mX \leq \SI{1000}{\GeV}$.}

%%% Local Variables:
%%% mode: latex
%%% TeX-master: "../../phd_thesis"
%%% End:
