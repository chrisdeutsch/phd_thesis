\section{Systematic Uncertainties}
\label{sec:uncertainties}

\subsection{Experimental Uncertainties}

The uncertainty on the integrated luminosity of the $pp$ collision
dataset collected with the ATLAS detector in the period from 2015 to
2018 is \SI{1.7}{\percent}~\cite{ATLAS-CONF-2019-021}. This
uncertainty is applied to all processes estimated using simulation
that are normalised using theory cross sections.

An uncertainty on the reweighting of the instantaneous luminosity
distribution used for the pile-up overlay in simulated data to match
the conditions observed in data is considered.

Uncertainties on electrons in simulation are obtained from dedicated
calibration measurements~\cite{EGAM-2018-01,TRIG-2018-05} updated for
the \SI{139}{\per\femto\barn} Run~2 dataset. The electron energy scale
and resolution is measured in $Z \to e^+e^-$ events and propagated to
predictions by shifting and smearing momenta of electrons in
simulation. Uncertainties on the calibration of electron
reconstruction, identification, isolation, and trigger efficiencies in
simulation are obtained from measurements in~$J/\psi \to e^+e^-$ and
$Z \to e^+e^-$ events and applied as alternative weights to electrons
in simulation.

Muons... \cite{MUON-2018-03} \\
Scale, (Sagitta) , (ID? MS?) \\
Efficiencies: Reconstruction, Isolation, Trigger

Taus... \\
Scale \\
Efficiencies: Reconstruction, Identification, eVeto (taus and
electrons reconstructed as taus), Trigger

Jets... (JVT, JES, JER) -- JES/JER \cite{JETM-2018-05}

MET

Flavour tagging... \cite{FTAG-2018-01}

\begin{table}[htbp]
  \centering

  \begin{tabular}{lS[table-format=2]}
  \toprule
  Uncertainty source & {Components} \\
  \midrule
  Integrated luminosity & 1 \\
  Pile-up reweighting & 1 \\[0.5em]
  \textbf{Electrons} & \\
  Energy scale \& resolution & 3 \\
  Efficiencies (Reco., ID, Isol., Trigger) & 4 \\
  \tauhadvis $e^\pm$-veto efficiency & 2 \\[0.5em]
  \textbf{Muons} & \\
  Energy scale, ... & 5 \\
  Efficiencies (Reco., Isol., TTVA, Trigger) & 10 \\[0.5em]
  \textbf{Taus} & \\
  Energy scale & 4 \\
  Efficiencies (Reco., ID, $e$-Veto, Trigger) & 32 \\[0.5em]
  \textbf{Jets} & \\
  Energy scale & 33 \\
  Energy resolution & 14 \\
  Efficiency (jet vertex tagger) & 1 \\[0.5em]
  \textbf{Flavour tagging} & \\
  Efficiences (tag and mis-tag) & 13 \\[0.5em]
  \textbf{\pTmiss} & \\
  Momentum scale \& resolution & 3 \\
  \bottomrule
\end{tabular}

%%% Local Variables:
%%% mode: latex
%%% TeX-master: "../phd_thesis"
%%% End:


  \caption{Table of CP uncertainties}
  \label{tab:bla}
\end{table}

\todo[inline]{Where applicable, dedicated calibrations for ATLAS fast
  simulation are used.}


\subsubsection{Experimental?}


\todo[inline]{Split in experimental \& theory uncertainties}

\todo[inline]{Experimental: Jets -- JES, JER, JVT, b-tag}

\todo[inline]{Experimental: TAUS -- TES, Reco, ID}

\todo[inline]{Experimental: electrons / muons -- ?}

\todo[inline]{Experimental: MET -- all momentum / energy uncertainties are
  propagated to MET, additionally uncertainties related to the soft term}


\subsection{Extrapolation Uncertainties}

The normalisation of the \ZHF and \ttbar backgrounds are measured in
the combined likelihood fit of control and signal regions. Constraints
on the normalisation of the \ZHF background primarily originate from
the dedicated control region. The \ttbar background normalisation is
constrained in the signal region of the \lephad SLT channel and the
\ZHF control region. Due to the determination of the normalisation of
these processes from the fit to data, any uncertainties on the overall
normalisation of these processes, for example uncertainties on cross
sections or the integrated luminosity, are omitted. However, the use
of these normalisation measurements in other regions requires the
introduction of extrapolation uncertainties.

The extrapolation uncertainties are defined with respect to a
reference region, for which the \ZHF control region is chosen due to
its ability to constrain the normalisation of both the \ZHF and the
\ttbar backgrounds. In all other regions extrapolation uncertainties
are assigned to account for uncertainties in the modelling of the
background process with simulation relative to the reference
region. This introduces additional degrees of freedom in the
background model by allowing the normalisation factors of background
templates in other regions to deviate from the reference normalisation
within their constraints given by the uncertainty.

Consider the general case of a physics process that is estimated using
simulation and normalised by a fit to observed data in two analysis
regions A and B, with region A being defined as the reference
region. The probability of an event originating from the process to be
recorded in a given region is given by the product of the acceptance
and efficiency, \AccTimesEff, for a given region. The nominal result
of the simulation predicts how \AccTimesEff relates between both
regions and one can define the ratio
\begin{align*}
  \mathcal{R} = \frac{(\AccTimesEff)_{\text{B}}}{(\AccTimesEff)_{\text{A}}} \,\text{,}
\end{align*}
which will be be referred to as the \emph{relative acceptance} between
regions A and B, herafter. Uncertainties on the relative acceptance
have to be assigned as uncertainties on the normalisation of the
background in region B\todo{Can one show this mathematically?}. These
uncertainties are estimated by performing variations of the background
prediction, for example by using alternative generator setups, and
estimating a relative uncertainty on~$\mathcal{R}$ according to
\begin{align*}
  X = \frac{\mathcal{R}(\text{variation}) - \mathcal{R}(\text{nominal})}{\mathcal{R}(\text{nominal})}
\end{align*}

% The shape effects of uncertainties are estimated separately and will be described in~\Cref{sec:modelling_uncertainties}.

% Note: Sign is important -> direction of effect when correlating systematics


\subsection{Modelling Uncertainties}
\label{sec:modelling_uncertainties}

\todo[inline]{Theory: PDF, \alphas, renormalization and factorization scale}

\todo[inline]{Theory: simualted processes not normalized in data \ra
  uncertainties on theory xsec is applied}

\todo[inline]{Theory: V+jets -- resummation, CKKW matching}

\todo[inline]{Theory: \ttbar -- ME, PS, ISR, FSR -- Generally good
  modelling in bulk of phase space but can be problematic in tails /
  phase space corners.}

\todo[inline]{Theory: Signal -- ?}


%%% Local Variables:
%%% mode: latex
%%% TeX-master: "../../phd_thesis"
%%% End:
