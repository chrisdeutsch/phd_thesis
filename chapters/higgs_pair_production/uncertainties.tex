\section{Systematic Uncertainties}%
\label{sec:uncertainties}

The systematic uncertainties affecting the search for non-resonant and resonant
Higgs boson pair production are discussed in the following. Experimental
uncertainties are described in \Cref{sec:experimental_uncertainties} but exclude
uncertainties related to the estimation of \faketauhadvis backgrounds in the
\hadhad channel. These were previously described
in~\Cref{sec:bkg_hadhad_ttbarfakes,sec:bkg_hadhad_ff}. Theory uncertainties are
described in~\Cref{sec:theory_uncertainties} and include uncertainties on the
modelling of physics processes with MC simulations.

The searches for Higgs boson pair production presented in this thesis are
generally limited by statistical uncertainties due to the finite number of
events observed in the SRs. Theory uncertainties play a lesser role and
instrumental uncertainties are only relevant for searches for scalar resonance
with low masses ($\mX \lesssim \SI{300}{\GeV}$). The impact of statistical and
systematic uncertainties on the results are discussed in the context of the
statistical interpretation in \Cref{sec:statistical_analysis}.


\subsection{Experimental Uncertainties}%
\label{sec:experimental_uncertainties}

Experimental uncertainties arise from the measurement of the integrated
luminosity, the re-weighting of the pile-up conditions in simulation, the
reconstruction of physics objects, and the efficiencies of selections applied to
these objects. Unless otherwise noted, all experimental uncertainties apply to
signal and background processes estimated using simulation.

The integrated luminosity of the \pp-collision dataset recorded with the ATLAS
detector during Run~2 of the LHC is measured with an uncertainty of
\SI{1.7}{\percent}~\cite{ATLAS-CONF-2019-021}. This uncertainty is assigned to
all processes normalised using theoretical cross section predictions. Moreover,
an uncertainty is assigned on the re-weighting of the simulated event samples to
match the pile-up conditions in the recorded \pp-collision dataset.

Uncertainties related to the reconstruction and selection of physics objects are
provided by dedicated calibration measurements performed by the ATLAS
collaboration. In this search, these calibrations are used for
electrons~\cite{EGAM-2018-01,TRIG-2018-05}, muons~\cite{MUON-2018-03},
\tauhadvis~\cite{ATLAS-CONF-2017-029}, jets~\cite{JETM-2018-05},
flavour-tagging~\cite{FTAG-2018-01,FTAG-2020-08,FTAG-2021-002}, and the
\pTmissAbs reconstruction~\cite{ATLAS-CONF-2018-023}. These measurements yield
uncertainties on the momentum scale, momentum resolution, and selection
efficiencies of reconstructed objects. The major categories of instrumental
uncertainties are summarised in
\Cref{tab:experimental_uncertainties_2}. Uncertainties on the efficiencies of
electron, muon, and \tauhadvis triggers are only considered in channels where
these triggers are used. All uncertainties affecting the four-momentum of
reconstructed and selected objects are also propagated to the object-based
\pTmissAbs reconstruction.
% Additional calibrations and uncertainties are derived for event samples using
% fast simulation of the ATLAS detector.

Except\todo{merge with previous paragraph.} for the luminosity uncertainty, any
of the previously mentioned uncertainties can affect the shape and/or
normalisation of the distributions used in the statistical
interpretation. Therefore, both the shape and normalisation effects of these
uncertainties are propagated to the relevant distributions.

\begin{table}[htbp]
  \centering

  \caption{Summary of instrumental uncertainties. The number of independent NPs
    describing the uncertainty, $N_{\text{NPs}}$, is given in the right-most
    column.}%
  \label{tab:experimental_uncertainties_2}

  { \renewcommand{\arraystretch}{1.5} \begin{tabular}{lp{9.6cm}c}
  \toprule
  Category    & Uncertainty sources & {$N_{\text{d.o.f.}}$} \\
  \midrule
  Electrons   & Momentum scale and resolution; Reconstruction, identification, isolation, and trigger efficiencies. & 7 \\
  Muons       & Momentum scale and resolution; Reconstruction, track-to-vertex-association, isolation, and trigger efficiencies. & 15 \\
  \tauhadvis  & Momentum scale; Reconstruction, identification, $e$-veto, and trigger efficiencies. Mis-tag rates for electrons of the $e$-veto algorithm.  & 38 \\
  Jets        & Momentum scale and resolution; Jet vertex tagging efficiency. & 48 \\
  $b$-tagging & Tagging efficiencies for $b$-jets and mis-tag rates for $c$- and light-jets. & 13 \\
  \pTmissAbs  & Momentum scale and resolution; Track soft-term uncertainties.$^\dagger$ & 3 \\
  \bottomrule
\end{tabular}


%%% Local Variables:
%%% mode: latex
%%% TeX-master: "../phd_thesis"
%%% End:
 }
\end{table}

Experimental uncertainties on the \faketauhadvis background estimate in the
\lephad channel are derived to account for the statistical uncertainties of the
\FFqcd, \FFttbar, and \rqcd estimates. Moreover, uncertainties on the
subtraction of \ttbar and non-\ttbar processes in the fake factor method are
accounted for. An additional uncertainty is assigned on the value of \rqcd due
to it being derived using a simulation-based estimate of \ttbarFakes in the
Anti-ID region. In the \lephad SLT and LTT channel the total uncertainties on
the normalisation of the \faketauhadvis background are $^{+18\,\%}_{-23\,\%}$
and $^{+28\,\%}_{-30\,\%}$, respectively. The leading contributions to these
uncertainties are the \ttbar subtraction and variations of \rqcd in the SLT
channel and the \ttbar and non-\ttbar subtraction, variations of \rqcd, and the
statistical precision of \FFttbar in the LTT channel. The effect of these
variations on the shape of final discriminants are considered in the background
model.


\subsection{Theory Uncertainties}%
\label{sec:modelling_uncertainties}%
\label{sec:theory_uncertainties}

A number of theoretical uncertainties have to be considered for signal and
background processes estimated using simulation. For a given process these
uncertainties are split, where applicable, into uncertainties on the cross
section of the process, uncertainties on the acceptance of an analysis selection
(e.g.\ the SR selection), and uncertainties affecting the acceptance in terms of
the shape of discriminants used for the statistical analysis. With few
exceptions, acceptance uncertainties on a given process that originate from the
same source are assumed to be fully correlated between all analysis regions.

The description of theory uncertainties is divided into three parts.  First, the
treatment of uncertainties on the major \ZHF and \ttbar backgrounds is
described, representing a special case since the cross section of these
processes is determined from the likelihood fit to observed data. Subsequently,
uncertainties on other background processes estimated with simulation and
normalised using theoretical predictions are briefly outlined. Finally,
uncertainties on the modelling of the SM \HH and BSM $X \to \HH$ signals are
presented.

%\subsubsection{Uncertainties on the relative acceptance of \ZHF and
% \ttbar between regions}
\subsubsection{Acceptance Uncertainties on \ZHF and \ttbar Backgrounds}

The normalisation of the \ZHF and \ttbar backgrounds are measured in the
combined likelihood fit of CRs and SRs. Constraints on the normalisation of the
\ZHF background primarily originate from the dedicated CR. The normalisation of
the \ttbar background is mainly constrained in the SR of the \lephad SLT channel
and the \ZHF CR. Due to the determination of the normalisation of these
processes from the fit to data, any uncertainties on their overall
normalisation, for example uncertainties on the total cross sections, are
omitted. Uncertainties changing the relative acceptance of \ZHF or \ttbar events
between regions have to be considered instead. These uncertainties will be
referred to as \emph{relative acceptance uncertainties}, hereafter. The approach
outlined in the following was originally adopted by the previously published
analysis in this channel~\cite{HIGG-2016-16-witherratum} from searches and
measurements of $VH$~($H\to\bbbar$)
production~\cite{HIGG-2016-29,HIGG-2018-04,HIGG-2018-51}.

Assuming a general case of a background process that is estimated using
simulation and normalised by a fit to data in two analysis regions A and B, with
region A being defined as the reference region. In this case, the probability of
an event to be selected in a given region~R is given by the product of
acceptance and efficiency, $(\AccTimesEff)_{\text{R}}$, for this region. The
simulation predicts how \AccTimesEff relates between both regions and one can
define the ratio
\begin{align*}
  \mathcal{R} = \frac{(\AccTimesEff)_{\text{B}}}{(\AccTimesEff)_{\text{A}}} \,\text{,}
\end{align*}
which will be be referred to as the \emph{relative acceptance} between regions A
and B. Uncertainties on the modelling of $\mathcal{R}$ in simulation are
assigned as uncertainties on the normalisation of the background process in
region B\todo{Show mathematically?} and no additional uncertainties are assigned
in the reference region, A. They are estimated by performing variations of the
background prediction, for example using alternative generator configurations,
and estimating a relative change of~$\mathcal{R}$ according to
\begin{align}
  \frac{\Delta \mathcal{R}}{\mathcal{R}} = \frac{\mathcal{R}(\text{variation}) - \mathcal{R}(\text{nominal})}{\mathcal{R}(\text{nominal})} \,\text{.}
  \label{eq:relative_acceptance_uncertainty}
\end{align}
The use of the relative acceptance for the definition of the modelling
uncertainties leads to a cancellation of variations that lead to the same
relative change in \AccTimesEff in both regions, which corresponds to an overall
change in normalisation that can be absorbed into the freely floating
normalisation factor.

% This uncertainty on the relative acceptance between the the
% reference region A and region B is included in the background model
% by assigning a normalisation uncertainty of
% $\Delta \mathcal{R} / \mathcal{R}$ on the background in region B.

% The absolute value\footnote{When introducing multiple regions, the
% relative sign of $\Delta \mathcal{R} / \mathcal{R}$ when comparing
% different regions to the same reference is important when
% correlating these uncertainties in the statistical analysis and is
% kept to allow to consistently define the variations.} of
% $\Delta \mathcal{R} / \mathcal{R}$ is considered the extrapolation
% uncertainty.

% The shape effects of uncertainties are estimated separately and will be described in~\Cref{sec:modelling_uncertainties}.

% Note: Sign is important -> direction of effect when correlating systematics

The relative acceptance uncertainties are defined using the \ZHF CR as a
reference region, providing constraints on both the \ZHF and \ttbar
backgrounds. The uncertainties are estimated separately for the \ZHF and \ttbar
backgrounds in the SRs (\hadhad, \lephad SLT, \lephad LTT). Uncertainties
originating from the same source are assumed to be fully correlated between all
analysis regions.

The relative acceptance uncertainties originating from the modelling of the \ZHF
background in simulation are estimated by performing variations of the event
generation process.

\todo[inline]{TODO: reference \Cref{app:zjets_uncertainties}.}

Relative acceptance uncertainties on the \ZHF background are derived using these
variations according to~\Cref{eq:relative_acceptance_uncertainty} for the three
SRs. The uncertainties are summarised in~\Cref{tab:uncertainties_zhf_extrapol}
separately for all sources of uncertainty.

The \lephad SLT SR selection is the closest to the \ZHF CR, resulting in a small
relative acceptance uncertainty of approximately \SI{9}{\percent}. The \hadhad
and \lephad LTT channels deviate yielding an uncertainty of approximately
\SI{16}{\percent}. In all cases the dominant sources of uncertainty are the
scale variations as well as the comparison with
\MGNLO+\PYTHIA[8]\todo{``Interpretation'' needs some work.}.

The relative acceptance uncertainties on the \ttbar background are estimated
using a similar approach. The prescriptions documented in
Ref.~\cite{ATL-PHYS-PUB-2020-023} are used to estimate modelling uncertainties
on the nominal prediction obtained from simulation with
\POWHEGBOX[v2]+\PYTHIA[8]. A revised version of these prescriptions was
previously presented in~\Cref{sec:bkg_hadhad_ttbarfakes} as part of the
\ttbarFakes scale factor measurement. In the following, the prescriptions are
summarised, highlighting the differences with respect to the previous
description.

The uncertainty of the modelling of the hard interaction (parton shower) is
evaluated by replacing \POWHEGBOX[v2] (\PYTHIA[8]) with \MGNLO (\HERWIG[7]) for
the matrix element (parton shower) generation. Explicit variations of the
scales, and \PYTHIA[8] damping parameter, $\hdamp$, are omitted. Instead they
are included in the variations of the initial-state radiation. The uncertainties
on the amount of final-state radiation and the PDF+\alphas uncertainties are
estimated using internal re-weighting in \PYTHIA[8]. The relative acceptance
uncertainty for \ttbar backgrounds in the three SRs is given
in~\Cref{tab:uncertainties_ttbar_extrapol}.

\begin{table}[htbp]
  \centering

  \caption{Relative acceptance uncertainties on the \ZHF (a) and \ttbar
    background (b) in the three SRs. The relative sign of the effect of
    variations between the SRs is indicated by the ``$\pm$'' and ``$\mp$''
    prefixes. The total uncertainty is given for illustration of the size of the
    uncertainties only.}

  \begin{subtable}[t]{0.95\textwidth}
    \centering
    \subcaption{\ZHF ($Z+bb$, $Z+bc$, $Z+cc$) background}%
    \label{tab:uncertainties_zhf_extrapol}

    \begin{tabular}{lccc}
  \toprule
  & \multicolumn{3}{c}{Signal region} \\
  \cline{2-4}
  Uncertainty source & {\hadhad} & {\lephad SLT} & {\lephad LTT} \\
  \midrule
  \MGNLO+\PYTHIA[8] & $\pm 7.0\,\%$ & $\mp 2.1\,\%$ & $\mp 11\,\%$ \\[0.2em]
  Factorisation and renormalisation scale & $\substack{+12 \\ -9.7}\,\%$ & $\substack{+5.4 \\ -3.0}\,\%$ & $\substack{+8.5 \\ -3.0}\,\%$ \\[0.2em]
  Multi-jet merging (CKKW) & $\pm 5.4\,\%$ & $\pm 7.0\,\%$ & $\pm 7.2\,\%$ \\[0.2em]
  Parton shower resummation scale (QSF) & $\mp 6.0\,\%$ & $\pm 1.7\,\%$ & $\pm 1.6\,\%$ \\[0.2em]
  Alternative PDF sets & $\pm 1.0\,\%$ & $ \pm 1.0\,\%$ & $\pm 1.1\,\%$ \\[0.2em]
  PDF+\alphas (\NNPDF[3.0nnlo]) & $\pm 0.77\,\%$ & $\pm 0.27\,\%$ & $\pm 0.35\,\%$ \\
  \midrule
  Total & $\substack{+16\\-15}\,\%$ & $\substack{+9.3\\-8.1}\,\%$ & $\substack{+16\\-14}\,\%$ \\
  \bottomrule
\end{tabular}

%%% Local Variables:
%%% mode: latex
%%% TeX-master: "../phd_thesis"
%%% End:

  \end{subtable}

  \vspace{10pt}

  \begin{subtable}[t]{0.95\textwidth}
    \centering
    \subcaption{\ttbar background}%
    \label{tab:uncertainties_ttbar_extrapol}

    \begin{tabular}{lccc}
  \toprule
  & \multicolumn{3}{c}{Region} \\
  \cline{2-4}
  Uncertainty source & {\hadhad} & {\lephad SLT} & {\lephad LTT} \\
  \midrule
  ME & $\mp 3.8\,\%$ & $\mp 0.3\,\%$ & $\pm 0.9\,\%$ \\[0.2em]
  PS & $\pm 2.2\,\%$ & $\pm 7.2\,\%$ & $\pm 8.8\,\%$ \\[0.2em]
  ISR & $\mp 0.3\,\%$ & $\mp 0.9\,\%$ & $\pm 1.3\,\%$ \\[0.2em]
  FSR & $\substack{-4.5\\+2.0}\,\%$ & $\substack{-1.0\\+1.5}\,\%$ & $\substack{-3.2\\+1.0}\,\%$ \\[0.2em]
  PDF+\alphas & $\pm 0.2\,\%$ & $\pm 0.6\,\%$ & $\pm 0.8\,\%$ \\
  \midrule
  Total & $\substack{+4.8\\-6.3}\,\%$ & $\pm 7.4\,\%$ & $\substack{+9.0\\-9.4}\,\%$ \\
  \bottomrule
\end{tabular}

%%% Local Variables:
%%% mode: latex
%%% TeX-master: "../phd_thesis"
%%% End:

  \end{subtable}
\end{table}

Relative acceptance uncertainties are implemented as uncertainties on the
normalisation of the respective backgrounds in SRs only. Variations of the
modelling in simulation can also change the shapes of discriminants used for
signal extraction, including $\mll$ in the \ZHF CR, which have to be considered.

The impact of modelling uncertainties on distributions of the MVA discriminants
in the SRs and the \mll distribution in the \ZHF CR are investigated by
performing shape comparisons of the varied and nominal distributions. In cases
where the shapes do not differ significantly, no additional uncertainties are
assigned.  When deviations are observed, the shape uncertainties are propagated
to the discriminants and correlated with the corresponding normalisation
uncertainty.

Shape uncertainties on the \ZHF background are considered for the comparison
with the alternative event generation setup changing both the matrix element and
the parton shower generator (\hadhad), and for variations of the factorisation
and renormalisation scales (\hadhad, \lephad SLT \& LTT). All other variations,
including variations in the \ZHF CR, are found to have negligible impact on the
shape of the discriminating variables (\mll / MVA scores).

Shape uncertainties on the \ttbar background are considered for the comparison
with alternative matrix element generator and parton shower programs (\hadhad,
\lephad SLT), and the uncertainty on the amount of initial- and final-state
radiation (\hadhad). All other variations have negligible impact of the shapes
of the relevant distributions.


\subsubsection{Uncertainties on Minor Backgrounds}

The minor backgrounds considered by the analysis are normalised using cross
sections from theory and thus both cross section as well as acceptance
uncertainties are considered. In contrast to the major backgrounds, shape
uncertainties on minor backgrounds are neglected and only uncertainties on the
normalisation are considered (with the exception of the $tW$ acceptance
uncertainties).  A brief description of the uncertainties on minor backgrounds
is given in the following:
\begin{description}

\item[$Z + (bl, cl, ll)$] A normalisation uncertainty of \SI{5}{\percent} is
  assigned to account for the uncertainty on the predicted cross section of
  \Zjets production at NNLO~\cite{Anastasiou:2003ds} used to normalise the
  sample. Additionally, an acceptance uncertainty of \SI{23}{\percent} is
  adopted from Ref.~\cite{HIGG-2018-51}.

\item[$W + \text{jets}$] A \SI{5}{\percent} cross section uncertainty is
  assigned to the NNLO cross section prediction~\cite{Anastasiou:2003ds}. A
  normalisation uncertainty of \SI{37}{\percent} is assigned in the \lephad
  channels adopted from Ref.~\cite{HIGG-2018-51}. This uncertainty is inflated
  to \SI{50}{\percent} in the \hadhad channel since the small \Wjets
  contribution is not part of the data-driven \faketauhadvis estimation.

\item[Diboson] Normalisation uncertainties of \SI{20}{\percent},
  \SI{26}{\percent}, and \SI{25}{\percent} are applied to $ZZ$, $WZ$, and $WW$
  production, respectively. These uncertainties are adopted from
  Ref.~\cite{HIGG-2018-51}.

\item[Single top] Uncertainties on the cross section used to normalise the
  predictions are taken from Ref.~\cite{stopxsec}. A \SI{20}{\percent}
  acceptance uncertainty is assigned to the minor contribution of single top
  produced in $s$- and $t$-channel diagrams which is adopted from
  Ref.~\cite{HIGG-2018-51}.

  Acceptance uncertainties in the phase space selected by the analysis are
  derived for $tW$ production, which is the dominant source of single top
  background in this analysis. The acceptance uncertainties on the NLO+PS
  matching and the choice of parton shower program are estimated separately by
  comparison with alternative simulation setups using \MGNLO[2.6.2] and
  \HERWIG[7], respectively. Uncertainties on PDF+\alphas and the amount of
  initial- and final-state radiation are obtained by re-weighting of the nominal
  simulation result. Finally, an uncertainty on the treatment of the removal of
  interference\todo{What happens to the interference? Neglected?  Included in
    \ttbar?} between $tW$ at NLO and \ttbar production is estimated by
  performing MC-to-MC comparisons of the nominal diagram removal and diagram
  subtraction schemes~\cite{Frixione:2008yi}.

  The total acceptance uncertainty on $tW$ production is \SI{34}{\percent},
  \SI{14}{\percent}, \SI{23}{\percent} in the \hadhad, \lephad SLT, and \lephad
  LTT SR, respectively. The shape effect of the FSR and $tW$-$\ttbar$
  interference uncertainties on the MVA discriminants are propagated using
  suitable parameterisation.

\item[Single SM $H$ production] Uncertainties on the total cross section of the
  SM $H$ production modes considered as backgrounds are assigned according to
  the recommendations in Ref.~\cite{deFlorian:2016spz} for
  $m_{H} = \SI{125.0}{\GeV}$. Analogously, uncertainties on the branching ratios
  of $\PHiggs \to \tautau$ and $\PHiggs \to \bbbar$ are assigned to the relevant
  processes and are taken from Ref.~\cite{deFlorian:2016spz}.

  An acceptance uncertainty of \SI{100}{\percent} is assigned to $H \to \tautau$
  backgrounds produced via ggF, VBF, and $WH$ to account for difficulties in the
  modelling of produced in association with heavy flavour quarks\todo{Citation?}
  and the lack of a dedicated background estimate for single Higgs bosons
  production via $bbH$.

  Acceptance uncertainties in the phase space selected by the analysis are
  derived for single Higgs boson backgrounds produced via $ZH$, $\ttbar H$ for
  both $H \to \tautau$ and $H \to \bbbar$. They are derived by varying the
  parton shower program (\HERWIG[7]), PDF+\alphas, and factorisation /
  renormalisation scales and assigned as normalisation uncertainties when
  non-negligible. Additionally for $\ttbar H$ production, an uncertainty on the
  NLO+PS matching is assigned by comparing with an alternative generator
  (\MGNLO[v2.6.0]), and uncertainties targeting the modelling of initial- and
  final-state radiation.
\end{description}

\todo[inline]{No uncertainties on $ttW$ and $ttZ$?}

% \begin{table}[htbp]
%   \centering

%   \missingfigure[figwidth=0.75\textwidth]{Summary of bkg uncertainties.}

%   \caption{Summary of minor background uncertainties}
% \end{table}

\subsubsection{Signal Modelling Uncertainties: SM $HH$}

% CROSS SECTION
Uncertainties on the cross section of the SM $HH$ production in the ggF and VBF
production modes are considered when the signal strength,
$\mu = \sigma / \sigma_{\text{SM}}$, is used as a parameter of interest
(POI). When the total SM $HH$ cross section is the POI, the cross section
uncertainties are omitted.

The uncertainty on the cross section of SM $HH$ in the ggF production mode at
NNLO~\FTapprox~\cite{Grazzini:2018bsd} consists of an uncertainty on PDF+\alphas
of \SI{\pm 3.0}{\percent}~\cite{LHCHWGHH} and an uncertainty from scale
variations and the scheme used for the treatment of the mass of the virtual top
in the heavy-quark loop of $^{+6\%}_{-23\%}$~\cite{Baglio:2020wgt}. For the VBF
production mode, which is normalised using cross section at
N$^3$LO~\cite{Dreyer:2018qbw}, the scale uncertainties
are~$^{+0.03\,\%}_{-0.04\,\%}$~\cite{LHCHWGHH} and the uncertainty from PDF and
\alphas is~\SI{\pm 2.1}{\percent}~\cite{LHCHWGHH}.

Uncertainties on the branching ratios of $H \to \tautau$ and $H \to \bbbar$ are
taken from Ref.~\cite{deFlorian:2016spz} and assigned as uncertainties on the
signal normalisation.

Signal acceptance uncertainties are derived in all regions by performing
variations of the signal prediction.
% The variations are ensured to leave the total cross section
% invariant to prevent double-counting of cross section uncertainties.
The uncertainty on the choice of the parton shower program is estimated by
comparing the nominal signal generation using \PYTHIA[8.244] for parton
showering with an alternative using \HERWIG[7.1.6]. An uncertainty from the
effect of missing higher orders in the perturbative expansion is estimated by
performing six variations of the factorisation and renormalisation scales using
internal re-weighting of the generator. Uncertainties on the PDF sets are
estimated using the prescriptions for the
\PDFforLHC[15nlo]~\cite{Butterworth:2015oua} and
\NNPDF[3.0nlo]~\cite{Ball:2014uwa} PDF sets for ggF and VBF, respectively. The
PDF uncertainties are provided in the form of 30 eigenvariations for
\PDFforLHC[15nlo] and 100 replica sets for \NNPDF[3.0nlo]. Finally, the value of
$\alphas(\mZ^2)$ is varied up and down by 0.0015 (0.001) about the central value
of $0.118$ in the \PDFforLHC[15nlo] (\NNPDF[3.0nlo]) PDF set.

% The \PDF4LHC[15nlo] prescription uses 30
% eigenvariations of the PDF set that are independently propagated to
% the observables and the deviations from the nominal value are added in
% quadrature to yield the uncertainty. For \NNPDF[3.0nlo] PDF set which
% provides uncertainties in the form of MC replicas, the uncertainty is
% defined by the sample standard deviation of the observable determined
% using 100 replicas of the PDF set.

The acceptance uncertainties on the SM $HH$ signal are summarised
in~\Cref{tab:theory_uncertainty_signal} for the considered production modes. The
uncertainties on PDF+\alphas are found to be negligible in most cases and thus
omitted. The impact of the variations on the shape of the MVA discriminants is
found to be negligible in most cases. The variations of the factorisation and
renormalisation scales for the VBF production mode in the \hadhad SR is found to
induce small changes in the shape (up to \SI{2}{\percent} in bins sensitive to
the SM $HH$ signal) of the SM $HH$ BDT score distribution which is implemented
as a shape uncertainty for later statistical evaluation.

\begin{table}[htbp]
  \centering

  \todo[inline]{Why is there a dagger?}

  \caption{Theory uncertainties on the acceptance of non-resonant SM $HH$
    signals in the three SRs. Uncertainties marked as ``--'' are negligible.}%
  \label{tab:theory_uncertainty_signal}

  \begin{tabular}{
  p{0.4\textwidth}
  S[retain-explicit-plus, table-format=-1.1]
  S[retain-explicit-plus, table-format=-1.1]
  S[retain-explicit-plus, table-format=-1.1]
  }
  \toprule
  & \multicolumn{3}{c}{Acceptance uncertainty / \%} \\
  \cmidrule{2-4}
  Uncertainty source & {\hadhad} & {\lephad SLT} & {\lephad LTT} \\
  \midrule
  \multicolumn{4}{c}{SM~$HH$ (\ggF)} \\
  \midrule
  Parton shower simulation                & \pm 4.3 & \pm 7.6 & \pm 7.5 \\[0.35em]
  Factorisation and renormalisation scale & \pm 1.4 & \pm 1.2 & \pm 1.0 \\[0.35em]
  PDF+\alphas                             & {--} & {--} & {--} \\
  \midrule
  \multicolumn{4}{c}{SM~$HH$ (VBF)} \\
  \midrule
  Parton shower simulation                & \pm 3.0 & \pm 6.3 & \pm 2.1 \\[0.35em]
  Factorisation and renormalisation scale & \pm 0.1 & \pm 1.0 & \pm 1.0 \\[0.35em]
  PDF+\alphas                             & \pm 1.0 & {--} & {--} \\
  \bottomrule
\end{tabular}

%%% Local Variables:
%%% mode: latex
%%% TeX-master: "../phd_thesis"
%%% End:

\end{table}

\todo[inline]{Acceptance uncertainties are assumed to be fully correlated across
  channels but uncorrelated across production modes.}

\subsubsection{Signal Modelling Uncertainties: BSM $gg \to X \to HH$}

Cross section uncertainties are not considered for the production of heavy
scalar resonances in BSM scenarios since the cross
section~$\sigma(gg \to X \to HH)$ is used as the POI. However, uncertainties on
the SM Higgs boson branching ratios are considered according to
Ref.~\cite{deFlorian:2016spz}.

Acceptance uncertainties on BSM \HH production are estimated for the choice of
parton shower program, factorisation and renormalisation scales, and
PDF+\alphas. The uncertainties are calculated for a subset of resonance masses
considered in the analysis (due to the availability of alternative samples) and
are subsequently extrapolated to the full set.

The acceptance uncertainty from the choice of parton shower program is estimated
by replacing the default configuration using \HERWIG[7.1.3] with \PYTHIA[8.235]
for signals with six different resonance masses:
\begin{align*}
  \mX / \si{\GeV} \in \left\{ 251, 260, 280, 400, 500, 1000
  \right\} \,\text{.}
\end{align*}
The uncertainty is estimated by comparing the acceptance of signal-like events,
which for this purpose are defined as all events exceeding the
15\textsuperscript{th} percentile of the PNN score distribution for the default
signal generation setup, between both parton shower programs. This approach is
taken due to large changes in acceptance of (background-like) signal events at
low PNN score to which this search is not sensitive, thus avoiding an inflation
of the parton shower uncertainty at high PNN score which the analysis can be
sensitive to. The uncertainties evaluated at six points of \mX are linearly
interpolated and extrapolated ($\mX > \SI{1000}{\GeV}$) in \mX to estimate
uncertainties for signal masses for which no alternative samples were produced.

\Cref{fig:resonant_partonshower} illustrates the acceptance uncertainty
estimated from the comparison of parton shower programs. The alternative
configuration using \PYTHIA for parton showering predicts acceptances that are
slightly larger than the nominal configuration for the \hadhad and \lephad SLT
channels at low resonance masses. For $\mX = \SI{1000}{\GeV}$ \HERWIG predicts
larger signal acceptances in all three channels. The linear extrapolation in \mX
to the largest resonance mass of \SI{1600}{\GeV} that is considered leads to
acceptance uncertainties ranging from \SIrange{16}{19}{\percent}.

% \begin{table}[htbp]
%   \centering

%       % // Calculated for hadhad by Alessandra: 2021-11-30
    % // Calculated for SLT / LTT by Nicholas: 2021-12-03
    % // mX     hadhad      SLT        LTT
    % // 251    +0.046      +0.038     +0.039
    % // 260    +0.060      +0.037     -0.0010
    % // 280    +0.12       +0.029     +0.0077
    % // 400    +0.085      +0.051     -0.010
    % // 500    +0.060      +0.043     -0.013
    % // 1000   -0.053      -0.049     -0.082
\begin{tabular}{l
  S[retain-explicit-plus]
  S[retain-explicit-plus]
  S[retain-explicit-plus]}
  \toprule
  & \multicolumn{3}{c}{Acceptance uncertainty / \%} \\
  \cmidrule{2-4}
  $\mX / \si{\GeV}$ & {\hadhad} & {\lephad SLT} & {\lephad LTT} \\
  \midrule
  251 & +4.6 & +3.8 & +3.9 \\
  260 & +6.0 & +3.7 & -0.10 \\
  280 & +12 & +2.9 & +0.77 \\
  400 & +8.5 & +5.1 & -1.0 \\
  500 & +6.0 & +4.3 & -1.3 \\
  1000 & -5.3 & -4.9 & -8.2 \\
  1600 (extrapol.) & & & \\
  \bottomrule
\end{tabular}

%%% Local Variables:
%%% mode: latex
%%% TeX-master: "../phd_thesis"
%%% End:


%   \caption{Acceptance uncertainty on the BSM production of $gg \to X \to HH$
%   in the \hadhad, \lephad SLT, and \lephad LTT signal regions estimated using
%   variations of the parton shower program.}
% \label{tab:resonant_partonshower}
% \end{table}

\begin{figure}[htbp]
  \centering

  \includegraphics[scale=0.85]{uncertainties/resonant_ps_acc}

  \caption{Acceptance uncertainty on the BSM production of $gg \to X \to HH$ in
    the SRs estimated from a comparison with an alternative parton shower
    program. A positive sign of the uncertainty indicates that the alternative
    configuration (\PYTHIA) predicts a larger \AccTimesEff than the nominal one
    (\HERWIG) and vice versa for negatively signed uncertainties. The lines
    indicate the linear inter- / extrapolation in \mX used to obtain the
    uncertainties for all other points of \mX considered in the analysis (hollow
    markers).}
  \label{fig:resonant_partonshower}
\end{figure}

Acceptance uncertainties from factorisation and renormalisation scales, and
PDF+\alphas are evaluated at generator-level for two mass points with
$\mX = \SI{500}{\GeV}$ and \SI{1000}{\GeV} after approximating the selections
applied in the analysis. The uncertainties are found to be negligible and are
therefore omitted.

An additional uncertainty is assigned in the \hadhad and \lephad LTT channels to
signal samples using fast simulation with
\textsc{ATLFAST-II}~\cite{SOFT-2010-01}. The efficiency of \tauhadvis triggers
is found to deviate between full and fast simulation, without dedicated
calibrations of \tauhadvis trigger efficiencies in fast simulation being
available. An uncertainty is estimated by comparing the acceptance between full
and fast simulation for a benchmark sample at a mass of \SI{400}{\GeV}. The
acceptance predicted using fast simulation is \SI{6.5}{\percent}
(\SI{3.6}{\percent}) larger than in full simulation for the \hadhad (\lephad
LTT) SR for the benchmark point. Distributions of kinematic and MVA input
variables are compared between fast and full simulation showing no significant
deviations in their shapes. Therefore, the difference in acceptance is assigned
as an additional normalisation uncertainty in the \hadhad and \lephad LTT
channel for all signal samples simulated using \textsc{ATLFAST-II}~(i.e.\
$\mX \leq \SI{1000}{\GeV}$).

%%% Local Variables:
%%% mode: latex
%%% TeX-master: "../../phd_thesis"
%%% End:
