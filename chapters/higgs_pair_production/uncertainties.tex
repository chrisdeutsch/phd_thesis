\section{Systematic Uncertainties}
\label{sec:uncertainties}

\subsection{Experimental Uncertainties}

\todo[inline]{Make this a 'description'?}

The uncertainty on the integrated luminosity of the $pp$ collision
dataset collected with the ATLAS detector in the period from 2015 to
2018 is \SI{1.7}{\percent}~\cite{ATLAS-CONF-2019-021}. This
uncertainty is applied to all processes estimated using simulation
that are normalised using theory cross sections.

An uncertainty on the reweighting of the instantaneous luminosity
distribution used for the pile-up overlay in simulated data to match
the conditions observed in data is considered.

Uncertainties on electrons in simulation are obtained from dedicated
calibration measurements~\cite{EGAM-2018-01,TRIG-2018-05} updated for
the \SI{139}{\per\femto\barn} Run~2 dataset. The electron energy scale
and resolution is measured in $Z \to e^+e^-$ events and propagated to
predictions by shifting and smearing momenta of electrons in
simulation. Uncertainties on the calibration of electron
reconstruction, identification, isolation, and trigger efficiencies in
simulation are obtained from measurements in~$J/\psi \to e^+e^-$ and
$Z \to e^+e^-$ events and applied as alternative weights to electrons
in simulation.

Muons... \cite{MUON-2018-03} \\
Scale, (Sagitta) , (ID? MS?) \\
Efficiencies: Reconstruction, Isolation, Trigger

Taus... \\
Scale \\
Efficiencies: Reconstruction, Identification, eVeto (taus and
electrons reconstructed as taus), Trigger

Jets... (JVT, JES, JER) -- JES/JER \cite{JETM-2018-05}

MET

Flavour tagging... \cite{FTAG-2018-01}

\begin{table}[htbp]
  \centering

  \begin{tabular}{lS[table-format=2]}
  \toprule
  Uncertainty source & {Components} \\
  \midrule
  Integrated luminosity & 1 \\
  Pile-up reweighting & 1 \\[0.5em]
  \textbf{Electrons} & \\
  Energy scale \& resolution & 3 \\
  Efficiencies (Reco., ID, Isol., Trigger) & 4 \\
  \tauhadvis $e^\pm$-veto efficiency & 2 \\[0.5em]
  \textbf{Muons} & \\
  Energy scale, ... & 5 \\
  Efficiencies (Reco., Isol., TTVA, Trigger) & 10 \\[0.5em]
  \textbf{Taus} & \\
  Energy scale & 4 \\
  Efficiencies (Reco., ID, $e$-Veto, Trigger) & 32 \\[0.5em]
  \textbf{Jets} & \\
  Energy scale & 33 \\
  Energy resolution & 14 \\
  Efficiency (jet vertex tagger) & 1 \\[0.5em]
  \textbf{Flavour tagging} & \\
  Efficiences (tag and mis-tag) & 13 \\[0.5em]
  \textbf{\pTmiss} & \\
  Momentum scale \& resolution & 3 \\
  \bottomrule
\end{tabular}

%%% Local Variables:
%%% mode: latex
%%% TeX-master: "../phd_thesis"
%%% End:


  \caption{Table of CP uncertainties}
  \label{tab:bla}
\end{table}

\todo[inline]{Where applicable, dedicated calibrations for ATLAS fast
  simulation are used.}


\subsubsection{Experimental?}


\todo[inline]{Split in experimental \& theory uncertainties}

\todo[inline]{Experimental: Jets -- JES, JER, JVT, b-tag}

\todo[inline]{Experimental: TAUS -- TES, Reco, ID}

\todo[inline]{Experimental: electrons / muons -- ?}

\todo[inline]{Experimental: MET -- all momentum / energy uncertainties are
  propagated to MET, additionally uncertainties related to the soft term}


\subsection{Uncertainties on the relative acceptance of \ZHF and
  \ttbar between regions}

The normalisation of the \ZHF and \ttbar backgrounds are measured in
the combined likelihood fit of control and signal regions. Constraints
on the normalisation of the \ZHF background primarily originate from
the dedicated control region. The \ttbar background normalisation is
constrained in the signal region of the \lephad SLT channel and the
\ZHF control region. Due to the determination of the normalisation of
these processes from the fit to data, any uncertainties on the overall
normalisation of these processes, for example uncertainties on the
total cross sections, are omitted. Uncertainties changing the relative
acceptance of \ZHF or \ttbar events between regions have to be
considered instead. These uncertainties will be referred to as
\emph{relative acceptance uncertainties}, hereafter. The approach
outlined in the following was originally adopted by the previously
published analysis in this channel~\cite{HIGG-2016-16-witherratum}
from searches and measurements of
$VH$~($H\to\bbbar$)~\cite{HIGG-2016-29,HIGG-2018-04,HIGG-2018-51}.

Relative acceptance uncertainties are defined with respect to a
reference region, for which the \ZHF control region is chosen due to
its ability to constrain the normalisation of both the \ZHF and the
\ttbar backgrounds. In non-reference regions normalisation
uncertainties are assigned to \ZHF and \ttbar backgrounds to account
for uncertainties in the modelling of the acceptance of these
processes in simulation relative to the reference region. This
introduces additional degrees of freedom in the background model by
allowing the normalisation factors of the background templates in
non-reference regions to deviate from the normalisation factors in the
reference region within constraints.

Consider a general case of a physics process that is estimated using
simulation and normalised by a fit to observed data in two analysis
regions A and B, with region A being defined as the reference
region. The probability of an event originating from the process to be
recorded in a given region is given by the product of the acceptance
and efficiency, $(\AccTimesEff)_{\text{R}}$, in a given region~R. The
nominal result of the simulation predicts how \AccTimesEff relates
between both regions and one can define the ratio
\begin{align*}
  \mathcal{R} = \frac{(\AccTimesEff)_{\text{B}}}{(\AccTimesEff)_{\text{A}}} \,\text{,}
\end{align*}
which will be be referred to as the \emph{relative acceptance} between
regions A and B. Uncertainties on the modelling of $\mathcal{R}$ in
simulation are assigned as additional uncertainties on the
normalisation of the background in region B\todo{Show
  mathematically?}. These uncertainties are estimated by performing
variations of the background prediction, for example by using
alternative generator setups, and estimating a relative change
of~$\mathcal{R}$ according to
\begin{align*}
  \frac{\Delta \mathcal{R}}{\mathcal{R}} = \frac{\mathcal{R}(\text{variation}) - \mathcal{R}(\text{nominal})}{\mathcal{R}(\text{nominal})} \,\text{.}
\end{align*}
% This uncertainty on the relative acceptance between the the reference
% region A and region B is included in the background model by assigning
% a normalisation uncertainty of $\Delta \mathcal{R} / \mathcal{R}$ on
% the background in region B.

% The absolute value\footnote{When introducing multiple regions, the
%   relative sign of $\Delta \mathcal{R} / \mathcal{R}$ when comparing
%   different regions to the same reference is important when
%   correlating these uncertainties in the statistical analysis and is
%   kept to allow to consistently define the variations.} of
% $\Delta \mathcal{R} / \mathcal{R}$ is considered the extrapolation
% uncertainty.

% The shape effects of uncertainties are estimated separately and will be described in~\Cref{sec:modelling_uncertainties}.

% Note: Sign is important -> direction of effect when correlating systematics

The relative acceptance uncertainties are estimated separately for the
\ZHF and \ttbar backgrounds for all three signal regions (\hadhad,
\lephad SLT, \lephad LTT) with respect to the \ZHF control
region. Uncertainties originating from the same source are assumed to
be fully correlated.

The following variations of the modelling of the \ZHF using simulation
are considered. The approach follows prescriptions developed by the
ATLAS collaboration:
\begin{description}
\item[Renormalisation and factorisation scales] Six variations of the
  renormalisation and factorisation scales are performed using
  internal reweighting implemented in \SHERPA[2.2.1], altering the
  scales by factors of $\frac{1}{2}$ and $2$. The following variations
  are considered:
  \begin{align*}
    \left( \frac{\muR}{\muR^{\text{nom.}}}, \frac{\muF}{\muF^{\text{nom.}}} \right) \in
    \left\{ (\tfrac{1}{2}, \tfrac{1}{2}), (\tfrac{1}{2}, 1), (1, \tfrac{1}{2}), (1, 2), (2, 1), (2, 2) \right\} \,\text{,}
  \end{align*}
  where $\muR^{\text{nom.}}$ and $\muF^{\text{nom.}}$ are the nominal
  values of the scales.

\item[Resummation scale] The resummation scale \todo{more explanation}
  is varied by factors $\frac{1}{2}$ and 2. The variation is provided
  as a reweighting of event weights by the
  collaboration~\cite{anders:2017} and is parametrised in the
  transverse momentum of the \PZ boson and the number of jets
  (anti-$k_{t}$ $R = 0.4$ with $\pT > \SI{20}{\GeV}$) at truth-level.

\item[Multi-jet merging] The \Vjets samples produced with
  \SHERPA[2.2.1] uses matrix elements of NLO accuracy for up to two
  jets and LO for up to four jets. The different jet multiplicities
  are merged using an extended CKKW
  algorithm~\cite{Catani:2001cc,Hoeche:2009rj}. The merging
  scale\footnote{What does it mean?}, which is a parameter of the
  algorithm, is varied from \SI{20}{\GeV} to \SI{30}{\GeV} and
  \SI{15}{\GeV}, respectively.

  The variations of the merging scale are also provided as weights to
  be applied to the nominal event generation following the approach
  for variations of the resummation scale.

\item[PDF+\alphas] Uncertainties on the \NNPDF[3.0nnlo] set of
  PDFs~\cite{Ball:2014uwa} are evaluated using 100 replica sets
  provided through the \textsc{LHAPDF6}
  library~\cite{Buckley:2014ana}. Similarly, the \NNPDF[3.0nnlo] PDF
  sets with $\alphas(Q^2=\mZ^2) = 0.117$ and $0.119$ are compared to
  the nominal value of $0.118$. Finally, an uncertainty from the
  choice of PDF set is considered by comparing with two alternative
  PDF sets, \MMHT[nnlo68cl]~\cite{Harland-Lang:2014zoa} and
  \CT[14nnlo]~\cite{Dulat:2015mca}.

\item[Alternative generator and parton shower] The nominal prediction
  obtained with \SHERPA[2.2.1] (\OPENLOOPS and Comix) is compared to
  an alternative setup using~\MGNLO[2.2.2]~\cite{Alwall:2014hca} for
  the calculation of LO matrix elements interfaced
  to~\PYTHIA[8.186]~\cite{Sjostrand:2007gs} for parton showering.

\end{description}

\begin{table}[htbp]
  \centering

  \begin{tabular}{lccc}
  \toprule
  & \multicolumn{3}{c}{Signal region} \\
  \cline{2-4}
  Uncertainty source & {\hadhad} & {\lephad SLT} & {\lephad LTT} \\
  \midrule
  \MGNLO+\PYTHIA[8] & $\pm 7.0\,\%$ & $\mp 2.1\,\%$ & $\mp 11\,\%$ \\[0.2em]
  Factorisation and renormalisation scale & $\substack{+12 \\ -9.7}\,\%$ & $\substack{+5.4 \\ -3.0}\,\%$ & $\substack{+8.5 \\ -3.0}\,\%$ \\[0.2em]
  Multi-jet merging (CKKW) & $\pm 5.4\,\%$ & $\pm 7.0\,\%$ & $\pm 7.2\,\%$ \\[0.2em]
  Parton shower resummation scale (QSF) & $\mp 6.0\,\%$ & $\pm 1.7\,\%$ & $\pm 1.6\,\%$ \\[0.2em]
  Alternative PDF sets & $\pm 1.0\,\%$ & $ \pm 1.0\,\%$ & $\pm 1.1\,\%$ \\[0.2em]
  PDF+\alphas (\NNPDF[3.0nnlo]) & $\pm 0.77\,\%$ & $\pm 0.27\,\%$ & $\pm 0.35\,\%$ \\
  \midrule
  Total & $\substack{+16\\-15}\,\%$ & $\substack{+9.3\\-8.1}\,\%$ & $\substack{+16\\-14}\,\%$ \\
  \bottomrule
\end{tabular}

%%% Local Variables:
%%% mode: latex
%%% TeX-master: "../phd_thesis"
%%% End:


  \caption{Uncertainties on \ZHF with reference \ZHF control region.}
  \label{tab:uncertainties_zhf_extrapol}
\end{table}


\begin{table}[htbp]
  \centering

  \begin{tabular}{lccc}
  \toprule
  & \multicolumn{3}{c}{Signal region} \\
  \cline{2-4}
  Uncertainty source & {\hadhad} & {\lephad SLT} & {\lephad LTT} \\
  \midrule
  ME & $\mp 3.8\,\%$ & $\mp 0.3\,\%$ & $\pm 0.9\,\%$ \\[0.2em]
  PS & $\pm 2.2\,\%$ & $\pm 7.2\,\%$ & $\pm 8.8\,\%$ \\[0.2em]
  ISR & $\mp 0.3\,\%$ & $\mp 0.9\,\%$ & $\pm 1.3\,\%$ \\[0.2em]
  FSR & $\substack{-4.5\\+2.0}\,\%$ & $\substack{-1.0\\+1.5}\,\%$ & $\substack{-3.2\\+1.0}\,\%$ \\[0.2em]
  PDF+\alphas & $\pm 0.2\,\%$ & $\pm 0.6\,\%$ & $\pm 0.8\,\%$ \\
  \midrule
  Total & $\substack{+4.8\\-6.3}\,\%$ & $\pm 7.4\,\%$ & $\substack{+9.0\\-9.4}\,\%$ \\
  \bottomrule
\end{tabular}

%%% Local Variables:
%%% mode: latex
%%% TeX-master: "../phd_thesis"
%%% End:


  \caption{Uncertainties on \ttbar with reference \ZHF control region.}
  \label{tab:uncertainties_ttbar_extrapol}
\end{table}




\subsection{Modelling Uncertainties}
\label{sec:modelling_uncertainties}

\todo[inline]{Theory: PDF, \alphas, renormalization and factorization scale}

\todo[inline]{Theory: simualted processes not normalized in data \ra
  uncertainties on theory xsec is applied}

\todo[inline]{Theory: V+jets -- resummation, CKKW matching}

\todo[inline]{Theory: \ttbar -- ME, PS, ISR, FSR -- Generally good
  modelling in bulk of phase space but can be problematic in tails /
  phase space corners.}

\todo[inline]{Theory: Signal -- ?}


%%% Local Variables:
%%% mode: latex
%%% TeX-master: "../../phd_thesis"
%%% End:
