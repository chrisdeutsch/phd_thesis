\section{Systematic Uncertainties}

\todo[inline]{Split in experimental \& theory uncertainties}

\subsection{Experimental Uncertainties}

\todo[inline]{Experimental: Jets -- JES, JER, JVT, b-tag}

\todo[inline]{Experimental: TAUS -- TES, Reco, ID}

\todo[inline]{Experimental: electrons / muons -- ?}

\todo[inline]{Experimental: MET -- all momentum / energy uncertainties are
  propagated to MET, additionally uncertainties related to the soft term}

\subsection{Extrapolation Uncertainties}

The normalization for the Z+HF and \ttbar backgrounds are measured in dedicated
control regions. An extrapolation uncertainty from the control region A to a
signal region B is estimated using the relative acceptance times efficiency between A and B:
\begin{align*}
  \mathcal{R} = \frac{ \left( \mathcal{A} \times \varepsilon \right)_\text{B}}{\left( \mathcal{A} \times \varepsilon \right)_\text{A}} = \frac{ \left( \sigma \times \mathcal{A} \times \varepsilon \right)_\text{B}}{\left( \sigma \times \mathcal{A} \times \varepsilon \right)_\text{A}} = \frac{N_\text{B}}{N_\text{A}}
\end{align*}
The uncertainty on $\mathcal{R}$ is used as an extrapolation uncertainty and is
estimated as follows:
\begin{align*}
  \text{Relative extrapolation uncertainty} = \frac{\mathcal{R}_\text{var} - \mathcal{R}_\text{nom}}{\mathcal{R}_\text{nom}}
\end{align*}
where the relative acceptance times efficiency is calculated for the nominal
$\mathcal{R}_\text{nom}$ and variation $\mathcal{R}_\text{var}$.

\todo[inline]{Theory: PDF, \alphas, renormalization and factorization scale}

\todo[inline]{Theory: simualted processes not normalized in data \ra
  uncertainties on theory xsec is applied}

\todo[inline]{Theory: V+jets -- resummation, CKKW matching}

\todo[inline]{Theory: \ttbar -- ME, PS, ISR, FSR}

\todo[inline]{Theory: Signal -- ?}


%%% Local Variables:
%%% mode: latex
%%% TeX-master: "../../phd_thesis"
%%% End:
