\section{Event Selection}
\label{sec:event_selection}

An event selection is applied to select events consistent with a \bbtthh final
state. Due to the chosen MVA analysis strategy, the selection does not aim to
reduce backgrounds significantly.

\subsection{Event Cleaning}

\todo[inline]{Primary vertex requirements \& jet cleaning}

\todo[inline]{Muon cleaning / muons from cosmics? Are we applying this?}

\todo[inline]{Analysis specific object counts \& thresholds from trigger}

\todo[inline]{Cuts aiming to reduce backgrounds, e.g.\ $m_{\tau\tau} >
  \SI{60}{\GeV}$ to reduce DY}

\todo[inline]{SR selection (not really applicable -- maybe talk about PNN)??? Anti-Tau CR}

\todo[inline]{b-jet energy correction}

\todo[inline]{Observables used to reduce backgrounds: dRTauTau, dRBB, mtautau,
  mbb, mHH}

\subsection{Trigger}
\label{sec:trigger}

\todo[inline]{Trigger \& trigger evolution}

In the \tauhad\tauhad channel, events are selected using single (STT) and di-\tauhad
triggers (DTT). The list of triggers used can be found in \cref{tab:triggers_hadhad}.

\begin{table}[h]
  \centering

  \scriptsize

  \begin{tabular}{ll}
    \toprule
    Trigger & Period \\

    \midrule
    \multicolumn{2}{c}{Single \tauhad triggers (STT)} \\
    \midrule

    HLT\_tau80\_medium1\_tracktwo\_L1TAU60 & 15 -- 16 A \\
    HLT\_tau125\_medium1\_tracktwo & 16 B -- 16 D3 \\
    HLT\_tau160\_medium1\_tracktwo & 16 D4 -- 17 B4 \\
    HLT\_tau160\_medium1\_tracktwo\_L1TAU100 & 17 B5 -- 17 end \\
    HLT\_tau160\_medium1\_tracktwoEF\_L1TAU100 & 18 -- \\
    HLT\_tau160\_mediumRNN\_tracktwoMVA\_L1TAU100 & 18 K -- \\

    \midrule
    \multicolumn{2}{c}{Di-\tauhad triggers (DTT)} \\
    \midrule

    HLT\_tau35\_medium1\_tracktwo\_tau25\_medium1\_tracktwo\_L1TAU20IM\_2TAU12IM & 15 \\
    HLT\_tau35\_medium1\_tracktwo\_tau25\_medium1\_tracktwo & 16 -- 17 B4 \\
    HLT\_tau35\_medium1\_tracktwo\_tau25\_medium1\_tracktwo\_L1TAU20IM\_2TAU12IM\_4J12 & 17 \\
    HLT\_tau35\_medium1\_tracktwo\_tau25\_medium1\_tracktwo\_L1DR-TAU20ITAU12I-J25 & 17 B5 -- 17 end \\
    HLT\_tau35\_medium1\_tracktwoEF\_tau25\_medium1\_tracktwoEF\_L1TAU20IM\_2TAU12IM\_4J12.0ETA23 & 18 -- \\
    HLT\_tau35\_medium1\_tracktwoEF\_tau25\_medium1\_tracktwoEF\_L1DR-TAU20ITAU12I-J25 & 18 -- \\
    HLT\_tau35\_mediumRNN\_tracktwoMVA\_tau25\_mediumRNN\_tracktwoMVA\_L1TAU20IM\_2TAU12IM\_4J12.0ETA23 & 18 K -- \\
    HLT\_tau35\_mediumRNN\_tracktwoMVA\_tau25\_mediumRNN\_tracktwoMVA\_L1DR-TAU20ITAU12I-J25 & 18 K -- \\

    \bottomrule
  \end{tabular}

  \caption{Triggers used for data taking in the \tauhad\tauhad channel.}
  \label{tab:triggers_hadhad}
\end{table}

Priority is given to STT events if the reconstructed \tauhad fulfil the
\pT-threshold of the trigger (\SI{100}{\GeV} for \verb|tau80|, \SI{140}{\GeV}
for \verb|tau125| and \SI{180}{\GeV} for \verb|tau160|) and are geometrically
matched to the HLT object that fired the trigger.

If the event does not fulfil the STT criteria, then the DTT is checked. The
\pT-thresholds for \tauhadvis is \SI{40}{\GeV} (\SI{30}{\GeV}) for the leading
(subleading) \tauhadvis candidate.

For three runs (336506, 336548, 336567) during 2017 data taking, L1Topo-based
triggers were mistakenly disabled in the trigger firmware also affecting the
\verb|L1DR-TAU20ITAU12I-J25| trigger. As a backup the almost unprescaled
\verb|HLT_tau35_medium1_tracktwo_tau25_medium1_tracktwo| trigger was used.

In 2017 / 2018 two different di-\tauhad triggers with different L1 seeds are
used. The L1 seeds are \verb|L1TAU20IM_2TAU12IM_4J12| and
\verb|L1TAU20IM_2TAU12IM_4J12.0ETA23| (4J12) and \verb|L1DR-TAU20ITAU12I-J25|
(L1Topo) and differ in the requirements on \tauhadvis and additional jets.

The 4J12 trigger requires two\footnote{Two \tauhadvis already satisfy 2J12}
additional jets at L1 with $\ET > \SI{12}{\GeV}$. Additionally, in 2018 the jets
are required to be in $|\eta| < 2.3$. The L1Topo trigger uses the ATLAS
topological trigger introduced in 2017 to require a $\Delta R(\tauhad, \tauhad)
< 2.8$ on both \tauhad as well as one additional jet with $\ET > \SI{25}{\GeV}$
at L1.

Orthogonality (and being on plateau of the trigger turn-on) between the 4J12 and
L1Topo trigger channel is ensured by offline cuts discussed in
Section.%~\ref{subsec:selhh_hadhad}.

During TS1 new \tauhadvis triggers employing RNN-based \tauhad identification
and MVA-based energy calibration were deployed. Starting from Period K, the
recommendation is to use a logical OR between the old
(\verb|medium1_tracktwoEF|) and new triggers (\verb|mediumRNN_tracktwoMVA|).


\subsection{Selection of $\pp \ra hh \ra \bbbar\tauhad\tauhad$ Events}


\subsubsection{Preselection}

Trigger selection follows \cref{sec:trigger}

\subsubsection{Signal Region}

The SR selection is as follows:
\begin{itemize}
\item Exactly two reconstructed \tauhadvis passing \textit{loose} identification
  (RNN)

\item Unit electric charge of \tauhadvis candidate with opposite sign as
  reconstructed from the tracks associated to the \tauhadvis candidates

\item Two or more jets

\item Exactly two \btagged jets with the \SI{77}{\percent} working point of the
  DL1r tagger

\item No reconstructed and identified electrons and muons in the event

\item Passing the trigger selection described in \cref{sec:trigger}

\item The \tauhadvis are geometrically matched to objects that fired either the
  STT or the DTT trigger.

\item \mMMC > \SI{60}{\GeV}

\item At least one \bjet with \pT > \SI{45}{\GeV} \todo{Can we drop this?}
\end{itemize}


Orthogonality between the L1Topo and the 4J12 DTT is ensured by offline cuts on
\tauhadvis and jet \pT:
\begin{itemize}
\item STT events: \tauhadvis are required to pass a trigger-dependent \pT
  threshold as described in \cref{sec:trigger}. STT events are prioritised over
  DTT events.

\item DTT events in 2015-2016: The leading jet is required to have \pT >
  \SI{80}{\GeV} due to the additional jet requirement (J25) at L1 in 2016. This
  ensures that the L1 trigger is close to its efficiency plateau, minimizing the
  impact of a mismatch in trigger efficiency in data / MC.

\item DTT events in 2017-2018: Are categorised into two categories depending on
  the triggers to be checked:\todo{Chris: need to check treatment of non-L1Topo
    trigger at the beginning of 2017}
  \begin{itemize}
  \item If leading and subleading jet $\pT > \SI{45}{\GeV}$: Event is considered a
    4J12 event. This cut ensures that the additional jet requirement at L1 (J12)
    is fulfilled.
  \item Else if leading jet $\pT > \SI{80}{\GeV}$ and $\dRtautau \leq 2.5$:
    Event is considered a L1Topo event. The jet \pT and \dRtautau cut ensure
    that the additional J25 and \dRtautau requirement at L1 are fulfilled.
  \end{itemize}
\end{itemize}


\todo[inline]{There is a \SI{23}{\GeV} subleading \tauhadvis \pT cut at
  derivation level which will affect STT events.}

\todo[inline]{Acceptance times efficiency plot for resonant analysis.}

\todo[inline]{Di-Tau Mass Reconstruction}

\todo[inline]{B-Jet Corrections}


%%% Local Variables:
%%% mode: latex
%%% TeX-master: "../../phd_thesis"
%%% End:
