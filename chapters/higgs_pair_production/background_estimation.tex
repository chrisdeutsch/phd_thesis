\section{Background Estimation}%
\label{sec:background_estimation}

The dominant background processes in the search for Higgs boson pair production
in $\bbtautau$ final states are \Zjets, \ttbar, and backgrounds with quark- or
gluon-initiated jets that are misidentified as \tauhadvis (\faketauhadvis
backgrounds).
% Backgrounds with \faketauhadvis primarily originate from multi-jet, \Wjets,
% and \ttbar processes.
Minor backgrounds originate from single-top, $\ttbar V$, diboson, and single
Higgs boson production.  The single Higgs boson production modes considered in
this search are \ggF, VBF, \VH and \ttH for $H \to \tautau$; $\VH$ and $\ttH$
for $H \to \bbbar$. Single Higgs boson production via $\bbbar H$ is found to be
negligible.
% While single Higgs boson production is classified as a minor background here,
% it has relevance to the extraction of \HH signals due to their similar
% signatures.

The \Zjets and \ttbar backgrounds are estimated using templates obtained from
simulation with their normalisations being determined in a simultaneous
likelihood fit to observed data in SRs and CRs. A CR enriched in $Z$~boson
production in association with jets from quarks of heavy flavour (\ZHF) is
defined in \Cref{sec:bkg_zjets} providing constraints on the normalisation of
this background. The normalisation of the \ttbar background is constrained by
the inclusion of the SR of the \lephad SLT channel and the \ZHF CR in the
simultaneous fit, both having a measurable contribution of events from \ttbar
production.
% The signal region of the \lephad SLT channel serves to constrain the
% normalisation of the \ttbar background as it selects \ttbar with high purity.

Major \faketauhadvis backgrounds are estimated using (semi-)data-driven methods,
while minor ones are estimated using simulation. In the \hadhad channel, the
primary sources of \faketauhadvis are \ttbar and multi-jet production for which
separate estimation techniques are used. The \faketauhadvis background from
\ttbar (\ttbarFakes) is estimated using simulation after applying corrections of
\jettotauhadvis misidentification efficiencies measured in a CR. This method is
described in \Cref{sec:bkg_hadhad_ttbarfakes}. The multi-jet background is
estimated using a fully data-driven \emph{fake factor method} that is introduced
in \Cref{sec:bkg_hadhad_ff}. Both methods were developed as part of this thesis
and differ from the approach adopted in the previous publication in this
channel~\cite{HIGG-2016-16-witherratum}. Lastly, the estimation of the
\faketauhadvis backgrounds in the \lephad channels uses a \emph{combined fake
  factor method} that simultaneously estimates \faketauhadvis backgrounds from
multi-jet and \ttbar processes. This method is briefly summarised
in~\Cref{sec:bkg_lephad_combined_ff}.

Minor background contributions are estimated using simulation and are normalised
to the integrated luminosity of the \pp~collision dataset using the highest
order cross-section predictions available. These backgrounds are not discussed
in detail in this section, however, theoretical uncertainties on the modelling
of these processes using simulation are discussed in
\Cref{sec:modelling_uncertainties}.

\subsection{Associated Production of $\PZ \to \ell\ell$ with Quarks of Heavy Flavour}%
\label{sec:bkg_zjets}
% References:
%
% PMG weak boson wiki
% https://twiki.cern.ch/twiki/bin/view/AtlasProtected/PmgWeakBosonProcesses#Normalisation_discrepancies_due
%
% Differential cross-sections for Z + b-jets at 13 TeV
% https://link.springer.com/content/pdf/10.1007/JHEP07(2020)044.pdf

The production of \PZ bosons in association with jets is estimated
using events simulated with \SHERPA[2.2.1]~\cite{Bothmann:2019yzt}
interfaced to the matrix element generators
\OPENLOOPS~\cite{Buccioni:2019sur,Cascioli:2011va,Denner:2016kdg} and
Comix~\cite{Gleisberg:2008fv} (cf.\ \Cref{tab:monte_carlo} in
\Cref{sec:data_and_simulation}). This generator configuration merges
hard-scatter matrix elements at NLO for final states with up to two
partons with matrix elements at LO for up to four partons. Prior to
the fit, inclusive \Zjets cross sections at
NNLO~\cite{Anastasiou:2003ds} are used for the normalisation of the
background prediction.

% Why IS HF difficult?
% Theory context here: https://arxiv.org/pdf/2204.12355.pdf
The requirement of having two \btagged jets in the signal regions
leads to an enhancement of \PZ bosons produced in association with
quarks of heavy flavour. The modelling of the \ZHF background is
difficult due to its sensitivity to the flavour structure of the
proton and to gluons splitting to bottom or charm
quarks~\cite{Maltoni:2012pa,Napoletano:2018euk,Napoletano:2019tla}. The
nominal prediction of the \ZHF background with \SHERPA, which employs
a five flavour number scheme\footnote{TODO: Explain} for the treatment
of $b$-quarks in the proton, is known to underestimate the \ZHF
contribution~\cite{STDM-2017-38}.
% by \SIrange{10}{30}{\percent} depending on the selected phase
% space~\cite{STDM-2017-38}.
Therefore, the normalisation of the \ZHF background is measured in a
dedicated control region.

% Instead of relying on the normalisation predicted by simulation, the
% normalisation of the \ZHF background is measured in a dedicated
% control region targeting $Z + bb$ production which is then
% extrapolated to the signal regions.

The approach of estimating the \ZHF background described in the
following is adopted with few modifications~\cite{bokan} from the
previous publication in this channel~\cite{HIGG-2016-16-witherratum}
which built on findings from searches of $VH$ ($\PHiggs \to \bbbar$)
production~\cite{HIGG-2016-29}.

% Control region definition
A dedicated control region is defined targeting the production of
$\PZ \ra \Plp\Plm$ ($\ell = e , \mu$) in association with
$b$-jets. The definitions of selected objects used in the control
region and requirements on event quality criteria remain the same as
previously described
in~\Cref{sec:object_reconstruction,sec:event_selection} for
consistency with the signal region selection. Events with same flavour
lepton pairs are recorded using single- and di-lepton
triggers. Thresholds are applied to the \pT of electrons and muons
after offline reconstruction to ensure that the triggers operate close
to their trigger efficiency plateau. Depending on the run conditions
of the LHC, the \pT-thresholds range from \SIrange{25}{27}{\GeV} for
single-electron and \SIrange{21}{28}{\GeV} for single-muon
triggers. Events selected by di-electron triggers need to pass
symmetric \pT-thresholds on both the leading and sub-leading electron
ranging from \SIrange{13}{25}{\GeV}. Events selected by di-muon
triggers are required to pass asymmetric thresholds of
\SIrange{19}{24}{\GeV} on the leading and \SI{10}{\GeV} on the
sub-leading muon. All events are required to be consistent with the
decay of a \PZ boson into electrons or muons in association with
$b$-jets. Leptons are required to be of same flavour with opposite
electric charges and a di-lepton invariant mass between \SI{75}{\GeV}
and \SI{110}{\GeV}. Exactly two \btagged jets are required with the
invariant mass between both jets
fulfilling~$\mBB \not\in [\SI{40}{\GeV}, \SI{210}{\GeV}]$. This
requirement on \mBB had to be introduced to ensure orthogonality with
signal regions of searches for Higgs boson pair production in final
states with $\bbbar\Plp\Plm + \pTmissAbs$.\footnote{For future
  iterations of this search it would be justifiable, based on
  arguments of the larger SM \HH sensitivity of the
  $\bbbar\tau^{+}\tau^{-}$ channel, to forgo the orthogonality
  requirement thus allowing the use a \ZHF control region selection
  more similar to the selections applied for the signal regions of the
  $\bbbar\tau^{+}\tau^{-}$ channel.} After the \ZHF control region
selection, the electron and muon channels are combined for further
analysis.

% Labeling and fit
% https://twiki.cern.ch/twiki/bin/view/AtlasProtected/FlavourTaggingLabeling
Simulated \Zjets events entering the \ZHF control region are
categorised according to a generator-level flavour label assigned to
the pair of $b$-jet candidates. Reconstructed jets are labelled as
either $b$, $c$, or \emph{light} ($l$) according to the presence of
hadrons within a cone of $\Delta R < 0.3$ around the jet axis. If a
$b$- or $c$-flavoured hadron with a transverse momentum of at least
\SI{5}{\GeV} is found within the cone, the jet is labelled $b$ or $c$,
respectively. When a hadron matches multiple jets the ambiguity is
resolved by giving precedence to the closest jet in $\Delta R$. Jets
that are not matched to any $b$- or $c$-flavoured hadrons are labelled
as \emph{light}. Six categories are defined based on the flavour label
of the $b$-jet candidate pair:~$Z + bb$, $Z + bc$, $Z + cc$, $Z + bl$,
$Z + cl$, and $Z + ll$. Contributions from $Z + bb$, $Z + bc$, and
$Z + cc$ are combined and collectively referred to as \ZHF. The
remaining \Zjets events with at least one \emph{light}-jet are
combined into a sample referred to as \ZLF.

% \Zjets events from simulation are partitioned according to the
% flavour labels of the $b$-jet candidates into six categories:
% $Z + bb$, $Z + bc$, $Z + cc$, $Z + bl$, $Z + cl$, and $Z +
% ll$. Contributions from $Z + bb$, $Z + bc$, and $Z + cc$, where both
% $b$-jet candidates are matched to hadrons of heavy flavour at
% generator-level, are combined and collectively referred to as
% \ZHF. The remaining \Zjets events with at least one \emph{light}-jet
% are combined into a sample referred to as \ZLF.

The pre-fit event yield in the \ZHF control region is given
in~\Cref{tab:zcr_yields}. The majority of events in the control region
originate from \ZHF\footnote{About \SI{90}{\percent} of events in the
  inclusive \ZHF sample are $Z + bb$ events.} or top quark pair
production.
% Only about
% 3% of top quark is single top The production of $Z + bb$ accounts for \SI{90}{\percent} of events in the inclusive \ZHF sample.
To distinguish between the \ZHF and \ttbar contributions in the
likelihood fit, the invariant di-lepton mass, \mll, is used as a
discriminant. The \mll distribution prior to the fit is depicted
in~\Cref{fig:zcr_mll_prefit} showing the expected discrepancy between
data and the pre-fit prediction.

\begin{table}[htbp]
  \centering

  \caption{Event yields in the \ZHF control region before (pre-fit)
    and after (post-fit) the binned maximum likelihood fit of the \mll
    distribution in the control region. The \emph{Other} category
    summarises smaller backgrounds and largely consists of events from
    di-boson processes. The uncertainties on the event yield include
    all experimental and systematic uncertainties.}%
  \label{tab:zcr_yields}

  % Pre-fit:
% Other contains:
% ttH & 32.6 $\pm$ 2.8\\
% VBFHtautau & 0.041 $\pm$ 0.041\\
% diboson & 412 $\pm$ 91\\
% W & 21.9 $\pm$ 3.4\\
% DY & 74.8 $\pm$ 5.7\\
% DYtt & 0.052 $\pm$ 0.011\\

% Post-fit of CR only
%
% Other:
% ttH & 32.5 $\pm$ 2.8\\
% VBFHtautau & 0.04 $\pm$ 0.04\\
% DY & 72.6 $\pm$ 5.2\\
% diboson & 402 $\pm$ 88\\
% W & 21.2 $\pm$ 3.2\\
% DYtt & 0.0485 $\pm$ 0.0091\\

\begin{tabular}{l@{\hskip 20pt}S[table-format=5.0(4)]@{\hskip 20pt}S[table-format=5.0(4)]}
  \toprule
  & \multicolumn{2}{c}{Event yield} \\
  \cmidrule{2-3}
  Process & {Pre-fit} & {Post-fit} \\
  \midrule
  $Z \to \ell^+\ell^- + \text{HF}$ & 41200 \pm 3200 & 55700 \pm 1300 \\
  Top quark & 36600 \pm 1400 & 35260 \pm 370 \\
  $Z \to \ell^+\ell^- + \text{LF}$ & 5300 \pm 1800 &  4500 \pm 1300 \\
  Other & 541 \pm 94 & 528 \pm 90 \\
  \midrule
  Total prediction & 83600 \pm 5200 & 96030 \pm 320 \\
  \midrule
  Observed data & \multicolumn{2}{c}{\num{96032}} \\
  \bottomrule
\end{tabular}


%%% Local Variables:
%%% mode: latex
%%% TeX-master: "../phd_thesis"
%%% End:

\end{table}

% Since the normalisations of \ttbar and \ZHF are extracted from a
% fit to data, a discriminant distinguishing between both components is
% required. The invariant di-lepton mass, \mll, which is shown prior to
% the fit in~\Cref{fig:zcr_mll_prefit}, is used for this purpose. A
% discrepancy in the normalisation of the \ZHF background can be seen
% when comparing the pre-fit prediction with the data observed in the
% control region.

\begin{figure}[htbp]
  \centering


  \begin{subfigure}{.485\textwidth}
    \includegraphics[width=\textwidth]{zhfcr/Region_BMin0_incJet1_Y2015_DZllbbCR_T2_L2_distmLL_J2_Prefit_fixed}
    \subcaption{Pre-fit}
    \label{fig:zcr_mll_prefit}
  \end{subfigure}\hfill%
  \begin{subfigure}{.485\textwidth}
    \includegraphics[width=\textwidth]{zhfcr/Region_BMin0_incJet1_Y2015_DZllbbCR_T2_L2_distmLL_J2_GlobalFit_conditionnal_mu0_fixed}
    \subcaption{Post-fit (\ZHF control region only)}
    \label{fig:zcr_mll_postfit}
  \end{subfigure}

  \caption{Distribution of the invariant di-lepton mass for the
    combination of electron- and muon-channel in the Z+HF control
    region before (a) and after (b) the likelihood fit restricted to
    the control region. The contribution of \Zjets is sub-divided into
    cases where both $b$-jet candidates are matched to heavy flavour
    quarks ($b$ or $c$) and cases where at most one candidate is
    matched to heavy flavour quarks at generator-level. Prior to the
    fit the \Zjets background is normalised to cross section
    predictions at NNLO~\cite{Anastasiou:2003ds}. All statistical and
    systematic uncertainties are included.}
\end{figure}

% ATLAS_norm_Zhf    1.3856e+00 +/-  1.19e-01
% ATLAS_norm_ttbar    9.7290e-01 +/-  3.92e-02
% Included in the SR fits: Systematic uncertainties are introduced at a later stage...
The \ZHF control region is included in simultaneous fits of signal and
control regions to provide constraints on the normalisation of the
\ZHF background. Details on systematic uncertainties and the fit model
are discussed
in~\Cref{sec:uncertainties,sec:statistical_analysis}. Restricting the
maximum likelihood fit to the control region yields estimates of the
normalisation factors of \num{1.39 \pm 0.12} and \num{0.97 \pm 0.04}
for \ZHF and \ttbar, respectively. The quoted normalisation factors
include all statistical and systematic
uncertainties. \Cref{tab:zcr_yields} and \Cref{fig:zcr_mll_postfit}
show the event yields and \mll distribution after the fit.



% When performing the likelihood fit in the control region only, the
% estimated normalisation factors are~\num{1.39 \pm 0.12} for the \ZHF
% and~\num{0.97 \pm 0.04} for the \ttbar background. The abundance of
% \ttbar events in the control region provides stringent constraints
% on the normalisation of the \ttbar background in addition to
% constraining the normalisation of the \ZHF background. The post-fit
% event yields and \mll distribution is shown in~\Cref{tab:zcr_yields}
% and~\Cref{fig:zcr_mll_postfit}, respectively.

%%% Local Variables:
%%% mode: latex
%%% TeX-master: "../../phd_thesis"
%%% End:


\subsection{Fake-\tauhadvis Background from \ttbar Production in the \hadhad Channel}%
\label{sec:bkg_hadhad_ttbarfakes}
In the fully-hadronic \hadhad channel, \ttbar with \faketauhad are a
significant background contribution. In contrast to \faketauhad from
multi-jet where both reconstructed \tauhad candidates are originating
from jets, in $\ttbar \ra \Pbottom \PWp \APbottom \PWm$ frequently
only a single \tauhadvis candidate originates from jets of hadronic
decays of the \PW.

The quality of the modelling of \faketauhad in simulation is generally
unknown and needs to be checked. The pre-fit distributions\todo{Show?}
showed no significant indication of large mismodelling. However, a
data-driven method is still employed to get an idea about the level of
agreement between data and MC and to provide uncertainties on these
contributions.

Given the reasonable agreement of the Monte Carlo simulation with the
data, an approach of correcting the misidentification efficiencies in
simulation using a data-driven measurement is chosen. Two approaches
were investigated, one being the direct measurement of the
misidentification efficiency
\begin{align*}
  \text{Fake rate:}\quad \varepsilon_\text{mis-ID}^\text{data} = \frac{N_\text{post-ID}^\text{data}}{N_\text{pre-ID}^\text{data}}
\end{align*}
and the other being a measurement of the misidentification efficiency
relative to the one in simulation
\begin{align*}
  \text{Fake scale factor:}\quad
  \text{SF} =
  \frac{\varepsilon_\text{mis-ID}^\text{data}}{\varepsilon_\text{mis-ID}^\text{MC}} =
  \frac{N_\text{post-ID}^\text{data} / N_\text{pre-ID}^\text{data}}{N_\text{post-ID}^\text{MC} / N_\text{pre-ID}^\text{MC}}
  = \frac{N_\text{post-ID}^\text{data}}{N_\text{post-ID}^\text{MC}} \cdot \frac{N_\text{pre-ID}^\text{MC}}{N_\text{pre-ID}^\text{data}}
\end{align*}
The second term is completely independent of the modelling of the
\tauhadvis identification efficiency and describes the overall
modelling of \ttbar in Monte Carlo simulation. This term is assumed to
be a constant overall normalisation difference between MC and
data. Differential shape differences will be accounted for in the
systematic uncertainties on \ttbar modelling when extracting the scale
factors.

\todo[inline]{Main point: scale factor method provides a correction
  that is applied post-ID.}

\todo[inline]{How would these be applied any differently? What are the
  advantages of either method?}


\subsubsection{Measurement of scale factors}

The transverse mass of the \PW boson is used as a variable
discriminating between. The main idea is to distinguish between
semi-leptonic and di-leptonic \ttbar since two true taus would be
expected in di-leptonic and jets faking taus in semi-leptonic. The
all-hadronic mode is negligible due to the presence of one electron or
muon.

To distinguish between semi-leptonic and di-leptonic \ttbar, one can
try to reconstruct the transverse mass of the \PW boson
\begin{align*}
  \mTW = \sqrt{\left( | \myvec{p}_{\text{T}, \ell} | + | \pTmiss | \right)^2
               - \myvec{p}_{\text{T}, \ell} \cdot \pTmiss}
\end{align*}
where $\myvec{p}_{\text{T}, \ell}$ and \pTmiss are the vectors of the
lepton ($e$ or $\mu$) momentum and missing transverse momentum in the
transverse plane. For semi-leptonic \ttbar the event rate drops
significantly beyond \SI{100}{\GeV} while for di-leptonic \ttbar, due
to the presence of additional neutrinos, the transverse mass extends
to larger values.

A top control region is defined in the $\Plepton + \tauhadvis$ final
state, where \Plepton can be either electrons or muons. The control
region is similar to the top control region used in the \lephad
channel (c.f.\ \cref{sec:}) with minor alterations. It is defined by
requiring exactly one \tauhadvis and exactly one electron or muon
passing the identification criteria\todo{reference}, and exactly two
\btagged jets. Only events passing the single lepton trigger selection
are considered. The \tauhadvis and the other lepton is required to
have opposite electric charge. The \tauhadvis selection is adapted to
more closely follow the selection of the \hadhad channel, that is the
\tauhadvis candidate is required to have $\pT > \SI{25}{\GeV}$ and
\tauhadvis are considered up to $|\eta| < 2.5$ in pseudorapidity
(instead of 2.3 in the \lephad channel). A selection of
$\mBB > \SI{150}{\GeV}$ ensures orthogonality with the signal region
of the \lephad channel.\todo{Have this in a table?}

\todo[inline]{Talk about offline reconstructed taus and their identification}

The \tauhadvis misidentification efficiencies depend on the \tauhadvis
identification algorithm and the associated working point that is
used. In this analysis the loose working point of the RNN \tauhadvis
identification algorithm is used as the baseline \tauhadvis
selection. In the \hadhad channel events are selected employing
single- and di-\tauhadvis triggers which also employ identification
algorithms to reduce trigger-rates at the HLT. These algorithms are
developed to be similar to their counterparts in the \tauhadvis
reconstruction\todo{Mention that they might diverge to some extend
  esp.\ given we now use RNN?}. Differences between the identification
at the HLT and during the offline reconstruction are expected and due
to limitations in the read-out of the detector and the time available
to make a decision on whether the event is accepted or rejected.

The effect of this two-stage selection of \tauhadvis based on
identification criteria, first at the HLT and then during offline
reconstruction, needs to be taken into account when measuring
corrections to the \tauhadvis misidentification efficiencies. The top
control region, which is collected using single electron and muon
triggers, allows to estimate the corrections for \tauhadvis without
any requirements at the HLT.

The requirements at the HLT can be emulated by requiring that the
event passes appropriately chosen single-\tauhadvis triggers. The
triggers with the lowest thresholds on the \tauhadvis \pT are chosen
that use the same chain of algorithms\todo{Energy scale,
  Identification} as is used by the trigger-selection in the signal
region of the \hadhad channel. Generally, only prescaled versions of
these triggers were available during data-taking but the trigger
decision can be recalculated in retrospect
(\textit{resurrected}). Three sub-regions of the top control region
are defined by requiring that the events pass the decision of the
resurrected triggers outlined in~\Cref{tab:triggers_ttbar_fake_sf} and
that the reconstructed \tauhadvis is geometrically matched to the
\tauhadvis candidate at the HLT (off the corresponding \tauhadvis
trigger leg).

\begin{table}[htbp]
  \centering

  \begin{tabular}{lp{7cm}p{5cm}}
    \toprule
    Offline ID & HLT chain & Relevance \\
    \midrule
    loose      & -- & {Subleading \tauhadvis candidates for events selected by single-\tauhadvis triggers.} \\
               && \\
    loose      & \verb|HLT_tau25_medium1_tracktwo| & {\tauhadvis candidates selected by di-\tauhadvis triggers in 2015-2017.} \\
               && \\
    loose      & \verb|HLT_tau25_medium1_tracktwoEF| & {\tauhadvis candidates selected by di-\tauhadvis triggers in 2018 until period K.} \\
               && \\
    loose & \verb|HLT_tau25_medium1_tracktwoEF| \par \textbf{or} \par \verb|HLT_tau25_mediumRNN_tracktwoMVA|  & {\tauhadvis candidates selected by di-\tauhadvis triggers in 2018 from period K.}\\
    \bottomrule
  \end{tabular}

  \todo[inline]{Describe the identification cuts applied at the HLT?}

  \caption{Combinations of \tauhadvis identification algorithms at the
    HLT and the offline reconstruction.}
  \label{tab:triggers_ttbar_fake_sf}
\end{table}


The available dataset for the measurement is as follows:
\begin{itemize}

\item \verb|HLT_tau25_medium1_tracktwo|: \SI{139}{\ifb}

\item \verb|HLT_tau25_medium1_tracktwoEF|: \SI{58}{\ifb}

\item \verb|HLT_tau25_mediumRNN_tracktwoMVA|: \SI{37}{\ifb}

\end{itemize}

The measurement regions are subdivided by the decay mode of the
\tauhadvis (1-prong or 3-prong) and in bins of \tauhadvis \pT:
\begin{itemize}
\item 1-prong \tauhadvis with $\pT / \si{\GeV}$: $[25, 30)$, $[30, 35)$,
  $[35, 40)$, $[40, 45)$, $[45, 55)$, $[55, 70)$, $[70, \infty)$

\item 3-prong \tauhadvis with $\pT / \si{\GeV}$: $[25, 30)$, $[30, 40)$,
  $[40, 50)$, $[50, 70)$, $[70, \infty)$
\end{itemize}
these regions are chosen such that their size and allows for a
determination of corrections with limited impact of statistical
uncertainty while allowing to extract potential \pT dependencies of
the correction.


% Fit model
Fit \mTW

Backgrounds:\\
\ttbar and \ttbar with fake \tauhad\\
single top \\
V+jets \\
Multi-jet is neglected

Regions \& normalisation factors



%%% Local Variables:
%%% mode: latex
%%% TeX-master: "../../phd_thesis"
%%% End:


\subsection{Fake-\tauhadvis Background from Multi-jet Production in the \hadhad Channel}%
\label{sec:bkg_hadhad_ff}%
\label{sec:hadhad_multijet}
\label{sec:hadhad_multijet}

Multi-jet production is a source of background in the \hadhad signal
region where both \tauhadvis candidates originate from the
misidentification of quark- or gluon-initiated jets. It represents the
second largest background with \faketauhadvis in the \hadhad SR after
the dominant \ttbarFakes contribution.

\subsubsection{The fake factor method}

The multi-jet background is estimated using the fake factor method
which is a data-driven method for background estimation. It is
applicable in cases where two event observables exist that are
statistically independent for the background process to be estimated,
while also being strong discriminators between the background and
other processes (signal and non-multi-jet processes). Four disjoint
regions can be defined, three background-enriched control regions and
a signal-like region, by performing a binary categorisation for each
observable. The assumption of statistical independence allows to
relate the expected number of events for the background process
between the control regions and the signal-like region, thus yielding
a data-driven estimate of the expected background contribution in the
signal-like region.

In the \hadhad channel, two observables allowing to define control
regions enriched in multi-jet events are the \tauid requirements
applied to \tauhadvis candidates and the sign of the electric charges
of both candidates.

In the signal region, \tauhadvis candidates are required to pass the
loose \tauid working point. This requirement defines regions where
both \tauhadvis candidates pass \tauid which are herafter referred to
as ID regions. The selection is partially inverted to obtain control
regions enhanced in multi-jet events by requiring that exactly one
\tauhadvis candidate is failing the loose \tauid working point, while
still passing a working point corresponding to an efficiency loss of
\tauhad of about 1\,\% ($\text{RNN score} > 0.01$). The \tauhadvis
candidates fulfilling this selection are referred to as
anti-\tauhadvis and the regions defined by the inversion of the
identification criterion as Anti-ID regions. The identification
criterion cannot be fully inverted due to
pre-selections\footnote{Datasets targeting $\PHiggs \to \hadhad$
  require events with at least one \tauhadvis passing the loose \tauid
  working point and one \tauhadvis with an RNN \tauid score exceeding
  0.01.} applied in the data reduction pipeline of the ATLAS
experiment. However, the fake factor method is still valid in the
presence of these constraints.

The electric charge of both \tauhadvis candidates produced from signal
processes and dominant sources of backgrounds with two \tauhadvis
orginating for hadronic $\tau$~decays ($\PZ \to \tautau$,
$\PHiggs \to \tautau$, \ttbar) are expected to be reconstructed with
opposite-sign (OS) electric charge. The OS requirement is inverted
yielding regions with \tauhadvis candidates of same-sign (SS) electric
charge, depleting the region of processes where both \tauhadvis
orginate from hadronic $\tau$ decays. In contrast, the multi-jet
background contributes similarly to the OS and SS regions since
\tauhadvis charge reconstruction has little sensitivity to the
relative sign of the electric charge between partons initiating jets
that are being misidentified as \tauhadvis.

With the previously defined control regions and assumption of
independence of both categorical observables, the expected multi-jet
contribution in regions with \tauhadvis passing loose identification
and with \tauhadvis pairs of opposite-sign electric charge can be
estimated using
\begin{align*}
  N_\text{multi-jet}^{\text{OS, ID}} =
  N_\text{multi-jet}^{\text{OS, Anti-ID}}
  \cdot
  \underbrace{\frac{N_\text{multi-jet}^{\text{SS, ID}}}
  {N_\text{multi-jet}^{\text{SS, Anti-ID}}}}
  _{= \text{FF}_\text{SS}} \,\text{,}
\end{align*}
where $N_\text{multi-jet}$ is the number of multi-jet events in a
given region. The fake factor (FF) measures the ratio of multi-jet
events in the ID and Anti-ID region\footnote{The use of identification
  or isolation criteria to define the ratio is the main difference
  between the fake factor method and the more general ABCD method.}.

The probability of misidentifying a quark- or gluon-jet as a hadronic
$\tau$ decay depends strongly on the properties of the reconstructed
\tauhadvis candidate. Particularly, the reconstructed decay mode and
visible transverse momentum affect the probability of a jet
reconstructed as a \tauhadvis to pass \tauid (cf.\
\Cref{sec:tauid}). To control for this effect, fake factor
measurements are frequently performed in bins of observables related
to properties of reconstructed \tauhadvis.

The control regions defined by inverting the OS and ID requirements on
\tauhadvis do not provide pure samples of multi-jet events. Therefore,
number of multi-jet events is estimated according to
\begin{align*}
  N_\text{multi-jet} = N_\text{data} - N_\text{non-multi-jet} \,\text{,}
\end{align*}
where $N_\text{data}$ is the observed number of events in the
multi-jet enriched region and $N_\text{non-multi-jet}$ the expected
contribution of non-multi-jet events estimated using simulation.

In~\Cref{tab:mjfakes_yields} the multi-jet and non-multi-jet yields in
the regions relevant for the \faketauhadvis estimation are
summarised. The 2 $b$-tag region, while most similar to the signal
region, is not well suited to estimate fake factors:
\begin{itemize}

\item The 2 $b$-tag regions used in the fake factor method have a
  large contamination from non-multi-jet backgrounds, primarily
  \ttbarFakes, that have to be subtracted. The large size of the
  subtraction leads to a degradation of the statistical precision of
  measured fake factors and introduce large systematic uncertainties
  originating from the modelling of the subtracted components.

\item The \btag requirement suppresses the multi-jet contribution in
  the control regions preventing a differential measurement of fake
  factors in properties of the \tauhadvis.

\item The multi-jet estimate cannot be validated in the 2 $b$-tag
  region due to the absence of a region with high multi-jet purity
  that is similar to the signal region.

  % Resultingly, the statistical independence of the charge sign and ID
  % observables employed by the FF method cannot be verified.
\end{itemize}
These issues are addressed by performing the fake factor measurement
in the 1 $b$-tag region, which has a higher abundance and purity of
multi-jet events, and extrapolating the measurement into the 2 $b$-tag
region to obtain a multi-jet background estimate in the signal region.

\begin{table}[htbp]
  \centering

  \begin{subtable}[t]{\textwidth}
    \centering
    % Size of subtraction and multi-jet purity:
%                         multi_jet  non_multi_jet  multi_jet_error  non_multi_jet_error  multi_jet_purity
% anti_id charge_sign
% False   OS           16067.048497   16443.951503       204.558258            96.607872          0.494203
%         SS           14040.394005    1971.605995       129.147367            25.827164          0.876867
% True    OS           91582.182374   13677.817626       334.090987            79.729466          0.870057
%         SS           78399.983641    5707.016359       296.470480            61.544664          0.932146

\begin{tabular}{
  ll
  S[table-format=5.0(3)]
  S[table-format=5.0(3)]
  c}
  \toprule
  \multicolumn{2}{l}{Region} & {$N_\text{multi-jet}$} & {$N_\text{non-multi-jet}$} & {Multi-jet purity} \\
  \midrule
  SS & ID      & 14040 +- 130 & 1970 +- 30   & 88\,\% \\
  SS & Anti-ID & 78400 +- 300 & 5710 +- 70   & 93\,\% \\
  OS & ID      & 16070 +- 210 & 16440 +- 100 & 49\,\% \\
  OS & Anti-ID & 91580 +- 340 & 13680 +- 80  & 87\,\% \\
  \bottomrule
\end{tabular}



%%% Local Variables:
%%% mode: latex
%%% TeX-master: "../phd_thesis"
%%% End:

    \subcaption{1 $b$-tag regions}
    \label{tab:mjfakes_yields_1tag}
  \end{subtable}

  \begin{subtable}[t]{\textwidth}
    \centering
    % Size of subtraction and multi-jet purity:
%                        multi_jet  non_multi_jet  multi_jet_error  non_multi_jet_error  multi_jet_purity
% anti_id charge_sign
% False   OS            408.197943    7971.802057       105.950917            53.344135          0.048711
%         SS           1299.622259    1001.377741        50.345854            15.287412          0.564808
% True    OS           8429.603396    8864.396604       139.699303            47.136984          0.487429
%         SS           7653.735896    3338.264104       108.557939            28.157166          0.696301

\begin{tabular}{
  ll
  S[table-format=5.0(3)]
  S[table-format=5.0(3)]
  c}
  \toprule
  \multicolumn{2}{l}{Region} & {$N_\text{multi-jet}$} & {$N_\text{non-multi-jet}$} & {Multi-jet purity} \\
  \midrule
  \multirow{2}{*}{SS} & ID      & 1300 +- 60  & 1000 +- 20 & 56\,\% \\
                             & Anti-ID & 7650 +- 110 & 3340 +- 30 & 70\,\% \\
  \midrule
  \multirow{2}{*}{OS} & ID      & \multicolumn{3}{c}{\rule[3pt]{5.2em}{0.3pt}\hspace{1em}Signal Region\hspace{1em}\rule[3pt]{5.2em}{0.3pt}} \\
                             & Anti-ID & 8430 +- 140 & 8860 +- 50 & 49\,\% \\
  \bottomrule
\end{tabular}




%%% Local Variables:
%%% mode: latex
%%% TeX-master: "../phd_thesis"
%%% End:

    \subcaption{2 $b$-tag regions}
  \end{subtable}

  \caption{Number of multi-jet events in regions defined by different
    \tauhadvis pair electric charge (OS, SS) and \tauhadvis
    identification (ID, Anti-ID). The number of multi-jet events,
    $N_\text{multi-jet}$, is estimated by subtracting the number
    non-multi-jet events, $N_\text{non-multi-jet}$, estimated using
    Monte Carlo simulation from the observed number of events in this
    region. In (a) the breakdown is shown after a 1 $b$-tag
    requirement; in (b) after a 2 $b$-tag requirement (the SR -- 2
    $b$-tag OS ID -- is omitted). The uncertainties are from
    statistical sources only.}
  \label{tab:mjfakes_yields}
\end{table}

A schematic illustration of this approach is given
in~\Cref{fig:fakefactor_regions}. Fake factors measured in the 1
$b$-tag regions ($\text{FF}_\text{SS}^\text{1-tag}$) are applied to
events in the 2 $b$-tag OS Anti-ID region after subtraction of
non-multi-jet contributions to obtain multi-jet templates in the
signal region. Multiplicative transfer factors
($\text{TF}_{1 \ra 2\,b\text{-tag}}$) are applied to
$\text{FF}_\text{SS}^\text{1-tag}$ when used in 2 $b$-tag regions,
absorbing possible differences between fake factors measured in 1 and
2 $b$-tag regions and the uncertainties associated with this
extrapolation.

\begin{figure}[htbp]
  \centering

  \includegraphics[scale=1]{fakefactors/regions}

  \caption{Schematic description of the fake factor method employed to
    estimate the multi-jet background in the signal region of the
    \hadhad channel. The squares represent the multi-jet events
    ($N_\text{multi-jet} = N_\text{data} - N_\text{non-multi-jet}$) in
    a particular region.}
  \label{fig:fakefactor_regions}
\end{figure}

The 1 $b$-tag OS ID region serves as a validation region to check the
closure of the multi-jet estimate and thus verify the independence of
the categorical observables related to \tauid (ID / Anti-ID) and
electric charge of the \tauhadvis pair (OS / SS). This approach is
equivalent\todo{Is it really? I don't think so.} to a comparison of
fake factors measured in the OS and SS
regions\footnote{\Cref{tab:mjfakes_yields_1tag} can be used to
  calculate inclusive fake factors in the OS and SS regions, yielding
  $\text{FF}_\text{SS}^\text{1-tag} \approx
  \text{FF}_\text{OS}^\text{1-tag} \approx 0.18$, which is a
  sufficient condition for statistical independence of the fake factor
  observables at the level of the inclusive selection.}, which have to
agree under the assumptions of the method.


\subsubsection{Measurement of fake factors}

% - STT / DTT binning
The fake factor measurement is performed separately for events
selected by single- and di-\tauhadvis triggers as well as the years of
data collection. During Run~2 of the LHC, different \tauhadvis trigger
chains were used by the ATLAS experiment to collect the data used in
this analysis. As a result, the topologies of the selected events and
the \tauid applied at the high-level trigger changed as Run~2
progressed. To account for possible differences resulting from the
change in trigger-selection, the fake factor measurement is subdivided
into three major data collection periods: 2015-2016, 2017, and
2018. The 1 $b$-tag regions relevant to the measurement of fake
factors are shown in~\Cref{fig:mjfakes_1tag_ss_plots}.

\begin{figure}[htbp]
  \centering

  \begin{subfigure}{0.49\textwidth}
    \includegraphics[width=\textwidth]{fakefactors/region_plots/tau0pt_1tag_ss_id}
    \subcaption{Leading \tauhadvis \pT in the 1 $b$-tag SS ID region}
  \end{subfigure}
  \begin{subfigure}{0.49\textwidth}
    \includegraphics[width=\textwidth]{fakefactors/region_plots/tau1pt_1tag_ss_id}
    \subcaption{Sub-leading \tauhadvis \pT in the 1 $b$-tag SS ID
      region}
  \end{subfigure}

  \begin{subfigure}{0.49\textwidth}
    \includegraphics[width=\textwidth]{fakefactors/region_plots/tau0pt_1tag_ss_antiid}
    \subcaption{Leading \tauhadvis \pT in the 1 $b$-tag SS Anti-ID region}
  \end{subfigure}
  \begin{subfigure}{0.49\textwidth}
    \includegraphics[width=\textwidth]{fakefactors/region_plots/tau1pt_1tag_ss_antiid}
    \subcaption{Sub-leading \tauhadvis \pT in the 1 $b$-tag SS Anti-ID
      region}
  \end{subfigure}

  \caption{1-tag SS ID and Anti-ID region. MC background is stat.\
    uncertainty only. At pre-selection level inclusive in years,
    triggers etc. Correspond to yields in
    \Cref{tab:mjfakes_yields_1tag} SS.}
  \label{fig:mjfakes_1tag_ss_plots}
\end{figure}

% Binning based on anti-tau
The fake factors are primarily determined by the properties of the
anti-\tauhadvis


{% Group for extra definitions
  \newcommand*{\ffargs}{\ensuremath{( \myvec{x}_{\tau} )}\xspace}

  \newcommand*{\NmjID}[2]{\ensuremath{N_\text{multi-jet}^{\text{#1, loose }\tau_{#2}}}\xspace}
  \newcommand*{\NmjIDIncl}[1]{\ensuremath{N_\text{multi-jet}^{\text{#1, ID}}}\xspace}

  \newcommand*{\NmjAntiIDIncl}[1]{\ensuremath{N_\text{multi-jet}^{\text{#1, Anti-ID}}}\xspace}
  \newcommand*{\NmjAntiID}[2]{\ensuremath{N_\text{multi-jet}^{\text{#1, anti-}\tau_{#2}}}\xspace}

  \todo[inline]{Assumption: For DTT both taus can be treated
    independently as the only difference is \pT (not the case for STT
    thus different treatment).}

  The Anti-ID region can partitioned into two regions differing in
  whether the \tauhadvis candidate leading or sub-leading in \pT is
  reconstructed as the anti-\tauhadvis. Provided the conditions for
  the fake factor method are fulfilled, both regions can be used to
  obtain individual estimates of the multi-jet background in the OS ID
  region. The notation used to describe the fake factor measurement is
  introduced in the following:
  \begin{description}[style=standard]
  \item[$\tau_0$ ($\tau_1$)] The \tauhadvis candidate leading (sub-leading) in \pT.

  \item[$\myvec{x}_\tau$] Categorical observables of a \tauhadvis
    candidate that define the bin of the fake factor measurement. The
    observables used for the di-\tauhadvis trigger fake factors are
    the reconstructed decay mode ($N_\text{tracks}$), the bin of
    \tauhadvis \pT, and the bin of \tauhadvis $\eta$.

  \item[$\NmjID{SS(OS)}{i}\ffargs$] Number of multi-jet events in the
    SS (OS) ID region where~$\tau_i$ has
    observables~$\myvec{x}_\tau$. \todo{Maybe distinguish better from
      the FF estimate?}

  \item[$\NmjAntiID{SS(OS)}{i}\ffargs$] Number of multi-jet events in
    the SS (OS) Anti-ID region where $\tau_i$ is the anti-\tauhadvis
    with observables~$\myvec{x}_\tau$.
  \end{description}
  With these definitions, two sets of fake factors can be defined as
  \begin{align*}
    \FF_{i}\ffargs &= \frac{\NmjID{SS}{i} \ffargs}{\NmjAntiID{SS}{i}\ffargs}
                     \quad \text{for} \quad i = 0, 1 \,\text{,}
  \end{align*}
  where $\FF_{i}$ is the fake factor relating the ID region with the
  partition of the Anti-ID region where $\tau_i$ is the
  anti-\tauhadvis. These can be used to obtain two multi-jet estimates
  in the OS region given by
  \begin{align*}
    \NmjID{OS}{i}\ffargs = \FF_{i}\ffargs \cdot \NmjAntiID{OS}{i}\ffargs
    \quad \text{for} \quad i = 0, 1 \,\text{.}
  \end{align*}
  An average of both estimates can be calculated, yielding fake
  factors that are inclusive in whether the anti-\tauhadvis is the
  leading or sub-leading \tauhadvis candidate. The inclusive fake
  factors can be expressed as
  \begin{align*}
    \FF_\text{incl.}\ffargs = \frac{1}{2} \left[ f_0\ffargs \cdot \FF_0\ffargs
    + f_1\ffargs \cdot \FF_1\ffargs \right] \,\text{,}
  \end{align*}
  with $f_i\ffargs$ being the fraction of Anti-ID events where
  $\tau_i$ is the anti-\tauhadvis and has
  observables~$\myvec{x}_\tau$. The inclusive fake factor can be
  measured directly according to
  \begin{align*}
    \FF_\text{incl.}\ffargs
    = \frac{1}{2} \frac{ \NmjID{SS}{0}\ffargs + \NmjID{SS}{1}\ffargs }
                       { \NmjAntiID{SS}{0}\ffargs + \NmjAntiID{SS}{1}\ffargs }
  \end{align*}
  and the multi-jet estimate in the OS region obtained by
  \begin{align*}
    \NmjIDIncl{OS}\ffargs = \FF_\text{incl.}\ffargs \cdot \NmjAntiIDIncl{OS}\ffargs \,\text{,}
  \end{align*}
  where $\NmjAntiIDIncl{OS}\ffargs$ is the number of multi-jet events
  in the OS Anti-ID region\todo{Emphasise that this is now the
    inclusive region?} with anti-\tauhadvis $\myvec{x}_\tau$.

  Benefits of this approach: The inclusive fake factor is parametrised
  directly in the properties of the anti-\tauhadvis while maximally
  using the available number of events in the Anti-ID region.
}


% - DTT fake factors (differential)
Events selected by di-\tauhadvis triggers are additionally measured
separately for

In DTT, both taus can be treated on the same level (aside from pt
ordering).



\begin{figure}[htbp]
  \centering

  \begin{subfigure}{0.49\textwidth}
    \includegraphics[width=\textwidth]{fakefactors/fake_factors_dtt_1516}
    \subcaption{DTT fake factors measured using 2015-2016 data}
  \end{subfigure}
  \begin{subfigure}{0.49\textwidth}
    \includegraphics[width=\textwidth]{fakefactors/fake_factors_dtt_17}
    \subcaption{DTT fake factors measured using 2017 data}
  \end{subfigure}

  \begin{subfigure}{0.49\textwidth}
    \includegraphics[width=\textwidth]{fakefactors/fake_factors_dtt_18}
    \subcaption{DTT fake factors measured using 2018 data}
  \end{subfigure}

  \caption{Combined Fake Factors DTT. The last bin is inclusive in all
    \pT larger than \SI{150}{\GeV}.}
  \label{fig:mjfakes_fake_factors}
\end{figure}



% - STT fake factors
Here it makes a difference whether the leading or the subleading tau
is an anti-\tauhadvis since this entails a particular \pT distribtion
(and the FFs are not binned in \pT).


% Additional info that would be interesting:
% - Subtraction

\begin{figure}[htbp]
  \centering

  \includegraphics[width=0.45\textwidth]{fakefactors/fake_factors_stt}

  \todo[inline]{Largest deviation for leading 1-prong tau possibly due
    to looser pT threshold.}

  \caption{STT fake factors}
  \label{fig:mjfakes_stt_ffs}
\end{figure}


\begin{figure}[htbp]
  \centering

  \includegraphics[width=0.45\textwidth]{fakefactors/transfer_factors}

  \caption{Transfer Factors}
  \label{fig:mjfakes_transfer_factor}
\end{figure}


\subsubsection{Estimation of multi-jet backgrounds in the signal region}

% - Transfer factor calculation
% - Subtraction in Anti-ID region

\subsubsection{Systematic Uncertainties}

% - Uncertainties


% The fake factors are not expected to depend strongly
% on the applied \btag requirement

% Dominant subtraction is ttbar, execpt for 1-tag OS ID where it is
% Ztautau.

% Signal contamination / other background contamination

% Yield table / plots of regions

% Checked in 1-tag OS. Does agreement in 1-tag OS confirm that charge
% and ID are independent? Closure, yes

\todo[inline]{Old stuff below:}

The fake factor (FF) measures the ratio of events with fake \tauhadvis in ID and
Anti-ID region:
\begin{align*}
  \FF = \frac{N\left( \text{fake} \, \tauhadvis, \text{ID} \right)}{N\left( \text{fake} \, \tauhadvis, \text{Anti-ID} \right)}
\end{align*}
The probability of a jet faking a \tauhadvis depends strongly on \tauhadvis \pT
and decay mode. Therefore, the fake factor is frequently parametrised in these
quantities. It is also affected by the \tauhadvis identification already applied
in the high-level \tauhadvis-triggers which also needs to be taken into account.

The fake factors are measured in the fake enriched SS-region by subtracting
non-fake-\tauhadvis background using their estimates from simulation:
\begin{align*}
  \FF_\text{SS} = \frac{N(\text{SS}, \text{ID}) - N_\text{non-fake}(\text{SS}, \text{ID})}{N(\text{SS}, \text{Anti-ID}) - N_\text{non-fake}(\text{SS}, \text{Anti-ID})}
\end{align*}
where $N$ is the total yield and $N_\text{non-fake}$ the yield of
non-fake-\tauhadvis backgrounds in the corresponding region.

To obtain the fake \tauhadvis estimate in the OS ID-region, the assumption is
made that the fake factors are independent of the reconstructed charge of the
fake \tauhadvis candidates. Therefore, the SS fake factors $\FF_\text{SS}$ are
applied to events in the OS Anti-ID region after subtracting any
non-fake-\tauhadvis contributions.
\begin{align*}
  N(\text{fake}, \text{OS ID}) = \FF_\text{SS} \times \left[ N(\text{OS}, \text{Anti-ID}) - N_\text{non-fake}(\text{OS}, \text{Anti-ID}) \right]
\end{align*}
Systematic uncertainties are assigned to
cover possible difference betwen OS and SS fake factors, varying the
subtraction in the OS Anti-ID region. Moreover, the fake factors are
independently varied by their statistical uncertainty.

Due to the limited acceptance of fake \tauhadvis in the 2 \btag region, the 1
\btag region is used to calculate the fake factors instead. The 1 \btag fake
factors are applied to the 2 \btag OS Anti-ID region and an additional 1 to 2
\btag transfer factor is applied.

The binning of the fake factors is dependent on the trigger that selected the
event. For STT events the fake factor is binned in whether the anti-\tauhadvis
is leading or subleading in \pT, and the decay mode of the \tauhadvis
($N_\text{track}$). Due to low statistics in the STT category the fake factors
are inclusive in \tauhadvis \pT. The \tauhadvis identification at the HLT is
only applied to one of the two \tauhadvis candidates affecting the probability
of jets faking \tauhadvis, motivating the binning in whether the leading /
subleading \tauhadvis fails the identification.

For DTT events, HLT \tauhadvis identification is applied to both \tauhadvis
candidates. Therefore, the fake factors do not need to distinguish between cases
where the leading and subleading \tauhadvis fails the loose identification. The
fake factors for DTT events are parametrised in the \pT and decay mode of the
the \tauhadvis candidate failing the identification requirement.

Moreover, all fake factors are binned by data-taking period (${\text{2015-2016},
  \text{2017}, \text{2018}}$) which takes into account the different triggers
being used to select events used for the analysis.

\todo[inline]{Could we use nOS to enhance statistics? Maybe flip the FF method
  so that we use events in SS ID to build the template instead of OS Anti-ID.}

\todo[inline]{Can we make the STT FF depend on the trigger-match instead of
  the leading / subleading binning?}









%%% Local Variables:
%%% mode: latex
%%% TeX-master: "../../phd_thesis"
%%% End:


\subsection{Fake-\tauhadvis Backgrounds in the \lephad Channel}%
\label{sec:bkg_lephad_combined_ff}
The estimation of \jettotauhadvis backgrounds in the \lephad channels
is outlined in the following. A data-driven background estimation
method is adopted that yields a combined estimate of the multi-jet and
\ttbar background with a \faketauhadvis. The method is an extension of
the fake factor method, previously introduced in
\Cref{sec:hadhad_multijet}, to account for multiple sources of
\faketauhadvis that can differ in their process-specific fake factors.

Events in the \lephad channel where the selected \tauhadvis candidate
is an \antitau define the Anti-ID region used for the fake factor
method. Two control regions are defined that are enhanced in multi-jet
and \ttbar events, respectively, each with a corresponding ID and
Anti-ID region which are orthogonal to the signal regions. These
control regions are used to calculate fake factors specifically for
\faketauhadvis from multi-jet and \ttbar events. They are calculated
as follows:
\begin{description}

\item[Multi-jet fake factors] are determined in a region defined by
  requiring that the electron/muon fails the loose isolation working
  point. Otherwise, the selection is identical to the \lephad signal
  region selection. This control region has high multi-jet purity and
  is therefore ued to calculate multi-jet fake factors, \FFqcd, as the
  ratio of multi-jet events in ID and Anti-ID region. The number of
  multi-jet events is estimated by subtracting the number of
  non-multi-jet events estimated using simulation from the observed
  number of events in ID and Anti-ID region.

\item[\ttbar fake factors] are determined in a region defined by
  requiring $\mBB > \SI{150}{\GeV}$ while keeping other selections
  identical to the ones of the \lephad signal regions. This control
  region has high \ttbar purity but is not pure in \ttbarFakes events
  that are used for the fake factor calculation. Similar to the
  multi-jet fake factors, the \ttbar fake factor, \FFttbar, is
  calculated as the ratio of \ttbarFakes events in ID and Anti-ID
  regions. The number of \ttbarFakes events are estimated, assuming
  negligible contribution of multi-jet events, by subtracting the
  number of non-\ttbarFakes events (excl.\ multi-jet events) estimated
  using simulation from the observed number of events in ID and
  Anti-ID region.
\end{description}

Both sets of fake factors, \FFqcd and \FFttbar, can be combined
according to the expected fraction of \faketauhadvis events in the Ant


The Anti-ID region corresponding to the signal region selection
consists of a mixture of multi-jet and \ttbar events,
however. Therefore, one perform a weighted combination of \FFqcd and
\FFttbar to define the combined fake factor \FFcomb.

\begin{align*}
  \FFcomb = \rqcd \, \FFqcd + (1 - \rqcd) \, \FFttbar
\end{align*}

For the most part $\rqcd = 0$ since the \lephad channel is dominated
by \faketauhadvis from \ttbar.


\rqcd is calculated in the Anti-ID region corresponding to the signal region selection:
\begin{align*}
  \rqcd = \frac
  {N_{\text{data}} - N(\text{non-multi-jet})}
  {N_{\text{data}} - N(\text{non-multi-jet or non-ttbar-fake})}
\end{align*}

\todo[inline]{Why can't we do this in the \hadhad channel?}

Why this is not so easy in the \hadhad channel:
\begin{itemize}
\item In \hadhad there are two \tauhadvis
\item There is no \ttbar control region accessible in a \hadhad final
  state thus one has to go to \lephad
\end{itemize}

%%% Local Variables:
%%% mode: latex
%%% TeX-master: "../../phd_thesis"
%%% End:



%%% Local Variables:
%%% mode: latex
%%% TeX-master: "../../phd_thesis"
%%% End:
