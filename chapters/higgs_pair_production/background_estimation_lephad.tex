The estimation of \jettotauhadvis backgrounds in the \lephad channels
is outlined in the following. A data-driven background estimation
method is adopted that yields a combined estimate of the multi-jet and
\ttbar background with a \faketauhadvis. The method is an extension of
the fake factor method, previously introduced in
\Cref{sec:hadhad_multijet}, to account for multiple sources of
\faketauhadvis that can differ in their process-specific fake factors.

Events in the \lephad channel where the selected \tauhadvis candidate
is an \antitau define the Anti-ID region used for the fake factor
method. Two control regions are defined that are enhanced in multi-jet
and \ttbar events, respectively, each with a corresponding ID and
Anti-ID region which are orthogonal to the signal regions. These
control regions are used to calculate fake factors specifically for
\faketauhadvis from multi-jet and \ttbar events. They are calculated
as follows:
\begin{description}

\item[Multi-jet fake factors] are determined in a region defined by
  requiring that the electron/muon fails the loose isolation working
  point. Otherwise, the selection is identical to the \lephad signal
  region selection. This control region has high multi-jet purity and
  is therefore ued to calculate multi-jet fake factors, \FFqcd, as the
  ratio of multi-jet events in ID and Anti-ID region. The number of
  multi-jet events is estimated by subtracting the number of
  non-multi-jet events estimated using simulation from the observed
  number of events in ID and Anti-ID region.

\item[\ttbar fake factors] are determined in a region defined by
  requiring $\mBB > \SI{150}{\GeV}$ while keeping other selections
  identical to the ones of the \lephad signal regions. This control
  region has high \ttbar purity but is not pure in \ttbarFakes events
  that are used for the fake factor calculation. Similar to the
  multi-jet fake factors, the \ttbar fake factor, \FFttbar, is
  calculated as the ratio of \ttbarFakes events in ID and Anti-ID
  regions. The number of \ttbarFakes events are estimated, assuming
  negligible contribution of multi-jet events, by subtracting the
  number of non-\ttbarFakes events (excl.\ multi-jet events) estimated
  using simulation from the observed number of events in ID and
  Anti-ID region.
\end{description}

Both sets of fake factors, \FFqcd and \FFttbar, can be combined
according to the expected fraction of \faketauhadvis events in the Ant


The Anti-ID region corresponding to the signal region selection
consists of a mixture of multi-jet and \ttbar events,
however. Therefore, one perform a weighted combination of \FFqcd and
\FFttbar to define the combined fake factor \FFcomb.

\begin{align*}
  \FFcomb = \rqcd \, \FFqcd + (1 - \rqcd) \, \FFttbar
\end{align*}

For the most part $\rqcd = 0$ since the \lephad channel is dominated
by \faketauhadvis from \ttbar.


\rqcd is calculated in the Anti-ID region corresponding to the signal region selection:
\begin{align*}
  \rqcd = \frac
  {N_{\text{data}} - N(\text{non-multi-jet})}
  {N_{\text{data}} - N(\text{non-multi-jet or non-ttbar-fake})}
\end{align*}

\todo[inline]{Why can't we do this in the \hadhad channel?}

Why this is not so easy in the \hadhad channel:
\begin{itemize}
\item In \hadhad there are two \tauhadvis
\item There is no \ttbar control region accessible in a \hadhad final
  state thus one has to go to \lephad
\end{itemize}

%%% Local Variables:
%%% mode: latex
%%% TeX-master: "../../phd_thesis"
%%% End:
