The estimation of \faketauhadvis backgrounds in the \lephad channels is outlined
in the following. A data-driven background estimation technique yielding a
combined estimate of the multi-jet and \ttbar background with a \faketauhadvis
is adopted. The method is an extension of the FF~method
% , previously introduced in \Cref{sec:hadhad_multijet},
that accounts for multiple sources of \faketauhadvis, differing in their
process-specific FFs.

Events in the \lephad channel in which the selected \tauhadvis candidate is an
\antitau define the Anti-ID region used for the FF method. Two CRs are defined
that are enhanced in multi-jet and \ttbar events, respectively. Both regions can
be divided into an ID and an Anti-ID region. These CRs are used to determine FFs
specifically for \faketauhadvis from multi-jet and \ttbar events. The CR
definitions and FF measurements are described in the following:
\begin{description}

\item[Multi-jet fake factors] are determined in a region defined by the
  requirement that the electron/muon fails the loose isolation working
  point. Moreover, the $\mBB < \SI{150}{\GeV}$ requirement is dropped. The
  remainder of the CR selection is identical to the SR selection of the \lephad
  channels. This CR has high multi-jet purity and allows calculating multi-jet
  FFs, \FFqcd, as the ratio of multi-jet events in the ID and Anti-ID
  region. The number of multi-jet events is estimated by subtracting the
  expected number of non-multi-jet events, which is estimated using simulation,
  from the observed number of events.

\item[\ttbar fake factors] are determined in a region defined by requiring
  $\mBB > \SI{150}{\GeV}$ with the other selections remaining identical to the
  SR selection. This CR has high \ttbar purity but is not pure in \ttbarFakes
  events due to contributions of \ttbar events with a \truetauhadvis that have
  to be subtracted. FFs for \ttbar, \FFttbar, are calculated as the ratio of
  \ttbarFakes events in ID and Anti-ID regions. The number of \ttbarFakes events
  is estimated, assuming negligible contribution of multi-jet events, by
  subtracting the expected number of non-\ttbarFakes events, which is estimated
  using simulation, from the observed number of events.

\end{description}
The FF measurement is performed separately for the \lephad SLT and LTT channels,
and separately for 1- and 3-prong \tauhadvis candidates. Moreover, the FFs are
measured in bins of \tauhadvis candidate \pT.

When estimating the \faketauhadvis background in one of the \lephad SRs, it
needs to be considered that the corresponding Anti-ID region consists of a
mixture of multi-jet and \ttbarFakes events. Let \rqcd be the fraction of
\faketauhadvis backgrounds that originate from multi-jet events in the Anti-ID
region. \emph{Combined fake factors} are defined as the weighted combination of
\FFqcd and \FFttbar:
\begin{align*}
  \FFcomb = \rqcd \, \FFqcd + (1 - \rqcd) \, \FFttbar \,\text{.}
\end{align*}
The combined FFs can be applied to events with \faketauhadvis, irrespective of
whether the event originates from multi-jet or \ttbar processes, to yield the
background estimate in the ID region. The determination of \rqcd proceeds
according to
\begin{align}
  \rqcd =
  \frac{N_{\text{multi-jet}}}{N_{\text{multi-jet}} + N_{\text{\ttbarFakes}}}
  = \frac
  { N_{\text{data}} - N_{\text{non-multi-jet}} }
  { N_{\text{data}} - N_{\text{non-(multi-jet or \ttbarFakes)}} } \,\text{,}
  \label{eq:rqcd}
\end{align}
where $N_{\text{data}}$ is the number of events observed in the Anti-ID region
and $N_{p}$ the number of events expected from process $p$ in the Anti-ID
region. The subtractions on the right-hand side of \Cref{eq:rqcd} use the
expected number of events predicted using simulation. This includes the
subtraction of \ttbarFakes events in the numerator.

The determination of \rqcd is performed separately for the \lephad SLT and LTT
channels, 1- and 3-prong \tauhadvis candidates, and events containing electrons
and muons. In addition, \rqcd is determined in bins of \pT of the \tauhadvis
candidate. In the SLT channel, \rqcd is typically small or zero showing that the
majority of \faketauhadvis backgrounds originate from \ttbar. The multi-jet
contribution in the LTT channel is larger with \rqcd ranging from
\SIrange{10}{30}{\percent} depending on \tauhadvis candidate \pT and
\Ntracks. Uncertainties on the \rqcd estimate have little impact on the
\faketauhadvis background prediction since \FFqcd and \FFttbar tend to be of
similar size in most bins.

The use of a similar method in the \hadhad channel would be preferred compared
to separately estimating the \faketauhadvis background from multi-jet and
\ttbar. The combined FF method does not need to distinguish between events with
\faketauhadvis from multi-jet and \ttbar when applying the FFs to events in the
Anti-ID region. In contrast, the multi-jet estimate in the \hadhad channel
requires a large subtraction of \ttbarFakes events, which is a dominant source
of systematic uncertainty. In the combined FF method, this uncertainty is
restricted to an uncertainty on \rqcd. Despite possibly large uncertainties on
\rqcd, the uncertainty on the \faketauhadvis background estimate from the
combined FF method would be small since \FFqcd and \FFttbar are of similar
size. The search presented in this thesis is not limited by uncertainties
related to the \faketauhadvis background estimation and therefore this approach
was not pursued. However, in the future systematic uncertainties will become
more relevant at which point the combined FF method should also be considered
for the \hadhad channel.

% The \faketauhadvis estimation in the \lephad channel is concluded by briefly
% discussing the reason for not using a combined FF method in the \hadhad
% channel. In general, using a similar method in the \hadhad channel would be
% preferred compared to performing separate estimates of the \faketauhadvis
% backgrounds from multi-jet and \ttbar. This is because the combined FF method
% does not require to distinguish between events with \faketauhadvis from
% multi-jet and \ttbar when applying the FFs to events from the Anti-ID
% region. Currently, the multi-jet estimate in the \hadhad channel requires a
% large subtraction of \ttbar events with \faketauhadvis, which is a large
% source of systematic uncertainty on the multi-jet estimate. For the combined
% FF method, this uncertainty would be restricted to an uncertainty on the \rqcd
% determination. However, under the assumption that \FFqcd and \FFttbar are
% similar, a large uncertainty on \rqcd would only have a small effect on the
% resulting \faketauhadvis background estimate.

% Difficulties arise when trying to adopt the method to the \hadhad channel
% which were previously discussed in \Cref{sec:bkg_hadhad_ttbarfakes}. It is
% possible that these can be overcome in the future. At the current stage the
% search is not limited by uncertainties related to the \faketauhadvis
% estimation methods and therefore this approach was not pursued. However, the
% method should be considered for future iterations of this search for which
% systematic uncertainties will become more relevant.

%%% Local Variables:
%%% mode: latex
%%% TeX-master: "../../phd_thesis"
%%% End:
