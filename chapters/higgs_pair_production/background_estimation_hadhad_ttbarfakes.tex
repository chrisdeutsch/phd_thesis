In the fully-hadronic \hadhad channel, \ttbar with \faketauhad are a
significant background contribution. In contrast to \faketauhad from
multi-jet where both reconstructed \tauhad candidates are originating
from jets, in $\ttbar \ra \Pbottom \PWp \APbottom \PWm$ frequently
only a single \tauhadvis candidate originates from jets of hadronic
decays of the \PW.

The quality of the modelling of \faketauhad in simulation is generally
unknown and needs to be checked. The pre-fit distributions\todo{Show?}
showed no significant indication of large mismodelling. However, a
data-driven method is still employed to get an idea about the level of
agreement between data and MC and to provide uncertainties on these
contributions.

Given the reasonable agreement of the Monte Carlo simulation with the
data, an approach of correcting the misidentification efficiencies in
simulation using a data-driven measurement is chosen. Two approaches
were investigated, one being the direct measurement of the
misidentification efficiency
\begin{align*}
  \text{Fake rate:}\quad \varepsilon_\text{mis-ID}^\text{data} = \frac{N_\text{post-ID}^\text{data}}{N_\text{pre-ID}^\text{data}}
\end{align*}
and the other being a measurement of the misidentification efficiency
relative to the one in simulation
\begin{align*}
  \text{Fake scale factor:}\quad
  \text{SF} =
  \frac{\varepsilon_\text{mis-ID}^\text{data}}{\varepsilon_\text{mis-ID}^\text{MC}} =
  \frac{N_\text{post-ID}^\text{data} / N_\text{pre-ID}^\text{data}}{N_\text{post-ID}^\text{MC} / N_\text{pre-ID}^\text{MC}}
  = \frac{N_\text{post-ID}^\text{data}}{N_\text{post-ID}^\text{MC}} \cdot \frac{N_\text{pre-ID}^\text{MC}}{N_\text{pre-ID}^\text{data}}
\end{align*}
The second term is completely independent of the modelling of the
\tauhadvis identification efficiency and describes the overall
modelling of \ttbar in Monte Carlo simulation. This term is assumed to
be a constant overall normalisation difference between MC and
data. Differential shape differences will be accounted for in the
systematic uncertainties on \ttbar modelling when extracting the scale
factors.

\todo[inline]{Main point: scale factor method provides a correction
  that is applied post-ID.}

\todo[inline]{How would these be applied any differently? What are the
  advantages of either method?}


\subsubsection{Measurement of scale factors}

The transverse mass of the \PW boson is used as a variable
discriminating between. The main idea is to distinguish between
semi-leptonic and di-leptonic \ttbar since two true taus would be
expected in di-leptonic and jets faking taus in semi-leptonic. The
all-hadronic mode is negligible due to the presence of one electron or
muon.

To distinguish between semi-leptonic and di-leptonic \ttbar, one can
try to reconstruct the transverse mass of the \PW boson
\begin{align*}
  \mTW = \sqrt{\left( | \myvec{p}_{\text{T}, \ell} | + | \pTmiss | \right)^2
               - \myvec{p}_{\text{T}, \ell} \cdot \pTmiss}
\end{align*}
where $\myvec{p}_{\text{T}, \ell}$ and \pTmiss are the vectors of the
lepton ($e$ or $\mu$) momentum and missing transverse momentum in the
transverse plane. For semi-leptonic \ttbar the event rate drops
significantly beyond \SI{100}{\GeV} while for di-leptonic \ttbar, due
to the presence of additional neutrinos, the transverse mass extends
to larger values.

A top control region is defined in the $\Plepton + \tauhadvis$ final
state, where \Plepton can be either electrons or muons. The control
region is similar to the top control region used in the \lephad
channel (c.f.\ \cref{sec:}) with minor alterations. It is defined by
requiring exactly one \tauhadvis and exactly one electron or muon
passing the identification criteria\todo{reference}, and exactly two
\btagged jets. Only events passing the single lepton trigger selection
are considered. The \tauhadvis and the other lepton is required to
have opposite electric charge. The \tauhadvis selection is adapted to
more closely follow the selection of the \hadhad channel, that is the
\tauhadvis candidate is required to have $\pT > \SI{25}{\GeV}$ and
\tauhadvis are considered up to $|\eta| < 2.5$ in pseudorapidity
(instead of 2.3 in the \lephad channel). A selection of
$\mBB > \SI{150}{\GeV}$ ensures orthogonality with the signal region
of the \lephad channel.\todo{Have this in a table?}

\todo[inline]{Talk about offline reconstructed taus and their identification}

The \tauhadvis misidentification efficiencies depend on the \tauhadvis
identification algorithm and the associated working point that is
used. In this analysis the loose working point of the RNN \tauhadvis
identification algorithm is used as the baseline \tauhadvis
selection. In the \hadhad channel events are selected employing
single- and di-\tauhadvis triggers which also employ identification
algorithms to reduce trigger-rates at the HLT. These algorithms are
developed to be similar to their counterparts in the \tauhadvis
reconstruction\todo{Mention that they might diverge to some extend
  esp.\ given we now use RNN?}. Differences between the identification
at the HLT and during the offline reconstruction are expected and due
to limitations in the read-out of the detector and the time available
to make a decision on whether the event is accepted or rejected.

The effect of this two-stage selection of \tauhadvis based on
identification criteria, first at the HLT and then during offline
reconstruction, needs to be taken into account when measuring
corrections to the \tauhadvis misidentification efficiencies. The top
control region, which is collected using single electron and muon
triggers, allows to estimate the corrections for \tauhadvis without
any requirements at the HLT.

The requirements at the HLT can be emulated by requiring that the
event passes appropriately chosen single-\tauhadvis triggers. The
triggers with the lowest thresholds on the \tauhadvis \pT are chosen
that use the same chain of algorithms\todo{Energy scale,
  Identification} as is used by the trigger-selection in the signal
region of the \hadhad channel. Generally, only prescaled versions of
these triggers were available during data-taking but the trigger
decision can be recalculated in retrospect
(\textit{resurrected}). Three sub-regions of the top control region
are defined by requiring that the events pass the decision of the
resurrected triggers outlined in~\Cref{tab:triggers_ttbar_fake_sf} and
that the reconstructed \tauhadvis is geometrically matched to the
\tauhadvis candidate at the HLT (off the corresponding \tauhadvis
trigger leg).

\begin{table}[htbp]
  \centering

  \begin{tabular}{lp{7cm}p{5cm}}
    \toprule
    Offline ID & HLT chain & Relevance \\
    \midrule
    loose      & -- & {Subleading \tauhadvis candidates for events selected by single-\tauhadvis triggers.} \\
               && \\
    loose      & \verb|HLT_tau25_medium1_tracktwo| & {\tauhadvis candidates selected by di-\tauhadvis triggers in 2015-2017.} \\
               && \\
    loose      & \verb|HLT_tau25_medium1_tracktwoEF| & {\tauhadvis candidates selected by di-\tauhadvis triggers in 2018 until period K.} \\
               && \\
    loose & \verb|HLT_tau25_medium1_tracktwoEF| \par \textbf{or} \par \verb|HLT_tau25_mediumRNN_tracktwoMVA|  & {\tauhadvis candidates selected by di-\tauhadvis triggers in 2018 from period K.}\\
    \bottomrule
  \end{tabular}

  \todo[inline]{Describe the identification cuts applied at the HLT?}

  \caption{Combinations of \tauhadvis identification algorithms at the
    HLT and the offline reconstruction.}
  \label{tab:triggers_ttbar_fake_sf}
\end{table}


The available dataset for the measurement is as follows:
\begin{itemize}

\item \verb|HLT_tau25_medium1_tracktwo|: \SI{139}{\ifb}

\item \verb|HLT_tau25_medium1_tracktwoEF|: \SI{58}{\ifb}

\item \verb|HLT_tau25_mediumRNN_tracktwoMVA|: \SI{37}{\ifb}

\end{itemize}

The measurement regions are subdivided by the decay mode of the
\tauhadvis (1-prong or 3-prong) and in bins of \tauhadvis \pT:
\begin{itemize}
\item 1-prong \tauhadvis with $\pT / \si{\GeV}$: $[25, 30)$, $[30, 35)$,
  $[35, 40)$, $[40, 45)$, $[45, 55)$, $[55, 70)$, $[70, \infty)$

\item 3-prong \tauhadvis with $\pT / \si{\GeV}$: $[25, 30)$, $[30, 40)$,
  $[40, 50)$, $[50, 70)$, $[70, \infty)$
\end{itemize}
these regions are chosen such that their size and allows for a
determination of corrections with limited impact of statistical
uncertainty while allowing to extract potential \pT dependencies of
the correction.


% Fit model
Fit \mTW

Backgrounds:\\
\ttbar and \ttbar with fake \tauhad\\
single top \\
V+jets \\
Multi-jet is neglected

Regions \& normalisation factors



%%% Local Variables:
%%% mode: latex
%%% TeX-master: "../../phd_thesis"
%%% End:
