In the fully-hadronic \hadhad channel, \ttbar with \faketauhad are a
significant background contribution. In contrast to \faketauhad from
multi-jet where both reconstructed \tauhad candidates are originating
from jets, in $\ttbar \ra \Pbottom \PWp \APbottom \PWm$ frequently
only a single \tauhadvis candidate originates from jets of hadronic
decays of the \PW.

The quality of the modelling of \faketauhad in simulation is generally
unknown and needs to be checked. The pre-fit distributions\todo{Show?}
showed no significant indication of large mismodelling. However, a
data-driven method is still employed to get an idea about the level of
agreement between data and MC and to provide uncertainties on these
contributions.

Given the reasonable agreement of the Monte Carlo simulation with the
data, an approach of correcting the misidentification efficiencies in
simulation using a data-driven measurement is chosen. Two approaches
were investigated, one being the direct measurement of the
misidentification efficiency
\begin{align*}
  \text{Fake rate:}\quad \varepsilon_\text{mis-ID}^\text{data} = \frac{N_\text{post-ID}^\text{data}}{N_\text{pre-ID}^\text{data}}
\end{align*}
and the other being a measurement of the misidentification efficiency
relative to the one in simulation
\begin{align*}
  \text{Fake scale factor:}\quad
  \text{SF} =
  \frac{\varepsilon_\text{mis-ID}^\text{data}}{\varepsilon_\text{mis-ID}^\text{MC}} =
  \frac{N_\text{post-ID}^\text{data} / N_\text{pre-ID}^\text{data}}{N_\text{post-ID}^\text{MC} / N_\text{pre-ID}^\text{MC}}
  = \frac{N_\text{post-ID}^\text{data}}{N_\text{post-ID}^\text{MC}} \cdot \frac{N_\text{pre-ID}^\text{MC}}{N_\text{pre-ID}^\text{data}}
\end{align*}
The second term is completely independent of the modelling of the
\tauhadvis identification efficiency and describes the overall
modelling of \ttbar in Monte Carlo simulation. This term is assumed to
be a constant overall normalisation difference between MC and
data. Differential shape differences will be accounted for in the
systematic uncertainties on \ttbar modelling when extracting the scale
factors.

\todo[inline]{How would these be applied any differently? What are the
  advantages of either method?}


\subsubsection{Measurement of scale factors}

The transverse mass of the \PW boson is used as a variable
discriminating between. The main idea is to distinguish between
semi-leptonic and di-leptonic \ttbar since two true taus would be
expected in di-leptonic and jets faking taus in semi-leptonic. The
all-hadronic mode is negligible due to the presence of one electron or
muon.

To distinguish between semi-leptonic and di-leptonic \ttbar, one can try to reconstruct the transverse mass of the \PW boson
\begin{align*}
  \mTW = \sqrt{\left( | \myvec{p}_{\text{T}, \ell} | + | \pTmiss | \right)^2
               - \myvec{p}_{\text{T}, \ell} \cdot \pTmiss}
\end{align*}
where $\myvec{p}_{\text{T}, \ell}$ and \pTmiss are the vectors of the
lepton ($e$ or $\mu$) momentum and missing transverse momentum in the
transverse plane. For semi-leptonic \ttbar the event rate drops
significantly beyond \SI{100}{\GeV} while for di-leptonic \ttbar, due
to the presence of additional neutrinos, the transverse mass extends
to larger values.

%%% Local Variables:
%%% mode: latex
%%% TeX-master: "../../phd_thesis"
%%% End:
