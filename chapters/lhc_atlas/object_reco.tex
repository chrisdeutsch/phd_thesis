\section{Reconstruction of Collision Events in the ATLAS Detector}%
\label{sec:object_reco_at_atlas}

\todo[inline]{Introductory sentence}


\subsection{Tracking and Vertexing}

The reconstruction of charged particle trajectories is referred to as
tracking. This section focuses on tracking using information from the ID only
and is not concerned with the extension of tracking to the MS. The inputs to the
tracking algorithms of the ATLAS experiment are \emph{space-points} from the
pixel and SCT detector, and \emph{drift circles} from the TRT. Space-points are
measurements of location in three-dimensional space obtained by clustering the
charge signals of adjacent segments in the pixel and SCT detectors. Drift
circles are measurements of distance from the anode wires of individual
straw-tubes in the TRT. Two different tracking algorithms are used by the ATLAS
experiment, \emph{inside-out} and \emph{outside-in} tracking:
\begin{description}

\item[Inside-out algorithm] The inside-out algorithm starts by finding track
  seeds in the pixel and SCT detector. Track seeds are short track candidate
  segments constructed from three space-points in the ID. These track seeds are
  used as a starting point for a combinatorial Kalman filter~\todo{citation}
  that iteratively extends the track candidate with space-points from the
  inside-out. An ambiguity resolution step, see for example
  Ref.~\cite{PERF-2015-08}, is applied on all track candidates to remove tracks
  from combinations of unrelated space-points and duplicated tracks. Moreover, a
  precision fit is performed...  \todo[inline]{TRT extension?}

\item[Outside-in algorithm] The outside-in algorithm is used to recover
  efficiencies for non-promptly produced charged particles (e.g.\ secondary
  particles from $b$-hadron decays) which might not have a corresponding track
  seed.

\end{description}

% https://cds.cern.ch/record/2018442/files/pdf.pdf

\cite{PERF-2015-08}

\cite{Salzburger:2015sgq}

% - formation of localised clusters of charge depositions in the silicon detectors
% -> three-dimensional space-points

% - drift circles for the TRT

% - inside-out:
% 1. Seeding from three space points
% 2. Combinatorial Kalman filter builds first track candidates
% 3. Resolution of ambiguities and application of quality cuts
% 4. Track fit:

% - outside-in:
% -> If no track seed exists (e.g. tracks from secondary vertices) -> start in the TRT and go inwards


Tracks are described by five parameters (parameterisation at the perigee --
point of closest approach): $d_0$, $z_0$, $\phi$, $\theta$, $q / p$



\subsection{Topological Clustering of Energy in Calorimeter Cells}

\subsection{Electrons}

\subsection{Muons}

\subsection{Jets and $b$-tagging}

\subsection{Hadronic Decays of Tau Leptons}

\begin{figure}[htb]
  \begin{subfigure}[b]{0.47\textwidth}
    \centering

    \includegraphics{figs/tauid/tau_decay_feynman}

    \vspace*{3em}
    \subcaption{a}%
    \label{fig:tau_feynman}
  \end{subfigure}\hfill
  \begin{subfigure}[b]{0.47\textwidth}
    \centering

    \begin{overpic}[scale=0.9]{figs/tauid/tau_branching_pie_chart}
      \put (31, 83) {$\pi^- \nu_\tau$}
      \put (-5.5, 45) {$\pi^- \pi^0 \nu_\tau$}
      \put (16, 7) {$\pi^- 2 \pi^0 \nu_\tau$}
      \put (40.5, 2) {$2 \pi^- \pi^+ \nu_\tau$}
      \put (65, 6.5) {$2 \pi^- \pi^+ \pi^0 \nu_\tau$}
      \put (76.5, 15.5) {other}
      \put (70, 77.5) {$e^- \bar{\nu}_e \nu_\tau$}
      \put (88.5, 41.5) {$\mu^- \bar{\nu}_\mu \nu_\tau$}
    \end{overpic}

    \subcaption{}%
    \label{fig:tau_branching_ratios}
  \end{subfigure}
  \caption{Decay and branching ratios of the tau
    lepton. Charge-conjugate decay modes are omitted.}
\end{figure}


\subsubsection{Seed Jet}

Seeded with AntiKt 0.4 jets on TopoClusters at the LC scale.

\subsubsection{Tau Vertex Association}

TJVA

\subsubsection{Track Association}

\cite{duschinger}

\todo[inline]{Make sure to point out the difference between ``tau
  tracks'' and all other tracks.}

\subsubsection{Energy Calibration}

MVA TES

\subsubsection{Electron Veto}
\subsubsection{Tau Identification}

\subsection{Missing Transverse Energy}%
\label{sec:atlas_met}


%%% Local Variables:
%%% mode: latex
%%% TeX-master: "../../phd_thesis"
%%% End:
