\section{The ATLAS Detector}%
\label{sec:atlas}

The ATLAS detector~\cite{PERF-2007-01}, shown in
\Cref{fig:atlas_detector_overview}, is a cylindrical particle detector
surrounding the LHC beamline at one of the IPs. The detector covers most of the
solid angle around the IP to ensure that all detectible particles from
collisions can be observed. The central part of the ATLAS detector is referred
to as the \emph{barrel}, while the two sections covering solid angles close to
the LHC beamline are referred to as the \emph{endcaps}. Different layers of
detector technologies are concentrically arranged around the IP that, in
conjunction, allow to detect and identify different types of particles, enabling
an almost full interpretation of collision events.

\begin{figure}[htbp]
  \centering

  \includegraphics[width=0.76\textwidth]{atlas/atlas_overview}

  \caption{Overview of the ATLAS detector. Image taken from
    Ref.~\cite{Pequenao:1095924}.}%
  \label{fig:atlas_detector_overview}
\end{figure}

The ATLAS experiment uses a right-handed cartesian coordinate system with the
origin being located in the centre of the detector at the nominal IP. The axes
of the coordinate system are given as follows: the $x$-axis points to the centre
of the LHC, the $y$-axis points upwards, and the $z$-axis points along the LHC
beamline. The plane spanned by the $x$- and $y$-axes is referred to as the
transverse plane. A spherical coordinate system is used to specify directions in
three-dimensional space. The azimuthal angle, $\phi$, of this coordinate system
is defined as the angle in the transverse plane measured with respect to the
$x$-axis, and the polar angle, $\theta$, being the angle with respect to the
$z$-axis. With these coordinate systems, transverse momenta and energies are
defined as $\pT = \sqrt{p_x^2 + p_y^2} = p \sin\theta$ and $\ET = E \sin\theta$,
respectively. At hadron colliders, the polar angle is frequently given in terms
of the pseudorapidity $\eta$, which is defined as
\begin{align*}
  \eta = - \ln\tan\left( \frac{\theta}{2} \right) \,\text{.}
\end{align*}
Similarly, the angular separation between two particles is defined as
\begin{align*}
  \Delta R = \sqrt{\Delta \eta^2 + \Delta \phi^2} = \sqrt{(\eta_2 - \eta_1)^2 +
  (\phi_2 - \phi_1)^2} \,\text{,}
\end{align*}
where $\eta_1$ and $\eta_2$ are the pseudorapidities and $\phi_1$ and $\phi_2$
the azimuthal angle of both particles, respectively.

The main components of the ATLAS detector, going from the IP outwards, are the
\emph{Inner Detector} (ID) used for measuring the trajectories of charged
particles, the \emph{Calorimeters} used to destructively measure the energy of
most charged and neutral particles, and the \emph{Muon Spectrometer} (MS) used
to measure the trajectories of muons that can pass the calorimeters. Particles
in the ID are bent in the transverse plane due to a magnetic field pointing
along the $z$-axis that is produced by a superconducting solenoid surrounding
the ID.  The MS is surrounded by a superconducting toroid magnets that bend the
trajectories of muons in or against the $z$-direction. The bending of charged
particle trajectories in the ID and MS allows to determine the momenta of the
particles. The following sections summarise the most important parts of the
ATLAS detector.


\subsection{The Inner Detector}

The ID, shown in~\Cref{fig:atlas_inner_detector},

non-destructive measurements of charged particle trajectories of up to
$|\eta| < 2.5$.

Different detector technologies are used due to the increasing particle flux closest to

\begin{figure}[htbp]

  \begin{subfigure}[b]{0.55\textwidth}
    \includegraphics[width=\textwidth]{atlas/atlas_indet_1}%
    \subcaption{}
  \end{subfigure}\hfill%
  \begin{subfigure}[b]{0.45\textwidth}
    \includegraphics[width=\textwidth, trim=0 2.5cm 0 2cm, clip]{atlas/atlas_indet_2}%
    \subcaption{}
  \end{subfigure}

  \caption{Schematic view of the Inner Detector including the two endcaps (b). A
    zoomed in view of the barrel region is given in (b). The Insertable B-Layer
    (IBL), which was introduced into the ATLAS detector after Run~1 of the LHC,
    is not displayed. The IBL is located closest to the beampipe at a radius of
    $r = \SI{33.5}{\milli\metre}$~\cite{PIX-2018-001}. Images taken from
    Ref.~\cite{Pequenao:1095926}.}%
  \label{fig:atlas_inner_detector}
\end{figure}


\subsection{Calorimeters}

\begin{figure}[htbp]
  \centering

  \includegraphics[width=0.65\textwidth]{atlas/atlas_calo}

  \caption{Calorimeters. Image taken from Ref.~\cite{Pequenao:1095927}.}%
  \label{fig:atlas_calorimeters}
\end{figure}


\subsection{Muon Spectrometer}

\begin{figure}[htbp]
  \centering

  \includegraphics[width=0.65\textwidth]{atlas/atlas_muon}

  \caption{Muon subsystems. Image taken from Ref.~\cite{Pequenao:1095929}.}%
  \label{fig:atlas_muon_system}

  \todo[inline]{Is this really needed?}
\end{figure}

\subsection{The ATLAS Trigger System}

%%% Local Variables:
%%% mode: latex
%%% TeX-master: "../../phd_thesis"
%%% End:
