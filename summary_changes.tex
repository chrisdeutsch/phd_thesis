\documentclass[fontsize=11pt, paper=a4]{scrartcl}
\usepackage[twoside=false, texlive=2020]{ubonn-thesis}
\usepackage{thesis_defs}
\usepackage{atlasmisc}
\usepackage{atlasparticle}
\usepackage{atlasunit}
\usepackage[margin=2.5cm]{geometry}

\title{Search for Higgs Boson Pair Production in the\\$\bbbar\tautau$ Final State
  with the ATLAS\\Experiment at the LHC}
\subtitle{Synopsis of Editorial Changes}
\author{Christopher Deutsch}
\date{}

\begin{document}
\maketitle
\thispagestyle{empty}


% \textbf{Grammatical errors:}
% \begin{itemize}

% \item 3 occurances of ``with few modifications'' changed to ``with \textbf{a} few
%   modifications'' (pp.\ 49, 100, 177)

% \item ``The sample of simulated events is normalised using \textbf{an} $\pp \to \HH$
%   cross section of \SI{31.05}{\femto\barn} \dots'' changed to ``The sample of
%   simulated events is normalised using \textbf{a} $\pp \to \HH$ cross section of
%   \SI{31.05}{\femto\barn} \dots'' (p.\ 82)

% \end{itemize}


% \textbf{Edits for clarity:}
% \begin{itemize}

% \item Minor change in sentence structure for footnote on p.\ 131:

%   ``The fraction of events with a misreconstructed $H \to \bbbar$ or
%   $H \to \tautau$ candidate, i.e.\ $b$-jet candidates or \tauhadvis not being
%   matched to Higgs boson decay products at generator-level, are about
%   \SI{2}{\percent} and \SI{0.2}{\percent} \textbf{in SR of the \hadhad channel},
%   respectively.'' changed to ``\textbf{In the SR of the \hadhad channel,} the
%   fraction of events with a misreconstructed $H \to \bbbar$ or $H \to \tautau$
%   candidate, i.e.\ $b$-jet candidates or \tauhadvis not being matched to Higgs
%   boson decay products at generator-level, are about \SI{2}{\percent} and
%   \SI{0.2}{\percent}, respectively.''

%   The intent is to clarify that \SI{2}{\percent} refers to the
%   misreconstruction rate of the $H \to \bbbar$ candidate and \SI{0.2}{\percent}
%   to the misreconstruction rate of the $H \to \tautau$ candidate.

% \item ``The precise measurement of charged-particle tracks is important to
%   reconstruct the primary vertex (PV) of the hard interaction with \textbf{high}
%   spatial resolution.'' changed to ``The precise measurement of charged-particle
%   tracks is important to reconstruct the primary vertex (PV) of the hard
%   interaction with \textbf{good} spatial
%   resolution.'' (p.\ 33)

%   The intent is to remove any potential ambiguity surrounding the term ``high
%   spatial resolution''. This clarification is done because the terms ``low'' and
%   ``high'' can have varying implications depending on the specific metric used
%   to quantify resolution.
% \end{itemize}

\textbf{Edits to correct grammatical errors (changes are marked in bold):}\\
\begin{center}
  \begin{tabular}{p{0.45\textwidth}@{\hskip 0.05\textwidth}p{0.45\textwidth}}
  \toprule
  Submitted version & After editorial changes\\
  \midrule

  p.\ 9: &\\
  An appropriate weak isospin grouping of SM fermions is chosen in anticipation of \textbf{a} $SU(2)$ symmetry. &
  An appropriate weak isospin grouping of SM fermions is chosen in anticipation of \textbf{an} $SU(2)$ symmetry.\\\\


  p.\ 10: &\\
  Since the electric charge differs between the upper and lower component of the $SU(2)_{\text{L}}$ doublets, \textbf{a} $SU(2)_{\text{L}}$ transformation would violate the $U(1)_Q$ symmetry of QED. Therefore, the weak hypercharge is defined such that both components of \textbf{a} $SU(2)_{\text{L}}$ doublet have the same value of $Y$. &
  Since the electric charge differs between the upper and lower component of the $SU(2)_{\text{L}}$ doublets, \textbf{an} $SU(2)_{\text{L}}$ transformation would violate the $U(1)_Q$ symmetry of QED. Therefore, the weak hypercharge is defined such that both components of \textbf{an} $SU(2)_{\text{L}}$ doublet have the same value of $Y$.\\\\

  p.\ 12: &\\
  The aim of the BEH mechanism is to embed the doublet of complex scalar fields into the electroweak theory with \textbf{a} $SU(2)_{\text{L}}\otimes U(1)_Y$ symmetry.&
  The aim of the BEH mechanism is to embed the doublet of complex scalar fields into the electroweak theory with \textbf{an} $SU(2)_{\text{L}}\otimes U(1)_Y$ symmetry.\\\\

  p.\ 13: &\\
  While the full Lagrangian still possesses \textbf{a} $SU(2)_{\text{L}} \otimes U(1)_Y$ symmetry, the spontaneous choice of a vacuum state with non-vanishing VEV appears to break the symmetry, a process referred to as \emph{spontaneous symmetry breaking}. &
  While the full Lagrangian still possesses \textbf{an} $SU(2)_{\text{L}} \otimes U(1)_Y$ symmetry, the spontaneous choice of a vacuum state with non-vanishing VEV appears to break the symmetry, a process referred to as \emph{spontaneous symmetry breaking}.\\\\

  p.\ 49: &\\
  The \textsc{HistFactory} model is introduced in the following, adopting the notation of Ref.~[140] with few modifications. &
  The \textsc{HistFactory} model is introduced in the following, adopting the notation of Ref.~[140] with \textbf{a} few modifications.\\

  \bottomrule
\end{tabular}
\end{center}

\begin{center}
  \begin{tabular}{p{0.45\textwidth}@{\hskip 0.05\textwidth}p{0.45\textwidth}}
  \toprule
  Submitted version & After editorial changes\\
    \midrule

  p.\ 82: &\\
  The sample of simulated events is normalised using \textbf{an} $\pp \to \HH$ cross section of \SI{31.05}{\femto\barn} at $\text{NNLO}_{\text{FTapprox}}$~[46], which is a combination of the full-theory prediction at NLO with additional NNLO corrections derived in the large top-quark mass limit. &
  The sample of simulated events is normalised using \textbf{a} $\pp \to \HH$ cross section of \SI{31.05}{\femto\barn} at $\text{NNLO}_{\text{FTapprox}}$~[46], which is a combination of the full-theory prediction at NLO with additional NNLO corrections derived in the large top-quark mass limit.\\\\

  p.\ 93: &\\
  An event is considered to pass the STT selection if the following conditions are true: the event was selected by \textbf{a} STT,\dots &
  An event is considered to pass the STT selection if the following conditions are true: the event was selected by \textbf{an} STT, \dots\\\\

  p.\ 97: &\\
  This tagger is replaced by the \textsc{DL1r} tagger~[129] operating at \textbf{the the} \SI{77}{\percent} efficiency working point. &
  This tagger is replaced by the \textsc{DL1r} tagger~[129] operating at \textbf{the} \SI{77}{\percent} efficiency working point.\\\\

  p.\ 100: &\\
  This approach is adopted with few modifications~[258] from the previous publication in this channel~[184], which built on findings from searches for $VH$ ($\PHiggs \to \bbbar$) production~[236]. &
  This approach is adopted with \textbf{a} few modifications~[258] from the previous publication in this channel~[184], which built on findings from searches for $VH$ ($\PHiggs \to \bbbar$) production~[236]. \\\\

  p.\ 177: &\\
  The reinterpretation adopts the statistical framework presented in Section~6.7 with few modifications. &
  The reinterpretation adopts the statistical framework presented in Section~6.7 with \textbf{a} few modifications.\\

  \bottomrule
\end{tabular}
\end{center}

\clearpage

\textbf{Edits for clarity:}\\
\begin{center}
  \begin{tabular}{p{0.45\textwidth}@{\hskip 0.05\textwidth}p{0.45\textwidth}}
  \toprule
  Submitted version & After editorial changes\\
  \midrule

  p.\ 33: &\\
  The precise measurement of charged-particle tracks is important to reconstruct the primary vertex (PV) of the hard interaction with high spatial resolution. &
  The precise measurement of charged-particle tracks is important to reconstruct the primary vertex (PV) of the hard interaction with good spatial resolution.\\\\

  p.\ 131 (footnote 12): &\\
  The fraction of events with a misreconstructed $H \to \bbbar$ or $H \to \tautau$ candidate, i.e.\ $b$-jet candidates or \tauhadvis not being matched to Higgs boson decay products at generator-level, are about \SI{2}{\percent} and \SI{0.2}{\percent} in SR of the \hadhad channel, respectively. &
  In the SR of the \hadhad channel, the fraction of events with a misreconstructed $H \to \bbbar$ or $H \to \tautau$ candidate, i.e.\ $b$-jet candidates or \tauhadvis not being matched to Higgs boson decay products at generator-level, are about \SI{2}{\percent} and \SI{0.2}{\percent}, respectively.\\\\


  p.\ 82 (added footnote): &\\
  The theoretical uncertainties on the cross section prediction are $^{\hspace{0.25pt}+\hspace{0.25pt}\phantom{0}6\,\%}_{-23\,\%}$ from scale variations and the treatment of the finite top-quark mass~[47] and $\pm\SI{3}{\percent}$ from uncertainties on PDFs and $\alphas$~[48]. &
 The theoretical uncertainties on the cross section prediction are $^{\hspace{0.25pt}+\hspace{0.25pt}\phantom{0}6\,\%}_{-23\,\%}$ from scale variations and the treatment of the finite top-quark mass$^2$~[47] and $\pm\SI{3}{\percent}$ from uncertainties on PDFs and $\alphas$~[48].\newline
 {\footnotesize $^2$ This uncertainty is defined by a comparison of the pole mass and the $\overline{\text{MS}}$ scheme for the mass of virtual top-quarks.}\\

  \bottomrule
\end{tabular}
\end{center}

% TODO: pt jet seed in Tau-ID

% TODO: Why ggF xsec uncertainty asymmetric?

% TODO: Why VBF xsec uncertainty so small?

% TODO: ROC-AUC values seem high for the plot. Answer: The ROC-AUC cannot be
% read off this plot that easily (only a crude approximation after further
% computation). Also the y-axis is given in log-scale which might suggest that
% there is more overlap between background & signal than there actually is.

\end{document}

%%% Local Variables:
%%% mode: latex
%%% TeX-master: "summary_changes.tex"
%%% TeX-master: t
%%% End:
